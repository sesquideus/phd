\subsection{Charakteristika vedeckých cieľov
projektu}\label{charakteristika-vedeckuxfdch-cieux13eov-projektu}

eteoroidných častíc na Zem je dôležité pre poznanie

Dominantným zdrojom meteoroidov sú malé telesá Slnečnej sústavy, najmä
kométy a blízkozemské asteroidy, ktoré pri preletoch blízko Slnka
uvoľňujú drobné telieska, prípadne fragmentujú pri vzájomných zrážkach
alebo tesných priblíženiach k Slnku. Mierne odlišnosti počiatočných
rýchlostí telies, gravitačné perturbácie pochádzajúce od planét a
negravitačné efekty pôsobiace na častice spôsobujú, že ich dráhy sa na
dlhých časových škálach menia a diverzifikujú. Na škálach stoviek až
tisícov rokov vytvoria široký prstenec okolo orbity pôvodného materského
telesa.

Tieto častice nie je kvôli ich malým rozmerom pri súčasnom stave
techniky možné pozorovať priamo v otvorenom medziplanetárnom priestore.
Naše znalosti o nich pochádzajú najmä zo sledovania ich interakcie s
vrchnými vrstvami atmosféry Zeme, kde ich pozorujeme ako meteory.

Potreba presných dát o priestorovej koncentrácii a toku meteoroidných
častíc v okolí Zeme a jej dráhy sa bude s narastajúcim vedeckým a
komerčným využívaním kozmického priestoru neustále zvyšovať. Detailné
poznanie dráhových charakteristík, početnosti a distribúcie veľkostí
častí je dôležité aj pre zhodnotenie ohrozenia povrchu Zeme. Objekty s
rozmermi viac ako 10 metrov predstavujú nebezpečenstvo a majú potenciál
spôsobiť značné škody, od lokálnych materiálnych škôd až po katastrofy
globálneho charakteru.

Primárnym zdrojom použitých dát sú kamery systému AMOS, ktorý bol
vyvinutý a je prevádzkovaný Oddelením astronómie a astrofyziky KAFZM
FMFI UK (Zigo, Tóth, and Kalmančok 2013) (Tóth and others 2015). Týmto
projektom nadväzujeme na diplomovú prácu (Baláž 2018), v ktorej sme sa
zaoberali tokom meteoroidných častíc v hornej atmosfére. Ďalším krokom
je rozšírenie práce telies na

Vhodným nástrojom je paralelná \(N\)-body simulácia malých častíc.
Častice skúmaných rozmerov na seba pôsobia len zanedbateľne malými
silami a preto je možné simuláciu efektívne paralelizovať pomocou
grafickej karty (GPU).

V rámci projektu sa plánujeme zúčastniť na konferencii
\textbf{Meteoroids 2019}, ktorú v tomto roku organizuje Univerzita
Komenského v Bratislave, a prezentovať doterajšie výsledky.

(Baláž 2018)

\subsubsection*{Referencie}\label{referencie}
\addcontentsline{toc}{subsubsection}{Referencie}

\hypertarget{refs}{}
\hypertarget{ref-balaz2018}{}
Baláž, Martin. 2018. ``Determination of Total Meteoroid Flux in
Millimetre to Metre Size Range.'' Master's thesis, Bratislava, Slovakia:
Comenius University in Bratislava.

\hypertarget{ref-toth2015}{}
Tóth, Juraj, and others. 2015. ``All-sky Meteor Orbit System AMOS and
Preliminary Analysis of Three Unusual Meteor Showers.'' \emph{Planetary
and Space Science} 118: 102--6.
\url{https://doi.org/10.1016/j.pss.2015.07.007}.

\hypertarget{ref-zigo2013}{}
Zigo, Pavol, Juraj Tóth, and Dušan Kalmančok. 2013. ``All-Sky Meteor
Orbit System AMOS.'' In \emph{Proceedings of the International Meteor
Conference, 31st Imc, La Palma, Canary Islands, Spain, 2012}, edited by
M. Gyssens and P. Roggemans, 18--20.

\hypertarget{charakteristika-vedeckuxfdch-cieux13eov-projektu}{%
\subsection{Charakteristika vedeckých cieľov
projektu}\label{charakteristika-vedeckuxfdch-cieux13eov-projektu}}

Meteoroidy, čiže malé kamenné alebo železné telieska obiehajúce okolo
Slnka, sú jedným z najcennejších zdrojov informácií o pôvode a vývoji
Slnečnej sústavy. Za meteoroidy považujeme telesá s rozmermi v rozpätí
\SI{30}{\micro\metre} až \SI{1}{\metre} (Perlerin, 2018); pričom väčšie
telesá nazývame asteroidmi a pri menších telesách hovoríme o
medziplanetárnom prachu. Pri súčasnom stave techniky nie je možné
meteoroidy možné pozorovať priamo v otvorenom medziplanetárnom
priestore. Naše znalosti o nich pochádzajú najmä zo sledovania ich
interakcie s vrchnými vrstvami atmosféry Zeme. Ak dráha meteoroidu
križuje dráhu Zeme a teleso vstúpi do zemskej atmosféry, jeho kinetická
energia je dostatočná na roztavenie a ionizáciu materiálu. Vznikajúcu
svetelnú stopu následne môžeme pozorovať ako meteor.

Zdanlivý pohyb meteorov je oproti ostatným nebeským telesám veľmi
rýchly. Pravdepodobne aj preto boli meteory historicky považované skôr
za meteorologický, než astronomický úkaz, a ich kozmický pôvod bol
definitívne potvrdený až v roku 1798 (Czegka, 2000). Pozorovaním
meteorov zo zemského povrchu je možné zistiť, že sú často združené do
\emph{meteorických rojov}, teda zoskupení telies s podobnými
geocentrickými dráhami. Projekcia na oblohu vytvára ilúziu bodového
zdroja, čiže \emph{radiantu}.

Dominantným zdrojom meteoroidov sú malé telesá Slnečnej sústavy, najmä
kométy a blízkozemské asteroidy, ktoré pri preletoch blízko Slnka
uvoľňujú drobné telieska, prípadne fragmentujú pri vzájomných zrážkach.
Mierne odlišnosti počiatočných rýchlostí telies, gravitačné perturbácie
pochádzajúce od planét a negravitačné efekty pôsobiace na častice
spôsobujú, že ich dráhy sa na dlhých časových škálach pomaly menia a
diverzifikujú. Na škálach stoviek až tisícov rokov vytvoria široký
prstenec okolo pôvodnej orbity svojho materského telesa. Na ešte dlhších
škálach sa prstence rozpadajú a prispievajú do \emph{sporadického
pozadia}, teda zdroja rozptýlených meteoroidov na zdanlivo náhodných
dráhach (Jenniskens, 1998).

Mimo zemskej atmosféry tieto telieska predstavujú vážne nebezpečenstvo
pre ľudské misie a komerčné satelity. Potreba presných dát o
priestorovej koncentrácii a toku meteoroidných častíc v okolí Zeme a jej
dráhy sa bude s narastajúcim vedeckým a komerčným využívaním kozmického
priestoru neustále zvyšovať. Detailné poznanie dráhových charakteristík,
početnosti a distribúcie veľkostí častí je dôležité aj pre zhodnotenie
ohrozenia povrchu Zeme. Objekty s rozmermi viac ako 10 metrov
predstavujú nebezpečenstvo a majú potenciál spôsobiť značné škody, od
lokálnych materiálnych škôd až po katastrofy globálneho charakteru.

V práci sa zameriame na pôvod a dynamický vývoj prúdov meteoroidov od
ich vzniku až po zánik v~atmosfére. Cieľom projektu je vytvoriť ucelený
model ich priestorového rozloženia vo vnútornej Slnečnej sústave.
Vhodným nástrojom na výskum dráhovej dynamiky a evolučných ciest
meteorických rojov sú numerické \(N\)-body simulácie. Výsledok simulácie
je možné štatisticky porovnať s~observačnými dátami a následne
optimalizačnými metódami nájsť najlepšiu možnú zhodu. Takto získaná
virtuálna populácia dobre popisuje skutočné rozloženie častíc. Primárnym
zdrojom observačných dát na našom pracovisku sú kamery systému AMOS
(All-sky Meteor Orbit System), ktorý bol vyvinutý a je prevádzkovaný
Oddelením astronómie a astrofyziky KAFZM FMFI UK (Tóth et al., 2015;
Zigo et al., 2013). Počas riešenia projektu očakávame splnenie
nasledujúcich úloh:

\begin{itemize}
\tightlist
\item
  návrh \(N\)-body simulácie s využitím masívnej paralelizácie pomocou
  grafického procesora (GPU),
\item
  implementácia simulácie v jazyku Python alebo C++,
\item
  vývoj porovnávacích algoritmov a optimalizačných procedúr,
\item
  porovnanie výsledkov s inými modelmi, napríklad (Ryabova, 2013).
\end{itemize}

V rámci projektu sa taktiež plánujeme zúčastniť na konferencii
\textbf{Meteoroids 2019}, ktorú v tomto roku organizuje Univerzita
Komenského v Bratislave, a prezentovať dovtedy získané výsledky vo forme
príspevku alebo postera; prípadne tiež na konferencii \textbf{IMC 2019}
v Bollmannsruh v Nemecku.

\hypertarget{referencie}{%
\subsubsection*{Referencie}\label{referencie}}
\addcontentsline{toc}{subsubsection}{Referencie}

\hypertarget{refs}{}
\leavevmode\hypertarget{ref-czegka2000}{}%
Czegka, W., 2000. Lichenberg, Benzenberg, Brandes, and Their Meteor
Height Determination in 1798--1800: An Empirical Approach to Solove the
Meteorite Enigma. Meteoritics and Planetary Science Supplement 35, A45.

\leavevmode\hypertarget{ref-jenniskens1998}{}%
Jenniskens, P., 1998. On the dynamics of meteoroid streams. Earth,
Planets, and Space 50, 555--567.
\url{https://doi.org/10.1186/BF03352149}

\leavevmode\hypertarget{ref-imo-meteor}{}%
Perlerin, V., 2018. Definition of Terms in Meteor Astronomy (IAU).

\leavevmode\hypertarget{ref-ryabova2013}{}%
Ryabova, G., 2013. Modeling of meteoroid streams: The velocity of
ejection of meteoroids from comets (a review). Solar System Research 47,
219--238. \url{https://doi.org/10.1134/S0038094613030052}

\leavevmode\hypertarget{ref-toth2015}{}%
Tóth, J., Kornoš, L., Zigo, P., others, 2015. All-sky Meteor Orbit
System AMOS and preliminary analysis of three unusual meteor showers.
Planetary and Space Science 118, 102--106.
\url{https://doi.org/10.1016/j.pss.2015.07.007}

\leavevmode\hypertarget{ref-zigo2013}{}%
Zigo, P., Tóth, J., Kalmančok, D., 2013. All-Sky Meteor Orbit System
AMOS, v: Gyssens, M., Roggemans, P. (Ed.), Proceedings of the
International Meteor Conference, 31st IMC, La Palma, Canary Islands,
Spain, 2012. s. 18--20.

\subsection{Charakteristika vedeckých cieľov
projektu}\label{charakteristika-vedeckuxfdch-cieux13eov-projektu}

Meteory sú pozorované a, neboli však považované za súčasť astronómie.
Kozmický pôvod tohoto javu bol definitívne potvrdený až v roku 1798
(Czegka 2000). Meteory sú často združené do \emph{meteorických rojov},
teda zoskupení telies s podobnými dráhami, ktoré pri pozorovaní z
konkrétneho miesta na zemskom povrchu vytvárajú ilúziu bodového zdroja,
čiže \emph{radiantu}.

Dominantným zdrojom meteoroidov sú malé telesá Slnečnej sústavy, najmä
kométy a blízkozemské asteroidy, ktoré pri preletoch blízko Slnka
uvoľňujú drobné telieska, prípadne fragmentujú pri vzájomných zrážkach
alebo tesných priblíženiach k Slnku. Mierne odlišnosti počiatočných
rýchlostí telies, gravitačné perturbácie pochádzajúce od planét a
negravitačné efekty pôsobiace na častice spôsobujú, že ich dráhy sa na
dlhých časových škálach menia a diverzifikujú. Na škálach stoviek až
tisícov rokov vytvoria široký prstenec okolo orbity pôvodného materského
telesa; na ešte dlhších škálach sa prstence rozpadajú a prispievajú do
tzv. \emph{sporadického pozadia}, teda zdroja rozptýlených meteoroidov
na zdanlivo náhodných dráhach (Jenniskens 1998).

Meteoroidy nie je kvôli ich malým rozmerom pri súčasnom stave techniky
možné pozorovať priamo v otvorenom medziplanetárnom priestore. Naše
znalosti o nich pochádzajú najmä zo sledovania ich interakcie s vrchnými
vrstvami atmosféry Zeme. Ak dráha meteoroidu križuje dráhu Zeme a teleso
vstúpi do zemskej atmosféry, jeho kinetická energia je dostatočná na
roztavenie a ionizáciu materiálu. Tento jav následne môžeme pozorovať
ako meteor.

Mimo zemskej atmosféry tieto telieska predstavujú vážne nebezpečenstvo
pre ľudské misie a komerčné satelity. Potreba presných dát o
priestorovej koncentrácii a toku meteoroidných častíc v okolí Zeme a jej
dráhy sa s narastajúcim vedeckým a komerčným využívaním kozmického
priestoru bude neustále zvyšovať. Detailné poznanie dráhových
charakteristík, početnosti a distribúcie veľkostí častí je dôležité aj
pre zhodnotenie ohrozenia povrchu Zeme. Objekty s rozmermi viac ako 10
metrov predstavujú nebezpečenstvo a majú potenciál spôsobiť značné
škody, od lokálnych materiálnych škôd až po katastrofy globálneho
charakteru.

V práci sa zameriame na pôvod a dynamický vývoj prúdov meteoroidov od
ich vzniku až po zánik v atmosfére. Cieľom projektu je preskúmať
(\ldots{}) a vytvoriť ucelený model ich priestorového rozloženia vo
vnútornej Slnečnej sústave. Primárnym zdrojom dát na našom pracovisku sú
kamery systému AMOS, ktorý bol vyvinutý a je prevádzkovaný Oddelením
astronómie a astrofyziky KAFZM FMFI UK (Zigo, Tóth, and Kalmančok 2013)
(Tóth and others 2015). Vhodným nástrojom na výskum dráhovej dynamiky a
evolučných ciest meteorických rojov sú numerické \(N\)-body simulácie.

V rámci projektu sa plánujeme zúčastniť na konferencii
\textbf{Meteoroids 2019}, ktorú v tomto roku organizuje Univerzita
Komenského v Bratislave, a prezentovať doterajšie výsledky vo forme
príspevku alebo postera.

\subsubsection*{Referencie}\label{referencie}
\addcontentsline{toc}{subsubsection}{Referencie}

\hypertarget{refs}{}
\hypertarget{ref-czegka2000}{}
Czegka, W. 2000. ``Lichenberg, Benzenberg, Brandes, and Their Meteor
Height Determination in 1798--1800: An Empirical Approach to Solove the
Meteorite Enigma.'' \emph{Meteoritics and Planetary Science Supplement}
35 (July): A45.

\hypertarget{ref-jenniskens1998}{}
Jenniskens, P. 1998. ``On the Dynamics of Meteoroid Streams.''
\emph{Earth, Planets, and Space} 50 (June): 555--67.
\url{https://doi.org/10.1186/BF03352149}.

\hypertarget{ref-toth2015}{}
Tóth, Juraj, and others. 2015. ``All-sky Meteor Orbit System AMOS and
Preliminary Analysis of Three Unusual Meteor Showers.'' \emph{Planetary
and Space Science} 118: 102--6.
\url{https://doi.org/10.1016/j.pss.2015.07.007}.

\hypertarget{ref-zigo2013}{}
Zigo, Pavol, Juraj Tóth, and Dušan Kalmančok. 2013. ``All-Sky Meteor
Orbit System AMOS.'' In \emph{Proceedings of the International Meteor
Conference, 31st Imc, La Palma, Canary Islands, Spain, 2012}, edited by
M. Gyssens and P. Roggemans, 18--20.

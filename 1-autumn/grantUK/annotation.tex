\subsection{Anotácia}\label{anotuxe1cia}

Primárnym cieľom vedeckej práce je vytvoriť priestorový model rozloženia
a dynamiky malých meteoroidných častíc vo vnútornej Slnečnej sústave.
Ťažiskovou časťou práce je numerická simulácia, pomocou ktorej budeme
schopní vytvárať virtuálne meteoroidy, sledovať ich dynamiku vo
vnútornej Slnečnej sústave a interakciu so zemskou atmosférou. Výsledky
simulácie porovnáme s observačnými dátami získanými pomocou systému
celooblohových kamier AMOS.

Opakovanou variáciou parametrov simulácie až do dosiahnutia optimálnej
zhody s experimentálnymi dátami budeme schopní určiť skutočnú
priestorovú distribúciu a početnosť meteoroidov v oblasti orbity Zeme,
ako aj vyhodnotiť celkový početný a hmotnostný tok častíc dopadajúcich
na Zem. Porovnaním výsledkov s inými publikáciami dokážeme zistiť, do
akej miery sú staršie dáta získané priamou redukciou observačných dát
zaťažené výberovými efektami a prispejeme k tvorbe predstavy o populácii
meteoroidov vo vnútornej Slnečnej sústave.

V rámci riešenia projektu sa plánujeme zúčastniť na konferencii
Meteoroids 2019 a prezentovať doterajšie výsledky vo forme príspevku
alebo postera.

\subsection{Návrh metódy riešenia
projektu}\label{nuxe1vrh-metuxf3dy-rieux161enia-projektu}

Projekt nadväzuje na našu diplomovú prácu (Baláž 2018) a rozširuje
výsledky

Výsledky sú priamo porovnateľné s hodnotami získanými inými metódami,
napríklad vizuálnymi pozorovaniami vykonanými skupinou pozorovateľov,
priamou redukciou dát z kamerových systémov (AMOS (Zigo, Tóth, and
Kalmančok 2013) a CILBO (Koschny and others 2013)) alebo inými metódami
(Blaauw, Campbell-Brown, and Kingery 2016). Špeciálnu pozornosť venujeme
porovnaniu s analýzami konkrétnych pozorovacích nocí, publikovanými
napríklad portálom \texttt{MeteorFlux.io} (Molau and Barentsen 2018).

V nasledujúcej fáze navrhujeme metódu, ktorou je možné získané výsledky
rozšíriť na oblasť vnútornej Slnečnej sústavy, resp. oblasť dráhy Zeme,
keďže pozemské pozorovanie meteorov neumožňujú vykonávať priame merania
vo vzdialenejších oblastiach. Dáta je však možné priamo extrapolovať do
blízkeho okolia. Základom metódy je numerická \(N\)-body simulácia s
využitím masívnej paralelizácie pomocou GPU.

Simulácia vytvára jednotlivé častice pri známych materských telesách
meteoroidov a následne numerickou integráciu pohybových rovníc určuje
ich budúcu polohu. Mierne odlišnosti počiatočných rýchlostí, gravitačné
perturbácie pochádzajúce planét a negravitačné efekty pôsobiace na
častice spôsobujú, že dráhy častíc sa na dlhých časových škálach líšia,
až vytvoria široký prstenec okolo orbity pôvodného materského telesa.

Celkové silové pôsobenie je dané najmä gravitačným pôsobením Slnka ako
centrálneho telesa, rušiacimi gravitačnými vplyvmi planét a krátkodobo
aj pôsobením materského objektu. Pre častice s veľmi malými rozmermi sú
dôležité aj negravitačné vplyvy, najmä tlak slnečného žiarenia a
Poynting-Robertsonov efekt ({\textbf{???}}). Jednotlivé častice sú
nezávislé a ich vzájomné silové pôsobenie môžeme úplne zanedbať. Pri
využití masívnej paralelizácie pomocou GPU očakávame podstatné zvýšenie
výpočtového výkonu na úroveň miliárd integračných krokov za sekundu, čo
umožní simulovať milióny častíc ({\textbf{???}}). Vysoký počet
simulovaných častíc je dôležitý, keďže k zrážkam so Zemou dochádza
pomerne zriedkavo, kým pre účely štatistického vyhodnotenia súboru
potrebujeme získať dostatočne početnú populáciu.

Ak počas integrácie dôjde ku kolízii niektorej z častíc so Zemou, daná
častica je označená ako spozorovaná. Sumárny štatistický súbor všetkých
týchto častíc je po aplikácii výberových efektov možné porovnať s
pozemskými pozorovaniami a určiť zhodu s experimentálnymi dátami.
Variáciou parametrov simulácie a minimalizáciou odchýlok sme schopní
určiť skutočnú distribúciu a pôvodnú dráhu telies.

\subsubsection*{Referencie}\label{referencie}
\addcontentsline{toc}{subsubsection}{Referencie}

\hypertarget{refs}{}
\hypertarget{ref-balaz2018}{}
Baláž, Martin. 2018. ``Determination of Total Meteoroid Flux in
Millimetre to Metre Size Range.'' Master's thesis, Bratislava, Slovakia:
Comenius University in Bratislava.

\hypertarget{ref-blaauw2016}{}
Blaauw, R. C., M. Campbell-Brown, and A. Kingery. 2016. ``Optical Meteor
Fluxes and Application to the 2015 Perseids.'' \emph{Monthly Notices of
the Royal Astronomical Society} 463 (1): 441--48.
\url{https://doi.org/10.1093/mnras/stw1979}.

\hypertarget{ref-koschny2013}{}
Koschny, Detlef, and others. 2013. ``A Double-Station Meteor Camera
Set-up in the Canary Islands.'' \emph{Geoscientific Instrumentation
Methods and Data Systems} 2: 339--48.

\hypertarget{ref-meteorflux}{}
Molau, Sirko, and Geert Barentsen. 2018. ``MeteorFlux.io.''
\url{http://meteorflux.io/}.

\hypertarget{ref-zigo2013}{}
Zigo, Pavol, Juraj Tóth, and Dušan Kalmančok. 2013. ``All-Sky Meteor
Orbit System AMOS.'' In \emph{Proceedings of the International Meteor
Conference, 31st Imc, La Palma, Canary Islands, Spain, 2012}, edited by
M. Gyssens and P. Roggemans, 18--20.

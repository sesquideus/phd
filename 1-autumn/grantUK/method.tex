\subsection{Návrh metódy riešenia
projektu}\label{nuxe1vrh-metuxf3dy-rieux161enia-projektu}

Projekt nadväzuje na našu diplomovú prácu (Baláž 2018) a rozširuje
platnosť výsledkov

V nasledujúcej fáze navrhujeme metódu, ktorou je možné získané výsledky
rozšíriť na oblasť vnútornej Slnečnej sústavy, resp. oblasť dráhy Zeme,
keďže pozemské pozorovanie meteorov neumožňujú vykonávať priame merania
vo vzdialenejších oblastiach. Dáta je však možné priamo extrapolovať do
blízkeho okolia. Základom metódy je numerická \(N\)-body simulácia s
využitím masívnej paralelizácie pomocou GPU.

Simulácia vytvára jednotlivé častice pri známych materských telesách
meteoroidov a následne numerickou integráciu pohybových rovníc určuje
ich budúcu polohu.

Celkové silové pôsobenie je dané najmä gravitačným pôsobením Slnka ako
centrálneho telesa, rušiacimi gravitačnými vplyvmi planét a krátkodobo
aj pôsobením materského objektu. Pre častice s veľmi malými rozmermi sú
dôležité aj negravitačné vplyvy, najmä tlak slnečného žiarenia a
Poynting-Robertsonov efekt ({\textbf{???}}). Jednotlivé častice sú
nezávislé a ich vzájomné silové pôsobenie môžeme úplne zanedbať. Pri
využití masívnej paralelizácie pomocou GPU očakávame podstatné zvýšenie
výpočtového výkonu na úroveň miliárd integračných krokov za sekundu, čo
umožní simulovať milióny častíc ({\textbf{???}}). Vysoký počet
simulovaných častíc je dôležitý, keďže k zrážkam so Zemou dochádza
pomerne zriedkavo, kým pre účely štatistického vyhodnotenia súboru
potrebujeme získať dostatočne početnú populáciu.

Ak počas integrácie dôjde ku kolízii niektorej z častíc so Zemou, daná
častica je označená ako spozorovaná. Sumárny štatistický súbor všetkých
týchto častíc je po aplikácii výberových efektov možné porovnať s
pozemskými pozorovaniami a určiť zhodu s experimentálnymi dátami.
Variáciou parametrov simulácie a minimalizáciou odchýlok sme schopní
určiť skutočnú distribúciu a pôvodnú dráhu telies.

\subsubsection*{Referencie}\label{referencie}
\addcontentsline{toc}{subsubsection}{Referencie}

\hypertarget{refs}{}
\hypertarget{ref-balaz2018}{}
Baláž, Martin. 2018. ``Determination of Total Meteoroid Flux in
Millimetre to Metre Size Range.'' Master's thesis, Bratislava, Slovakia:
Comenius University in Bratislava.

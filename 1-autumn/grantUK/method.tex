\subsection{Návrh metódy riešenia
projektu}\label{nuxe1vrh-metuxf3dy-rieux161enia-projektu}

Projektom nadväzujeme na diplomovú prácu (Baláž 2018), v ktorej sme sa
zaoberali tokom meteoroidných častíc v hornej atmosfére. Ďalším krokom
je rozšírenie zamerania práce na pôvod a vývoj týchto telies, teda na
obdobie od ich uvoľnenia z materského telesa až po zánik v zemskej
atmosfére. Základom metódy je numerická \(N\)-body simulácia s využitím
masívnej paralelizácie pomocou GPU. Simulácia vytvára jednotlivé častice
pri známych materských telesách meteoroidov a následne numerickou
integráciu pohybových rovníc určuje ich budúcu polohu.

Celkové silové pôsobenie je dané najmä gravitačným pôsobením Slnka ako
centrálneho telesa, rušiacimi gravitačnými vplyvmi planét a krátkodobo
aj pôsobením materského objektu. Pre častice s veľmi malými rozmermi sú
dôležité aj negravitačné vplyvy, najmä tlak slnečného žiarenia a
Poynting-Robertsonov efekt. Jednotlivé častice sú nezávislé a ich
vzájomné silové pôsobenie môžeme úplne zanedbať. Pri využití masívnej
paralelizácie pomocou GPU očakávame podstatné zvýšenie výpočtového
výkonu na úroveň miliárd integračných krokov za sekundu, čo umožní
simulovať milióny častíc (Nguyen 2007). Vysoký počet simulovaných častíc
je dôležitý, keďže k zrážkam so Zemou dochádza pomerne zriedkavo, kým
pre účely štatistického vyhodnotenia súboru potrebujeme získať
dostatočne početnú populáciu.

Ak počas integrácie dôjde ku kolízii niektorej z častíc so Zemou, daná
častica je označená ako spozorovaná. Sumárny štatistický súbor všetkých
takýchto častíc je po aplikácii výberových efektov možné porovnať s
pozemskými pozorovaniami a určiť zhodu s experimentálnymi dátami.
Variáciou parametrov simulácie a minimalizáciou odchýlok sme schopní
určiť skutočnú distribúciu a pôvodnú dráhu telies. Opätovné spustenie
simulácie s optimálnymi hodnotami parametrov spolu so znalosťou dráhy
skutočného materského telesa nám umožnia identifikovať prúdy častíc a
predpovedať aktivitu meteorických rojov.

\subsubsection*{Referencie}\label{referencie}
\addcontentsline{toc}{subsubsection}{Referencie}

\hypertarget{refs}{}
\hypertarget{ref-balaz2018}{}
Baláž, Martin. 2018. ``Determination of Total Meteoroid Flux in
Millimetre to Metre Size Range.'' Master's thesis, Bratislava, Slovakia:
Comenius University in Bratislava.

\hypertarget{ref-nguyen2007}{}
Nguyen, Hubert. 2007. \emph{GPU Gems 3}. Addison-Wesley Professional.

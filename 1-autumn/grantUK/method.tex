\hypertarget{nuxe1vrh-metuxf3dy-rieux161enia-projektu}{%
\subsection{Návrh metódy riešenia
projektu}\label{nuxe1vrh-metuxf3dy-rieux161enia-projektu}}

Projektom nadväzujeme na diplomovú prácu (Baláž, 2018), v ktorej sme sa
zaoberali tokom meteoroidných častíc v hornej atmosfére. Prirodzeným
ďalším krokom je rozšírenie zamerania práce na pôvod a vývoj týchto
telies, teda na obdobie od ich uvoľnenia z materského telesa až po zánik
v zemskej atmosfére.

Základom metódy je numerická \(N\)-body simulácia s využitím masívnej
paralelizácie pomocou grafického procesora (GPU). V~simulácii budeme
vytvárať jednotlivé častice pri známych materských telesách meteoroidov
a následne numerickou integráciou pohybových rovníc určíme ich budúcu
polohu. Celkové silové pôsobenie je dané najmä gravitačným pôsobením
Slnka ako centrálneho telesa, rušiacimi gravitačnými vplyvmi planét a
krátkodobo aj pôsobením materského objektu. Pre častice s veľmi malými
rozmermi sú dôležité aj negravitačné vplyvy, najmä tlak slnečného
žiarenia a Poyntingov-Robertsonov efekt.

Jednotlivé častice sú nezávislé a ich vzájomné silové pôsobenie môžeme
úplne zanedbať. Pri využití masívnej paralelizácie pomocou GPU očakávame
podstatné zvýšenie výpočtového výkonu na úroveň miliárd integračných
krokov za sekundu, čo umožní simulovať milióny častíc (Nyland et al.,
2007). Vysoký počet simulovaných častíc je dôležitý, keďže k zrážkam so
Zemou dochádza pomerne zriedkavo, kým pre účely štatistického
vyhodnotenia súboru potrebujeme získať dostatočne početnú populáciu.

Ak počas integrácie dôjde ku kolízii niektorej z častíc so Zemou, daná
častica je označená ako spozorovaná. Sumárny štatistický súbor všetkých
takýchto častíc je po aplikácii výberových efektov možné porovnať s
pozemskými pozorovaniami a určiť zhodu s experimentálnymi dátami.
Variáciou parametrov simulácie a minimalizáciou odchýlok sme schopní
určiť skutočnú distribúciu a pôvodnú dráhu telies. Opätovné spustenie
simulácie s optimálnymi hodnotami parametrov spolu so znalosťou dráhy
skutočného materského telesa nám umožnia identifikovať jednotlivé prúdy
častíc, predpovedať aktivitu zodpovedajúcich meteorických rojov a určiť
tok a distribúciu častíc v danej oblasti Slnečnej sústavy.

\hypertarget{referencie}{%
\subsubsection*{Referencie}\label{referencie}}
\addcontentsline{toc}{subsubsection}{Referencie}

\hypertarget{refs}{}
\leavevmode\hypertarget{ref-balaz2018}{}%
Baláž, M., 2018. Determination of total meteoroid flux in millimetre to
metre size range (Master's thesis). Comenius University in Bratislava,
Bratislava, Slovakia.

\leavevmode\hypertarget{ref-nyland2007}{}%
Nyland, L., Harris, M., Prins, J., 2007. GPU Gems 3, chapter 31: Fast
N-Body Simulation with CUDA. Addison-Wesley Professional.

\subsection{Rozbor aktuálneho stavu riešenia témy
projektu}\label{rozbor-aktuuxe1lneho-stavu-rieux161enia-tuxe9my-projektu}

Moderné siete fotografických staníc a videostaníc poskytujú dostatočne
presné dáta na určenie dráhových charakteristík pozorovaných telies
(Borovička, 1990). Nemožnosť reprodukovania meteorov v laboratórnych
podmienkach však necháva veľké neistoty v určení hodnôt niektorých
fyzikálnych parametrov, ako napríklad skutočnej hmotnosti meteoroidov
alebo distribúcie hmotností v populácii.

Základným parametrom populácie meteoroidov je \emph{hmotnostný exponent}
\(s\), ktorým popisujeme distribúciu hmotností častíc ako mocninnú
závislosť \(N(m) \propto m^{-s}\) (Pokorný a Brown, 2016). Súčasné
modely určenia hmotnostného exponentu sú však vo viacerých ohľadoch
nedostatočné. Pri redukcii dát z tradičných vizuálnych alebo
prístrojových pozorovacích metód sa obvykle neuvažujú výberové efekty
(\emph{selection bias}). Do zaznamenaných pozorovaní vstupuje rad
prirodzených aj inštrumentálnych vplyvov, ktoré skresľujú výslednú
štatistickú vzorku. Automatické videostanice, fotografické prístroje aj
vizuálne pozorovania prirodzene preferujú meteory s väčšou zdanlivou
jasnosťou, väčšou zdanlivou dĺžkou dráhy a menšou vzdialenosťou od
zenitu. Tieto efekty je potrebné zmerať a vyvinúť procedúry na ich
odstránenie. Ďalšie parametre populácie je možné určiť až po oprave
štatistického súboru o výberové efekty.

Alternatívnym spôsobom korekcie je numerická simulácia, v ktorej
vytvoríme populáciu meteoroidov, simulujeme ich vstup do atmosféry a
pozorovaný súbor meteorov následne štatisticky vyhodnocujeme. Porovnaním
výstupu simulácie so skutočnými pozorovaniami za rovnakých podmienok
získavame mieru zhody. Postupnými zmenami parametrov simulácie a
veľkosti výberových efektov sme schopní určiť hodnoty parametrov, pre
ktoré je dosiahnutá najlepšia možná zhoda medzi simuláciou a pozorovaním
a simulovanú populáciu vyhlásiť za štatistický model skutočného
rozloženia meteoroidných častíc.

Podobné postupy sú využívané na odstránenie výberových efektov pri
astronomických pozorovaniach blízkozemských asteroidov (Chesley a Vereš,
2017), ale dosiaľ neboli aplikované na opravu štatistických dát pri
pozemských pozorovaniach meteorov. Úspešne však boli použité na odhad
celkového počtu viditeľných meteorov (Gural, 2002). Výsledky sú priamo
porovnateľné s hodnotami získanými inými metódami, napríklad vizuálnymi
pozorovaniami vykonanými skupinou pozorovateľov, priamou redukciou dát z
kamerových systémov AMOS (Zigo et al., 2013) a CILBO (Koschny et al.,
2013) alebo inými metódami (Blaauw et al., 2016). Špeciálnu pozornosť
sme venovali porovnaniu s analýzami konkrétnych pozorovacích nocí,
publikovanými napríklad portálom \texttt{MeteorFlux.io} (Molau a
Barentsen, 2018).

Predbežné výsledky (Baláž, 2018) indikujú, že udávané hodnoty
hmotnostného exponentu \(s\) sú všeobecne skreslené ignorovaním
pôsobiacich výberových efektov. Publikované dáta udávajú hodnoty \(s\)
približne \num{1.85} (Hughes, 1995; Krisciunas, 1980), prípadne iba
\num{1.7} (Bel'kovich a Ishmukhametova, 2006). Tieto hodnoty však nie sú
konzistentné s výsledkami simulácií -- zhoda s experimentálnymi dátami
je dosiahnutá až pri podstatne vyššej hodnote \(s = \num{2.15}\).
Pravdepodobným vysvetlením je práve nízka detekčná schopnosť
pozorovateľov pri zaznamenávaní menej hmotných častíc a z nej
vyplývajúce podhodnotenie ich skutočného počtu.

\subsubsection*{Referencie}\label{referencie}
\addcontentsline{toc}{subsubsection}{Referencie}

\hypertarget{refs}{}
\hypertarget{ref-balaz2018}{}
Baláž, M., 2018. Determination of total meteoroid flux in millimetre to
metre size range (Master's thesis). Comenius University in Bratislava,
Bratislava, Slovakia.

\hypertarget{ref-belkovich2006}{}
Bel'kovich, O.I., Ishmukhametova, M.G., 2006. Mass Distribution of
Perseid Meteoroids. Solar System Research 40, 208--213.
\url{https://doi.org/10.1134/S003809460603004X}

\hypertarget{ref-blaauw2016}{}
Blaauw, R.C., Campbell-Brown, M., Kingery, A., 2016. Optical meteor
fluxes and application to the 2015 Perseids. Monthly Notices of the
Royal Astronomical Society 463, 441--448.
\url{https://doi.org/10.1093/mnras/stw1979}

\hypertarget{ref-borovicka1990}{}
Borovička, J., 1990. The comparison of two methods of determining meteor
trajectories from photographs. Bulletin of the Astronomical Institutes
of Czechoslovakia 41, 391--396.

\hypertarget{ref-chesley2017}{}
Chesley, S.R., Vereš, P., 2017. Projected Near-Earth Object Discovery
Performance of the Large Synoptic Survey Telescope, v: 2016 IEEE
Aerospace Conference. \url{https://doi.org/10.1109/AERO.2016.7500539}

\hypertarget{ref-gural2002}{}
Gural, P., 2002. Meteor Observation Simulation Tool, v: Proceedings of
the International Meteor Conference 2001, Cerkno, Slovenia. s. 29--35.

\hypertarget{ref-hughes1995}{}
Hughes, D.W., 1995. The Perseid Meteor Shower. Earth, Moon and Planets
31--70. \url{https://doi.org/10.1007/BF00671498}

\hypertarget{ref-koschny2013}{}
Koschny, D., Bettonvil, F., Licandro, J., others, 2013. A double-station
meteor camera set-up in the Canary Islands. Geoscientific
Instrumentation Methods and Data Systems 2, 339--348.

\hypertarget{ref-krisciunas1980}{}
Krisciunas, K., 1980. The luminosity functions of the 1969 Perseid and
Orionid meteor showers. Icarus 43, 381--384.
\url{https://doi.org/10.1016/0019-1035(80)90182-7}

\hypertarget{ref-meteorflux}{}
Molau, S., Barentsen, G., 2018. MeteorFlux.io.

\hypertarget{ref-pokorny-brown2016}{}
Pokorný, P., Brown, P.G., 2016. A reproducible method to determine the
meteoroid mass index. Astronomy \& Astrophysics 592.
\url{https://doi.org/10.1051/0004-6361/201628134}

\hypertarget{ref-zigo2013}{}
Zigo, P., Tóth, J., Kalmančok, D., 2013. All-Sky Meteor Orbit System
AMOS, v: Gyssens, M., Roggemans, P. (Ed.), Proceedings of the
International Meteor Conference, 31st IMC, La Palma, Canary Islands,
Spain, 2012. s. 18--20.

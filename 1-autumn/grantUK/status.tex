\subsection{Rozbor aktuálneho stavu riešenia témy
projektu}\label{rozbor-aktuuxe1lneho-stavu-rieux161enia-tuxe9my-projektu}

Moderné siete fotografických staníc a videostaníc poskytujú dáta
dostatočne presné na určenie pozorovaných aj pôvodných dráh telies
(Borovička 1990). Nemožnosť reprodukovania meteorov v laboratórnych
podmienkach však necháva pomerne veľké neistoty v určení hodnôt
niektorých základných parametrov, ako sú skutočná hmotnosť meteoroidov
alebo distribúcia hmotností v populácii.

Základným parametrom konkrétnej populácie meteoroidov je
\emph{hmotnostný exponent} \(s\), ktorým popisujeme distribúciu
hmotností častíc ako mocninnú závislosť \(N(m) \propto m^{-s}\) (Pokorný
and Brown 2016). Súčasné modely určenia hmotnostného exponentu sú však
vo viacerých ohľadoch nedostatočné. Pri redukcii dát z tradičných
vizuálnych alebo prístrojových pozorovacích metód sa obvykle neuvažujú
výberové efekty (selection bias). Do zaznamenaných pozorovaní vstupuje
rad prirodzených aj inštrumentálnych vplyvov, ktoré skresľujú výslednú
štatistickú vzorku. Automatické videostanice, fotografické prístroje aj
priame vizuálne pozorovania prirodzene preferujú meteory s väčšou
zdanlivou jasnosťou, väčšou zdanlivou dĺžkou dráhy a menšou
vzdialenosťou od zenitu. Tieto efekty je potrebné zmerať a vyvinúť
procedúry na ich odstránenie, a všetky ďalšie závery činiť až po
očistení a spracovaní štatistického súboru.

Alternatívnym spôsobom korekcie je numerická simulácia, v ktorej
vytvoríme populáciu meteoroidov, simulujeme ich vstup do atmosféry a
následne štatisticky vyhodnocujeme pozorovaný súbor meteorov. Porovnaním
výstupu simulácie so skutočnými pozorovaniami za rovnakých podmienok
získavame mieru zhody. Postupnými zmenami parametrov simulácie a
veľkosti výberových efektov sme schopní určiť hodnoty parametrov, pre
ktoré je dosiahnutá najlepšia možná zhoda medzi simuláciou a pozorovaním
a simulovanú populáciu vyhlásiť za model rozloženia meteoroidných
častíc.

Podobné postupy sú využívané na odstránenie výberových efektov pri
astronomických pozorovaniach blízkozemských asteroidov (Chesley and
Vereš 2017), ale dosiaľ neboli aplikované na opravu štatistických dát
pri pozemských pozorovaniach meteorov. Úspešne však boli použité na
odhad celkového počtu viditeľných meteorov (Gural 2002). Výsledky sú
priamo porovnateľné s hodnotami získanými inými metódami, napríklad
vizuálnymi pozorovaniami vykonanými skupinou pozorovateľov, priamou
redukciou dát z kamerových systémov AMOS (Zigo, Tóth, and Kalmančok
2013) a CILBO (Koschny and others 2013) alebo inými metódami (Blaauw,
Campbell-Brown, and Kingery 2016). Špeciálnu pozornosť sme venovali
porovnaniu s analýzami konkrétnych pozorovacích nocí, publikovanými
napríklad portálom \texttt{MeteorFlux.io} (Molau and Barentsen 2018).

Predbežné výsledky (Baláž 2018) indikujú, že udávané hodnoty
hmotnostného exponentu \(s\) sú všeobecne skreslené ignorovaním
pôsobiacich výberových efektov. Publikované dáta udávajú hodnoty \(s\)
približne \num{1.85} (Hughes 1995) (Krisciunas 1980), prípadne iba
\num{1.7} (Bel'kovich and Ishmukhametova 2006). Tieto hodnoty však nie
sú konzistentné s výsledkami simulácií -- zhoda s experimentálnymi
dátami je dosiahnutá až pri podstatne vyššej hodnote \(s = \num{2.15}\).
Pravdepodobným vysvetlením je práve nízka detekčná schopnosť
pozorovateľov pri zaznamenávaní menej hmotných častíc a z nej
vyplývajúce podhodnotenie ich skutočného počtu.

\subsubsection*{Referencie}\label{referencie}
\addcontentsline{toc}{subsubsection}{Referencie}

\hypertarget{refs}{}
\hypertarget{ref-balaz2018}{}
Baláž, Martin. 2018. ``Determination of Total Meteoroid Flux in
Millimetre to Metre Size Range.'' Master's thesis, Bratislava, Slovakia:
Comenius University in Bratislava.

\hypertarget{ref-belkovich2006}{}
Bel'kovich, O. I., and M. G. Ishmukhametova. 2006. ``Mass Distribution
of Perseid Meteoroids.'' \emph{Solar System Research} 40 (3). Pleiades
Publishing: 208--13. \url{https://doi.org/10.1134/S003809460603004X}.

\hypertarget{ref-blaauw2016}{}
Blaauw, R. C., M. Campbell-Brown, and A. Kingery. 2016. ``Optical Meteor
Fluxes and Application to the 2015 Perseids.'' \emph{Monthly Notices of
the Royal Astronomical Society} 463 (1): 441--48.
\url{https://doi.org/10.1093/mnras/stw1979}.

\hypertarget{ref-borovicka1990}{}
Borovička, J. 1990. ``The Comparison of Two Methods of Determining
Meteor Trajectories from Photographs.'' \emph{Bulletin of the
Astronomical Institutes of Czechoslovakia} 41 (December): 391--96.

\hypertarget{ref-chesley2017}{}
Chesley, Steven R., and Peter Vereš. 2017. ``Projected Near-Earth Object
Discovery Performance of the Large Synoptic Survey Telescope.'' In
\emph{2016 Ieee Aerospace Conference}.
\url{https://doi.org/10.1109/AERO.2016.7500539}.

\hypertarget{ref-gural2002}{}
Gural, Peter. 2002. ``Meteor Observation Simulation Tool.'' In
\emph{Proceedings of the International Meteor Conference 2001, Cerkno,
Slovenia}, 29--35.

\hypertarget{ref-hughes1995}{}
Hughes, David W. 1995. ``The Perseid Meteor Shower.'' \emph{Earth, Moon
and Planets}, nos. 1-3 (January): 31--70.
\url{https://doi.org/10.1007/BF00671498}.

\hypertarget{ref-koschny2013}{}
Koschny, Detlef, and others. 2013. ``A Double-Station Meteor Camera
Set-up in the Canary Islands.'' \emph{Geoscientific Instrumentation
Methods and Data Systems} 2: 339--48.

\hypertarget{ref-krisciunas1980}{}
Krisciunas, K. 1980. ``The Luminosity Functions of the 1969 Perseid and
Orionid Meteor Showers.'' \emph{Icarus} 43 (3): 381--84.
\url{https://doi.org/10.1016/0019-1035(80)90182-7}.

\hypertarget{ref-meteorflux}{}
Molau, Sirko, and Geert Barentsen. 2018. ``MeteorFlux.io.''
\url{http://meteorflux.io/}.

\hypertarget{ref-pokorny-brown2016}{}
Pokorný, P., and P. G. Brown. 2016. ``A Reproducible Method to Determine
the Meteoroid Mass Index.'' \emph{Astronomy \& Astrophysics} 592.
\url{https://doi.org/10.1051/0004-6361/201628134}.

\hypertarget{ref-zigo2013}{}
Zigo, Pavol, Juraj Tóth, and Dušan Kalmančok. 2013. ``All-Sky Meteor
Orbit System AMOS.'' In \emph{Proceedings of the International Meteor
Conference, 31st Imc, La Palma, Canary Islands, Spain, 2012}, edited by
M. Gyssens and P. Roggemans, 18--20.

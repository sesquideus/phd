\subsection{Rozbor aktuálneho stavu riešenia témy
projektu}\label{rozbor-aktuuxe1lneho-stavu-rieux161enia-tuxe9my-projektu}

súčasné modely sú vo viacerých ohľadoch nedostatočné.

Bežné vizuálne alebo prístrojové pozorovacie metódy obvykle neberú do
úvahy výberové efekty (selection bias). Do zaznamenaných pozorovaní
vstupuje rad efektov, ktoré skresľujú výslednú štatistickú vzorku. Tieto
vplyvy môžeme rozdeliť na

\begin{itemize}
\tightlist
\item
  prirodzené, teda spôsobené fyzikálnymi danosťami prostredia, napríklad

  \begin{itemize}
  \tightlist
  \item
    rôzne vzdialenosti medzi pozorovateľom a meteorom,
  \item
    pokles intenzity kvôli absorpcii v atmosfére,
  \item
    prírodné alebo umelé svetelné znečistenie;
  \end{itemize}
\item
  inštrumentálne, čiže spôsobené nedokonalosťou prístroja, ako napríklad

  \begin{itemize}
  \tightlist
  \item
    vinetácia (pokles osvitu senzora smerom k okrajom zorného poľa),
  \item
    softvérové chyby a chyby pri spracovaní obrazu.
  \end{itemize}
\end{itemize}

Automatické videostanice, fotografické prístroje aj priame vizuálne
pozorovania prirodzene preferujú meteory s väčšou zdanlivou jasnosťou,
väčšou zdanlivou dĺžkou dráhy a menšou vzdialenosťou od zenitu. Tieto
efekty je potrebné zmerať a vyvinúť procedúry na ich odstránenie.
Alternatívnym spôsobom korekcie je numerická simulácia, v ktorej
vytvoríme populáciu meteoroidov, simulujeme ich vstup do atmosféry a
následne štatisticky vyhodnocujeme pozorovaný súbor meteorov. Porovnaním
výstupu simulácie so skutočnými pozorovaniami za rovnakých podmienok
získavame mieru zhody. Postupnými zmenami parametrov simulácie a
veľkosti výberových efektov sme schopní určiť hodnoty parametrov, pre
ktoré je dosiahnutá najlepšia možná zhoda medzi simuláciou a pozorovaním
a simulovanú populáciu vyhlásiť za model rozloženia meteoroidných
častíc.

Podobné postupy sú využívané na odstránenie výberových efektov pri
astronomických pozorovaniach blízkozemských asteroidov (Chesley and
Vereš 2017), ale dosiaľ neboli aplikované na opravu štatistických dát
pri pozemských pozorovaniach meteorov (Gural 2002).

nadväzuje na numerickú simuláciu častíc, ktorá bola vyvinutá v rámci
diplomovej práce (Baláž 2018)

podozrenie, že udávané hodnoty hmotnostného exponentu \(s\) sú všeobecne
skreslené nedostatočným zahrnutím všetkých pôsobiacich výberových
efektov. Publikované dáta následne často udávajú podstatne nižšie
hodnoty \(s\), ako je reálne.

\subsubsection*{Referencie}\label{referencie}
\addcontentsline{toc}{subsubsection}{Referencie}

\hypertarget{refs}{}
\hypertarget{ref-balaz2018}{}
Baláž, Martin. 2018. ``Determination of Total Meteoroid Flux in
Millimetre to Metre Size Range.'' Master's thesis, Bratislava, Slovakia:
Comenius University in Bratislava.

\hypertarget{ref-chesley2017}{}
Chesley, Steven R., and Peter Vereš. 2017. ``Projected Near-Earth Object
Discovery Performance of the Large Synoptic Survey Telescope.'' In
\emph{2016 Ieee Aerospace Conference}.
\url{https://doi.org/10.1109/AERO.2016.7500539}.

\hypertarget{ref-gural2002}{}
Gural, Peter. 2002. ``Meteor Observation Simulation Tool.'' In
\emph{Proceedings of the International Meteor Conference 2001, Cerkno,
Slovenia}, 29--35.

\documentclass[12pt, aspectratio=1610]{beamer}
\RequirePackage{
    amsmath,
    amssymb,
    calc,
    cancel,
    booktabs,
    color,
    siunitx,
    tikz,
    wrapfig,
    array,
    leftidx,
    float,
    etoolbox,
    fancyhdr,
    longtable,
    hyperref,
    ltcaption,
    ulem,
    wasysym,
    accents,
}

\sisetup{
    detect-all              = true,
    separate-uncertainty    = true,                                   % 7.2 ± 0.5
    multi-part-units        = single,
    per-mode                = reciprocal,                   % symbol for "m/s", reciprocal for "ms^{-1}"
    group-separator         = {\,},
    group-minimum-digits    = 5,
    inter-unit-product      = {\kern 0.10em},
    exponent-product        = \cdot,                        % \times for 5 × 10^7, \cdot for 5 . 1O^7    
    number-unit-product     = {\ },
    output-decimal-marker   = {\text{.}},
    range-units             = single,
    range-phrase            = {\text{ -- }},
    list-units              = single,
    list-final-separator    = {\text{\ a\ }},
    retain-explicit-plus    = true,
}

\hypersetup{
    hidelinks,
    breaklinks              = true,
}

\usepackage[
    final
]{pdfpages}

\usepackage[many]{tcolorbox}
\RenewDocumentCommand{\vec}{m}{\overrightarrow{#1}}

\makeatletter
    \def\new@mathgroup{\alloc@8\mathgroup\mathchardef\@cclvi}
    \patchcmd{\document@select@group}{\sixt@@n}{\@cclvi}{}{}
    \patchcmd{\select@group}{\sixt@@n}{\@cclvi}{}{}
\makeatother

\RequirePackage{mathspec}                                   % includes fontspec
\RequirePackage{polyglossia}                                % multi-language support
\RequirePackage{xunicode}
\setdefaultlanguage{slovak}

% Setup fonts -- see fontspec/mathspec documentation.
% Fonts are loaded from .ttf and .otf files in the core/fonts/ directory. We NEVER use system fonts.
\defaultfontfeatures{
    Mapping         = tex-text,
    Scale           = MatchLowercase,
    Ligatures       = TeX
}

\DeclareSIUnit\au{AU}
\DeclareSIUnit\pixel{px}
\DeclareSIUnit\lightyear{ly}
\DeclareSIUnit\parsec{pc}
\DeclareSIUnit\earthmass{M_{\earth}}
\DeclareSIUnit\speedoflight{c}
\DeclareSIUnit\foe{foe}
\DeclareSIUnit\year{yr}
\DeclareSIUnit\eur{€}
\DeclareSIUnit\solarmass{M_{\astrosun}}
\DeclareSIUnit\solarluminosity{L_{\astrosun}}
\DeclareSIUnit{\byte}{B}

\DeclareMathOperator{\diff}{\mathrm{d}\!}
\DeclareMathOperator{\pdiff}{\partial\!}

\NewDocumentCommand{\slashfrac}{m m}{\left.#1\middle/#2\right.}
\NewDocumentCommand{\derive}{O{} m m}{\frac{\mathop{\mathrm{d}^{#1}#2}}{\mathop{\mathrm{d}#3^{#1}}}}
%\NewDocumentCommand{\derive}{O{} m m}{\frac{\diffe^{#1}#2}{\diffe#3^{#1}}}
\NewDocumentCommand{\pderive}{O{} m m}{\frac{\partial^{#1} #2}{\partial #3^{#1}}}

\NewDocumentCommand{\integrate}{O{} O{} m m}{\int\limits_{#1}^{#2} \! #3 \, \diff#4}
\NewDocumentCommand{\iintegrate}{O{} O{} m m m}{\iint\limits_{#1}^{\quad#2} #3 \d#4\!\d#5}
\NewDocumentCommand{\iiintegrate}{O{} O{} m m m m}{\iiint\limits_{#1}^{\quad#2} #3 \d#4\!\d#5\!\d#6}

\NewDocumentCommand{\labelmath}{m +m}{%
    \begin{equation}%
        #2%
        \label{#1}%
    \end{equation}%
}

\NewDocumentCommand{\labelalign}{m +m}{%
    \begin{align}%
        #2%
        \label{#1}%
    \end{align}%
}

\linespread{1.0}
\setlength{\parindent}{0cm}
\setlength{\parskip}{6pt}
\setlength{\abovedisplayskip}{0mm}
\setlength{\belowdisplayskip}{0mm}
\setlength{\abovedisplayshortskip}{0mm}
\setlength{\belowdisplayshortskip}{0mm}
\setlength{\itemindent}{0pt}
\setlength{\textfloatsep}{0mm}
\setlength{\tabcolsep}{3mm}
\setlength{\LTcapwidth}{0.8\textwidth}
\renewcommand{\arraystretch}{1.2}

\setcounter{secnumdepth}{2}
\linespread{1.0}
\setlength{\parindent}{0cm}
\setlength{\parskip}{6pt}
\setlength{\abovedisplayskip}{0mm}
\setlength{\belowdisplayskip}{0mm}
\setlength{\abovedisplayshortskip}{0mm}
\setlength{\belowdisplayshortskip}{0mm}
\setlength{\itemindent}{0pt}
\setlength{\textfloatsep}{0mm}
\setlength{\tabcolsep}{3mm}
\renewcommand{\arraystretch}{1.2}

\setcounter{secnumdepth}{0}

\NewDocumentCommand{\fspicture}{m O{W} O{black}}{
    {
        \setbeamertemplate{navigation symbols}{}
        \setbeamercolor{background canvas}{bg = #3}
        \begin{frame}[plain]
            \begin{tikzpicture}[remember picture, overlay]
                \node[at=(current page.center)] {
                    \ifstrequal{H}{#2}{                                  
                        \includegraphics[height=\paperheight]{#1}%
                    }{%
                        \includegraphics[width=\paperwidth]{#1}%
                    }
                };
            \end{tikzpicture}
        \end{frame}
    }
}

\NewDocumentCommand{\frejm}{m +m}{
    \begin{frame}
        \frametitle{#1}
        #2
    \end{frame}
}
\defbeamertemplate{description item}{align center}{\hfill\insertdescriptionitem\hfill}
\definecolor{desc}{rgb}{0.66, 0, 0}
\definecolor{citem}{rgb}{0.72, 0, 0}
\definecolor{csitem}{rgb}{0.90, 0, 0}
\definecolor{cssitem}{rgb}{1, 0.1, 0.1}
\definecolor{qprimarybg}{rgb}{0.95, 0.95, 0.95}
\definecolor{check}{rgb}{0, 0.8, 0}

\setbeamertemplate{navigation symbols}{}
\newfontfamily{\semibold}{Segoe UI Semibold}
\RenewDocumentCommand{\emph}{m}{{\semibold#1}}

\mode<presentation> {
    \usetheme{Szeged}
    \usecolortheme{beaver}
    
    \usefonttheme{professionalfonts}
    \setallmainfonts{Minion Pro}
    \setmathrm{Minion Pro}
    
    \setsansfont{Segoe UI}
    \setmonofont{Consolas}
    \setbeamercolor*{enumerate item}{fg = citem}
    \setbeamercolor*{enumerate subitem}{fg = csitem}
    \setbeamercolor*{enumerate subsubitem}{fg = cssitem}
    \setbeamercolor*{description item}{fg = desc}
    \setbeamercolor*{itemize item}{fg = citem}
    \setbeamercolor*{itemize subitem}{fg = csitem}
    \setbeamercolor*{itemize subsubitem}{fg = cssitem}
    \setbeamercolor*{palette primary}{fg = red, bg = qprimarybg}
}

\AtBeginSection[]{
    \subsection{\insertsection}
    \begin{frame}
        \vfill
        \centering
        \begin{beamercolorbox}[sep = 8pt, center, shadow = true, rounded = true]{title}
            \usebeamerfont{title}\insertsectionhead\\[1.5mm]%
            \vfill
        \end{beamercolorbox}
        \vfill
    \end{frame}
}
\makeatletter

% Render percent sign with nice font, not ugly Computer modern
    \mathcode`\%="7025

% Fixes mathspec bug -- URL numbers are rendered with wrong font
    \ernewcommand\eu@MathPunctuation@symfont{Latin:m:n}
    \DeclareMathSymbol{,}{\mathpunct}{\eu@MathPunctuation@symfont}{`,}
    \DeclareMathSymbol{.}{\mathord}{\eu@MathPunctuation@symfont}{`.}
    \DeclareMathSymbol{<}{\mathrel}{\eu@MathPunctuation@symfont}{`<}
    \DeclareMathSymbol{>}{\mathrel}{\eu@MathPunctuation@symfont}{`>}
    \DeclareMathSymbol{/}{\mathord}{\eu@MathPunctuation@symfont}{`/}
    \DeclareMathSymbol{;}{\mathpunct}{\eu@MathPunctuation@symfont}{`;}
    \DeclareMathSymbol{(}{\mathopen}{\eu@DigitsArabic@symfont}{`(}
    \DeclareMathSymbol{)}{\mathclose}{\eu@DigitsArabic@symfont}{`)}
    \XeTeXDeclareMathSymbol{^^^^2026}{\mathinner}{\eu@MathPunctuation@symfont}{"2026}[\mathellipsis]
    \DeclareMathSymbol{0}{\mathalpha}{\eu@DigitsArabic@symfont}{`0}
    \DeclareMathSymbol{1}{\mathalpha}{\eu@DigitsArabic@symfont}{`1}
    \DeclareMathSymbol{2}{\mathalpha}{\eu@DigitsArabic@symfont}{`2}
    \DeclareMathSymbol{3}{\mathalpha}{\eu@DigitsArabic@symfont}{`3}
    \DeclareMathSymbol{4}{\mathalpha}{\eu@DigitsArabic@symfont}{`4}
    \DeclareMathSymbol{5}{\mathalpha}{\eu@DigitsArabic@symfont}{`5}
    \DeclareMathSymbol{6}{\mathalpha}{\eu@DigitsArabic@symfont}{`6}
    \DeclareMathSymbol{7}{\mathalpha}{\eu@DigitsArabic@symfont}{`7}
    \DeclareMathSymbol{8}{\mathalpha}{\eu@DigitsArabic@symfont}{`8}
    \DeclareMathSymbol{9}{\mathalpha}{\eu@DigitsArabic@symfont}{`9}
\makeatother


\usepackage{tabularx}


\title{The AMOS database system}
\subtitle{Storing data, one meteor at a time}
\author{Mgr. Martin Baláž}
\institute{DAPEM FMPH UK}
\date{2019--04--17}

\begin{document}
    \begin{frame}
        \titlepage
    \end{frame}
                
    \section{Overview}
    \subsection{Overview}
        \frejm{Contents}{
            \setbeamersize{description width = 30mm}
            \begin{description}
                \item[objective]        What are we doing?
                \item[motivation]       Why?
                \item[implementation]   How?
                \item[results]          What have we done?
            \end{description}
        }
        
        \frejm{Objective}{
            \begin{itemize}
                \item to build a database system for storing real meteor data
                
                \item it must be
                \setbeamersize{description width = 30mm}
                \begin{description}
                    \item[web-based]        accessible from anywhere
                    \item[comprehensible]   we are able to identify interesting events easily
                    \item[autonomous]       integrate the entire processing pipeline
                \end{description}
                
            \end{itemize}
        }
        
        \frejm{Motivation}{
            Current state is a disaster
            \begin{itemize}
                \item no unified file format
                \item data spread across multiple computers
                \item analysis impossible
                \begin{itemize}
                    \item and we badly missed it in diploma thesis
                \end{itemize}
            \end{itemize}
        }
        
    \section{Data}{
    \subsection{Data}
        \frejm{Acquisition}{
            Data are acquired by cameras...
            \begin{itemize}
                \item 
            \end{itemize}
        }
        
        \frejm{Pre-processing}{
            (at remote machines)
        }
        
        \frejm{Retrieval}{
            (VPN or what, networking issue, wtf)
        }
        
        \frejm{Processing}{
            (Fero)
        }
        
        \frejm{Storage}{
            At this point data are ready to be stored
            \begin{itemize}
                \item in a \emph{structured} and \emph{semantic} way
                \item consistency is \emph{enforced} at all times
                \item auxiliary data
                \item 
            \end{itemize}

            To prevent problems down the line, we should
            \begin{itemize}
                \item keep \emph{all} data available
            \end{itemize}
        }
    }
        
    \section{Database}
    \subsection{Database}
        \frejm{Design}{
            Underlying data are well-structured and suitable for an \emph{object-relational} database
        
            \begin{itemize}
                \item we require
                \begin{itemize}
                    \item full ACID compliance
                    \item free
                    \item PostgreSQL
                \end{itemize}
            \end{itemize} 
        }
        
        
        
        
        \frejm{The database}{
            \emph{Problem}: We need a way to visualize and comprehend the data
            \begin{itemize}
                \item at least well-formed data from AMOS
            \end{itemize}
            \pause
            \emph{Solution}: A relational database and a simple web interface
            \begin{itemize}
                \item provides a basic framework for operations on the data
                \item much more user-friendly than a bare directory listing
                \begin{itemize}
                    \item Django
                    \item PostgreSQL
                    \item AstroPy
                    \item Gnuplot / matplotlib
                    \item \dots
                \end{itemize}
                \item needs import scripts and utilities
            \end{itemize}
        }
        
        \frejm{Models}{
            In an object-relational database, rows (records) are objects
            \begin{itemize}
                \item \texttt{Meteor}
                \item \texttt{Sighting}
                \item \texttt{Observer}
                \item \texttt{Location}
                \item \texttt{VideoFrame}... not yet, maybe later
            \end{itemize}
        }
        
        \frejm{Model \texttt{Sighting}}{
            Describes the \emph{sighting} of a single meteor by an AMOS camera
            \begin{itemize}    
                \item maximum apparent magnitude            
                \item timestamp
                \item observed projected position on the sky
                    \item three coordinates
                    \begin{itemize}
                        \item azimuth
                        \item altitude
                        \item distance (from \texttt{Meteor})
                    \end{itemize}
                    \item at three timestamps
                        \begin{itemize}
                            \item beginning
                            \item maximum brightness
                            \item end
                        \end{itemize}      
            \end{itemize}
        }
        
        \frejm{Model \texttt{Sighting} -- miscellanea}{
            \begin{itemize}

                \item miscellaneous computed information
                    \begin{itemize}
                        \item arc length
                        \item duration
                        \item Sun and Moon info
                        \begin{itemize}
                            \item position
                            \item elongation
                            \item magnitude
                        \end{itemize}
                    \end{itemize}
                \item visualisation
                    \begin{itemize}
                        \item real photograph from AMOS
                        \item corresponding simulation (?)
                    \end{itemize}                    
            \end{itemize}
            
            (link)
        }
        
        \frejm{Model \texttt{Meteor}}{  
            Describes the actual \emph{event}, an atmospheric entry of a meteoroid particle
            \begin{itemize}
                \item \emph{timestamp}
                \item true geographic location
                \begin{itemize}
                    \item again, three coordinates
                    \begin{itemize}
                        \item \emph{latitude}
                        \item \emph{longitude}
                        \item \emph{altitude}
                    \end{itemize}
                    \pause
                    \item at three timestamps
                    \begin{itemize}
                        \item beginning
                        \item maximum brightness
                        \item end
                    \end{itemize}
                \end{itemize}
            \end{itemize}
        }
        
        \frejm{Model \texttt{Meteor} -- auxiliary data}{
            \begin{itemize}                
                \item true \emph{trail length}
                \begin{itemize}
                    \item this is not well-defined
                \end{itemize}
                \item computed \emph{absolute magnitude} (least-squares)
                \item \emph{visualisation}
                \begin{itemize}
                    \item KML file for Google Earth
                    \item online map (OpenLayers)
                \end{itemize}
            \end{itemize}
            
            (link)
        }
        
        \frejm{System in operation}{
            \begin{itemize}
                \item \href{http://192.168.0.177:4805/meteors/}{Meteor list}
                \item \href{http://192.168.0.177:4805/sightings/}{Sighting list}
                \item \href{http://192.168.0.177:4805/sighting/2537}{Example of a sighting}
                \item \href{http://192.168.0.177:4805/meteor/example}{Example of a meteor}
                \item \href{http://192.168.0.177:4805/meteor/example/path}{Example of a meteor path}

            \end{itemize}                 
                admin!
        }
        
    
    
    \section{Visualisation}
        \subsection{Visualisation}
    
    
    
        

    
        \frejm{Thank you for your attention}{
            \textit{Above all else, show the data.}
            \scriptsize
            \begin{flushright}
                Edward R. Tufte\\
                The Visual Display of Quantitative Information, 1983
            \end{flushright}
        }      
                
        \frejm{References}{
            \begin{itemize}
                \item \textbf{Jones, W.; Halliday, I.}:
                    Effects of Excitation and Ionization in Meteor Trains. MNRAS 320, 4, 417--423 (2001)
                \item \textbf{Luciuk, M.}:
                    Meteor Showers. {\footnotesize \url{http://www.asterism.org/tutorials/tut36\%20Meteor\%20Showers.pdf}}.
                \item \textbf{Hill, K. A.; Rogers, L. A.; Hawkes, R. L.}:
                    High geocentric velocity meteor ablation. Astronomy \& Astrophysics 444, 615--624 (2005) 
                \item \textbf{Öpik, E. J.}:
                    Physics of meteor flight in the atmosphere. Interscience Publishers, 1958.  
            \end{itemize}
        }
            
\end{document}

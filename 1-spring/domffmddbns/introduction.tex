\section{Introduction}
    \subsection{Motivation}
        While data collected by all-sky cameras are not particularly precise compared to those recorded by high-resolution photographic cameras,
        the relatively high sky coverage of such systems and long total accumulated observation time
        result in a large databases of meteor records that can be used for statistical analyses.
        
        Such extensive datasets are well suited for a multitude of tasks, such as identification of less prominent meteor
        showers or determination of total flux of meteoroid particles impinging on the surface of the Earth.
        These are often much more difficult to accomplish using traditional methods.
        
        As these systems usually operate automatically, data are available in almost real time and may be
        used for monitoring of short-term meteor activity outbursts. Data from all-sky cameras significantly
        contribute to our understanding of the dynamics of the Solar System and are also highly valuable for space agencies.
        
        Most all-sky system are able to observe meteors with apparent magnitude down to $5^{\mathrm{m}}$, although
        the efficiency in this magnitude range is very low. Meteors brighter than about $0^{\mathrm{m}}$ are detected relatively reliably.
        While there is no practical upper limit on brightness, very bright meteors occur only infrequently
        and thus are not suitable for statistical analyses.

        \subsubsection{Determination of total meteoroid flux} \label{imf}
            As all-sky cameras capture images of the sky continuously, recording properties of all detected meteors,
            along with precise timestamps and positions, it is possible to reconstruct the original
            population and estimate the total particle count in the appropriate size range.
            Moreover, if the original mass of each particle $m_{\infty}$ is known or estimated,
            we are able to compute the total mass flux.

        \subsubsection{Millimetre to metre size range} \label{imr}
            The size interval was chosen so that it best suits the available tools, namely the all-sky video meteor camera network AMOS.
            Generally speaking, meteoroids smaller than about one millimetre in diameter produce
            meteor streaks that are typically too faint for automated visual observations and are better suited for radar methods.
            However, at very high entry velocities even smaller particles may be well visible. The lower bound of the interval is thus somewhat arbitrary.

            Meteoroids larger than one metre enter the atmosphere only very sparsely. It is virtually impossible to collect any statistically valid
            dataset from a small numbers of fixed ground-based stations, each of which is only able to cover a minuscule fraction of the surface of the Earth. 
            Methods for determining the mass flux at this particle size thus often have to rely on witness reports and dashboard or security cameras.
            Fortunately, once they do occur, these events are rather noticeable and witness reports are usually plentiful.
            Statistical methods can be used to determine the particle flux in this size range as well.

    \subsection{AMOS}
        Since 2011, Department of Astronomy of KAFZM FMFI UK
        
        The captured video sequences are processed by the \textsc{UFOAnalyzerV2} software package.
        For each frame the position on the sky, apparent velocity vector, brightness, angular
        speed, deceleration and various auxiliary data are provided.
        
        
gural

asteroid equivalents (Jedicke / Vereš)

AMOS camera details \cite{zigo2013}
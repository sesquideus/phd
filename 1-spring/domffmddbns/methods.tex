\section{Methods}
    Here we describe the used methods.
    
    \subsection{Meteor flight model}
        Describe the used model by Öpik
        
        equations of motion (keep it simple)
        
        equation of luminosity + tau(v)
        
        Runge-Kutta integrator with variable mass

        supplementary quantities (air mass, density, angular speed) -- very briefly
    
        \begin{equation}
            \mathrm{d}v = -\frac{\Gamma A \rho_{\mathrm{air}} v^2}{m^{1/3} \rho^{2/3}} \mathrm{d}t\text{.}
            \label{m-dv}
        \end{equation}
        
        \begin{equation}
            \mathrm{d}m = -\frac{\Lambda}{2Q} \frac{Am^{2/3}}{\rho^{2/3}} \rho_{\mathrm{air}} v^3 \mathrm{d}t\text{.}
            \label{m-dm}
        \end{equation}
        
        \cite{jones-halliday2001} defined the excitation coefficient~$\zeta$, which represents
        the sum of all excitation probabilities over collisions and obtained the following relationship between $\zeta$ and $\tau$:
        \begin{equation}
            \tau = \frac{2\epsilon\zeta}{mv^2}\text{,}
            \label{m-tau}
        \end{equation}
    
        where $\epsilon$ is the mean excitation energy. For estimation of $\zeta$ we used a slightly improved version of the
        model compiled by \cite{hill2005}. In the formulae speed $v$ is stated in metres per second.
        To obtain the total flux of emitted visible light $F_0$, we take the time derivative of the particle's kinetic energy.
        Neglecting the second (deceleration) term yields the following \emph{equation of luminance}:
        \begin{equation}
            F_0 = \tau(v) \frac{\Lambda}{4Q} \rho_{\mathrm{air}} v^5\text{.}
            \label{m-f0}
        \end{equation}
        
    \subsection{Generating the meteoroids}
        material properties, spherical Earth model
        
        initial masses
        
        initial velocities and positions (ensure uniform distribution)
        
        initial times
        
    \subsection{Simulation of observations}
        apparent magnitude
        
    \subsection{Application of selection bias}
        
\section{Statistical analysis}
    \subsection{Detection probability functions}

    \subsection{Calibration of AMOS cameras}

    
    




    
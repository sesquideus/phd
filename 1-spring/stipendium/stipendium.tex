\documentclass[12pt, a4paper, oneside]{report}

\RequirePackage{
    amsmath,
    amssymb,
    calc,
    cancel,
    booktabs,
    color,
    siunitx,
    tikz,
    wrapfig,
    array,
    leftidx,
    float,
    etoolbox,
    fancyhdr,
    longtable,
    hyperref,
    ltcaption,
    ulem,
    wasysym,
    accents,
}

\sisetup{
    detect-all              = true,
    separate-uncertainty    = true,                                   % 7.2 ± 0.5
    multi-part-units        = single,
    per-mode                = reciprocal,                   % symbol for "m/s", reciprocal for "ms^{-1}"
    group-separator         = {\,},
    group-minimum-digits    = 5,
    inter-unit-product      = {\kern 0.10em},
    exponent-product        = \cdot,                        % \times for 5 × 10^7, \cdot for 5 . 1O^7    
    number-unit-product     = {\ },
    output-decimal-marker   = {\text{.}},
    range-units             = single,
    range-phrase            = {\text{ -- }},
    list-units              = single,
    list-final-separator    = {\text{\ a\ }},
    retain-explicit-plus    = true,
}

\hypersetup{
    hidelinks,
    breaklinks              = true,
}

\usepackage[
    final
]{pdfpages}

\usepackage[many]{tcolorbox}
\RenewDocumentCommand{\vec}{m}{\overrightarrow{#1}}

\makeatletter
    \def\new@mathgroup{\alloc@8\mathgroup\mathchardef\@cclvi}
    \patchcmd{\document@select@group}{\sixt@@n}{\@cclvi}{}{}
    \patchcmd{\select@group}{\sixt@@n}{\@cclvi}{}{}
\makeatother

\RequirePackage{mathspec}                                   % includes fontspec
\RequirePackage{polyglossia}                                % multi-language support
\RequirePackage{xunicode}
\setdefaultlanguage{slovak}

% Setup fonts -- see fontspec/mathspec documentation.
% Fonts are loaded from .ttf and .otf files in the core/fonts/ directory. We NEVER use system fonts.
\defaultfontfeatures{
    Mapping         = tex-text,
    Scale           = MatchLowercase,
    Ligatures       = TeX
}

\DeclareSIUnit\au{AU}
\DeclareSIUnit\pixel{px}
\DeclareSIUnit\lightyear{ly}
\DeclareSIUnit\parsec{pc}
\DeclareSIUnit\earthmass{M_{\earth}}
\DeclareSIUnit\speedoflight{c}
\DeclareSIUnit\foe{foe}
\DeclareSIUnit\year{yr}
\DeclareSIUnit\eur{€}
\DeclareSIUnit\solarmass{M_{\astrosun}}
\DeclareSIUnit\solarluminosity{L_{\astrosun}}
\DeclareSIUnit{\byte}{B}

\DeclareMathOperator{\diff}{\mathrm{d}\!}
\DeclareMathOperator{\pdiff}{\partial\!}

\NewDocumentCommand{\slashfrac}{m m}{\left.#1\middle/#2\right.}
\NewDocumentCommand{\derive}{O{} m m}{\frac{\mathop{\mathrm{d}^{#1}#2}}{\mathop{\mathrm{d}#3^{#1}}}}
%\NewDocumentCommand{\derive}{O{} m m}{\frac{\diffe^{#1}#2}{\diffe#3^{#1}}}
\NewDocumentCommand{\pderive}{O{} m m}{\frac{\partial^{#1} #2}{\partial #3^{#1}}}

\NewDocumentCommand{\integrate}{O{} O{} m m}{\int\limits_{#1}^{#2} \! #3 \, \diff#4}
\NewDocumentCommand{\iintegrate}{O{} O{} m m m}{\iint\limits_{#1}^{\quad#2} #3 \d#4\!\d#5}
\NewDocumentCommand{\iiintegrate}{O{} O{} m m m m}{\iiint\limits_{#1}^{\quad#2} #3 \d#4\!\d#5\!\d#6}

\NewDocumentCommand{\labelmath}{m +m}{%
    \begin{equation}%
        #2%
        \label{#1}%
    \end{equation}%
}

\NewDocumentCommand{\labelalign}{m +m}{%
    \begin{align}%
        #2%
        \label{#1}%
    \end{align}%
}

\linespread{1.0}
\setlength{\parindent}{0cm}
\setlength{\parskip}{6pt}
\setlength{\abovedisplayskip}{0mm}
\setlength{\belowdisplayskip}{0mm}
\setlength{\abovedisplayshortskip}{0mm}
\setlength{\belowdisplayshortskip}{0mm}
\setlength{\itemindent}{0pt}
\setlength{\textfloatsep}{0mm}
\setlength{\tabcolsep}{3mm}
\setlength{\LTcapwidth}{0.8\textwidth}
\renewcommand{\arraystretch}{1.2}

\setcounter{secnumdepth}{2}

\RequirePackage[
    paper                   = a4paper,
    left                    = 20mm,
    right                   = 20mm,
    top                     = 20mm,
    bottom                  = 20mm,
    headheight              = 16pt,
    headsep                 = 16pt,
    footskip                = 32pt,
    includeheadfoot,                                        % we wish to include header and footer into page dimensions
    %showframe                                              % display visual frame (must be turned off for production)
]{geometry}

\usepackage{titlesec}
\usepackage{enumitem}

% Setup enumitem options for *description*, *enumerate* and *itemize*

\setlist[enumerate]{
    topsep          =   0mm,
    itemsep         =   0mm,
}
\setlist[itemize]{
    topsep          =   0mm,
    itemsep         =   0mm,
}
\setlist[description]{
    style           = multiline,
    labelindent     =       8mm,
    leftmargin      =      50mm,
    itemsep         =       0mm,
}

\definecolor{colour-url}{RGB}{0, 137, 162}
\definecolor{colour-link}{RGB}{0, 137, 162}
\definecolor{colour-cite}{RGB}{0, 137, 49}

\hypersetup{
    colorlinks              = true,
    linkcolor               = colour-link,                  % custom Trojsten link colour
    urlcolor                = colour-url,                   % custom Trojsten URL link colour
    citecolor               = colour-cite,                  % custom Trojsten URL link colour
}

\setallmainfonts{Minion Pro}
\setmonofont{Consolas}
\newfontfamily{\semibold}{Adobe Garamond Pro Semibold}

\RenewDocumentCommand{\implies}{}{\quad\Rightarrow\quad}
\NewDocumentCommand{\dt}{m}{\skew{3}\dot{#1}}
\NewDocumentCommand{\ddt}{m}{\skew{3}\ddot{#1}}
\RenewDocumentCommand{\emph}{m}{{\semibold#1}}

\titleformat{\section}[hang]{\bfseries\LARGE}{}{0pt}{}
\titleformat{\subsection}[hang]{\bfseries\large}{}{0pt}{}[]
\titleformat{\subsubsection}[hang]{\bfseries}{}{0pt}{}[]
\titleformat{\paragraph}[hang]{\semibold}{}{0pt}{}[]

\makeatletter

\def\new@mathgroup{\alloc@8\mathgroup\mathchardef\@cclvi}
\patchcmd{\document@select@group}{\sixt@@n}{\@cclvi}{}{}
\patchcmd{\select@group}{\sixt@@n}{\@cclvi}{}{}

\mathcode`\%="7025
\makeatother


\fancypagestyle{first}{
    \fancyhf{}
    \fancyhead[C]{\textsc{Mgr. Martin Baláž, Jurigovo námestie 13, 841 04 Bratislava, Slovenská republika}}
    \renewcommand\headrulewidth{0.5pt}
}

\begin{document}
    \linespread{1.3}
    \setcounter{secnumdepth}{0}
    \setlength{\parindent}{0cm}
    \setlength{\parskip}{3mm}
    \setlength{\baselineskip}{6mm}
    \setlength{\abovedisplayskip}{0mm}
    \setlength{\belowdisplayskip}{0mm}
    \setlength{\abovedisplayshortskip}{0mm}
    \setlength{\belowdisplayshortskip}{5mm}
    \renewcommand{\arraystretch}{1.2}
    
    \pagestyle{empty}
    \thispagestyle{first}    
        
    \vspace*{6mm}
    \hfill
    \begin{minipage}{0.4 \linewidth}
        \linespread{1.6}
        prof. RNDr. Daniel Ševčovič, DrSc. \\[1mm]
        Fakulta matematiky, fyziky a informatiky \\[1mm]
        Univerzita Komenského v Bratislave \\[1mm]
        Mlynská dolina, 842 48 Bratislava
        
        \vspace*{6mm}
        Bratislava, 22. marca 2019
    \end{minipage}
    
    \vspace{10mm}
    
    \underline{\textbf{Vec: Žiadosť o udelenie jednorazového mimoriadneho štipendia}}
    
    \vspace{6mm}

    \begin{tabular}{l l}
        \textbf{Žiadateľ:}          & Mgr. Martin Baláž \\
        \textbf{Študijný program:}  & astronómia a astrofyzika (dAAF) \\
        \textbf{Rok štúdia:}        & prvý \\
        \textbf{Názov aktivity:}    & IT infraštruktúra systému AMOS \\
        \textbf{Ciele aktivity:}    & vytvoriť nástroj na správu infraštruktúry systému AMOS \\
        \textbf{Finančné nároky:}   & \SI{500}{\eur} \\
    \end{tabular}
    
    \vspace{8mm}
    \hyphenpenalty=1000
    \exhyphenpenalty=1000
    
    Vážený pán dekan,
    
    touto formou reagujem na výzvu na podávanie žiadostí o mimoriadne štipendium podľa
    článku 10 odseku 6 Štipendijného poriadku FMFI.
    
    Cieľmi projektu sú detailné zoznámenie sa s prevádzkou systému AMOS,    
    údržba staníc systému AMOS prostredníctvom softvéru na vzdialenú správu
    a tvorba administračného systému, umožňujúceho sledovať ich stav v reálnom čase a odstraňovať prípadné problémy.
    Medzi aktivity patrí najmä inštalácia a kontrola softvéru, zabezpečenie prevádzky systému AMOS
    a riešenie mimoriadnych situácií (výpadky napájania alebo spojenia so stanicou).
    V rámci projektu taktiež plánujem pokračovať vo vývoji databázy a informačného systému na vizualizáciu spracovaných dát.
    Aktivity vyžadujú cestovanie na Astronomické a Geofyzikálne observatórium
    v Modre za účelom oboznámenia sa s fungovaním systému AMOS.
    Celkovú časovú náročnosť projektu odhadujeme na približne 80 hodín.

    V prípade poskytnutia finančných prostriedkov budú použité na krytie nákladov vzniknutých
    v súvislosti s cestovaním na AGO Modra a ako kompenzácia za aktivity presahujúce okruh študijných povinností
    v rámci doktorandského štúdia.
        
    \vspace*{6mm}
    
    S pozdravom,    
    \hfill Mgr. Martin Baláž

    
\end{document}

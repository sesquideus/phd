\documentclass[12pt,aspectratio=1610]{beamer}
\linespread{1.0}
\setlength{\parindent}{0cm}
\setlength{\parskip}{6pt}
\setlength{\abovedisplayskip}{0mm}
\setlength{\belowdisplayskip}{0mm}
\setlength{\abovedisplayshortskip}{0mm}
\setlength{\belowdisplayshortskip}{0mm}
\setlength{\itemindent}{0pt}
\setlength{\textfloatsep}{0mm}
\setlength{\tabcolsep}{3mm}
\renewcommand{\arraystretch}{1.2}

\setcounter{secnumdepth}{0}

\NewDocumentCommand{\fspicture}{m O{W} O{black}}{
    {
        \setbeamertemplate{navigation symbols}{}
        \setbeamercolor{background canvas}{bg = #3}
        \begin{frame}[plain]
            \begin{tikzpicture}[remember picture, overlay]
                \node[at=(current page.center)] {
                    \ifstrequal{H}{#2}{                                  
                        \includegraphics[height=\paperheight]{#1}%
                    }{%
                        \includegraphics[width=\paperwidth]{#1}%
                    }
                };
            \end{tikzpicture}
        \end{frame}
    }
}

\NewDocumentCommand{\frejm}{m +m}{
    \begin{frame}
        \frametitle{#1}
        #2
    \end{frame}
}
\defbeamertemplate{description item}{align center}{\hfill\insertdescriptionitem\hfill}
\definecolor{desc}{rgb}{0.66, 0, 0}
\definecolor{citem}{rgb}{0.72, 0, 0}
\definecolor{csitem}{rgb}{0.90, 0, 0}
\definecolor{cssitem}{rgb}{1, 0.1, 0.1}
\definecolor{qprimarybg}{rgb}{0.95, 0.95, 0.95}
\definecolor{check}{rgb}{0, 0.8, 0}

\setbeamertemplate{navigation symbols}{}
\newfontfamily{\semibold}{Segoe UI Semibold}
\RenewDocumentCommand{\emph}{m}{{\semibold#1}}

\sisetup{
    output-decimal-marker = {,},
}

\title{Nobelova cena za fyziku 2019}
\subtitle{James "Jim" Peebles}
\author{Martin Baláž}
\institute{FMFI UK}
\date{2019--10--24}

\begin{document}
    {
        \begin{frame}
            \titlepage
        \end{frame}
    }
                
    \section{Nobelova cena}
        \frejm{James Peebles}{
            {\large ,,For contributions to our understanding of the evolution of the universe and Earth's place in the cosmos``}
            \begin{columns}
                \begin{column}{0.5\textwidth}
                    \begin{itemize}
                        \item 1935, Kanada
                        \item Princeton University
                        \item Einstein Professor of Science
                    \end{itemize}
                \end{column}
                \begin{column}{0.5\textwidth}
                    \includegraphics[width = \textwidth]{Peebles.jpg}
                \end{column}
            \end{columns}
        }
        
        \frejm{Dielo}{
            \begin{itemize}
                \item jeden zo zakladateľov modernej kozmológie
                \item Princeton, \emph{Robert Dicke}
                \pause
                \item teoretický pohľad na mladý vesmír
                \pause
                \item celoživotné dielo...
            \end{itemize}
        }


    \section{História}
        \frejm{Expanzia vesmíru}{
            \begin{itemize}
                \item \emph{Edwin Hubble} (1929): červený posun
                \item $\Implies$ čím je galaxia ďalej, tým rýchlejšie sa od nás vzďaľuje
                \pause
                \item vesmír sa rozpína
                \pause
                \item asi \SI{70}{\kilo\metre\per\second\per\mega pc} $\approx$ \num{2.2e-18}
            \end{itemize}
        }

        \frejm{Veľký tresk}{
            \begin{itemize}
                \item \emph{George Gamow} a \emph{Georges Lemaître}
                \pause
                \item poďme späť v čase
                \item ak sa vesmír rozpína, kedysi musel byť menší a horúcejší
                \pause
                \item asi pred \SI{1.38e-10}{rokov} musel byť takmer bodový
            \end{itemize}
        }
        
        \fspicture{inflation.jpg}

        \frejm{Expanzia vesmíru}{
            \begin{columns}
                \begin{column}{0.5 \textwidth}
                    \begin{itemize}
                        \item asi 380000 rokov po Veľkom tresku
                        \item teplota klesá pod \SI{3000}{\kelvin}
                        \pause
                        \item formujú sa neutrálne atómy
                        \item vesmír je priehľadný
                    \end{itemize}
                    \pause
                    \begin{itemize}
                        \item svetlo sa môže voľne šíriť
                        \item expanzia, červený posun...
                        \pause
                        \item dnes asi 1100-krát menej, \SI{2.725}{\kelvin}
                        \item žiarenie čierneho telesa
                    \end{itemize}
                \end{column}
                \begin{column}{0.5 \textwidth}
                    \includegraphics[width = \textwidth]{spectrum.png}
                \end{column}
            \end{columns}
        }

        \frejm{Objav}{
            \begin{itemize}
                \item \emph{Arno Penzias} a \emph{Robert Wilson}, 1964
                \item Holmdel Horn Antenna
                \pause
                \item holuby v anténe...
                \pause
                \item prvé priame pozorovanie CMB (Dicke)
                \item Nobelova cena za fyziku 1978
            \end{itemize}
        }
        
        \fspicture{Holmdel.jpeg}
        \fspicture{WMAP_2010.png}

        \frejm{Tmavá hmota + energia}{
            \begin{columns}
                \begin{column}{0.5 \textwidth}
                    \begin{itemize}
                        \item expanzia vesmíru sa \emph{zrýchľuje}
                        \pause
                        \item \emph{ΛCDM model}
                        \item vesmír je homogénny a izotropný
                        \item odchýlky sú viditeľné vo WMAP
                    \end{itemize}
                \end{column}
                \begin{column}{0.5 \textwidth}
                    \includegraphics[width = \textwidth]{pie.jpg}
                \end{column}
            \end{columns}
        }

        \fspicture{cosmic-web.jpg}

    \section{Alternatívne teórie}
        \frejm{Bez tmavej hmoty}{
            Veľa pozorovaní sa dá vysvetliť aj inak...
            \begin{itemize}
                \item rotačné krivky galaxií (\emph{Zwicky} 1930)
                \item gravitačné šošovkovanie
            \end{itemize}
        }

        \fspicture{rotation-curve.png}

        \frejm{MOND}{
            \begin{itemize}
                \item Čo ak je gravitácia iná?
                \item \emph{Milgrom} 1983
            \end{itemize}
            $$
                \cancel{\vec{F} = Gm_1m_2 \frac{\vec{r}}{r^3}}
            $$
            \pause
            $$
                \vec{F} = \frac{G m_1 m_2}{1 - \exp\left(-\sqrt{\frac{g}{g+}}\right)} \frac{\vec{r}}{r^3}
            $$
            \begin{itemize}
                \item $g+ \approx \SI{1.2e-10}{\metre\per\second\squared}$
            \end{itemize}
        }
        
        \frejm{Referencie}{
            \begin{itemize}
                \item NASA (WMP 2010, Holmdel Horn Antenna)
                \item Princeton University/EPA (James Peebles)
                \item Mario de Leo (Rotation curves)
            \end{itemize}
        }
            
\end{document}

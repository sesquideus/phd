\documentclass[12pt,aspectratio=1610]{beamer}
\linespread{1.0}
\setlength{\parindent}{0cm}
\setlength{\parskip}{6pt}
\setlength{\abovedisplayskip}{0mm}
\setlength{\belowdisplayskip}{0mm}
\setlength{\abovedisplayshortskip}{0mm}
\setlength{\belowdisplayshortskip}{0mm}
\setlength{\itemindent}{0pt}
\setlength{\textfloatsep}{0mm}
\setlength{\tabcolsep}{3mm}
\renewcommand{\arraystretch}{1.2}

\setcounter{secnumdepth}{0}

\NewDocumentCommand{\fspicture}{m O{W} O{black}}{
    {
        \setbeamertemplate{navigation symbols}{}
        \setbeamercolor{background canvas}{bg = #3}
        \begin{frame}[plain]
            \begin{tikzpicture}[remember picture, overlay]
                \node[at=(current page.center)] {
                    \ifstrequal{H}{#2}{                                  
                        \includegraphics[height=\paperheight]{#1}%
                    }{%
                        \includegraphics[width=\paperwidth]{#1}%
                    }
                };
            \end{tikzpicture}
        \end{frame}
    }
}

\NewDocumentCommand{\frejm}{m +m}{
    \begin{frame}
        \frametitle{#1}
        #2
    \end{frame}
}
\defbeamertemplate{description item}{align center}{\hfill\insertdescriptionitem\hfill}
\definecolor{desc}{rgb}{0.66, 0, 0}
\definecolor{citem}{rgb}{0.72, 0, 0}
\definecolor{csitem}{rgb}{0.90, 0, 0}
\definecolor{cssitem}{rgb}{1, 0.1, 0.1}
\definecolor{qprimarybg}{rgb}{0.95, 0.95, 0.95}
\definecolor{check}{rgb}{0, 0.8, 0}

\setbeamertemplate{navigation symbols}{}
\newfontfamily{\semibold}{Segoe UI Semibold}
\RenewDocumentCommand{\emph}{m}{{\semibold#1}}

\usepackage{transparent}

\title{Nobelova cena za fyziku 2019}
\subtitle{James "Jim" Peebles}
\author{Martin Baláž}
\institute{FMFI UK}
\date{2019--10--24}

\begin{document}
    {
        \begin{frame}
            \titlepage
        \end{frame}
    }
                
    \section{Overview}
        \frejm{}{
        }
        

        \fspicture{inflation.jpg}

        \frejm{Expanzia vesmíru}{
            \begin{itemize}
                \item asi 380000 rokov po Veľkom tresku
                \item teplota klesá pod \SI{3000}{\kelvin}
                \pause
                \item formujú sa neutrálne atómy
                \item vesmír je priehľadný
            \end{itemize}
            \pause
            \begin{itemize}
                \item svetlo sa môže voľne šíriť
                \item expanzia...
                \pause
                \item dnes
            \end{itemize}
        }

        \fspicture{WMAP_2010.png}
        \fspicture{cosmic-web.jpg}
        

        \frejm{Tmavá hmota / energia}{
        }

        \fspicture{pie.jpg}

    \section{Alternatívne teórie}
        \frejm{MOND}{
            \begin{itemize}
                \item
                    $$
                        \cancel{\vec{F} = Gm_1m_2 \frac{\vec{r}}{r^3}}
                    $$
                \pause
                \item
                    $$
                        \vec{F} = \frac{G m_1 m_2}{1 - \exp\left(-\sqrt{\frac{g}{g+}}\right)} \frac{\vec{r}}{r^3}
                    $$
            \end{itemize}
        }
        

        \frejm{References}{
            \begin{itemize}
                \item 
            \end{itemize}
        }
            
\end{document}

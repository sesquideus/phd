\documentclass[12pt,aspectratio=1610]{beamer}
\linespread{1.0}
\setlength{\parindent}{0cm}
\setlength{\parskip}{6pt}
\setlength{\abovedisplayskip}{0mm}
\setlength{\belowdisplayskip}{0mm}
\setlength{\abovedisplayshortskip}{0mm}
\setlength{\belowdisplayshortskip}{0mm}
\setlength{\itemindent}{0pt}
\setlength{\textfloatsep}{0mm}
\setlength{\tabcolsep}{3mm}
\renewcommand{\arraystretch}{1.2}

\setcounter{secnumdepth}{0}

\NewDocumentCommand{\fspicture}{m O{W} O{black}}{
    {
        \setbeamertemplate{navigation symbols}{}
        \setbeamercolor{background canvas}{bg = #3}
        \begin{frame}[plain]
            \begin{tikzpicture}[remember picture, overlay]
                \node[at=(current page.center)] {
                    \ifstrequal{H}{#2}{                                  
                        \includegraphics[height=\paperheight]{#1}%
                    }{%
                        \includegraphics[width=\paperwidth]{#1}%
                    }
                };
            \end{tikzpicture}
        \end{frame}
    }
}

\NewDocumentCommand{\frejm}{m +m}{
    \begin{frame}
        \frametitle{#1}
        #2
    \end{frame}
}
\defbeamertemplate{description item}{align center}{\hfill\insertdescriptionitem\hfill}
\definecolor{desc}{rgb}{0.66, 0, 0}
\definecolor{citem}{rgb}{0.72, 0, 0}
\definecolor{csitem}{rgb}{0.90, 0, 0}
\definecolor{cssitem}{rgb}{1, 0.1, 0.1}
\definecolor{qprimarybg}{rgb}{0.95, 0.95, 0.95}
\definecolor{check}{rgb}{0, 0.8, 0}

\setbeamertemplate{navigation symbols}{}
\newfontfamily{\semibold}{Segoe UI Semibold}
\RenewDocumentCommand{\emph}{m}{{\semibold#1}}

\title{Investigation of meteor models with numerical simulation}
\subtitle{Toying with models of meteor flight for a deeper understanding}
\author{\small \emph{Martin Baláž} \\ Juraj Tóth, PhD. \\ Peter Vereš, PhD.}
\institute{IMC 2019, Bollmannsruh}
\date{2019--10--05}

\begin{document}
    {
        \usebackgroundtemplate{\includegraphics[width=\paperwidth]{fireworks-i.png}}
        \begin{frame}
            \titlepage
        \end{frame}
    }
                
    \section{Overview}
        \frejm{What am I developing}{
            \begin{itemize}
                \item a (more-or-less) universal meteor simulator
                \pause
                \item we can
                \begin{itemize}
                    \item estimate \emph{meteor flux}
                    \item validate \emph{physical models}
                    \item fit \emph{observations}
                    \item ...
                \end{itemize}
                \pause
                \item work in progress
            \end{itemize}
            \begin{itemize}
                \item let's take it beyond statistics
            \end{itemize}

        }

        \frejm{Solution}{
        }
        
        \frejm{Algorithm}{
            \begin{enumerate}
                \item generate the meteoroid population
                \pause
                \item simulate atmospheric entry and create meteor objects
                \pause
                \item compute meteor sightings as seen by predefined observers
                    \begin{itemize}
                        \item position in the sky, magnitude, entry angle, ...
                    \end{itemize}
                \pause
                \item (optionally) apply observational bias 
                    \begin{itemize}
                        \item any parameter
                        \item limiting magnitude, altitude, angular speed, ...
                    \end{itemize}
                \pause
                \item calculate the statistic and compare it to AMOS data 
                \item adjust bias parameters 
                \pause
                \item repeat
            \end{enumerate}
        }
						
    \setbeamersize{description width = 5mm}
    \section{Simulation}
        % This time we are investigating the models
        %\frejm{Model}{            
        %    \emph{Whipple} (1938), improved by \emph{Öpik} (1955) and \emph{Ceplecha} (2001)
        %    
        %    We assume
        %    \begin{itemize}
        %        \item spherical, continuously ablating particles
        %    \end{itemize}
        %    
        %    We need
        %    \begin{itemize}
        %        \item equations of motion
        %        \item equation of luminance
        %        \item atmospheric and instrumental effects
        %        \item to compute the statistic
        %    \end{itemize}            
        %}
    
        %\frejm{Equations of motion}{
        %    \begin{itemize}
        %        \item braking equation
        %        $$
        %            \diff{v} = -\frac{\Gamma A}{m^{1/3} \rho^{2/3}} \rho_{\mathrm{air}} v^2 \diff{t}
        %        $$
        %        \item equation of ablation
        %        $$
        %            \diff{m} = -\frac{\Lambda A}{2Q} \frac{m^{2/3}}{\rho^{2/3}} \rho_{\mathrm{air}} v^3 \diff{t}
        %        $$
        %        \item equation of luminance
        %        $$
        %            L = \tau(v) \frac{\Lambda A}{4Q} \frac{m^{2/3}}{\rho^{2/3}} \rho_{\mathrm{air}} v^5
        %        $$
        %        \begin{itemize}
        %            \item $\tau(v)$ determined by \emph{Jones \& Halliday (2001)}
        %        \end{itemize}
        %    \end{itemize}
        %}
        
        \frejm{Simulation of flight}{
            Customized \emph{Runge--Kutta} integrator (RK4)
            \begin{itemize}
                \item run until complete ablation of the particle
                \item \emph{snapshots} taken $N$ times per second
                \item multiple integration steps between snapshots
            \end{itemize}
        }
        
        \fspicture{angularSpeed-teplicne-streaks.png}[H][black]
                 
        %\fspicture{mjd-entryAngle-appMag.png}
        %\fspicture{logInitMass-absMag-appMag.png}
        %\fspicture{absMag-elevation-appMag.png}

    \section{Population generators}
        \frejm{Random generator}{
            \begin{itemize}
                \item try to approximate the real population
                \begin{itemize}
                    \item easy for \emph{showers}
                    \item \emph{very difficult} for sporadic background
                \end{itemize}
                \pause
                \item constant $\vec{v}$
                \item slowly varying activity
            \end{itemize}
        }

        \frejm{Grid generator}{
            Good old method
            \begin{itemize}
                \item fix all parameters of meteoroids
                \pause
                \item slowly vary one parameter
                \pause
                \item simulate, observe, analyze...
                \item look at the changes in the output
            \end{itemize}
        }

        \fspicture{grid-sky.png}[H]

    \section{Entry angle}

    \section{Density}
    
    \section{}
        \frejm{Deficiencies of the model}{
            \begin{itemize}
                \item not dependent on $\Gamma$
            \end{itemize}
        }

        \frejm{Variations in entry angle}{
            \begin{itemize}
                \item North pole, varying declination
            \end{itemize}
        }

        \fspicture{ea-tme.png}
        \fspicture{ea-the.png}
        \fspicture{ea-Mae.png}
        \fspicture{ea-tLe.png}
							
    \section{Conclusion}
        \frejm{Validation}{
            Revise High velocity meteor ablation Hill et al.
            Revise other simulations (Gural)
        }

        \frejm{Ultimate goal}{
            Revise the models and design a better one
        }

        \frejm{Summary}{
            \begin{itemize}
                \item versatile, universal tool
                \item applicable to \emph{almost any} meteor observing system
            \end{itemize}
            \pause
            \begin{itemize}
                \item the tools are \emph{open-source}
                \begin{itemize}
                    \item \url{https://github.com/sesquideus/asmodeus}
                \end{itemize}
                \item suggestions or comments welcome
            \end{itemize}
        }
        
        \frejm{References}{
            \begin{itemize}
                \item \textbf{Öpik, E. J.}:
                    Physics of meteor flight in the atmosphere. Interscience Publishers, 1958.  
                \item \textbf{Jenniskens, P.}:
                    Meteor Showers and Their Parent Comets. Cambridge University Press, Cambridge, 2006.
                \item \textbf{Blaauw, R. C. et al.}:
                    Optical meteor fluxes and application to the 2015 Perseids. MNRAS vol. 463, 2016, pp441--448.
                \item \textbf{Hill, K. A. -- Rogers, L. A. -- Hawkes, R. L.}:
                    High geocentric velocity meteor ablation. Astronomy \& Astrophysics 444, 615--624 (2005) 
                \item \textbf{Jones, W. -- Halliday, I.}:
                    Effects of Excitation and Ionization in Meteor Trains. MNRAS vol. 321, 2001, pp417--423.
            \end{itemize}
        }
            
\end{document}

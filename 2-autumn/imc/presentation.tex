\documentclass[12pt,aspectratio=1610]{beamer}
\linespread{1.0}
\setlength{\parindent}{0cm}
\setlength{\parskip}{6pt}
\setlength{\abovedisplayskip}{0mm}
\setlength{\belowdisplayskip}{0mm}
\setlength{\abovedisplayshortskip}{0mm}
\setlength{\belowdisplayshortskip}{0mm}
\setlength{\itemindent}{0pt}
\setlength{\textfloatsep}{0mm}
\setlength{\tabcolsep}{3mm}
\renewcommand{\arraystretch}{1.2}

\setcounter{secnumdepth}{0}

\NewDocumentCommand{\fspicture}{m O{W} O{black}}{
    {
        \setbeamertemplate{navigation symbols}{}
        \setbeamercolor{background canvas}{bg = #3}
        \begin{frame}[plain]
            \begin{tikzpicture}[remember picture, overlay]
                \node[at=(current page.center)] {
                    \ifstrequal{H}{#2}{                                  
                        \includegraphics[height=\paperheight]{#1}%
                    }{%
                        \includegraphics[width=\paperwidth]{#1}%
                    }
                };
            \end{tikzpicture}
        \end{frame}
    }
}

\NewDocumentCommand{\frejm}{m +m}{
    \begin{frame}
        \frametitle{#1}
        #2
    \end{frame}
}
\defbeamertemplate{description item}{align center}{\hfill\insertdescriptionitem\hfill}
\definecolor{desc}{rgb}{0.66, 0, 0}
\definecolor{citem}{rgb}{0.72, 0, 0}
\definecolor{csitem}{rgb}{0.90, 0, 0}
\definecolor{cssitem}{rgb}{1, 0.1, 0.1}
\definecolor{qprimarybg}{rgb}{0.95, 0.95, 0.95}
\definecolor{check}{rgb}{0, 0.8, 0}

\setbeamertemplate{navigation symbols}{}
\newfontfamily{\semibold}{Segoe UI Semibold}
\RenewDocumentCommand{\emph}{m}{{\semibold#1}}

\title{Investigation of meteor models with numerical simulation}
\author{\small \emph{Martin Baláž} \\ Juraj Tóth, PhD. \\ Peter Vereš, PhD.}
\institute{IMC 2019, Bollmannsruh}
\date{2019--10--05}

\begin{document}
    {
        \usebackgroundtemplate{\includegraphics[width=\paperwidth]{fireworks-i.png}}
        \begin{frame}
            \titlepage
        \end{frame}
    }
                
    \section{Overview}
        \frejm{Objective}{

            \begin{itemize}
                \item we are left with a highly capable simulation
            \end{itemize}
        }

        \frejm{Solution}{
        }
        
        \frejm{Algorithm}{
            \begin{enumerate}
                \item generate the meteoroid population
                \pause
                \item simulate atmospheric entry and create meteor objects
                \pause
                \item compute meteor sightings as seen by predefined observers
                    \begin{itemize}
                        \item position in the sky, magnitude, entry angle, ...
                    \end{itemize}
                \pause
                \item apply observational bias 
                    \begin{itemize}
                        \item limiting magnitude, altitude, angular speed, ...
                    \end{itemize}
                \pause
                \item calculate the statistic and compare it to AMOS data 
                \item adjust bias parameters 
                \pause
                \item repeat
            \end{enumerate}
        }
						
    \setbeamersize{description width = 5mm}
    \section{Simulation}
        \frejm{Model}{            
            \emph{Whipple} (1938), improved by \emph{Öpik} (1955) and \emph{Ceplecha} (2001)
            
            We assume
            \begin{itemize}
                \item spherical, continuously ablating particles
                \item no gravity
            \end{itemize}
            
            We need
            \begin{itemize}
                \item equations of motion
                \item equation of luminance
                \item atmospheric and instrumental effects
                \item to compute the statistic
            \end{itemize}            
        }
    
        %\frejm{Equations of motion}{
        %    \begin{itemize}
        %        \item braking equation
        %        $$
        %            \diff{v} = -\frac{\Gamma A}{m^{1/3} \rho^{2/3}} \rho_{\mathrm{air}} v^2 \diff{t}
        %        $$
        %        \item equation of ablation
        %        $$
        %            \diff{m} = -\frac{\Lambda A}{2Q} \frac{m^{2/3}}{\rho^{2/3}} \rho_{\mathrm{air}} v^3 \diff{t}
        %        $$
        %        \item equation of luminance
        %        $$
        %            L = \tau(v) \frac{\Lambda A}{4Q} \frac{m^{2/3}}{\rho^{2/3}} \rho_{\mathrm{air}} v^5
        %        $$
        %        \begin{itemize}
        %            \item $\tau(v)$ determined by \emph{Jones \& Halliday (2001)}
        %        \end{itemize}
        %    \end{itemize}
        %}
        
        \frejm{Simulation of flight}{
            Customized \emph{Runge--Kutta} integrator (RK4)
            \begin{itemize}
                \item run until complete ablation of the particle
                \item \emph{snapshots} taken 20 times per second
                \item multiple integration steps between snapshots
            \end{itemize}
        }
        
        \frejm{Virtual observations}{
            Next, we create observations
            \begin{itemize}
                \item observers on the ground
                \begin{itemize}
                    \item each represents an AMOS camera
                \end{itemize}
                \item sequence of frames
            \end{itemize}           
        }
            
        \fspicture{angularSpeed-teplicne-streaks.png}[H][black]
        
        \frejm{Virtual observations}{
            Next, we create observations
            \begin{itemize}
                \item observers on the ground
                \begin{itemize}
                    \item each represents an AMOS camera
                \end{itemize}
                \item sequence of frames
                \item only the \emph{brightest frame} is analyzed
            \end{itemize}           
        }
            
        \frejm{Selection bias}{
            \emph{Detection efficiency is not constant!}\\[4mm]
            \begin{itemize}
                \item for each meteor:
                \begin{itemize}
                    \item \emph{decide} whether it is detected
                    \item based on its \emph{properties}
                \end{itemize}
                \item compare only visible meteors to real data
            \end{itemize}
        }

                        
        \frejm{Tool: ASMODEUS}{
            \textbf{A}ll-\textbf{S}ky \textbf{M}eteor \textbf{O}ptical \textbf{D}etection \textbf{E}fficiency \textbf{S}imulator\\[5mm]
            A multi-purpose virtual meteor observatory
            \begin{itemize}
                \item suite of seven scripts in \emph{Python}
                \item implements the described model
                \item numerous analytic tools and visualisations
            \end{itemize}
        }
        
        %\fspicture{mjd-entryAngle-appMag.png}
        %\fspicture{logInitMass-absMag-appMag.png}
        %\fspicture{absMag-elevation-appMag.png}

    \section{Generators}
        \frejm{Grid generator}{
            \begin{itemize}
                \item fix all properties except one or two
                \begin{itemize}
                    \item initial position
                    \item mass
                \end{itemize}
            \end{itemize}
        }

        \fspicture{grid.png}

    \section{Results}    
							
    \section{Conclusion}
        \frejm{Summary}{
            \begin{itemize}
                \item it is a \emph{surprisingly good} method
                \begin{itemize}
                    \item versatile, universal tool
                    \item applicable to \emph{almost any} meteor observing system
                \end{itemize}
            \end{itemize}
            \pause
            \begin{itemize}
                \item the tools will be \emph{shared} and free to use
                \item any comments or suggestions are welcome
            \end{itemize}
        }
        
        \frejm{References}{
            \begin{itemize}
                \item \textbf{Öpik, E. J.}:
                    Physics of meteor flight in the atmosphere. Interscience Publishers, 1958.  
                \item \textbf{Jenniskens, P.}:
                    Meteor Showers and Their Parent Comets. Cambridge University Press, Cambridge, 2006.
                \item \textbf{Blaauw, R. C. et al.}:
                    Optical meteor fluxes and application to the 2015 Perseids. MNRAS vol. 463, 2016, pp441--448.
                \item \textbf{Hill, K. A. -- Rogers, L. A. -- Hawkes, R. L.}:
                    High geocentric velocity meteor ablation. Astronomy \& Astrophysics 444, 615--624 (2005) 
                \item \textbf{Jones, W. -- Halliday, I.}:
                    Effects of Excitation and Ionization in Meteor Trains. MNRAS vol. 321, 2001, pp417--423.
            \end{itemize}
        }
            
\end{document}

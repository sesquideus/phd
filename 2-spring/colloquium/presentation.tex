\documentclass[12pt,aspectratio=1610]{beamer}
\linespread{1.0}
\setlength{\parindent}{0cm}
\setlength{\parskip}{6pt}
\setlength{\abovedisplayskip}{0mm}
\setlength{\belowdisplayskip}{0mm}
\setlength{\abovedisplayshortskip}{0mm}
\setlength{\belowdisplayshortskip}{0mm}
\setlength{\itemindent}{0pt}
\setlength{\textfloatsep}{0mm}
\setlength{\tabcolsep}{3mm}
\renewcommand{\arraystretch}{1.2}

\setcounter{secnumdepth}{0}

\NewDocumentCommand{\fspicture}{m O{W} O{black}}{
    {
        \setbeamertemplate{navigation symbols}{}
        \setbeamercolor{background canvas}{bg = #3}
        \begin{frame}[plain]
            \begin{tikzpicture}[remember picture, overlay]
                \node[at=(current page.center)] {
                    \ifstrequal{H}{#2}{                                  
                        \includegraphics[height=\paperheight]{#1}%
                    }{%
                        \includegraphics[width=\paperwidth]{#1}%
                    }
                };
            \end{tikzpicture}
        \end{frame}
    }
}

\NewDocumentCommand{\frejm}{m +m}{
    \begin{frame}
        \frametitle{#1}
        #2
    \end{frame}
}
\defbeamertemplate{description item}{align center}{\hfill\insertdescriptionitem\hfill}
\definecolor{desc}{rgb}{0.66, 0, 0}
\definecolor{citem}{rgb}{0.72, 0, 0}
\definecolor{csitem}{rgb}{0.90, 0, 0}
\definecolor{cssitem}{rgb}{1, 0.1, 0.1}
\definecolor{qprimarybg}{rgb}{0.95, 0.95, 0.95}
\definecolor{check}{rgb}{0, 0.8, 0}

\setbeamertemplate{navigation symbols}{}
\newfontfamily{\semibold}{Segoe UI Semibold}
\RenewDocumentCommand{\emph}{m}{{\semibold#1}}

\title{Meteor(oid?) models and density estimation}
\subtitle{}
\author{\small \emph{Martin Baláž}}
\institute{DAA colloquium}
\date{2020--02--26}

\begin{document}
    {
        \usebackgroundtemplate{\includegraphics[width=\paperwidth]{pictures/fireworks-i.png}}
        \begin{frame}
            \titlepage
        \end{frame}
    }
    \section{Overview}
        \frejm{Motivation}{
            {\large wtf}\\[10mm]
            \hfill \textit{English proverb}
        }


    \setbeamersize{description width = 5mm}

    \section{Parametric estimation}
        \frejm{Introduction}{
            Sometimes we know what to expect...
            \begin{itemize}
                \item establish the values of $\sigma$ and $\mu$
            \end{itemize}
        }

        \frejm{Algorithm}{
            \begin{itemize}
                \item define the class of distributions
            \end{itemize}
        }

        \frejm{Histograms}{

        }

        \frejm{Bin width}{
            \begin{itemize}
                \item usually we set bin width manually
                \pause
                \item there are ±rigorous methods of determination of optimal bin width
            \end{itemize}
        }

        \frejm{Scaling}{
            \begin{itemize}
                \item histograms are very dependent on bin width
                \item optimal bin width...?
                \pause
            \end{itemize}
            $$
                w = 2 \cdot \mathrm{IQR} \cdot N^{-1/3}
            $$
        }

        \frejm{Shifting}{
            Bin position is also important
            \centering
            \includegraphics[width = 0.9\linewidth]{pictures/shift.png}
        }


        \frejm{Case study I: heights of people}{
        }

       % \fspicture{gauss-1.png}

        \frejm{Case study II: height by sex}{
            We can extract information from the distribution
            \begin{itemize}
                \item assume males and females, both normally distributed
            \end{itemize}
            $$
                F(x; c_\text{\male}, \mu_\text{\male}, \sigma_\text{\male}, c_\text{\female}, \mu_\text{\female}, \sigma_\text{\female}) =
                \frac{c_\text{\male}}{\sqrt{2\pi}\sigma_\text{\male}} e^{\frac{-\left(x - \mu_\text{\male}\right)^2}{2\sigma_\text{\male}^2}}
            $$
        }

        \frejm{Summary}{
            \begin{itemize}
                \item advantages
                \begin{itemize}
                    \item simple
                    \item can extract information
                \end{itemize}
                \item disadvantages
                \begin{itemize}
                    \item we have to know the distribution
                    \item dependent on binning
                    \begin{itemize}
                        \item both \emph{scale} and \emph{shift}
                    \end{itemize}
                    \item there must be many data points
                \end{itemize}
            \end{itemize}
        }

    \section{Kernel density estimation}
        \frejm{No parameters}{
            Sometimes this all is not applicable...
            \pause
            \begin{itemize}
                \item we cannot use parametric methods
                \item the distribution is not \emph{known}
                \item there are \emph{too many} parameters or minima
                \item ...
            \end{itemize}
        }

        \frejm{One dimension}{
        }

        \frejm{Bandwidth}{
            \begin{itemize}
                \item similar to bin width in histograms
            \end{itemize}
            $$
                \Hat[-1pt]{f}(\vec{x}) = \frac{1}{nh} \Sum[i = 1][n]{K\left(h \left(x - x_i\right)\right)}
            $$
        }

        \frejm{Goodness of fit}{
            If we know the true distribution, we can design a metric
            \begin{itemize}
                \item \emph{MISE}
            \end{itemize}
            $$
                \mathrm{MISE}(\Hat[-1pt]{f}(x), f\,(x)) = \Int[\mathbb{R}]{\left(\Hat[-1pt]{f}(x) - f\,(x) \right)^2}{x}
            $$

            \begin{itemize}
                \item samples still \textbf{can} originate in a weird distribution
            \end{itemize}
        }

        \frejm{Multiple dimensions}{
            KDE easily \emph{generalizes} to multivariate functions
            \begin{itemize}
                \item but the variables may be \emph{correlated}
            \end{itemize}
        }

        \frejm{Example}{
            \includegraphics[width = \linewidth]{pictures/kde.png}
        }

        \frejm{Correlation}{
            $$
                \Hat[-1pt]{f}(\vec{x}) = \frac{1}{n \Abs{\vec{H}}} \Sum[i = 1][n]{K\left(\vec{H}^{-1}\left(\vec{x} - \vec{x}_i\right)\right)}
            $$
        }

        \frejm{Convolution}{
            KDE can be defined as a convolution
        }

        \frejm{Adaptive KDE}{
            Bandwidth need not be constant
            \begin{itemize}
                \item sometimes it is beneficial to vary $h$
                \pause
                \item generally
                \begin{itemize}
                    \item narrow bandwidth where data are \emph{abundant}
                    \item wide bandwidth where data are \emph{sparse}
                \end{itemize}
            \end{itemize}
        }

    \section{Models}
        \frejm{What is this good for?}{
            \begin{itemize}
                \item we want to find PDFs for \emph{meteors} and \emph{meteoroids}
                \pause
            \item devise \emph{two models}
            \end{itemize}
        }

        \frejm{Observational model}{
            We may estimate the distribution of meteors in the sky
            \begin{itemize}
                \item time (1)
                \item not \emph{positions}, but \emph{radiants} (3 or 2+1)
                \item magnitude spectrum (1)
            \end{itemize}
            \pause
            $$
                \mathcal{A}(t, \vec{v}, m) \equiv \mathcal{A}(t, \delta_\mathrm{R}, \alpha_\mathrm{R}, v_\infty, m).
            $$
        }

        \frejm{Orbital model}{
            Observational model tells \emph{when} and \emph{what}...
            \pause
            but does not tell \emph{why}
            \pause
            \begin{itemize}
                \item time (1)
                \item position (coarse vs fine) (\cancel{3})
                \item velocity (3)
                \item mass spectrum (1)
            \end{itemize}
            $$
                \mathcal{M}_{\mathrm{rv}}(t, \vec{r}, \vec{v}, \mu) \quad\text{or}\quad
                \mathcal{M}_{\mathrm{oe}}(t, a, e, i, \omega, \Omega, T, \mu)
            $$
            \pause
            or better
            $$
                \mathcal{M}_{\mathrm{rv}}(t, \vec{r}, \vec{v}, \log \mu) \quad\text{or}\quad
                \mathcal{M}_{\mathrm{oe}}(t, 1/a, e, i, \omega, \Omega, T, \log \mu)
            $$
        }

        \frejm{Curse of dimensionality}{
            $N$-volume of parameter space increases exponentially
            \begin{itemize}
                \item the space is \emph{vast}
                \pause
                \item $\ang{360} \times \ang{180} \times 8766 \times 50 \times 50 \approx \num{1.42e12}$
                \pause
                \item almost all of it is ``in the corner''
                \item distance metrics are not very useful
            \end{itemize}
        }


    \section{Summary}

        \frejm{References}{
            \begin{itemize}
                \item \textbf{Hwang, J.-N. -- Lay, S.-R. and Lippman, A.}:
                    Nonparametric multivariate density estimation: a case study.
                    IEEE Transactions on Signal Processing 42, 1994.
                \item \textbf{Vida, D. -- Brown, P. -- Campbell-Brown, M.}:
                    Modeling the measurement accuracy of pre-atmosphere velocities of meteoroids. MNRAS 479, 2018
            \end{itemize}
        }
\end{document}

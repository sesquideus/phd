\documentclass[12pt, a4paper, oneside]{report}

\RequirePackage{
    amsmath,
    amssymb,
    calc,
    cancel,
    booktabs,
    color,
    siunitx,
    tikz,
    wrapfig,
    array,
    leftidx,
    float,
    etoolbox,
    fancyhdr,
    longtable,
    hyperref,
    ltcaption,
    ulem,
    wasysym,
    marvosym,
    accents,
    listings,
    tabularx,
}

\hypersetup{
    hidelinks,
    breaklinks              = true,
}

\usepackage[final]{pdfpages}
\usepackage[many]{tcolorbox}

\RenewDocumentCommand{\vec}{m}{\mathbf{#1}}

\makeatletter
    \def\new@mathgroup{\alloc@8\mathgroup\mathchardef\@cclvi}
    \patchcmd{\document@select@group}{\sixt@@n}{\@cclvi}{}{}
    \patchcmd{\select@group}{\sixt@@n}{\@cclvi}{}{}
\makeatother

\RequirePackage{mathspec}                                   % includes fontspec
\RequirePackage{polyglossia}                                % multi-language support
\RequirePackage{xunicode}
\setdefaultlanguage{slovak}

% Setup fonts -- see fontspec/mathspec documentation.
\defaultfontfeatures{
    Mapping         = tex-text,
    Scale           = MatchLowercase,
    Ligatures       = TeX
}


\NewDocumentCommand{\labelmath}{m +m}{%
    \begin{equation}%
        #2%
        \label{#1}%
    \end{equation}%
}

\NewDocumentCommand{\labelalign}{m +m}{%
    \begin{align}%
        #2%
        \label{#1}%
    \end{align}%
}

\linespread{1.0}
\setlength{\parindent}{0cm}
\setlength{\parskip}{6pt}
\setlength{\abovedisplayskip}{0mm}
\setlength{\belowdisplayskip}{0mm}
\setlength{\abovedisplayshortskip}{0mm}
\setlength{\belowdisplayshortskip}{0mm}
\setlength{\itemindent}{0pt}
\setlength{\textfloatsep}{0mm}
\setlength{\tabcolsep}{3mm}
\setlength{\LTcapwidth}{0.8\textwidth}
\renewcommand{\arraystretch}{1.2}

\setcounter{secnumdepth}{2}

/home/amos/dgs/core/tex/math.tex
/home/kvik/dgs/core/tex/siunitx.tex


\RequirePackage[
    paper                   = a4paper,
    left                    = 20mm,
    right                   = 20mm,
    top                     = 20mm,
    bottom                  = 20mm,
    headheight              = 16pt,
    headsep                 = 16pt,
    footskip                = 32pt,
    includeheadfoot,                                        % we wish to include header and footer into page dimensions
    %showframe                                              % display visual frame (must be turned off for production)
]{geometry}

\usepackage{titlesec}
\usepackage{enumitem}

% Setup enumitem options for *description*, *enumerate* and *itemize*

\setlist[enumerate]{
    topsep          =   0mm,
    itemsep         =   0mm,
}
\setlist[itemize]{
    topsep          =   0mm,
    itemsep         =   0mm,
}
\setlist[description]{
    style           = multiline,
    labelindent     =       8mm,
    leftmargin      =      50mm,
    itemsep         =       0mm,
}

\definecolor{colour-url}{RGB}{0, 137, 162}
\definecolor{colour-link}{RGB}{0, 137, 162}
\definecolor{colour-cite}{RGB}{0, 137, 49}

\hypersetup{
    colorlinks              = true,
    linkcolor               = colour-link,                  % custom Trojsten link colour
    urlcolor                = colour-url,                   % custom Trojsten URL link colour
    citecolor               = colour-cite,                  % custom Trojsten URL link colour
}

\setallmainfonts{Minion Pro}
\setmonofont{Consolas}
\newfontfamily{\semibold}{Adobe Garamond Pro Semibold}

\RenewDocumentCommand{\implies}{}{\quad\Rightarrow\quad}
\NewDocumentCommand{\dt}{m}{\skew{3}\dot{#1}}
\NewDocumentCommand{\ddt}{m}{\skew{3}\ddot{#1}}
\RenewDocumentCommand{\emph}{m}{{\semibold#1}}

\titleformat{\section}[hang]{\bfseries\LARGE}{}{0pt}{}
\titleformat{\subsection}[hang]{\bfseries\large}{}{0pt}{}[]
\titleformat{\subsubsection}[hang]{\bfseries}{}{0pt}{}[]
\titleformat{\paragraph}[hang]{\semibold}{}{0pt}{}[]

\makeatletter

\def\new@mathgroup{\alloc@8\mathgroup\mathchardef\@cclvi}
\patchcmd{\document@select@group}{\sixt@@n}{\@cclvi}{}{}
\patchcmd{\select@group}{\sixt@@n}{\@cclvi}{}{}

\mathcode`\%="7025
\makeatother


\fancypagestyle{first}{
    \fancyhf{}
    \fancyhead[C]{\textsc{Mgr. Martin Baláž, Jurigovo námestie 13, 841 04 Bratislava, Slovenská republika}}
    \renewcommand\headrulewidth{0.5pt}
}

\begin{document}
    \linespread{1.3}
    \setcounter{secnumdepth}{0}
    \setlength{\parindent}{0cm}
    \setlength{\parskip}{3mm}
    \setlength{\baselineskip}{6mm}
    \setlength{\abovedisplayskip}{0mm}
    \setlength{\belowdisplayskip}{0mm}
    \setlength{\abovedisplayshortskip}{0mm}
    \setlength{\belowdisplayshortskip}{5mm}
    \renewcommand{\arraystretch}{1.2}

    \pagestyle{empty}
    \thispagestyle{first}

    \vspace*{6mm}
    \hfill
    \begin{minipage}{0.4 \linewidth}
        \linespread{1.6}
        prof. RNDr. Daniel Ševčovič, DrSc. \\[1mm]
        Fakulta matematiky, fyziky a informatiky \\[1mm]
        Univerzita Komenského v Bratislave \\[1mm]
        Mlynská dolina, 842 48 Bratislava

        \vspace*{12mm}
        Santa Cruz de La Palma, 17. júna 2020
    \end{minipage}

    \vspace{10mm}

    \underline{Vec: \textbf{Žiadosť o možnosť prekročenia hranice 90 zapísaných kreditov v akademickom roku 2019/2020}}

    \vspace{6mm}

    \begin{tabular}{l l}
        \textbf{Žiadateľ:}          & Mgr. Martin Baláž \\
        \textbf{Študijný program:}  & astronómia a astrofyzika (dAAF) \\
        \textbf{Rok štúdia:}        & druhý \\
    \end{tabular}

    \vspace{8mm}
    \hyphenpenalty=1000
    \exhyphenpenalty=1000

    Vážený pán dekan,

    dovoľujem si Vás požiadať o povolenie prekročiť maximálny dovolený počet 90 zapísaných
    kreditov v jednom akademickom roku doktorandského štúdia.

    V akademickom roku 2019/2020 som sa aktívne zúčastnil dvoch medzinárodných konferencií
    (International Meteor Conference, Bollmannsruh, Nemecko; a Meteoroids 2019, Bratislava).
    Následne sme v spolupráci s mojím školiteľom doc. Tóthom, dr. Verešom z Harvard-Smithsonian Institute for Astrophysics
    a dr. Jedickem z University of Hawaii publikovali článok v časopise Planetary and Space Science.
    Zároveň sa ako spoluriešiteľ projektu podieľam na vývoji softvéru pre systém celooblohových kamier AMOS.
    V letnom semestri som sa v rámci projektu ERASMUS+ zúčastnil študijného pobytu na Kanárskych ostrovoch
    pod vedením školiteľky Dr. Julie de León.

    Vopred ďakujem za zváženie mojej žiadosti.

    \vspace*{6mm}

    S pozdravom,

    \hfill Mgr. Martin Baláž


\end{document}

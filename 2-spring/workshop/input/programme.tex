\documentclass[a4paper]{report}

\usepackage[
    left    = 15mm,
    right   = 15mm,
    top     = 15mm,
    bottom  = 15mm,
]{geometry}

\usepackage{siunitx}
\usepackage{xcolor}
\usepackage{tcolorbox}
\usepackage{booktabs}
\usepackage{titlesec}
\usepackage{enumitem}
\usepackage{tabularx, tabu, ragged2e}
\usepackage[MnSymbol]{mathspec}


\setcounter{secnumdepth}{0}

\setallmainfonts{Minion Pro}

\pagestyle{empty}
\renewcommand{\arraystretch}{1.5}
\linespread{1.3}
\aboverulesep=1.6ex
\belowrulesep=1.6ex

\setlist[description]{
    font=\Large,
    style=multiline,
    leftmargin=20mm,
}

\titleformat{\section}[hang]{\Huge\centering\bfseries}{}{6pt}{}[]

\begin{document}
    \begin{minipage}{\textwidth}
        \begin{minipage}{0.2\textwidth}
            \includegraphics[width=32mm]{input/logo.jpg}
        \end{minipage}
        \begin{minipage}{0.58\textwidth}
            \centering
            \fontsize{40}{50}\selectfont
            Astroworkshop\\
            \Large
            AGO Modra, 2020-03-05 -- 2020-03-08
        \end{minipage}
        \begin{minipage}{0.2\textwidth}
            \includegraphics[width=32mm]{input/fmfi.png}
        \end{minipage}
    \end{minipage}
    \vspace*{10mm}

    \begin{tabularx}{\textwidth}{>{}p{2cm} >{\RaggedRight}X}
        \textbf{Organizátori}: &
            Martin Baláž,             Patrik Čechvala,             Karol Havrila,             Emil Puha,             Matej Zigo             \\
    \end{tabularx}

            \section{Štvrtok}


                    \begin{tcolorbox}[
                                    colback=green!10,
                    colframe=green!50!black,
                                fonttitle=\Large\bfseries,
                title=18:00
            ]
                {\Large Otvorenie 7. ročníka ASTRO workshopu}
                                                            \\ \textbf{\textit{Roman Nagy}}
                                                                                \end{tcolorbox}
                    \begin{tcolorbox}[
                                    colback=green!10,
                    colframe=green!50!black,
                                fonttitle=\Large\bfseries,
                title=18:30
            ]
                {\Large Prekvapenie}
                                                            \\ \textbf{\textit{...a možná přijde i kouzelník...}}
                                                                                \end{tcolorbox}
                    \begin{tcolorbox}[
                                    colback=red!10,
                    colframe=red!50!black,
                                fonttitle=\Large\bfseries,
                title=20:00
            ]
                {\Large Večera}
                                                            \end{tcolorbox}
                    \section{Piatok}


                    \begin{tcolorbox}[
                                    colback=red!10,
                    colframe=red!50!black,
                                fonttitle=\Large\bfseries,
                title=09:00
            ]
                {\Large Raňajky}
                                                            \end{tcolorbox}
                    \begin{tcolorbox}[
                                    colback=white,
                    colframe=black!70!white,
                                fonttitle=\Large\bfseries,
                title=10:00
            ]
                {\Large Expedícia Hawaii 2020}
                                                            \\ \textbf{\textit{Juraj Tóth}}
                                                    \\ \textit{FMFI UK}                            \end{tcolorbox}
                    \begin{tcolorbox}[
                                    colback=white,
                    colframe=black!70!white,
                                fonttitle=\Large\bfseries,
                title=10:30
            ]
                {\Large Kozmochémia v praxi}
                                                            \\ \textbf{\textit{Justína Nováková}}
                                                    \\ \textit{PriF UK}                \\[2ex]Kozmochémia je vetva astronómie, ktorá prepája astronómiu, mineralógiu, petrológiu a chémiu pri štúdiu mimozemských materiálov. V súčasnosti najrýchlejšie napredujúcou analytickou metódou v tomto odvetví je hmotnostná spektrometria, ktorá je v rôznych podobách vhodná na vysokocitlivú analýzu chemického zloženia v podstate ľubovoľného materiálu. Táto prednáška bude venovaná predstaveniu hmotnostných spektrometrov, ktoré sú najvhodnejšie na analýzu meteoritov a porovnaniu ich špecifických prínosov.
            \end{tcolorbox}
                    \begin{tcolorbox}[
                                    colback=white,
                    colframe=black!70!white,
                                fonttitle=\Large\bfseries,
                title=11:00
            ]
                {\Large Dojmy z konferencie AstroEdu v Garchingu}
                                                            \\ \textbf{\textit{Patrik Čechvala}}
                                                    \\ \textit{FMFI UK}                \\[2ex]16.--18. septembra 2019 sa konala v Garchingu pri Mníchove konferencia zameraná na vyučovanie a popularizáciu astronómie organizovaná Medzinárodnou astronomickou úniou IAU. Konferencia sa odohrávala v návštevníckom centre Európskeho južného observatória ESO Supernova. Na tejto konferencii sme prezentovali príspevok, ktorý vznikol v spolupráci s doktorandkami z Katedry didaktiky. Ukážeme si niektoré zaujímavosti z tejto konferencie. Špeciálne sa budeme venovať samotnému návšteníckemu centru ESO Supernova, ktoré pozostáva z planetária a výstavných plôch.
            \end{tcolorbox}
                    \begin{tcolorbox}[
                                    colback=white,
                    colframe=black!70!white,
                                fonttitle=\Large\bfseries,
                title=11:45
            ]
                {\Large Structure of the outer Galactic disk}
                                                            \\ \textbf{\textit{Žofia Chrobáková}}
                                                    \\ \textit{Instituto de Astrofisica Canarías}                \\[2ex]The structure of the outer disk of our Galaxy is still not well described and there are many features that need to be better understood. Gaia DR2 provides data in unprecedented quality that can be analyzed to shed some light on the outermost parts of the Milky Way. We calculated stellar density using star counts obtained from Gaia DR2, up to galactocentric distance $R$ = 20 kpc using a deconvolution technique in the parallax errors. To recover star counts, we carry out the deconvolution, using the Lucy's inversion method of the Fredholm integral equations of the first kind, without assuming any prior. We analyzed the density in order to study the structure of the outer Galactic disk and created density maps, where we can see structural features, mainly the warp. We studied the warp in greater detail, fitting it with multiple models and analyzing its properties. When we study the northern and the southern warps separately, we get an asymmetry of \textasciitilde\SI{25}{\percent} larger amplitude in the north. We also study the flare -- the increase of the scale-height with galactocentric distance. We will show preliminary results of our analysis, which shed some light on possibility of flaring of the Galactic disk in the remote regions of the Galaxy.
            \end{tcolorbox}
                    \begin{tcolorbox}[
                                    colback=red!10,
                    colframe=red!50!black,
                                fonttitle=\Large\bfseries,
                title=12:30
            ]
                {\Large Obed}
                                                            \end{tcolorbox}
                    \begin{tcolorbox}[
                                    colback=blue!10,
                    colframe=black!50!blue,
                                fonttitle=\Large\bfseries,
                title=14:00
            ]
                {\Large Astronomický Geocaching}
                                                            \\ \textbf{\textit{Karol Havrila}}
                                                    \\ \textit{FMFI UK}                \\[2ex]Mnohí to poznáte: spojenie príjemného s užitočným prináša svoje čaro. A práve toto spojenie príjemného (pohyb mestských uličkách a v prírode) s užitočným (spoznávanie svojho okolia v kombinácii s logikou) prináša obľúbená celosvetová terénna hra, Geocaching. Cieľom hry je na základe vyriešených hádaniek, logických úloh v teréne a všestranných indícii získať súradnice miesta, kde je schovaná tzv. „keška“. V našich mestách a okolí sú stovky kešiek, ktoré prinášajú nie len zábavu z hrania, ale aj spoznávanie svojho okolia, jeho histórie a známych osobností, ktoré sú s danou oblasťou spojené. Okrem základných informácii o tom, ako funguje Geocaching, si hľadanie kešky vyskúšame priamo v teréne. Aby to nebolo také jednoduché, súradnice miesta budú ukryté v astronómii a astrofyzike. Pre nálezcov bude v keške ukryté malé prekvapenie.
            \end{tcolorbox}
                    \begin{tcolorbox}[
                                    colback=green!10,
                    colframe=green!50!black,
                                fonttitle=\Large\bfseries,
                title=15:00
            ]
                {\Large Spoločný výlet}
                                                            \end{tcolorbox}
                    \begin{tcolorbox}[
                                    colback=white,
                    colframe=black!70!white,
                                fonttitle=\Large\bfseries,
                title=17:00
            ]
                {\Large Inštalácia AMOSov ++}
                                                            \\ \textbf{\textit{Jaro Šimon}}
                                                    \\ \textit{AGO Modra}                \\[2ex]Celooblohový systeé na pozorovanie meteorov AMOS sa v súčasnosti okrem Slovenska nachádza aj na viacerých vyznamných observatoriách vo svete. Na prednáške si ukážeme zaujimavé fotografie z miest, kde je nainštalovaný, a porozprávame si o zážitkoch pri inštaláciách.
            \end{tcolorbox}
                    \begin{tcolorbox}[
                                    colback=white,
                    colframe=black!70!white,
                                fonttitle=\Large\bfseries,
                title=18:00
            ]
                {\Large Kerbal Space Program}
                                                            \\ \textbf{\textit{Martin Baláž}}
                                                    \\ \textit{FMFI UK}                \\[2ex]Kerbal Space Program is one of the most successful realistic space program simulation games. Apart from entertainment, it is very useful in teaching and intuitively understanding basic concepts of orbital mechanics, rocket construction and space mission design.
            \end{tcolorbox}
                    \begin{tcolorbox}[
                                    colback=red!10,
                    colframe=red!50!black,
                                fonttitle=\Large\bfseries,
                title=19:00
            ]
                {\Large Večera}
                                                            \end{tcolorbox}
                    \begin{tcolorbox}[
                                    colback=blue!10,
                    colframe=black!50!blue,
                                fonttitle=\Large\bfseries,
                title=21:00
            ]
                {\Large Ako s ďalekohľadom? -- praktické ukážky}
                                                            \\ \textit{\textbf{Patrik Čechvala}}, \textit{\textbf{Karol Havrila}}                                                    \\ \textit{FMFI UK}                \\[2ex]Prax nás naučila, že v súčasnosti študenti nemajú veľké znalosti o práci s astronomickým ďalekohľadom. Tento workshop bude zameraný na zoznámanie sa s tým, ako správne rozložiť prenosný ďalekohľad, nastaviť ho a ako prebieha pozorovanie s takýmto ďalekohľadom. Tieto zručnosti sa častokrát hodia pri verejných pozorovaniach, no mali by patriť k základnej výbave každého správneho astronóma.
            \end{tcolorbox}
                    \section{Sobota}


                    \begin{tcolorbox}[
                                    colback=red!10,
                    colframe=red!50!black,
                                fonttitle=\Large\bfseries,
                title=09:00
            ]
                {\Large Raňajky}
                                                            \end{tcolorbox}
                    \begin{tcolorbox}[
                                    colback=white,
                    colframe=black!70!white,
                                fonttitle=\Large\bfseries,
                title=10:00
            ]
                {\Large Výskum airglowu na Slovensku}
                                                            \\ \textbf{\textit{Šimon Mackovjak}}
                                                    \\ \textit{Oddelenie kozmickej fyziky, Ústav experimentálnej fyziky Slovenskej akadémie vied}                            \end{tcolorbox}
                    \begin{tcolorbox}[
                                    colback=white,
                    colframe=black!70!white,
                                fonttitle=\Large\bfseries,
                title=10:30
            ]
                {\Large ???}
                                                            \\ \textbf{\textit{Peter Vereš}}
                                                    \\ \textit{Minor Planet Center, Harvard-Smithsonian Institute for Astrophysics}                            \end{tcolorbox}
                    \begin{tcolorbox}[
                                    colback=white,
                    colframe=black!70!white,
                                fonttitle=\Large\bfseries,
                title=11:00
            ]
                {\Large ešte nevie}
                                                            \\ \textbf{\textit{Michal Šturc}}
                                                    \\ \textit{FMFI UK}                            \end{tcolorbox}
                    \begin{tcolorbox}[
                                    colback=red!10,
                    colframe=red!50!black,
                                fonttitle=\Large\bfseries,
                title=12:30
            ]
                {\Large Obed}
                                                            \end{tcolorbox}
                    \begin{tcolorbox}[
                                    colback=white,
                    colframe=black!70!white,
                                fonttitle=\Large\bfseries,
                title=16:00
            ]
                {\Large The diet of galaxies}
                                                            \\ \textbf{\textit{Pol Massana}}
                                                    \\ \textit{University of Surrey}                \\[2ex]Galaxies are gigantic complex structures and understanding how they are formed and evolve is a difficult matter. But one thing we do know, they feed from each other. Under the current paradigm of galaxy evolution, big galaxies have grown by assimilating smaller galaxies. These kind of events usually produce the most astonishingly beautiful images that can be taken with telescopes. But what is really happening when two galaxies collide? Is there really any violence involved at all? Should we be worried that our galaxy will be eaten as well? In these talk we will answer all these questions and learn about the latest developments in our understanding of galaxies.
            \end{tcolorbox}
                    \begin{tcolorbox}[
                                    colback=white,
                    colframe=black!70!white,
                                fonttitle=\Large\bfseries,
                title=16:30
            ]
                {\Large Prvý hnedý trpaslík z vesmírnej misie TESS objavený v Ondřejove}
                                                            \\ \textbf{\textit{Ján Šubjak}}
                                                    \\ \textit{Astronomický ústav AV ČR Ondřejov}                \\[2ex]V súčasnosti sú známe približne dve desiatky tranzitujúcich hnedých trpaslíkov s presne zmeranou hmotnosťou a polomerom. K nim sa nedávno pridal systém TOI-503 a pár ďalších čerstvo objavených systémov z vesmírnej misie TESS a potvrdených dodatočnou spektroskopiou. Napriek úspechom tejto misie však ostávame limitovaný pri štúdiu a pochopení populácie hnedých trpaslíkov vzhľadom na malý počet jej objavených členov a každý nový člen sa stáva testom pre súčasné modely a poznatky. TOI-503 je prvý hnedý trpaslík z vesmírnej misie TESS a vôbec prvý v okolí hviezdy spektrálneho typu Am. Podrobne sme analyzovali získané fotometrické a spektroskopické dáta pre tento systém a charakterizovali jeho parametre, ako aj možné spôsoby formovania.
            \end{tcolorbox}
                    \begin{tcolorbox}[
                                    colback=white,
                    colframe=black!70!white,
                                fonttitle=\Large\bfseries,
                title=17:00
            ]
                {\Large Optimisation of double-station balloon flights for meteor observation: Prediction of the MALBEC nacelle trajectory}
                                                            \\ \textbf{\textit{Danica Žilková}}
                                                    \\ \textit{FMFI UK}                \\[2ex]The “Meteor Automated Light Balloon Experimental Camera” (MALBEC) project aims at the observation of meteors from stratospheric altitude. The advantage is mainly to guarantee the success of an observation run of a meteor shower, even in presence of clouds. In order to fully exploit the scientific potential of a meteor observation (e.g. derive the internal structure and origin via the measure of tensile strength and orbit), double-station setup is required. The consequence for MALBEC is the necessity of stabilisation and we show that a 3-axis stabilisation is necessary. In addition, the two stations must be separated by a distance ranging from \textasciitilde\SIrange{40}{110}{\kilo\metre}, and the cameras must point towards the same portion of atmosphere. We show that under usual circumstances, double station stratospheric observation is possible since the distance and the azimuth between the two balloons (experiencing different atmospheric conditions) varies in small proportions. Under usual slow wind conditions, the distance between the stations varies by a few km and the elevation of the azimuth and elevation of the cameras needed to observe the same portion of atmosphere varies by a few deg only. In order to demonstrate the feasibility of stratospheric double station meteor observation we developed a tool to simulate the flight and predict the trajectory of the MALBEC nacelle. This will be further illustrated in the upcoming presentation.
            \end{tcolorbox}
                    \begin{tcolorbox}[
                                    colback=white,
                    colframe=black!70!white,
                                fonttitle=\Large\bfseries,
                title=18:00
            ]
                {\Large Popularizácia astronómie cez občianske združenie Slovenské planetáriá}
                                                            \\ \textbf{\textit{Juraj Kubica}}
                                                    \\ \textit{Slovenské planetáriá}                \\[2ex]Aký je stav popularizácie prírodných vied na Slovensku a ako sa v tom angažuje občianske združenie Slovenské planetáriá, ktoré sa zasadzuje za výstavbu prvého planetária v Bratislavskom kraji.
            \end{tcolorbox}
                    \begin{tcolorbox}[
                                    colback=red!10,
                    colframe=red!50!black,
                                fonttitle=\Large\bfseries,
                title=19:00
            ]
                {\Large Večera}
                                                            \end{tcolorbox}
                    \begin{tcolorbox}[
                                    colback=blue!10,
                    colframe=black!50!blue,
                                fonttitle=\Large\bfseries,
                title=20:00
            ]
                {\Large Kozmický rebrík}
                                                            \\ \textit{\textbf{Martin Baláž}}, \textit{\textbf{Patrik Čechvala}}, \textit{\textbf{Karol Havrila}}, \textit{\textbf{Roman Nagy}}                                                    \\ \textit{FMFI UK}                \\[2ex]Na tomto workshope si hravou formou ukážeme, akými spôsobmi sa dajú merať vzdialenosti v kozme.
            \end{tcolorbox}
                    \section{Nedeľa}


                    \begin{tcolorbox}[
                                    colback=red!10,
                    colframe=red!50!black,
                                fonttitle=\Large\bfseries,
                title=09:00
            ]
                {\Large Raňajky}
                                                            \end{tcolorbox}
                    \begin{tcolorbox}[
                                    colback=white,
                    colframe=black!70!white,
                                fonttitle=\Large\bfseries,
                title=10:00
            ]
                {\Large Star And Planet’s Characterisation Through High Spectral Resolution}
                                                            \\ \textbf{\textit{Maria Chiara Maimone}}
                                                    \\ \textit{Côte d’Azur observatory}                \\[2ex]With thousands of confirmed exoplanets and an increasing number of dedicated instruments, we are finally moving into an era where we can address fundamental questions concerning the diversity of their compositions, their atmospheric and interior processes, and their formation histories. How? Via their observable spectroscopic signatures. In the last decade, tremendous progress has been made in detecting and characterising atmospheric signatures of exoplanets through spectroscopic methods, allowing to unveil the composition for a dozen of them (Birkby, 2018).
Nevertheless these extraordinary results, we are only at the beginning: stellar and planetary models are still computed separately, and 1D models, largely used for the stars until now, do not reproduce the complexity of convection mechanism (Chiavassa \& Brogi, 2019). Our work could be the turning point: we aims at upgrading the already-in-place 3D radiative transfer code Optim3D (Chiavassa et al. 2009) -- largely used for stellar purposes so far -- to taking into account also the exoplanetary contribution. We propose to use simultaneously 3D Radiative Hydrodynamical simulations, performed for stars, and the innovative Global Climate Model (GCM), drawn up for exoplanets, in order to generate unprecedented precise synthetic spectra. As springboard to test the code, the analysis of CO and $\mathrm{H}_2\mathrm{O}$ molecules will be carried out on the well-know benchmark HD189733. Indeed, one of the most challenging problems is to disentangle star’s and its companion’s signals due to the same molecules. Hence, a complete dynamic characterisation is crucial: on one side, a precise knowledge of the stellar dynamic (i.e. convection-related surface structures) would allow to extract unequivocally the planetary signal; on the other one, a well-modelled dynamic of the planet (i.e. depth, shape, and position of spectral lines) would provide us with considerable information about the planetary atmospheric circulation.
            \end{tcolorbox}
            \end{document}

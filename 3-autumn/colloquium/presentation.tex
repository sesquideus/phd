\documentclass[12pt,aspectratio=1610]{beamer}
\linespread{1.0}
\setlength{\parindent}{0cm}
\setlength{\parskip}{6pt}
\setlength{\abovedisplayskip}{0mm}
\setlength{\belowdisplayskip}{0mm}
\setlength{\abovedisplayshortskip}{0mm}
\setlength{\belowdisplayshortskip}{0mm}
\setlength{\itemindent}{0pt}
\setlength{\textfloatsep}{0mm}
\setlength{\tabcolsep}{3mm}
\renewcommand{\arraystretch}{1.2}

\setcounter{secnumdepth}{0}

\NewDocumentCommand{\fspicture}{m O{W} O{black}}{
    {
        \setbeamertemplate{navigation symbols}{}
        \setbeamercolor{background canvas}{bg = #3}
        \begin{frame}[plain]
            \begin{tikzpicture}[remember picture, overlay]
                \node[at=(current page.center)] {
                    \ifstrequal{H}{#2}{                                  
                        \includegraphics[height=\paperheight]{#1}%
                    }{%
                        \includegraphics[width=\paperwidth]{#1}%
                    }
                };
            \end{tikzpicture}
        \end{frame}
    }
}

\NewDocumentCommand{\frejm}{m +m}{
    \begin{frame}
        \frametitle{#1}
        #2
    \end{frame}
}
\defbeamertemplate{description item}{align center}{\hfill\insertdescriptionitem\hfill}
\definecolor{desc}{rgb}{0.66, 0, 0}
\definecolor{citem}{rgb}{0.72, 0, 0}
\definecolor{csitem}{rgb}{0.90, 0, 0}
\definecolor{cssitem}{rgb}{1, 0.1, 0.1}
\definecolor{qprimarybg}{rgb}{0.95, 0.95, 0.95}
\definecolor{check}{rgb}{0, 0.8, 0}

\setbeamertemplate{navigation symbols}{}
\newfontfamily{\semibold}{Segoe UI Semibold}
\RenewDocumentCommand{\emph}{m}{{\semibold#1}}

\title{Hapke, OSIRIS \& Kvet ústavy 1}
\subtitle{Doing science in times of a pandemic}
\author{\small \emph{Martin Baláž}}
\institute{DAA colloquium}
\date{2020--10--21}

\begin{document}
    {
        \usebackgroundtemplate{\includegraphics[width=\paperwidth]{pictures/gtc.jpg}}
        \begin{frame}
            \titlepage
        \end{frame}
    }
    \section{Overview}
        \frejm{Gentle introduction}{
            Desperate times call for desperate measures...
            \pause
            \begin{itemize}
                \item last day exit from Slovakia
                \item 50 days of \emph{lockdown}
                \pause
                \item ...and also some work
                    \begin{itemize}
                        \item one whole day
                        \item home office until the end of time
                    \end{itemize}
                \pause
                \item trips and fun as well
            \end{itemize}
        }

    \section{OSIRIS}
        \frejm{OSIRIS}{
            Optical System for Imaging and low Resolution Integrated Spectroscopy
            \begin{itemize}
                \item long slit / multiple object spectroscopy
                \item two CCDs (2 × 2048 × 4096, binning 2 × 2)
            \end{itemize}
            \pause
            \begin{itemize}
                \item for us
                \begin{itemize}
                    \item long slit spectroscopy
                    \item reflectance spectra of faint asteroids
                    \item always use CCD2
                    \item design a pipeline to process the original output
                \end{itemize}
            \end{itemize}
        }

        \frejm{\texttt{IRAF} and \texttt{Starlink}}{
            \begin{columns}
                \begin{column}{0.66\textwidth}
                    \begin{itemize}
                        \item IRAF
                            \begin{itemize}
                                \item US NOAO, since 1985
                                \item horrible installation process
                                \item last release 2012
                                \item ``usage in new projects \emph{discouraged}''
                            \end{itemize}
                        \pause
                        \item Starlink
                            \begin{itemize}
                                \item the same but very British
                                \item \SI{1}{\giga\byte} of... something
                                \item slightly newer
                            \end{itemize}
                    \end{itemize}
                \end{column}
                \pause
                \begin{column}{0.33\textwidth}
                    \includegraphics[width=\linewidth]{pictures/dinosaur-computer.jpg}
                \end{column}
            \end{columns}
        }

        \frejm{\texttt{THELI} and \texttt{ccdproc}}{
            Let us use something newer...
            \begin{itemize}
                \item Theli
                    \begin{itemize}
                        \item new German software
                        \item highly recommended for general purpose reduction
                    \end{itemize}
                \pause
                \item \texttt{ccdproc}
                    \begin{itemize}
                        \item not a \emph{program}, but a \emph{Python library}
                        \item well documented
                        \item great for pipeline integration
                    \end{itemize}
            \end{itemize}
        }

        \frejm{Osiris reduction pipeline}{
            Only basic operations... but fine-tuned for this purpose
            \pause
            \begin{itemize}
                \item preprocessing
                    \begin{itemize}
                        \item master bias
                        \item master flat
                        \item trimming
                    \end{itemize}
                \pause
                \item extraction of the spectrum
                    \begin{itemize}
                        \item identification
                        \item background
                        \item cosmic rays
                    \end{itemize}
                \pause
                \item normalization
                    \begin{itemize}
                        \item spectral flat field
                        \item divide by solar analogue spectrum
                    \end{itemize}
            \end{itemize}
        }

    \section{Geofit}
        \frejm{Geofit}{
            \begin{itemize}
                \item program by Ted Roush
                \item FORTRAN77
                \item written in January 1992
                \pause
                \item uses \emph{Hapke model} to compute synthetic reflectance spectra
                \item comparison to observational spectra
                \pause
                \item virtually no comments
                \item untouched since 2012
            \end{itemize}
        }

        \frejm{Objective}{
            \begin{columns}
                \begin{column}{0.7\textwidth}
                    \begin{itemize}
                        \item make at least some sense of it
                        \item rewrite in \texttt{Python}
                        \item optimize
                        \pause
                        \item job halfway between \emph{programming} and \emph{archaeology}
                    \end{itemize}
                \end{column}
                \begin{column}{0.28\textwidth}
                    \includegraphics[width=\linewidth]{pictures/frankenstein.jpg}
                \end{column}
            \end{columns}
        }

        \frejm{Operation}{
            \begin{columns}
                \begin{column}{0.28\textwidth}
                    \includegraphics[width=\linewidth]{pictures/brdf.png}
                \end{column}
                \begin{column}{0.7\textwidth}
                    \begin{itemize}
                        \item \emph{ab initio} model of BRDF $f_r(\omega_i, \omega_r)$
                        \item material mixtures, interpolation of spectral values
                        \item Nelder-Mead minimization of the objective ($\chi^2$)
                        \begin{itemize}
                            \item \url{https://www.youtube.com/watch?v=j2gcuRVbwR0}
                        \end{itemize}
                        \pause
                        \item database of materials
                        \begin{itemize}
                            \item try various tuples
                            \item report minimum
                        \end{itemize}
                    \end{itemize}
                \end{column}
            \end{columns}
        }

    \section{Comet C/2020 F3 (NEOWISE)}
        \frejm{Comet}{
            The brighest comet visible from Slovakia since 1997
            \begin{itemize}
                \item naked eye
                \item even straight from Bratislava
                \item nice in binoculars
            \end{itemize}
        }

        %\fspicture{pictures/neowise.jpg}

    \section{2020 Perseids}
        \frejm{Dark skies}{
            \begin{itemize}
                \item Bortle scale
                \item magnitude per sq arcsec
                \item artificial to natural ratio...
            \end{itemize}

            \begin{table}
                \begin{tabular}{c r c}
                    \toprule
                        place & ratio & Bortle scale \\
                    \midrule
                        Bratislava  &       30  & 9 \\
                        AGO Modra   & \num{1.2} & 4 \\
                        rural sky   & \num{0.5} & 3 \\
                        Poloniny    & \num{0.045} & 2 \\
                        Atacama     & \num{0.0006} & 1 \\
                    \bottomrule
                \end{tabular}
            \end{table}
        }

        \frejm{Location}{
            \begin{itemize}
                \item first night 2020--08--11/12
                    \begin{itemize}
                        \item location: \emph{Grúň}, Hažín, MI, \SI{134}{\metre}
                        \item ratio \num{0.33} (Bortle 3)
                    \end{itemize}
                \item second night 2020--08--12/13
                    \begin{itemize}
                        \item location: \emph{Sninský kameň}, Vihorlatské vrchy, \SI{1006}{\metre}
                        \item ratio \num{0.13} (Bortle 2)
                    \end{itemize}
            \end{itemize}
        }

        \frejm{Observation setup}{
            In the end \emph{two nights} of observation
            \begin{itemize}
                \item six observers, covering $\pm$ whole sky
                \item less than optimal conditions this year
                \pause
                    \begin{itemize}
                        \item thousands of mosquitoes
                        \item rising Moon (01:00 -- 02:00)
                        \item COVID
                    \end{itemize}
                \pause
                \item multiple coverage
            \end{itemize}
        }

        \frejm{Earthquake!}{
            \begin{columns}
                \begin{column}{0.4\textwidth}
                    \begin{itemize}
                        \item 2020--08--12T18:07:09Z (sunset)
                        \item magnitude \num{2.1}
                        \item loud crash, rumbling sound
                    \end{itemize}
                \end{column}
                \begin{column}{0.59\textwidth}
                    \includegraphics[keepaspectratio, width=\textwidth]{pictures/seismo.png}
                \end{column}
            \end{columns}
        }

        \fspicture{pictures/starlink.jpg}[][black]

        \frejm{Meteors}{
            \begin{itemize}
                \item only differential recording
                \item no timestamps
                \item 18:30 -- 01:30 UTC
                \item summary
                    \begin{itemize}
                        \item $122 + 89 = 211$ FIX THIS
                        \item $341 + 235 = 576$
                    \end{itemize}
                \pause
                \item highlights
                    \begin{itemize}
                        \item mystery object
                        \item super-fragmenting yellow meteor
                    \end{itemize}
            \end{itemize}
        }

        \frejm{Future plans}{
            \begin{itemize}
                \item Orionids (tonight!)
                    \begin{itemize}
                        \item ZHR about 20
                        \item morning hours
                    \end{itemize}
                \pause
                \item Geminids
                    \begin{itemize}
                        \item new moon (!)
                        \item predictions of high ZHR (>150)
                    \end{itemize}
                \pause
                \item 2021 Perseids
                    \begin{itemize}
                        \item new moon
                        \item La Palma?
                    \end{itemize}
            \end{itemize}
        }

    \section{Kvet ústavy 1}
        \frejm{Objective}{
            \begin{itemize}
                \item mission by ??? team (Dušan Velič)
                \item collect meteoroid dust in the upper atmosphere
                \item first iteration
                    \begin{itemize}
                        \item proof of concept
                        \item test electronics, collector
                        \item obtain launch permission
                        \item preliminary analyses
                    \end{itemize}
            \end{itemize}
        }

        \frejm{History}{
            \begin{itemize}
                \item failed attempt December 2019 (misia sv. Mikuláša)
                \pause
                \item failed attempt 2020--09--03 (bad weather)
                \pause
                \item successful attempt 2020--09--13
                \begin{itemize}
                    \item reached altitude of \SI{32760}{\metre}
                    \item ground track \SI{52}{\kilo\metre}
                    \item successful recovery
                \end{itemize}
            \end{itemize}
        }

        \fspicture{pictures/2d_map.png}
        \fspicture{pictures/3dmap_1.png}
        \fspicture{pictures/3dmap_2.png}

        \frejm{Results}{
            \begin{itemize}
                \item lots of readily identifiable debris
                \pause
                    \begin{itemize}
                        \item pollen
                        \item fragments of maize leaves
                        \item insects
                        \item fragments of the device itself (?)
                    \end{itemize}
            \end{itemize}
        }

    \section{Summary}
        \frejm{Wake up, it's over}{
            If there are any questions...
        }

        \frejm{References}{
            \begin{itemize}
                \item \textbf{Slovenské seizmologické stránky}: Lokálne zemetrasenia
                    \url{http://www.seismology.sk/Local_Earthquakes/}
                \item \textbf{SatLeft}: image of Starlink train
                    \url{https://www.satflare.com/obspic/P822363815806934143.jpg}
                \item pictures of KVÚ-1 path by \textbf{Michal Valíček}
            \end{itemize}
        }
\end{document}

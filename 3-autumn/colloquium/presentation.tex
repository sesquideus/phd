\documentclass[12pt,aspectratio=1610]{beamer}
\linespread{1.0}
\setlength{\parindent}{0cm}
\setlength{\parskip}{6pt}
\setlength{\abovedisplayskip}{0mm}
\setlength{\belowdisplayskip}{0mm}
\setlength{\abovedisplayshortskip}{0mm}
\setlength{\belowdisplayshortskip}{0mm}
\setlength{\itemindent}{0pt}
\setlength{\textfloatsep}{0mm}
\setlength{\tabcolsep}{3mm}
\renewcommand{\arraystretch}{1.2}

\setcounter{secnumdepth}{0}

\NewDocumentCommand{\fspicture}{m O{W} O{black}}{
    {
        \setbeamertemplate{navigation symbols}{}
        \setbeamercolor{background canvas}{bg = #3}
        \begin{frame}[plain]
            \begin{tikzpicture}[remember picture, overlay]
                \node[at=(current page.center)] {
                    \ifstrequal{H}{#2}{                                  
                        \includegraphics[height=\paperheight]{#1}%
                    }{%
                        \includegraphics[width=\paperwidth]{#1}%
                    }
                };
            \end{tikzpicture}
        \end{frame}
    }
}

\NewDocumentCommand{\frejm}{m +m}{
    \begin{frame}
        \frametitle{#1}
        #2
    \end{frame}
}
\defbeamertemplate{description item}{align center}{\hfill\insertdescriptionitem\hfill}
\definecolor{desc}{rgb}{0.66, 0, 0}
\definecolor{citem}{rgb}{0.72, 0, 0}
\definecolor{csitem}{rgb}{0.90, 0, 0}
\definecolor{cssitem}{rgb}{1, 0.1, 0.1}
\definecolor{qprimarybg}{rgb}{0.95, 0.95, 0.95}
\definecolor{check}{rgb}{0, 0.8, 0}

\setbeamertemplate{navigation symbols}{}
\newfontfamily{\semibold}{Segoe UI Semibold}
\RenewDocumentCommand{\emph}{m}{{\semibold#1}}

\title{Hapke, OSIRIS \& Kvet ústavy 1}
\subtitle{Doing science in times of a pandemic}
\author{\small \emph{Martin Baláž}}
\institute{DAA colloquium}
\date{2020--10--21}

\begin{document}
    {
        \usebackgroundtemplate{\includegraphics[width=\paperwidth]{pictures/fireworks-i.png}}
        \begin{frame}
            \titlepage
        \end{frame}
    }
    \section{Overview}
        \frejm{Motivation}{
                    }


    \section{OSIRIS}
        \frejm{OSIRIS}{
            Optical System for Imaging and low Resolution Integrated Spectroscopy
            \begin{itemize}
                \item long slit / multiple object spectroscopy
                \item two CCDs (2 × 2048 × 4096, binning 2 × 2)
            \end{itemize}
            \pause
            \begin{itemize}
                \item for us
                \begin{itemize}
                    \item long slit spectroscopy
                    \item reflectance spectra of faint asteroids
                    \item always use CCD2
                    \item design a pipeline to process the original output
                \end{itemize}
            \end{itemize}
        }

        \frejm{\texttt{IRAF} and \texttt{Starlink}}{
            \begin{columns}
                \begin{column}{0.5\textwidth}
                    \begin{itemize}
                        \item IRAF
                            \begin{itemize}
                                \item US NOAO, since 1985
                                \item horrible installation
                                \item last release 2012
                                \item ``usage in new projects \emph{discouraged}''
                            \end{itemize}
                        \pause
                        \item Starlink
                            \begin{itemize}
                                \item the same but very British
                                \item \SI{1}{\giga\byte} of... something
                                \item slightly newer
                            \end{itemize}
                    \end{itemize}
                \end{column}
                \pause
                \begin{column}{0.4\textwidth}
                    \includegraphics[width=\linewidth]{pictures/dinosaur-computer.jpg}
                \end{column}
            \end{columns}
        }

        \frejm{\texttt{THELI} and \texttt{ccdproc}}{
            Let us use something newer...
            \begin{itemize}
                \item Theli
                    \begin{itemize}
                        \item new German software
                        \item highly recommended for general purpose reduction
                    \end{itemize}
                \pause
                \item \texttt{ccdproc}
                    \begin{itemize}
                        \item not a \emph{program}, but a \emph{Python library}
                        \item well documented
                        \item great for pipeline integration
                    \end{itemize}
            \end{itemize}
        }

        \frejm{Osiris reduction pipeline}{
            Basic operations but fine-tuned for this purpose
            \begin{itemize}
                \item preprocessing
                    \begin{itemize}
                        \item master bias
                        \item master flat
                        \item trimming
                    \end{itemize}
                \pause
                \item extraction of the spectrum
                    \begin{itemize}
                        \item identification
                        \item background
                        \item cosmic rays
                    \end{itemize}
                \item normalization
                    \begin{itemize}
                        \item spectral flat field
                        \item divide by solar analogue spectrum
                    \end{itemize}
            \end{itemize}
        }

    \section{Geofit}
        \frejm{Geofit}{
            \begin{itemize}
                \item program by Ted Roush
                \item FORTRAN77
                \item written in January 1992
                \pause
                \item uses \emph{Hapke model} to compute synthetic reflectance spectra
                \item comparison to observational spectra
                \pause
                \item virtually no comments
                \item untouched since 2012
            \end{itemize}
        }

        \frejm{Objective}{
            \begin{columns}
                \begin{column}{0.7\textwidth}
                    \begin{itemize}
                        \item make at least some sense of it
                        \item rewrite in \texttt{Python}
                        \item optimize
                        \pause
                        \item job halfway between \emph{programming} and \emph{archaeology}
                    \end{itemize}
                \end{column}
                \begin{column}{0.28\textwidth}
                    \includegraphics[width=\linewidth]{pictures/frankenstein.jpg}
                \end{column}
            \end{columns}
        }

        \frejm{Operation}{
            \begin{columns}
                \begin{column}{0.7\textwidth}
                    \begin{itemize}
                        \item \emph{ab initio} model of BRDF $f_r(\omega_i, \omega_r)$
                        \item material mixtures, interpolation of spectral values
                        \begin{itemize}
                            \item intimate
                            \item spatial
                        \end{itemize}
                        \pause
                        \item Nelder-Mead minimization of the objective function ($\chi^2$)
                        \item database of materials
                        \begin{itemize}
                            \item try various tuples
                            \item report minimum
                        \end{itemize}
                    \end{itemize}
                \end{column}
                \begin{column}{0.28\textwidth}
                    \includegraphics[width=\linewidth]{pictures/brdf.png}
                \end{column}
            \end{columns}
        }



    \section{Kvet ústavy 1}
        \frejm{Objective}{
            \begin{itemize}
                \item collect meteoroid dust in the upper atmosphere
                \item first iteration
                    \begin{itemize}
                        \item proof of concept
                        \item test electronics, collector
                        \item preliminary analyses
                    \end{itemize}
            \end{itemize}
        }

        \frejm{History}{
            \begin{itemize}
                \item failed attempt December 2019 (misia sv. Mikuláša)
                \item failed attempt 2020--09--03 (bad weather)
                \item successful attempt 2020--09--13
            \end{itemize}
        }




    \section{Summary}
        \frejm{References}{
            \begin{itemize}
                \item \textbf{Hwang, J.-N. -- Lay, S.-R. and Lippman, A.}:
                    Nonparametric multivariate density estimation: a case study.
                    IEEE Transactions on Signal Processing 42, 1994.
                \item \textbf{Vida, D. -- Brown, P. -- Campbell-Brown, M.}:
                    Modeling the measurement accuracy of pre-atmosphere velocities of meteoroids. MNRAS 479, 2018
                \item \textbf{Vida, D. -- Brown, P. -- Campbell-Brown, M.}:
                    Generating realistic synthetic meteoroid orbits. Icarus 296, 2017.
                \item Wikipedia, user \textbf{Drleft}:
                    Synthetic data 2D histograms, 2010.
                    \url{https://en.wikipedia.org/wiki/File:Synthetic_data_2D_histograms.png}
                \item Wikipedia, user \textbf{Drleft}:
                    Synthetic data 2D KDE, 2010.
                    \url{https://en.wikipedia.org/wiki/File:Synthetic_data_2D_KDE.png}
            \end{itemize}
        }
\end{document}

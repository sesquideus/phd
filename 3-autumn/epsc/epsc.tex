\documentclass[10pt]{beamer}
\RequirePackage{
    amsmath,
    amssymb,
    calc,
    cancel,
    booktabs,
    color,
    siunitx,
    tikz,
    wrapfig,
    array,
    leftidx,
    float,
    etoolbox,
    fancyhdr,
    longtable,
    hyperref,
    ltcaption,
    ulem,
    wasysym,
    accents,
    listings,
    tabularx,
}

\hypersetup{
    hidelinks,
    breaklinks              = true,
}

\usepackage[final]{pdfpages}
\usepackage[many]{tcolorbox}

\RenewDocumentCommand{\vec}{m}{\mathbf{#1}}

\makeatletter
    \def\new@mathgroup{\alloc@8\mathgroup\mathchardef\@cclvi}
    \patchcmd{\document@select@group}{\sixt@@n}{\@cclvi}{}{}
    \patchcmd{\select@group}{\sixt@@n}{\@cclvi}{}{}
\makeatother

\RequirePackage{mathspec}                                   % includes fontspec
\RequirePackage{polyglossia}                                % multi-language support
\RequirePackage{xunicode}
\setdefaultlanguage{slovak}

% Setup fonts -- see fontspec/mathspec documentation.
\defaultfontfeatures{
    Mapping         = tex-text,
    Scale           = MatchLowercase,
    Ligatures       = TeX
}


\NewDocumentCommand{\labelmath}{m +m}{%
    \begin{equation}%
        #2%
        \label{#1}%
    \end{equation}%
}

\NewDocumentCommand{\labelalign}{m +m}{%
    \begin{align}%
        #2%
        \label{#1}%
    \end{align}%
}

\linespread{1.0}
\setlength{\parindent}{0cm}
\setlength{\parskip}{6pt}
\setlength{\abovedisplayskip}{0mm}
\setlength{\belowdisplayskip}{0mm}
\setlength{\abovedisplayshortskip}{0mm}
\setlength{\belowdisplayshortskip}{0mm}
\setlength{\itemindent}{0pt}
\setlength{\textfloatsep}{0mm}
\setlength{\tabcolsep}{3mm}
\setlength{\LTcapwidth}{0.8\textwidth}
\renewcommand{\arraystretch}{1.2}

\setcounter{secnumdepth}{2}

/home/kvik/dgs/core/tex/math.tex
\DeclareSIUnit\au{AU}
\DeclareSIUnit\pixel{px}
\DeclareSIUnit\lightyear{ly}
\DeclareSIUnit\parsec{pc}
\DeclareSIUnit\earthmass{M_{\earth}}
\DeclareSIUnit\speedoflight{c}
\DeclareSIUnit\foe{foe}
\DeclareSIUnit\year{yr}
\DeclareSIUnit\eur{€}
\DeclareSIUnit\solarmass{M_{\astrosun}}
\DeclareSIUnit\solarluminosity{L_{\astrosun}}
\DeclareSIUnit{\byte}{B}




\linespread{1.0}
\setlength{\parindent}{0cm}
\setlength{\parskip}{6pt}
\setlength{\abovedisplayskip}{0mm}
\setlength{\belowdisplayskip}{0mm}
\setlength{\abovedisplayshortskip}{0mm}
\setlength{\belowdisplayshortskip}{0mm}
\setlength{\itemindent}{0pt}
\setlength{\textfloatsep}{0mm}
\setlength{\tabcolsep}{3mm}
\renewcommand{\arraystretch}{1.2}

\setcounter{secnumdepth}{0}

\NewDocumentCommand{\fspicture}{m O{W} O{black}}{
    {
        \setbeamertemplate{navigation symbols}{}
        \setbeamercolor{background canvas}{bg = #3}
        \begin{frame}[plain]
            \begin{tikzpicture}[remember picture, overlay]
                \node[at=(current page.center)] {
                    \ifstrequal{H}{#2}{                                  
                        \includegraphics[height=\paperheight]{#1}%
                    }{%
                        \includegraphics[width=\paperwidth]{#1}%
                    }
                };
            \end{tikzpicture}
        \end{frame}
    }
}

\NewDocumentCommand{\frejm}{m +m}{
    \begin{frame}
        \frametitle{#1}
        #2
    \end{frame}
}

\NewDocumentCommand{\fragfrejm}{m +m}{
    \begin{frame}[fragile]
        \frametitle{#1}
        #2
    \end{frame}
}

\defbeamertemplate{description item}{align center}{\hfill\insertdescriptionitem\hfill}
\definecolor{desc}{rgb}{0.66, 0, 0}
\definecolor{citem}{rgb}{0.72, 0, 0}
\definecolor{csitem}{rgb}{0.90, 0, 0}
\definecolor{cssitem}{rgb}{1, 0.1, 0.1}
\definecolor{qprimarybg}{rgb}{0.95, 0.95, 0.95}
\definecolor{check}{rgb}{0, 0.8, 0}
\definecolor{coded}{rgb}{0.9, 0.9, 0.9}
\definecolor{todo}{rgb}{1.0, 0.3, 0.3}
\definecolor{model}{rgb}{0.75, 0, 0}

\setbeamertemplate{navigation symbols}{}
\newfontfamily{\semibold}{Segoe UI Semibold}
\RenewDocumentCommand{\emph}{m}{{\semibold#1}}
\NewDocumentCommand{\code}{m}{\textcolor{desc}{\texttt{#1}}}
\NewDocumentCommand{\model}{m}{\colorbox{coded}{\textcolor{model}{\texttt{#1}}}}
\NewDocumentCommand{\todo}{m}{\colorbox{todo}{#1}}

\mode<presentation> {
    \usetheme{Szeged}
    \usecolortheme{beaver}
    
    \usefonttheme{professionalfonts}
    \setallmainfonts{Minion Pro}
    \setmathrm{Minion Pro}
    
    \setsansfont{Segoe UI}
    \setmonofont{Consolas}
    \setbeamercolor*{enumerate item}{fg = citem}
    \setbeamercolor*{enumerate subitem}{fg = csitem}
    \setbeamercolor*{enumerate subsubitem}{fg = cssitem}
    \setbeamercolor*{description item}{fg = desc}
    \setbeamercolor*{itemize item}{fg = citem}
    \setbeamercolor*{itemize subitem}{fg = csitem}
    \setbeamercolor*{itemize subsubitem}{fg = cssitem}
    \setbeamercolor*{palette primary}{fg = red, bg = qprimarybg}
}

\newcommand<>\highlightbox[2]{%
    \alt#3{\makebox[\dimexpr\width-2\fboxsep]{\colorbox{#1}{#2}}}{#2}%
}

\AtBeginSection[]{
    \subsection{\insertsection}
    \begin{frame}
        \vfill
        \centering
        \begin{beamercolorbox}[sep = 18pt, center, shadow = true, rounded = true]{title}
            \usebeamerfont{title}\insertsectionhead%
            \vfill
        \end{beamercolorbox}
        \vfill
    \end{frame}
}

\makeatletter
% Render percent sign with nice font, not ugly Computer modern
    \mathcode`\%="7025

% Fixes mathspec bug -- URL numbers are rendered with wrong font
    \ernewcommand\eu@MathPunctuation@symfont{Latin:m:n}
    \DeclareMathSymbol{,}{\mathpunct}{\eu@MathPunctuation@symfont}{`,}
    \DeclareMathSymbol{?}{\mathpunct}{\eu@MathPunctuation@symfont}{`?}
    \DeclareMathSymbol{.}{\mathord}{\eu@MathPunctuation@symfont}{`.}
    \DeclareMathSymbol{<}{\mathrel}{\eu@MathPunctuation@symfont}{`<}
    \DeclareMathSymbol{>}{\mathrel}{\eu@MathPunctuation@symfont}{`>}
    \DeclareMathSymbol{/}{\mathord}{\eu@MathPunctuation@symfont}{`/}
    \DeclareMathSymbol{;}{\mathpunct}{\eu@MathPunctuation@symfont}{`;}
    \DeclareMathSymbol{(}{\mathopen}{\eu@DigitsArabic@symfont}{`(}
    \DeclareMathSymbol{)}{\mathclose}{\eu@DigitsArabic@symfont}{`)}
    \XeTeXDeclareMathSymbol{^^^^2026}{\mathinner}{\eu@MathPunctuation@symfont}{"2026}[\mathellipsis]
    \DeclareMathSymbol{0}{\mathalpha}{\eu@DigitsArabic@symfont}{`0}
    \DeclareMathSymbol{1}{\mathalpha}{\eu@DigitsArabic@symfont}{`1}
    \DeclareMathSymbol{2}{\mathalpha}{\eu@DigitsArabic@symfont}{`2}
    \DeclareMathSymbol{3}{\mathalpha}{\eu@DigitsArabic@symfont}{`3}
    \DeclareMathSymbol{4}{\mathalpha}{\eu@DigitsArabic@symfont}{`4}
    \DeclareMathSymbol{5}{\mathalpha}{\eu@DigitsArabic@symfont}{`5}
    \DeclareMathSymbol{6}{\mathalpha}{\eu@DigitsArabic@symfont}{`6}
    \DeclareMathSymbol{7}{\mathalpha}{\eu@DigitsArabic@symfont}{`7}
    \DeclareMathSymbol{8}{\mathalpha}{\eu@DigitsArabic@symfont}{`8}
    \DeclareMathSymbol{9}{\mathalpha}{\eu@DigitsArabic@symfont}{`9}
\makeatother


\usepackage{tabularx}

\title{ASMODEUS Meteor Simulation Tool}
\subtitle{Universal Virtual Meteor Observatory}
\author{Martin Baláž}
\institute{DAPEM Comenius University}
\date{2020--09--07}

\begin{document}
    \begin{frame}
        \titlepage
    \end{frame}

    \section{Overview}
        \frejm{Objective}{
            To \emph{determine} the total meteoroid flux in millimetre to metre size range
            \begin{itemize}
                \item extrapolation from collected data
                \item we need to
                    \begin{itemize}
                        \item analyze the detection ability of AMOS
                        \item calibrate the system
                        \item \emph{de-bias} the observations
                    \end{itemize}
                \item estimate the flux
            \end{itemize}
        }

        \frejm{Algorithm}{
            So-called \textbf{holistic approach}

            \begin{itemize}
                \item suggested by Peter Vereš
                \begin{itemize}
                    \item originally used to determine the number of observable NEOs
                \end{itemize}
            \end{itemize}
            \pause
            \begin{itemize}
                \item simulate everything what could possibly be seen
                \item compare \emph{output} of the simulation to observations
                \begin{itemize}
                    \item change values of parameters (how?)
                    \item repeat until best match with observational data is found
                \end{itemize}
                \item look at the original \emph{input} of the simulation
                \begin{itemize}
                    \item declare it to be the true population
                    \item show or calculate whatever interests you
                \end{itemize}
            \end{itemize}
        }

        \frejm{Algorithm -- details}{
            \begin{enumerate}
                \item generate the meteoroid population
                \pause
                \item simulate atmospheric entry and create \texttt{Meteor} objects
                \pause
                \item compute virtual \texttt{Sighting}s using locations of \texttt{Observer}s
                \item filter visible events and apply observational bias
                    \begin{itemize}
                        \item distance
                        \item atmospheric attenuation
                        \item fill factor
                        \item limiting magnitude
                        \item activity variations
                        \item ...
                    \end{itemize}
                \pause
                \item calculate statistics and compare to AMOS data
                \item adjust the particle distribution and observational bias parameters
                \pause
                \item repeat \textit{ad libitum}
            \end{enumerate}
        }

    \section{\textsc{Asmodeus}}
        \frejm{What is it}{
            \textbf{A}ll-\textbf{S}ky \textbf{M}eteor \textbf{O}ptical \textbf{D}etection \textbf{E}fficiency \textbf{S}imulator
            \begin{itemize}
                \item a suite of five or six \emph{Python} scripts
                \begin{itemize}
                    \item custom-written vector transformations
                    \item low memory footprint
                    \item multithreaded
                \end{itemize}
                \item implements the developed model
                \item ready to be published on \texttt{GitHub}
            \end{itemize}
        }

        \frejm{How does it work?}{
            \begin{itemize}
                \item \texttt{asmodeus-generate}
                    \begin{itemize}
                        \item creates and saves a population
                        \item simulates atmospheric entry
                    \end{itemize}
                \item \texttt{asmodeus-observe}
                    \begin{itemize}
                        \item computes and stores meteor observations
                    \end{itemize}
                \item \texttt{asmodeus-analyze}
                    \begin{itemize}
                        \item computes the statistics
                    \end{itemize}
                \item \texttt{asmodeus-multifit}
                    \begin{itemize}
                        \item performs a multi-parametric fit
                        \item compares simulation to observational data
                    \end{itemize}
                \item \texttt{asmodeus-plot}
                    \begin{itemize}
                        \item plots the resulting data
                        \item \texttt{gnuplot} is required to produce images
                        \item we should switch to \texttt{matplotlib} instead
                    \end{itemize}
            \end{itemize}
        }

        \frejm{Model}{
            Designed by \emph{Whipple} (1938), improved by \emph{Öpik} (1955) and \emph{Ceplecha} (2001)

            We assume
            \begin{itemize}
                \item spherical particles
                \item moving in a straight line
                \begin{itemize}
                    \item no gravity
                    \item only drag / ablation
                \end{itemize}
            \end{itemize}

            And we need
            \begin{itemize}
                \item equations of motion
                \item equations of luminance
                \item to construct a virtual CCD image
                \item to compute the statistic
            \end{itemize}
        }

        \frejm{Equations of motion}{
            \begin{itemize}
                \item braking equation
                $$
                    \diff{v} = -\frac{\Gamma A}{m^{1/3} \rho^{2/3}} \rho_{\mathrm{air}} v^2 \diff{t}
                $$
                \item equation of ablation
                $$
                    \diff{m} = -\frac{\Lambda A}{2Q} \frac{m^{2/3}}{\rho^{2/3}} \rho_{\mathrm{air}} v^3 \diff{t}
                $$
                \item equation of luminance
                $$
                    L = \tau(v) \frac{\Lambda A}{4Q} \frac{m^{2/3}}{\rho^{2/3}} \rho_{\mathrm{air}} v^5
                $$
                \begin{itemize}
                    \item $\tau(v)$ determined by \emph{Jones \& Halliday (2001)}
                \end{itemize}
            \end{itemize}
        }

        \frejm{Simulation of flight}{
            \begin{itemize}
                \item equations solved by the Runge--Kutta method (RK4)
                \item until complete ablation of the particle
                \item properties recorded in every \texttt{Frame} (1/20 second)
            \end{itemize}
            \pause
            \begin{itemize}
                \item multiple integration steps between frames
                \begin{itemize}
                    \item greatly enhanced precision
                    \item trajectory data are very precise even without splitting
                    \item luminance increases by up to about 5\%
                \end{itemize}
            \end{itemize}
            \begin{itemize}
                \item more precise integrator is not necessary
                \begin{itemize}
                    \item Vaubaillon (2018)
                    \item \texttt{IAS15} or something symplectic
                \end{itemize}
            \end{itemize}
        }

    \section{\texttt{asmodeus-observe}}
        \frejm{Virtual observations}{
            In the next step we create observations
            \begin{itemize}
                \item multiple observers on the ground
                \item each represents an AMOS camera
                \item reduction of properties
                \begin{itemize}
                    \item entire \textit{streak} may be shown
                    \item but only a \textit{dot} is analyzed
                    \begin{itemize}
                        \item brightest frame
                        \item initial mass
                        \item initial and terminal elevation
                        \item flight time...
                    \end{itemize}
                \end{itemize}
            \end{itemize}
        }

        \fspicture{shower.png}[H][black]
        \fspicture{angularSpeed-ago.png}[H][black]

        \frejm{Selection bias}{
            \emph{Detection efficiency is not constant!}
            \begin{itemize}
                \item probability of detection is higher for meteors that are
                \begin{itemize}
                    \item bright
                    \item fast
                    \item close to zenith
                    \item ...
                \end{itemize}
            \end{itemize}
        }

        \frejm{Selection bias -- quantitative}{
            Bias summarized in \emph{detection probability functions}
            \begin{itemize}
                \item determine whether a meteor is detected
                \item magnitude dependence
                $$
                    D(m; f, m_0, \omega) = \frac{f}{1 + e^{\frac{m - m_0}{\omega}}}
                $$
                \item altitudinal dependence
                $$
                    A(\theta; \alpha) = \left(\sin \theta\right)^\alpha
                $$
                \item we need to establish values of parameters $f$, $m_0$, $\omega$, $\alpha$
                \item assume the effects are \emph{independent}
            \end{itemize}
        }
        \fspicture{limmag.pdf}[W][white]
        \fspicture{altitude-powersine.pdf}[][white]







        \frejm{Validation of the model}{
            \begin{itemize}
                \item we used a Perseid bolide \emph{EN120812}
                \item position and velocity by \emph{Spurný (2015)}
                \item one virtual observer
                \begin{itemize}
                    \item directly below point of maximal brightness
                \end{itemize}
                \vspace{5mm}
                \item a very good match was obtained
                \begin{itemize}
                    \item discrepancy likely caused by fragmentation
                \end{itemize}
                \item we could further vary initiall mass and shape
                \begin{itemize}
                    \item but this is not good science
                \end{itemize}
            \end{itemize}
        }

        \fspicture{validation-power.png}[][white]

        \frejm{Validation results}{
            \begin{tabularx}{\textwidth}{X r r}
                \toprule
                    property    & reference & simulation \\
                \midrule
                    latitude at maximum brightness              & \ang{48.785}          & \ang{48.766} \\
                    longitude at maximum brightness             & \ang{13.505}          & \ang{13.476} \\
                    elevation at maximum brightness             & \SI{82700}{\metre}    & \SI{80733}{\metre} \\
                    absolute magnitude ($\tau = \text{0.095}$)  & ---                   & -8.54 \\
                    absolute magnitude ($\tau = \text{0.18}$)   & -9.7 $\pm$ 0.2        & -9.41 \\
                \midrule
                    terminal latitude                           & \ang{48.747}          & \ang{48.704} \\
                    terminal longitude                          & \ang{13.457}          & \ang{13.396} \\
                    terminal elevation                          & \SI{78600}{\metre}    & \SI{73130}{\metre} \\
                    terminal speed                              & \SI{55 \pm 1}{\kilo\metre\per\second} & \SI{56.045}{\kilo\metre\per\second} \\
                \bottomrule
            \end{tabularx}
        }

    \section{\texttt{asmodeus-observe}}
        \frejm{Magnitude determination}{
            \begin{itemize}
                \item we know the luminous power
                \item we need to calculate the absolute magnitude
                \begin{itemize}
                    \item meteor in zenith
                    \item \SI{100}{\kilo\metre} high
                    \item no atmospheric attenuation
                \end{itemize}
            \end{itemize}
        }

        \frejm{Magnitude determination}{
            \begin{itemize}
                \item the model is not very precise
                \begin{itemize}
                    \item material constants are wild guesses
                    \item no fragmentation
                    \item luminous efficiency $\tau$ is questionable at best
                \end{itemize}
                \pause
                \item AMOS has not been calibrated
                \item \textsc{UFOAnalyzerV2} outputs are not reliable for bolides
                \begin{itemize}
                    \item Spurný vs Kaniansky, 2018 Poloniny bolide
                \end{itemize}
                \pause
            \end{itemize}

        }

    \section{\texttt{asmodeus-multifit}}
        \frejm{Parameter variation}{
            \begin{itemize}
                \item the process is inherently stochastic
                \begin{itemize}
                    \item multiple repetitions and averaging
                \end{itemize}
                \item there are multiple possible approaches
            \end{itemize}
            \begin{tabularx}{\textwidth}{X r r}
                \toprule
                    property    & advantage & disadvantage \\
                \midrule
                    exhaustive  & very precise & slow \\
                    gradient descent & converges quickly & finds only a local minimum \\
                    simulated annealing & converges quickly & \\
                \bottomrule
            \end{tabularx}
        }


    \section{Results}
        \frejm{Evaluation}{
            \begin{itemize}
                \item we fixed the model and the particle count
                \begin{itemize}
                    \item Perseids 2016 (August 11--12)
                    \item observed from \emph{Tepličné} (\ang{48.6822}~N, \ang{19.8580}~E, \SI{700}{\metre})
                    \item seven hours (19:00 -- 02:00 UTC)
                    \item mass index $s =$ 1.8
                \end{itemize}
                \item 100000 meteoroids are generated
                \pause
            \end{itemize}
            \begin{itemize}
                \item we varied the DPF parameters with \texttt{asmodeus-multifit}
                \item histograms evaluated using $\chi^2$ tests
                \begin{itemize}
                    \item determine similarity between histograms
                    \item normalized to 1
                \end{itemize}
            \end{itemize}
        }
        \fspicture{example-magnitude-nobias.pdf}[][white]
        \fspicture{example-magnitude-bias.pdf}[][white]

        \frejm{Magnitude DPF}{
            $$
                D(m; f, m_0, \omega) = \frac{f}{1 + e^{\frac{m - m_0}{\omega}}}
            $$
            \begin{itemize}
                \item a wide range of parameter combinations was searched
                \item fill factor $f$ does not contribute any information
            \end{itemize}
            \pause
            \begin{itemize}
                \item find values of parameters where $\chi^2$ is minimal
                \item account for statistical noise
            \end{itemize}
        }
        \fspicture{chiSquare-magnitude-1d8.pdf}[][white]
        \fspicture{chiSquare-magnitude-1d8z.pdf}[][white]
        \fspicture{histogram-magnitude-1d8-optimum.pdf}[][white]

        \frejm{Mass index $s$}{
            There are way too many bright meteors...
            \pause
            \begin{itemize}
                \item a natural reaction is to try another value of $s$
                \begin{itemize}
                    \item a full range \numrange{1.6}{2.8} was tried
                \end{itemize}
                \item best fit for $s = \text{2.15}$
                \item no value below 2 is consistent with observations
            \end{itemize}

        }
        \fspicture{histogram-magnitude-total-optimum.pdf}[][white]

        \frejm{Altitudinal DPF}{
            $$
                A(\theta; \alpha) = \left(\sin \theta\right)^\alpha
            $$
            \begin{itemize}
                \item only a simple 1D fit
                \item a very well defined minimum at $\alpha = \text{0.4}$
            \end{itemize}
        }
        \fspicture{chiSquare-altitude-2d15.pdf}[][white]

        \frejm{Total flux}{
            Finally, we may calculate the total flux
            \begin{itemize}
                \item simulation is run again with \emph{optimal DPF parameters}
                $$A(\theta) = \left(\sin \theta\right)^{0.4}$$
                $$D(m) = \frac{1}{1 + e^{\frac{m + 0.1}{0.35}}}$$
                \item number of meteors is \emph{scaled} to match observations
            \end{itemize}

            \begin{itemize}
                \item we obtain \num{135000} particles per \SI{1000000}{\kilo\metre\squared\hour}
                \item \SI{0.338}{\kilo\gram} per \SI{1000000}{\kilo\metre\squared\hour},
                \item about \SI{43}{\kilo\gram\per\hour} on Earth
            \end{itemize}
        }

    \section{Conclusion}
        \frejm{Summary}{
            \begin{itemize}
                \item \texttt{asmodeus} works as expected
            \end{itemize}
            \pause
            \begin{itemize}
                \item simulation is a \emph{surprisingly good} method
                \begin{itemize}
                    \item correct geometry and luminance data
                    \item observations \emph{comparable} to real data
                    \item and results are aesthetically pleasing
                \end{itemize}
            \end{itemize}
            \pause
            \begin{itemize}
                \item we were able to estimate the flux
                \begin{itemize}
                    \item by \emph{varying} the DPF parameters
                    \item a very good fit was found
                    \pause
                    \item mass index seems to be much higher than known values
                    \item a larger observational dataset is needed
                \end{itemize}
            \end{itemize}
        }

    \section{Known problems}
        \frejm{Temporal variation}{
        }

        \frejm{Unit tests}{
            Sometimes, it is a good idea to scrutinize everything you are doing
            \begin{itemize}
                \item every single concept
            \end{itemize}

        }


        \frejm{Conclusion}{
            \begin{itemize}
                \item Python is your friend
                \item Test, test, test everything
            \end{itemize}
        }


        \frejm{Thank you for your attention}{
            \textit{The scientist is not a person who gives the right answers, he's one who asks the right questions.}
            \scriptsize
            \begin{flushright}
                Claude Lévi-Strauss\\
                Le Cru et le Cuit, 1964
            \end{flushright}
        }

        \frejm{References}{
            \begin{itemize}
                \item \textbf{Baláž, M. et al.}
                    ASMODEUS Meteor Simulation Tool. Planetary \& Space Science, 2020
                \item \textbf{Luciuk, M.}:
                    Meteor Showers. {\footnotesize \url{http://www.asterism.org/tutorials/tut36\%20Meteor\%20Showers.pdf}}.
                \item \textbf{Hill, K. A. -- Rogers, L. A. -- Hawkes, R. L.}:
                    High geocentric velocity meteor ablation. Astronomy \& Astrophysics 444, 615--624 (2005)
                \item \textbf{Öpik, E. J.}:
                    Physics of meteor flight in the atmosphere. Interscience Publishers, 1958.
            \end{itemize}
        }

\end{document}

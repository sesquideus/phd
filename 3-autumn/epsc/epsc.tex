\documentclass[10pt,aspectratio=1610]{beamer}
\linespread{1.0}
\setlength{\parindent}{0cm}
\setlength{\parskip}{6pt}
\setlength{\abovedisplayskip}{0mm}
\setlength{\belowdisplayskip}{0mm}
\setlength{\abovedisplayshortskip}{0mm}
\setlength{\belowdisplayshortskip}{0mm}
\setlength{\itemindent}{0pt}
\setlength{\textfloatsep}{0mm}
\setlength{\tabcolsep}{3mm}
\renewcommand{\arraystretch}{1.2}

\setcounter{secnumdepth}{0}

\NewDocumentCommand{\fspicture}{m O{W} O{black}}{
    {
        \setbeamertemplate{navigation symbols}{}
        \setbeamercolor{background canvas}{bg = #3}
        \begin{frame}[plain]
            \begin{tikzpicture}[remember picture, overlay]
                \node[at=(current page.center)] {
                    \ifstrequal{H}{#2}{                                  
                        \includegraphics[height=\paperheight]{#1}%
                    }{%
                        \includegraphics[width=\paperwidth]{#1}%
                    }
                };
            \end{tikzpicture}
        \end{frame}
    }
}

\NewDocumentCommand{\frejm}{m +m}{
    \begin{frame}
        \frametitle{#1}
        #2
    \end{frame}
}
\defbeamertemplate{description item}{align center}{\hfill\insertdescriptionitem\hfill}
\definecolor{desc}{rgb}{0.66, 0, 0}
\definecolor{citem}{rgb}{0.72, 0, 0}
\definecolor{csitem}{rgb}{0.90, 0, 0}
\definecolor{cssitem}{rgb}{1, 0.1, 0.1}
\definecolor{qprimarybg}{rgb}{0.95, 0.95, 0.95}
\definecolor{check}{rgb}{0, 0.8, 0}

\setbeamertemplate{navigation symbols}{}
\newfontfamily{\semibold}{Segoe UI Semibold}
\RenewDocumentCommand{\emph}{m}{{\semibold#1}}

\usepackage{tabularx}

\title{ASMODEUS Meteor Simulation Tool}
\subtitle{A Universal Virtual Meteor Observatory}
\author{Martin Baláž}
\institute{DAPEM Comenius University}
\date{2020--09--07}

\begin{document}
    \begin{frame}
        \titlepage
    \end{frame}

    \section{Overview}
        \frejm{What is it?}{
            an universal virtual meteor observatory

            \pause
            \begin{itemize}
                \item simulate a set of meteors
                \item compute how they would be seen from the ground
                \item analyze the dataset
                \item visualize
                \item draw conclusions
            \end{itemize}
        }

    \section{Simulation}
        \frejm{Generating meteors}{
            \begin{itemize}
                \item select general area and time
                \item ensure \emph{homogeneity} over the observed surface
                \item velocity vector
                    \begin{itemize}
                        \item showers (fixed radiant)
                        \item sporadic (complex)
                    \end{itemize}
                \pause
                \item variable activity in time
                \item sample from pre-defined distributions
                    \begin{itemize}
                        \item mass
                        \item material constants
                    \end{itemize}
            \end{itemize}
        }

        \frejm{Simulating meteors}{
            \begin{itemize}
                \item pluggable model
                    \begin{itemize}
                        \item Whipple, Ceplecha, fragmentation, ...
                        \item equations of motion
                    \end{itemize}

                \item selection of integrators
                    \begin{itemize}
                        \item Euler, RK4, Dormand--Prince, ...
                        \item constant or adaptive step
                    \end{itemize}

                \item snapshots recorded at constant time intervals
            \end{itemize}
        }


    \section{Observation}
        \frejm{Observation}{
            Meteoroids are transformed to meteors
            \begin{itemize}
                \item observers usually correspond to real cameras
                \item \emph{streaks} or \emph{dots} depending on purpose
                \pause
                \item distance ($\propto d^{-2}$)
                \item atmospheric attenuation ($\propto e^{-\kappa d}$)
                \item fill factor
                \item ...
            \end{itemize}
        }
        \fspicture{pictures/streaks.png}[H]
        \fspicture{pictures/dots.png}[H]

        \frejm{Selection bias effects}{
            Detection efficiency is not constant!

            \pause
            \begin{itemize}
                \item cameras generally prefer meteors...
                    \begin{itemize}
                        \item brighter
                        \item faster
                        \item closer to zenith (or centre of FoV -- vignetting)
                    \end{itemize}
            \end{itemize}

            \pause
            \begin{itemize}
                \item determine whether a meteor is detected
                \item magnitude dependence
                $$
                    D(m; f, m_0, \omega) = \frac{f}{1 + e^{\frac{m - m_0}{\omega}}}
                $$
                \item altitudinal dependence
                $$
                    A(\theta; \alpha) = \left(\sin \theta\right)^\alpha
                $$
                \item light pollution
                \item cloud coverage
            \end{itemize}
        }

        \fspicture{pictures/dots.png}[H]
        \fspicture{pictures/biased.png}[H]


    \section{Applications}
        \frejm{Analyses}{
            \begin{itemize}
                \item most freedom is given to the user here
                \pause
                \item anything can be done...
                \pause
                \item ...but several scripts are provided
            \end{itemize}
        }

        \frejm{Optimal observing direction}{
            We may determine the optimal field of view...
            \\[5mm]
            \pause

            \emph{``Given a camera with FoV of \ang[parse-numbers=false]{x}, where should I point it to capture most meteors of shower~$y$ on night~$z$?''}
            \\[5mm]
            \pause

            \begin{itemize}
                \item kernel density estimation
                \item find minimum of the integral over all possible fields
                \item return its centre (alt / az)
            \end{itemize}
        }

        \frejm{Optimal observation location}{
            We may determine the optimal latitude for observing showers...

            \pause
            E. g. Perseids:
            \begin{itemize}
                \item $\delta_R \approx$ \ang{+56}
                \item at \ang{60} N
                    \begin{itemize}
                        \item nights are short in August
                        \item radiant high up in the morning
                    \end{itemize}
                \item at \ang{0} N
                    \begin{itemize}
                        \item longer nights
                        \item but radiant is low
                    \end{itemize}

                \item analytic solution difficult (sunlight, changing $z_R$, ...)
            \end{itemize}
        }

        \frejm{Analysis of flight models}{
            We may analyze models of meteoroid flight...
            \pause
            \begin{itemize}
                \item vary the properties of meteoroids systematically
                \item observe the variations in the \emph{output}
            \end{itemize}
        }
        \fspicture{pictures/Mhu.png}
        \fspicture{pictures/Twu.png}

        \frejm{Determination of flux}{
            We may determine the total \emph{number} or \emph{mass} of meteoroids...
            \pause
            \begin{itemize}
                \item simulate everything what could be seen
                \item compare the \emph{output} to observations
                \begin{itemize}
                    \item change values of parameters (how?)
                    \item repeat until best match is found
                \end{itemize}
                \item look at the original \emph{input}
                \begin{itemize}
                    \item declare it to be the true population
                \end{itemize}
            \end{itemize}
        }

        \frejm{Validation of a distribution model}{
            ...and we may go even deeper...
            \pause
            \begin{itemize}
                \item construct an \emph{orbital model} of meteoroids
                \item track atmospheric entry
                \item compare to observations, $\chi^2$ test, ...
            \end{itemize}
        }



    \section{\textsc{Asmodeus}}
        \frejm{What is it}{
            \textbf{A}ll-\textbf{S}ky \textbf{M}eteor \textbf{O}ptical \textbf{D}etection \textbf{E}fficiency \textbf{S}imulator
            \begin{itemize}
                \item a suite of five or six \emph{Python} scripts
                \item \url{https://github.com/sesquideus/asmodeus}
            \end{itemize}
        }

        \fspicture{shower.png}[H][black]
        \fspicture{angularSpeed-ago.png}[H][black]

        \fspicture{limmag.pdf}[W][white]


    \section{Results}
        \frejm{Evaluation}{
            \begin{itemize}
                \item we fixed the model and the particle count
                \pause
            \end{itemize}
            \begin{itemize}
                \item histograms evaluated using $\chi^2$ tests
                \begin{itemize}
                    \item determine similarity between histograms
                    \item normalized to 1
                \end{itemize}
            \end{itemize}
        }
        \fspicture{example-magnitude-nobias.pdf}[][white]
        \fspicture{example-magnitude-bias.pdf}[][white]

        \frejm{Magnitude DPF}{
            $$
                D(m; f, m_0, \omega) = \frac{f}{1 + e^{\frac{m - m_0}{\omega}}}
            $$
            \begin{itemize}
                \item a wide range of parameter combinations was searched
                \item fill factor $f$ does not contribute any information
            \end{itemize}
            \pause
            \begin{itemize}
                \item find values of parameters where $\chi^2$ is minimal
                \item account for statistical noise
            \end{itemize}
        }
        \fspicture{chiSquare-magnitude-1d8.pdf}[][white]
        \fspicture{chiSquare-magnitude-1d8z.pdf}[][white]
        \fspicture{histogram-magnitude-1d8-optimum.pdf}[][white]

        \frejm{Mass index $s$}{
            There are way too many bright meteors...
            \pause
            \begin{itemize}
                \item a natural reaction is to try another value of $s$
                \begin{itemize}
                    \item a full range \numrange{1.6}{2.8} was tried
                \end{itemize}
                \item best fit for $s = \text{2.15}$
                \item no value below 2 is consistent with observations
            \end{itemize}

        }
        \fspicture{histogram-magnitude-total-optimum.pdf}[][white]


    \section{Conclusion}
        \frejm{Thank you for your attention}{
            \textit{The scientist is not a person who gives the right answers, he's one who asks the right questions.}
            \scriptsize
            \begin{flushright}
                Claude Lévi-Strauss\\
                Le Cru et le Cuit, 1964
            \end{flushright}
        }

        \frejm{References}{
            \begin{itemize}
                \item \textbf{Baláž, M. et al.}
                    ASMODEUS Meteor Simulation Tool. Planetary \& Space Science, 2020
                \item \textbf{Luciuk, M.}:
                    Meteor Showers. {\footnotesize \url{http://www.asterism.org/tutorials/tut36\%20Meteor\%20Showers.pdf}}.
                \item \textbf{Hill, K. A. -- Rogers, L. A. -- Hawkes, R. L.}:
                    High geocentric velocity meteor ablation. Astronomy \& Astrophysics 444, 615--624 (2005)
                \item \textbf{Öpik, E. J.}:
                    Physics of meteor flight in the atmosphere. Interscience Publishers, 1958.
            \end{itemize}
        }

\end{document}

\documentclass[10pt,aspectratio=169]{beamer}
\RequirePackage{
    amsmath,
    amssymb,
    calc,
    cancel,
    booktabs,
    color,
    siunitx,
    tikz,
    wrapfig,
    array,
    leftidx,
    float,
    etoolbox,
    fancyhdr,
    longtable,
    hyperref,
    ltcaption,
    ulem,
    wasysym,
    accents,
    listings,
    tabularx,
}

\hypersetup{
    hidelinks,
    breaklinks              = true,
}

\usepackage[final]{pdfpages}
\usepackage[many]{tcolorbox}

\RenewDocumentCommand{\vec}{m}{\mathbf{#1}}

\makeatletter
    \def\new@mathgroup{\alloc@8\mathgroup\mathchardef\@cclvi}
    \patchcmd{\document@select@group}{\sixt@@n}{\@cclvi}{}{}
    \patchcmd{\select@group}{\sixt@@n}{\@cclvi}{}{}
\makeatother

\RequirePackage{mathspec}                                   % includes fontspec
\RequirePackage{polyglossia}                                % multi-language support
\RequirePackage{xunicode}
\setdefaultlanguage{slovak}

% Setup fonts -- see fontspec/mathspec documentation.
\defaultfontfeatures{
    Mapping         = tex-text,
    Scale           = MatchLowercase,
    Ligatures       = TeX
}


\NewDocumentCommand{\labelmath}{m +m}{%
    \begin{equation}%
        #2%
        \label{#1}%
    \end{equation}%
}

\NewDocumentCommand{\labelalign}{m +m}{%
    \begin{align}%
        #2%
        \label{#1}%
    \end{align}%
}

\linespread{1.0}
\setlength{\parindent}{0cm}
\setlength{\parskip}{6pt}
\setlength{\abovedisplayskip}{0mm}
\setlength{\belowdisplayskip}{0mm}
\setlength{\abovedisplayshortskip}{0mm}
\setlength{\belowdisplayshortskip}{0mm}
\setlength{\itemindent}{0pt}
\setlength{\textfloatsep}{0mm}
\setlength{\tabcolsep}{3mm}
\setlength{\LTcapwidth}{0.8\textwidth}
\renewcommand{\arraystretch}{1.2}

\setcounter{secnumdepth}{2}

/home/kvik/dgs/core/tex/math.tex
\DeclareSIUnit\au{AU}
\DeclareSIUnit\pixel{px}
\DeclareSIUnit\lightyear{ly}
\DeclareSIUnit\parsec{pc}
\DeclareSIUnit\earthmass{M_{\earth}}
\DeclareSIUnit\speedoflight{c}
\DeclareSIUnit\foe{foe}
\DeclareSIUnit\year{yr}
\DeclareSIUnit\eur{€}
\DeclareSIUnit\solarmass{M_{\astrosun}}
\DeclareSIUnit\solarluminosity{L_{\astrosun}}
\DeclareSIUnit{\byte}{B}




\linespread{1.0}
\setlength{\parindent}{0cm}
\setlength{\parskip}{6pt}
\setlength{\abovedisplayskip}{0mm}
\setlength{\belowdisplayskip}{0mm}
\setlength{\abovedisplayshortskip}{0mm}
\setlength{\belowdisplayshortskip}{0mm}
\setlength{\itemindent}{0pt}
\setlength{\textfloatsep}{0mm}
\setlength{\tabcolsep}{3mm}
\renewcommand{\arraystretch}{1.2}

\setcounter{secnumdepth}{0}

\NewDocumentCommand{\fspicture}{m O{W} O{black}}{
    {
        \setbeamertemplate{navigation symbols}{}
        \setbeamercolor{background canvas}{bg = #3}
        \begin{frame}[plain]
            \begin{tikzpicture}[remember picture, overlay]
                \node[at=(current page.center)] {
                    \ifstrequal{H}{#2}{                                  
                        \includegraphics[height=\paperheight]{#1}%
                    }{%
                        \includegraphics[width=\paperwidth]{#1}%
                    }
                };
            \end{tikzpicture}
        \end{frame}
    }
}

\NewDocumentCommand{\frejm}{m +m}{
    \begin{frame}
        \frametitle{#1}
        #2
    \end{frame}
}

\NewDocumentCommand{\fragfrejm}{m +m}{
    \begin{frame}[fragile]
        \frametitle{#1}
        #2
    \end{frame}
}

\defbeamertemplate{description item}{align center}{\hfill\insertdescriptionitem\hfill}
\definecolor{desc}{rgb}{0.66, 0, 0}
\definecolor{citem}{rgb}{0.72, 0, 0}
\definecolor{csitem}{rgb}{0.90, 0, 0}
\definecolor{cssitem}{rgb}{1, 0.1, 0.1}
\definecolor{qprimarybg}{rgb}{0.95, 0.95, 0.95}
\definecolor{check}{rgb}{0, 0.8, 0}
\definecolor{coded}{rgb}{0.9, 0.9, 0.9}
\definecolor{todo}{rgb}{1.0, 0.3, 0.3}
\definecolor{model}{rgb}{0.75, 0, 0}

\setbeamertemplate{navigation symbols}{}
\newfontfamily{\semibold}{Segoe UI Semibold}
\RenewDocumentCommand{\emph}{m}{{\semibold#1}}
\NewDocumentCommand{\code}{m}{\textcolor{desc}{\texttt{#1}}}
\NewDocumentCommand{\model}{m}{\colorbox{coded}{\textcolor{model}{\texttt{#1}}}}
\NewDocumentCommand{\todo}{m}{\colorbox{todo}{#1}}

\mode<presentation> {
    \usetheme{Szeged}
    \usecolortheme{beaver}
    
    \usefonttheme{professionalfonts}
    \setallmainfonts{Minion Pro}
    \setmathrm{Minion Pro}
    
    \setsansfont{Segoe UI}
    \setmonofont{Consolas}
    \setbeamercolor*{enumerate item}{fg = citem}
    \setbeamercolor*{enumerate subitem}{fg = csitem}
    \setbeamercolor*{enumerate subsubitem}{fg = cssitem}
    \setbeamercolor*{description item}{fg = desc}
    \setbeamercolor*{itemize item}{fg = citem}
    \setbeamercolor*{itemize subitem}{fg = csitem}
    \setbeamercolor*{itemize subsubitem}{fg = cssitem}
    \setbeamercolor*{palette primary}{fg = red, bg = qprimarybg}
}

\newcommand<>\highlightbox[2]{%
    \alt#3{\makebox[\dimexpr\width-2\fboxsep]{\colorbox{#1}{#2}}}{#2}%
}

\AtBeginSection[]{
    \subsection{\insertsection}
    \begin{frame}
        \vfill
        \centering
        \begin{beamercolorbox}[sep = 18pt, center, shadow = true, rounded = true]{title}
            \usebeamerfont{title}\insertsectionhead%
            \vfill
        \end{beamercolorbox}
        \vfill
    \end{frame}
}

\makeatletter
% Render percent sign with nice font, not ugly Computer modern
    \mathcode`\%="7025

% Fixes mathspec bug -- URL numbers are rendered with wrong font
    \ernewcommand\eu@MathPunctuation@symfont{Latin:m:n}
    \DeclareMathSymbol{,}{\mathpunct}{\eu@MathPunctuation@symfont}{`,}
    \DeclareMathSymbol{?}{\mathpunct}{\eu@MathPunctuation@symfont}{`?}
    \DeclareMathSymbol{.}{\mathord}{\eu@MathPunctuation@symfont}{`.}
    \DeclareMathSymbol{<}{\mathrel}{\eu@MathPunctuation@symfont}{`<}
    \DeclareMathSymbol{>}{\mathrel}{\eu@MathPunctuation@symfont}{`>}
    \DeclareMathSymbol{/}{\mathord}{\eu@MathPunctuation@symfont}{`/}
    \DeclareMathSymbol{;}{\mathpunct}{\eu@MathPunctuation@symfont}{`;}
    \DeclareMathSymbol{(}{\mathopen}{\eu@DigitsArabic@symfont}{`(}
    \DeclareMathSymbol{)}{\mathclose}{\eu@DigitsArabic@symfont}{`)}
    \XeTeXDeclareMathSymbol{^^^^2026}{\mathinner}{\eu@MathPunctuation@symfont}{"2026}[\mathellipsis]
    \DeclareMathSymbol{0}{\mathalpha}{\eu@DigitsArabic@symfont}{`0}
    \DeclareMathSymbol{1}{\mathalpha}{\eu@DigitsArabic@symfont}{`1}
    \DeclareMathSymbol{2}{\mathalpha}{\eu@DigitsArabic@symfont}{`2}
    \DeclareMathSymbol{3}{\mathalpha}{\eu@DigitsArabic@symfont}{`3}
    \DeclareMathSymbol{4}{\mathalpha}{\eu@DigitsArabic@symfont}{`4}
    \DeclareMathSymbol{5}{\mathalpha}{\eu@DigitsArabic@symfont}{`5}
    \DeclareMathSymbol{6}{\mathalpha}{\eu@DigitsArabic@symfont}{`6}
    \DeclareMathSymbol{7}{\mathalpha}{\eu@DigitsArabic@symfont}{`7}
    \DeclareMathSymbol{8}{\mathalpha}{\eu@DigitsArabic@symfont}{`8}
    \DeclareMathSymbol{9}{\mathalpha}{\eu@DigitsArabic@symfont}{`9}
\makeatother


\usepackage{tabularx}

\title{ASMODEUS Meteor Simulation Tool}
\subtitle{A Universal Virtual Meteor Observatory}
\author{\emph{Martin Baláž} \\ Juraj Tóth, PhD. \\ Peter Vereš, PhD. \\ Robert Jedicke, PhD.}
\institute{DAPEM Comenius University}
\date{2020--09--21}

\begin{document}
    \begin{frame}
        \titlepage
    \end{frame}

    \section{Overview}
        \frejm{What is it?}{
            a universal virtual meteor observatory

            \pause
            \begin{itemize}
                \item simulate a set of meteors
                \item compute how they would be seen from the ground
                \item analyze the dataset
                \item visualize
                \item draw conclusions
            \end{itemize}
        }

    \section{Simulation}
        \frejm{Generating meteors}{
            \begin{itemize}
                \item select general area and time
                \item ensure \emph{homogeneity} over the observed surface
                \pause
                \item select velocity vector
                    \begin{itemize}
                        \item showers (fixed radiant, easy)
                        \item sporadic (complex)
                    \end{itemize}
                \pause
                \item sample properties from pre-defined distributions
                    \begin{itemize}
                        \item time (activity profile)
                        \item mass (mass index $s$)
                        \item density
                        \item heat of ablation
                        \item ...
                    \end{itemize}
            \end{itemize}
        }

        \frejm{Simulating meteors}{
            \begin{columns}
                \begin{column}{0.65\textwidth}
                    \begin{itemize}
                        \item<1-> pluggable physical models
                            \begin{itemize}
                                \item model of atmosphere (NRLMSISE-00)
                                \item equations of motion (Whipple or Ceplecha, +fragmentation, ...)
                                \item equation of luminance (Hill et al.)
                            \end{itemize}

                        \item<3-> selection of integrators
                            \begin{itemize}
                                \item Euler, RK4, Dormand--Prince, ...
                                \item constant or adaptive step
                            \end{itemize}

                        \item<4-> snapshots recorded at constant time intervals
                    \end{itemize}
                \end{column}
                \begin{column}{0.35\textwidth}<2->
                    $$
                        \color{gray} \diff{v} = -\frac{\Gamma A}{m^{1/3} \rho^{2/3}} \rho_{\mathrm{air}} v^2 \diff{t}
                    $$
                    $$
                        \color{gray} \diff{m} = -\frac{\Lambda A}{2Q} \frac{m^{2/3}}{\rho^{2/3}} \rho_{\mathrm{air}} v^3 \diff{t}
                    $$
                    $$
                        \color{gray} L = \tau(v) \frac{\Lambda A}{4Q} \frac{m^{2/3}}{\rho^{2/3}} \rho_{\mathrm{air}} v^5
                    $$
                \end{column}
            \end{columns}
        }


    \section{Observation}
        \frejm{Observation}{
            Meteoroids are transformed to meteors
            \begin{itemize}
                \item observers usually correspond to real cameras
                \item \emph{streaks} or \emph{dots} depending on purpose
                \pause
            \item calculate geometry (ECEF $\to$ horizontal coordinates)\\[5mm]
                \pause
                \item account for various effects
                    \begin{itemize}
                        \item distance ($\propto d^{-2}$)
                        \item atmospheric attenuation (air mass, ...)
                        \item refraction
                    \end{itemize}
            \end{itemize}
        }
        \fspicture{pictures/streaks.png}[H]
        \fspicture{pictures/dots.png}[H]

        \frejm{Selection bias effects}{
            Detection efficiency is \emph{not constant}!

            \pause
            \begin{itemize}
                \item cameras generally prefer meteors...
                    \begin{itemize}
                        \item brighter
                        \item faster
                        \item closer to zenith (or centre of FoV -- vignetting)
                    \end{itemize}
            \end{itemize}

            \pause
            \begin{itemize}
                \item determine whether a meteor is \emph{detected}, depending on
                \begin{itemize}
                    \item apparent magnitude
                    \item altitude
                    \item light pollution
                    \item cloud coverage
                    \item ...
                \end{itemize}
            \end{itemize}
        }

        \fspicture{pictures/dots.png}[H]
        \fspicture{pictures/biased.png}[H]


    \section{Applications}
        \frejm{Optimal observing direction}{
            We may determine the optimal field of view...
            \\[5mm]
            \pause

            \emph{``We have a camera with field of view of \ang[parse-numbers=false]{x}. Where should we point it to capture most meteors of shower~$y$ on night~$z$?''}
            \\[5mm]
            \pause

            \begin{itemize}
                \item perform kernel density estimation of $D(\theta, \phi)$
                \item convolve with the camera FoV
                \item find maximum
            \end{itemize}
        }

        \frejm{Optimal observation location}{
            We may determine the optimal latitude for observing showers...

            \pause
            e. g. Perseids:
            \begin{itemize}
                \item $\delta_R \approx$ \ang{+56}
                \item at \ang{60} N
                    \begin{itemize}
                        \item nights are short in August
                        \item radiant high up in the morning
                    \end{itemize}
                \item at \ang{20} N
                    \begin{itemize}
                        \item nights are long
                        \item but radiant is hidden or very low
                    \end{itemize}

                \item analytic solution difficult (sunlight, changing $z_R$, ...)
            \end{itemize}
        }

        \frejm{Analysis of flight models}{
            We may analyze models of meteoroid flight...
            \pause
            \begin{itemize}
                \item vary the properties of meteoroids systematically
                \item observe the variations in the \emph{output}
            \end{itemize}
        }
        \fspicture{pictures/Mhu.png}[H][white]
        \fspicture{pictures/uqu.png}[H][white]

        \frejm{Determination of flux}{
            We may determine the total \emph{number} or \emph{mass} of meteoroids...
            \pause
            \begin{enumerate}
                \item simulate everything what could be seen
                \pause
                \item compare the \emph{output} to observations
                \begin{itemize}
                    \item change values of parameters (how?)
                    \item repeat until best match is found
                \end{itemize}
                \pause

                \item find best possible parameters
                    \begin{itemize}
                        \item magnitude dependence
                        $$
                            D(m; f, m_0, \omega) = \frac{f}{1 + e^{\frac{m - m_0}{\omega}}}
                        $$
                    \end{itemize}
                \pause
                \item look at the original \emph{input}
            \end{enumerate}
        }

        \fspicture{example-magnitude-nobias.pdf}[H][white]
        \fspicture{chiSquare-magnitude-1d8.pdf}[H][white]
        \fspicture{chiSquare-magnitude-1d8z.pdf}[H][white]
        \fspicture{histogram-magnitude-total-optimum.pdf}[H][white]

        \frejm{Validation of a distribution model}{
            ...and we may go even deeper...
            \pause
            \begin{itemize}
                \item construct an \emph{orbital model} of meteoroids
                \item track orbits and intersection with the Earth
                \item simulate atmospheric entry
                \item compare to observations, $\chi^2$ test, ...
                \pause
                \item computationally difficult
            \end{itemize}
        }

    \section{Conclusion}
        \frejm{Conclusion}{
            \begin{itemize}
                \item we have developed a universal Monte-Carlo simulator of meteors
                \item has a multitude of applications
                \item freely available at GitHub
                    \begin{itemize}
                        \item UI is not very friendly yet
                    \end{itemize}
            \end{itemize}
        }

        \frejm{Thank you for your attention}{
            \textit{The scientist is not a person who gives the right answers, he's one who asks the right questions.}
            \scriptsize
            \begin{flushright}
                Claude Lévi-Strauss\\
                Le Cru et le Cuit, 1964
            \end{flushright}
        }

        \frejm{References}{
            \begin{itemize}
                \item \textbf{Baláž, M. et al.}
                    ASMODEUS Meteor Simulation Tool. Planetary \& Space Science, 2020
                \item \textbf{Hill, K. A. -- Rogers, L. A. -- Hawkes, R. L.}:
                    High geocentric velocity meteor ablation. Astronomy \& Astrophysics 444, 615--624 (2005)
                \item \textbf{Öpik, E. J.}:
                    Physics of meteor flight in the atmosphere. Interscience Publishers, 1958.
                \item \textbf{Picone, J. M. et al.}
                    NRLMSISE-00 empirical model of the atmosphere: Statistical comparisons and scientific issues. Journal of Geophysical Research: Space Physics. 107 (A12): 1468, 2002.
            \end{itemize}
        }

\end{document}

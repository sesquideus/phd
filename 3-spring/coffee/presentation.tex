\documentclass[12pt,aspectratio=1610]{beamer}
\linespread{1.0}
\setlength{\parindent}{0cm}
\setlength{\parskip}{6pt}
\setlength{\abovedisplayskip}{0mm}
\setlength{\belowdisplayskip}{0mm}
\setlength{\abovedisplayshortskip}{0mm}
\setlength{\belowdisplayshortskip}{0mm}
\setlength{\itemindent}{0pt}
\setlength{\textfloatsep}{0mm}
\setlength{\tabcolsep}{3mm}
\renewcommand{\arraystretch}{1.2}

\setcounter{secnumdepth}{0}

\NewDocumentCommand{\fspicture}{m O{W} O{black}}{
    {
        \setbeamertemplate{navigation symbols}{}
        \setbeamercolor{background canvas}{bg = #3}
        \begin{frame}[plain]
            \begin{tikzpicture}[remember picture, overlay]
                \node[at=(current page.center)] {
                    \ifstrequal{H}{#2}{                                  
                        \includegraphics[height=\paperheight]{#1}%
                    }{%
                        \includegraphics[width=\paperwidth]{#1}%
                    }
                };
            \end{tikzpicture}
        \end{frame}
    }
}

\NewDocumentCommand{\frejm}{m +m}{
    \begin{frame}
        \frametitle{#1}
        #2
    \end{frame}
}
\defbeamertemplate{description item}{align center}{\hfill\insertdescriptionitem\hfill}
\definecolor{desc}{rgb}{0.66, 0, 0}
\definecolor{citem}{rgb}{0.72, 0, 0}
\definecolor{csitem}{rgb}{0.90, 0, 0}
\definecolor{cssitem}{rgb}{1, 0.1, 0.1}
\definecolor{qprimarybg}{rgb}{0.95, 0.95, 0.95}
\definecolor{check}{rgb}{0, 0.8, 0}

\setbeamertemplate{navigation symbols}{}
\newfontfamily{\semibold}{Segoe UI Semibold}
\RenewDocumentCommand{\emph}{m}{{\semibold#1}}

\title{Záhada koeficientu Γ}
\subtitle{??? Traja pátrači ???}
\author{\small \emph{Martin Baláž}}
\institute{MPH seminár}
\date{2021--04--23}

\begin{document}
    {
        \begin{frame}
            \titlepage
        \end{frame}
    }

    \section{Prehľad}
        \frejm{Odpor vzduchu}{
            \begin{itemize}
                \item Stagnation pressure: koľko ($\propto v$) a akú hybnosť nesú ($\propto v$)
            \end{itemize}
            $$
                F = \Derivative{p}{t} = k \frac{Mv}{t} = 2 e \frac{Mv}{t}
            $$
            \pause
            $$
                M = \rho V = \rho S v t
            $$
            \pause
            $$\vec{F} = c S \rho v^2 \Hat[-1.7mm]{v}$$
            \pause
            $$\vec{F} = -\frac{1}{2} \Gamma S \rho v^2 \Hat[-1.7mm]{v}$$
        }

        \frejm{Koľko je Γ?}{
            Klasické riešenie pre guľu: Γ = \num{0.47}
            \begin{itemize}
                \item ...lenže to platí pre Re = 10000
            \end{itemize}
            \pause
            Čo teraz?
            \begin{itemize}
                \item Karol: CFD
                \item experimentálne korelácie
            \end{itemize}
        }

        \frejm{Drag crisis / Eiffel paradox}{
            \begin{itemize}
                \item Prečo je golfová loptička poďobaná?
            \end{itemize}
            \hfill\includegraphics[keepaspectratio, width=0.2\textwidth]{pictures/golfball.jpg}
            \pause
            \begin{itemize}
                \item Prečo padajúci meteorit zrýchľuje?
                    \begin{itemize}
                        \item atmosféra predsa hustne $\propto e^{-h}$
                        \item Havrila \& Hrábek
                    \end{itemize}
            \end{itemize}
        }

    \section{Riešenia}
        \setbeamersize{description width = 30mm}
        \frejm{Reynoldsovo číslo}{
            $$\mathrm{Re} = \frac{L \rho_\infty v_\infty}{\mu}$$
            kde
            \begin{description}
                \item[$L$] charakteristický rozmer
                \item[$\rho_\infty$] hustota prostredia (unperturbed)
                \item[$v_\infty$] rýchlosť voči prostrediu (unperturbed)
                \item[$\mu$] koeficient dynamickej viskozity
            \end{description}
            \pause
            Uvažujeme
            \begin{itemize}
                \item inerciálne efekty pre veľké Re
                \item viskózne efekty pre malé Re
            \end{itemize}
        }

        \fspicture{pictures/morrison-drag.png}[][white]

        \frejm{...stále máme problém}{
            \begin{columns}
                \begin{column}{0.7\textwidth}
                    \begin{itemize}
                        \item častica \SI{e6}{\kilo\gram} vo výške \SI{180}{\kilo\metre}
                        \pause
                        \item terminálna rýchlosť \SI{65}{\metre\per\second}?
                    \end{itemize}
                    \pause
                \end{column}
                \begin{column}{0.29\textwidth}
                    \includegraphics[keepaspectratio, width=\textwidth]{pictures/wat.jpg}
                \end{column}
            \end{columns}
            \pause
            Čo s tým?
            \begin{itemize}
                \item model platí pre \emph{nestlačiteľnú} kvapalinu
                \item dva ďalšie efekty
                    \begin{itemize}
                        \item riedka atmosféra vo veľkých výškach -- non-continuum effects
                        \item vysoko nadzvuková rýchlosť a rázové vlny -- rázová vlna
                    \end{itemize}
            \end{itemize}
        }

        \frejm{Knudsenovo číslo}{
            Pomer strednej voľnej dráhy a charakteristického rozmeru
            $$
                \mathrm{Kn} = \frac{\lambda}{L}
            $$
            Pre ideálny plyn
            $$
                \mathrm{Kn} = \frac{kT}{\sqrt{2} \pi d^2 p L} = \frac{\mu}{\rho L}\sqrt{\frac{\pi m}{2 k T}}
            $$
            \pause
            \begin{description}
                \item[$\mathrm{Kn} < 0.01$]         continuum flow
                \item[$0.01 < \mathrm{Kn} < 0.1$]   slip flow
                \item[$0.1 < \mathrm{Kn} < 10$]     transitional flow
                \item[$10 < \mathrm{Kn}$]           free-molecular flow
            \end{description}
        }

        \frejm{Machovo číslo}{
            Pomer rýchlosti k rýchlosti zvuku v médiu
            \begin{itemize}
                \item a to je pri meteoroidoch dosť problém
                \item model iba s $\mathrm{Re}$ sa \emph{nedá} použiť
            \end{itemize}
            a platí vzťah...
            $$
                \mathrm{Kn} = \frac{\mathrm{Ma}}{\mathrm{Re}} \sqrt{\frac{\gamma\pi}{2}}
            $$

            \begin{description}
                \item[$\mathrm{Ma} < 0.3$]          incompressible
                \item[$0.3 < \mathrm{Ma} < 1$]      compressible
                \item[$1 < \mathrm{Ma} < 5$]        supersonic
                \item[$5 < \mathrm{Ma}$]            hypersonic
            \end{description}
        }

        \frejm{Singh+2020}{
            Článok...
        }

\end{document}

\hypertarget{dynamickuxe9-definovanie-dopadovuxfdch-oblasti-meteoritov-a-prachovuxfdch-ux10dastuxedc-pouux17eituxedm-duxe1t-z-celooblohovuxfdch-kamier-amos}{%
\section{Dynamické definovanie dopadových oblasti meteoritov a
prachových častíc použitím dát z celooblohových kamier
AMOS}\label{dynamickuxe9-definovanie-dopadovuxfdch-oblasti-meteoritov-a-prachovuxfdch-ux10dastuxedc-pouux17eituxedm-duxe1t-z-celooblohovuxfdch-kamier-amos}}

Predložená práca sa zaoberá tvarom trajektórie meteoroidov počas tmavej
fázy letu, predovšetkým jej určeniu na základe dát získaných systémom
AMOS vyvinutým na DAA KAFZM FMFI UK. Hlavným cieľom práce bolo oboznámiť
sa s problematikou dynamiky meteoroidov a mikrometeoroidov pri vstupe do
atmosféry Zeme a určiť tvar a polohu ich dopadovej oblasti.

Bakalárska práca je logicky členená do piatich hlavných kapitol. Prvá a
druhá kapitola na primeranej úrovni zhŕňajú teoretické poznatky o malých
telesách Slnečnej sústavy a o dynamike a fenomenológii meteoroidov pri
vstupe do atmosféry. V tretej a štvrtej kapitole autorka popisuje
fyzikálny model preletu telesa zemskou atmosférou a jeho implementáciu v
programe (\(\mu\))m-Trajectory. V záverečnej kapitole je dôkladne
popísaná aplikácia programu na konkrétny prípad bolidu
\texttt{EN170420\_223923}.

\hypertarget{pripomienky-k-obsahu-pruxe1ce}{%
\subsection{Pripomienky k obsahu
práce}\label{pripomienky-k-obsahu-pruxe1ce}}

K obsahu práce nemám závažné výhrady. Členenie a logická náväznosť textu
sú na výbornej úrovni a deklarované ciele práce boli naplnené. Po
dôkladnom preštudovaní nachádzam len dva výraznejšie nedostatky:

\hypertarget{prachovuxe9-ux10dastice}{%
\subsubsection{Prachové častice}\label{prachovuxe9-ux10dastice}}

Na obrázku 20a, resp. 23 sú znázornené miesta dopadu simulovaných
prachových častíc, ktoré ležia prakticky na priamke. Pravdepodobne nejde
o skutočný fyzikálny jav, ale len o artefakt simulácie spôsobený
konštantným smerom a rýchlosťou vetra v modeli.

Použitý model rýchlosti vetra zjavne nie je aplikovateľný na takúto
veľkú oblasť zemského povrchu. Vzdialenosť miesta dopadu častice od
počiatočného bodu je potom priamo určená jej hmotnosťou a smer je
prakticky nezávislý. Uvádzať vzdialenosť na štyri platné cifry je pri
takejto presnosti použitého modelu taktiež značne prehnané.

\hypertarget{diskusia}{%
\subsubsection{Diskusia}\label{diskusia}}

Tento nedostatok modelu je pomerne závažný -- neznehodnocuje síce celú
prácu, je však potrebné obmedzenia použitého modelu dôsledne popísať;
prípadne aspoň teoreticky navrhnúť, ako ich obísť alebo odstrániť (tu
napríklad očividne použitím lepšieho modelu vetra, ktorý by bol závislý
aj na zemepisnej šírke, dĺžke a čase, prípadne zahrnutím chýb samotného
modelu vetra a opakovanou simuláciou s mierne odlišnými hodnotami). Za
najproblematickejšiu časť práce preto považujem vyhodnotenie
experimentálnych dát. Na úrovni bakalárskej práce síce nejde o kritický
nedostatok, aj tu je však dôležité uvedomiť si doménu platnosti
vlastných výsledkov.

\hypertarget{pripomienky-k-forme-pruxe1ce}{%
\subsection{Pripomienky k forme
práce}\label{pripomienky-k-forme-pruxe1ce}}

Práca je písaná v slovenskom jazyku na dobrej gramatickej a syntaktickej
úrovni. Odporúčam viac pracovať na schopnosti precízne formulovať
myšlienky vo vedeckom texte a vyvarovať sa používania vágnych tvrdení.

Oceňujem množstvo a umiestnenie referencií na publikované práce iných
autorov. Pri niektorých rovniciach ale nie je z textu zrejmé, či ide o
vlastnú myšlienku autorky alebo o myšlienku prebratú. Naopak fakty, o
ktorých sa dá predpokladať, že sú čitateľovi zorientovanému v
meteorickej astronómii všeobecne známe, nie je nutné detailne citovať.

\hypertarget{vux161eobecnuxe1-technickuxe1-poznuxe1mka}{%
\subsubsection{Všeobecná technická
poznámka}\label{vux161eobecnuxe1-technickuxe1-poznuxe1mka}}

Citácie a odkazy na obrázky sú podľa všetkého písané ručne a nie
vytvorené automaticky. V~budúcich prácach silne odporúčam použiť
zodpovedajúce softvérové riešenie, predovšetkým v~záujme zníženia
množstva zbytočnej práce a kvôli predchádzaniu chybám v referenciách. Na
písanie vedeckých článkov odporúčam naučiť sa pracovať s~\LaTeX-om.

\hypertarget{abstrakt}{%
\subsubsection{Abstrakt}\label{abstrakt}}

\begin{itemize}
\tightlist
\item
  Viacero chýbajúcich členov v anglickej verzii.

  \begin{itemize}
  \tightlist
  \item
    \emph{„passage through \textbf{the} atmosphere“}
  \item
    \emph{„AMOS, \textbf{an} all-sky camera system“}
  \item
    \emph{„In \textbf{the} practical part {[}\ldots{}{]}“}
  \end{itemize}
\end{itemize}

\hypertarget{strana-13}{%
\subsubsection{Strana 13}\label{strana-13}}

\begin{itemize}
\tightlist
\item
  \emph{„Na rozdiel od planét však nie sú schopné ich pôsobením
  gravitácie“} --- vhodnejšie by bolo napísať „pôsobením svojej
  gravitácie“.
\end{itemize}

\hypertarget{strana-14}{%
\subsubsection{Strana 14}\label{strana-14}}

\begin{itemize}
\tightlist
\item
  Trpasličia planéta Ceres je v slovenčine ženského rodu.
\item
  Citácia (IAU, 2017) by nemala obsahovať názov publikácie.
\end{itemize}

\hypertarget{strana-15}{%
\subsubsection{Strana 15}\label{strana-15}}

\begin{itemize}
\tightlist
\item
  \emph{“{[}\ldots{}{]} sú uvoľnené excitované alebo ionizované
  častice, pozorovateľné optickými prístrojmi a radarovými meraniami.“}
  Priamo pozorovateľné tu nie sú častice, ale vyžiarené
  elektromagnetické žiarenie (pri pohľade na lampu nehovoríme o
  pozorovaní častíc žiarovky, ale vidíme jej svetlo).
\item
  \emph{„približne 42,5 km/s v mieste perihélia, ktorú dosahujú
  meteoroidy kometárneho pôvodu obiehajúce okolo Slnka“} --- tu treba
  zdôrazniť, že ide o teoretický horný limit, nie všeobecnú
  charakteristiku.
\item
  Nepresná formulácia: \emph{„padajú ďalej k Zemi rýchlosťou danou
  gravitačným zrýchlením“}.
\end{itemize}

\hypertarget{strana-16}{%
\subsubsection{Strana 16}\label{strana-16}}

\begin{itemize}
\tightlist
\item
  \emph{“{[}\ldots{}{]} kým nestratia väčšinu, prípadne celú hmotnosť,
  kedy končí svetlá fáza letu.“} Chýba kauzálna súvislosť --- svetlá
  fáza letu nie je definovaná stratou hmotnosti.
\item
  Vágne, resp. nepresné tvrdenie \emph{„jasnosť meteoru je úmerná miere
  ablácie“}.
\item
  \emph{„Veľmi jasné meteory {[}\ldots{}{]} sa nazývajú bolidy (Havrila,
  2018).“} Tento druh citácie nie je potrebný. Je to všeobecne uznávaná
  definícia, a takisto to nie je originálna myšlienka autora citovanej
  práce.
\end{itemize}

\hypertarget{strana-17}{%
\subsubsection{Strana 17}\label{strana-17}}

\begin{itemize}
\tightlist
\item
  \emph{„Prachové častice veľkosti od niekoľko mikrometrov až
  milimetrov“} --- pri veľkosti rádovo niekoľko milimetrov už nie je
  možné hovoriť o prachu.
\end{itemize}

\hypertarget{strana-21}{%
\subsubsection{Strana 21}\label{strana-21}}

\begin{itemize}
\tightlist
\item
  Správne meno autora v referencii je „Jean-Baptiste Kikwa\textbf{y}a“.
\end{itemize}

\hypertarget{strana-23}{%
\subsubsection{Strana 23}\label{strana-23}}

\begin{itemize}
\tightlist
\item
  Definícia v texte odporuje obrázku 5: \emph{“\(v_h\) je vertikálna
  zložka, kladná v smere nahor“}
\end{itemize}

\hypertarget{strana-26}{%
\subsubsection{Strana 26}\label{strana-26}}

\begin{itemize}
\tightlist
\item
  Netriviálna rovnica (6) nemá uvedený zdroj a nie je odvodená v texte,
  navyše nesedí jej fyzikálny rozmer.
\end{itemize}

\hypertarget{strana-27}{%
\subsubsection{Strana 27}\label{strana-27}}

\begin{itemize}
\tightlist
\item
  Netriviálna rovnica (11) nemá uvedený zdroj a nie je odvodená v texte.
\end{itemize}

\hypertarget{strana-31}{%
\subsubsection{Strana 31}\label{strana-31}}

\begin{itemize}
\tightlist
\item
  Programovací jazyk Python sa podľa štandardu píše s veľkým prvým
  písmenom.
\item
  \emph{„namiesto integračných krokov cez vzdialenosť (výšku)
  \(\mathrm{d}h\) prebieha výpočet cez časové kroky \(\mathrm{d}t\)“}
  -- z~tohto dôvodu by bolo vhodné v časti 3.2 uviesť rovnice s
  \(\mathrm{d}t\), nie všeobecný popis.
\item
  Obrázok 9: správny termín je \emph{používateľ}, nie \emph{užívateľ}
  programu.
\end{itemize}

\hypertarget{strana-32}{%
\subsubsection{Strana 32}\label{strana-32}}

\begin{itemize}
\tightlist
\item
  \emph{„poznať fyzikálne charakteristiky tekutiny“} -- vhodnejšie by
  bolo hovoriť o \emph{prostredí} alebo \emph{atmosfére}.
\end{itemize}

\hypertarget{strana-34}{%
\subsubsection{Strana 34}\label{strana-34}}

\begin{itemize}
\tightlist
\item
  Referencia má byť správne (ECMWF, 2021) (viď Technickú poznámku).
\end{itemize}

\hypertarget{strana-36}{%
\subsubsection{Strana 36}\label{strana-36}}

\begin{itemize}
\tightlist
\item
  Vyjadrenie \emph{„Do svetlej fázy vstúpil vo výške 91,8 km nad zemou
  západne od zenitu“} nemá zmysel bez udania polohy pozorovateľa.
\end{itemize}

\hypertarget{strana-41}{%
\subsubsection{Strana 41}\label{strana-41}}

\begin{itemize}
\tightlist
\item
  \emph{„Jedná sa o \ldots{}“} \(\rightarrow\) „Ide o \ldots{}“
\end{itemize}

\hypertarget{strana-43}{%
\subsubsection{Strana 43}\label{strana-43}}

\begin{itemize}
\tightlist
\item
  Obrázok 16b: chýba vyjadrenie konkrétnej polynomickej funkcie a
  vysvetlenie, aké sú jej nezávislé premenné.
\end{itemize}

\hypertarget{strana-44}{%
\subsubsection{Strana 44}\label{strana-44}}

\begin{itemize}
\tightlist
\item
  Vo vedeckej publikácii je primeranejšie uvádzať čas v jednotkách SI
  (tu v sekundách, prípadne iných jednotkách vhodných pre danú škálu).
\end{itemize}

\hypertarget{otuxe1zky}{%
\subsection{Otázky}\label{otuxe1zky}}

K práci mám nasledujúce otázky:

\hypertarget{ruxfdchlosux165-pri-vstupe-do-atmosfuxe9ry}{%
\subsubsection{Rýchlosť pri vstupe do
atmosféry}\label{ruxfdchlosux165-pri-vstupe-do-atmosfuxe9ry}}

Rozpätie možných rýchlostí meteoroidov pri vstupe do atmosféry je
autorkou na strane 15 uvedené ako
\SIrange{11.2}{72.8}{\kilo\metre\per\second}.

\begin{itemize}
\tightlist
\item
  Ide o vlastný výpočet autorky práce, alebo sú tieto hodnoty prevzaté?
\item
  Prečo nie je možná väčšia rýchlosť? Aký fyzikálny mechanizmus bráni
  meteoroidom pohybovať sa voči Zemi rýchlosťou napríklad
  \SI{100}{\kilo\metre\per\second}?
\end{itemize}

\hypertarget{modely}{%
\subsubsection{Modely}\label{modely}}

\begin{itemize}
\tightlist
\item
  Ako veľmi vplýva na presnosť určenia polohy prachovej častice pri
  dopade

  \begin{itemize}
  \tightlist
  \item
    presnosť určenia polohy meteoroidu na konci svetelnej fázy letu;
  \item
    rýchlosti meteoroidu;
  \item
    rýchlosti vetra?
  \end{itemize}
\item
  Ako by sa zmenili výsledky simulácie, keby sa model vetra modifikoval
  tak, že pre každý beh simulácie jednej častice jemne náhodne zmeníme
  zložky \(U\) a \(V\) (napríklad gaussovsky, so \(\sigma = 10\%\)
  veľkosti \(U\), resp. \(V\))?
\end{itemize}

\hypertarget{zuxe1ver}{%
\subsection{Záver}\label{zuxe1ver}}

Práca je napísaná zrozumiteľne a na odbornej úrovni. Študentka
preukázala porozumenie použitým matematickým metódam a algoritmom a je
schopná ich aplikovať na získané experimentálne dáta a vyvodiť z nich
vedecké závery. Napriek početným drobným formálnym nedostatkom
konštatujem, že práca dosiahla ciele vytýčené v zadaní a splnila
všeobecné požiadavky kladené na bakalársku prácu.

Po úspešnej obhajobe ju odporúčam hodnotiť známkou \textbf{A}.

~

V Bratislave, 2021--05--31

\hfill Mgr. Martin Baláž

\hfill KAFZM FMFI UK

\section*{Abstract}
    Deployment of multi-station video meteor networks presents a unique opportunity
    to measure the total mass flux of meteoroids impinging on the surface of the Earth.
    However, direct measurement of flux is not possible as raw data are heavily distorted by selection bias. 
    In this thesis, we present two possible methods of debiasing the data used for flux estimation.
    We used data obtained by AMOS, an all-sky video camera system developed and operated by Comenius University in Bratislava.
    
    The first method is based on sequential identification and elimination of possible sources of bias.
    Each identified effect is measured and adequate correction procedures are developed.
    The second presented approach is a simulation of meteoroid particles entering the atmosphere.
    The trajectory of each virtual meteoroid is tracked and after application of biases
    the resulting meteor observation is recorded. Once a sufficiently large dataset is obtained,
    statistical tests are performed and the distributions are compared to observational data.
    The entire procedure is repeated and the parameters of the simulation are gradually adjusted
    until best possible agreement with observational data is found.

    \emph{Keywords}: meteor, meteoroid, population, video, flux, model, debiasing, AMOS

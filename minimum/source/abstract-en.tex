\section*{Abstract}
    In this thesis we aim to design a suitable representation of spatial density
    and flux of meteoroids in interplanetary space and construct a model representing
    the actual distribution of positions and velocities of small bodies in the inner Solar System,
    supplemented by information about mass and material properties in different populations.
    The orbital model can be further reduced to an observational model,
    which describes the positions of radiants of meteors observable from the surface of the Earth.

    Two distinct approaches are outlined.
    The first method is based on direct reduction of observational data.
    De-biasing procedures are planned to be developed and applied to the datasets,
    after which we should be able to derive the original distribution by tracing the trajectories of meteors back in time.
    Another option is to deploy a parametric Monte-Carlo simulation of meteoroids in the interplanetary space
    and try to vary the parameters until its output matches the observational data.
    In both cases, multivariate kernel density estimation methods are applied to the discrete
    data to estimate the underlying distribution of meteors in a selected coordinate space.
    From this distribution we should be able to calculate the spatial density and flux of meteoroids
    in any volume of space covered by the model, or predict meteor shower activity as
    seen from the surface.

    \emph{Keywords}: meteor, meteoroid, population, observation, distribution, flux, model, debiasing, AMOS

\section*{Abstract}
    In this thesis we aim to design a suitable representation of spatial density
    and flux of meteoroids in interplanetary space and to construct a model representing
    the actual distribution of positions and velocities of small bodies in the inner Solar System,
    supplemented with information about mass and material properties in different populations.
    The orbital model can be reduced to an observational model,
    which describes the measurable properties of meteors observed from the surface of the Earth.

    Two distinct approaches are outlined.
    The first method is based on direct reduction of observational data.
    De-biasing procedures are planned to be developed and applied to the datasets,
    after which we should be able to derive the original distribution by tracing the trajectories of meteors back in time.
    The second method is based on Monte-Carlo simulations of meteoroids in interplanetary space and in the Earth's atmosphere.
    In both cases, multivariate kernel density estimation methods are applied to the discrete
    data in order to estimate the underlying distribution of meteors in a selected coordinate space.
    From this distribution we should be able to calculate the spatial density and flux of meteoroids
    in any volume of space covered by the model, or predict meteor shower activity as seen from the surface.

    \emph{Keywords}: meteor, meteoroid, population, observation, distribution, flux, model, debiasing, AMOS

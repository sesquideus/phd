\section*{Abstrakt}
    Inštalácia viacstaničných sietí meteorových kamier umožňuje okrem iného aj určenie celkového toku meteoroidných častíc.
    Priame meranie však nie je možné, pretože získané dáta sú zaťažené značnými systematickými chybami a výberovými efektami.
    V tejto práci predkladáme dve metódy odstránenia výberových efektov.
    V práci sme použili dáta z kamier AMOS, vyvinutých a prevádzkovaných na FMFI UK.

    Prvá metóda je založená na postupnej identifikácii a odstraňovaní výberových efektov,
    ktoré ovplyvňujú pozorované frekvencie meteorov v zemskej atmosfére.
    Vplyv uvažovaných efektov sme odhadli na základe kalibrácie systému s vizuálnymi pozorovateľmi.
    Druhou popísanou metódou je simulácia meteoroidov vstupujúcich do zemskej atmosféry.
    Na zaznamenané dráhy virtuálnych meteorov boli aplikované zvolené výberové kritériá
    a výsledný štatistický súbor bol uložený do databázy. Tieto dáta sme následne
    analyzovali štatistickými metódami a porovnávali s observačnými dátami z~kamier AMOS.
    Celý proces sme následne mnohokrát opakovali s modifikovanými parametrami
    až do nájdenia najlepšej možnej zhody s observačnými dátami
    a následne odhadli výsledný tok pre meteorický roj Perzeíd.

    \emph{Kľúčové slová}: meteor, meteoroid, tok, model, pozorovanie, výberové efekty

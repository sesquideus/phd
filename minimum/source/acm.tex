\Epigraph{
    Throughout the ages \\
    Of iron, bronze and stone \\
    We marvelled at the night sky \\
    And what may lie beyond
}{
    Children of the Sun \\
    \textsc{Dead Can Dance}
}

The study of interplanetary matter is 

While most of the large bodies of the Solar System were known since times immemorial,

\section{Asteroids} \label{aa}
    The first body of the Solar System that was not a planet, a moon or a comet was discovered at
    the very beginning of the \nth{19} century by Giuseppe Piazzi.

\section{Comets} \label{ac}
    Comets have historically been frowned upon as harbingers of doom and disaster.\footnote{Hence,
    a better term would likely be ``dis\textit{comet}''. The stars are innocent this time.}
    Many comets were soon identified as different apparitions of the same celestial body (who was this?)

    Unlike asteroids, comet

    this taxonomical boundary is not entirely set in stone, as there are objects which exhibit
    only very sporadic activity, such as (...).

    Comets are responsible for the creation of most of the contemporary meteor showers.
    Modern missions, such as Giotto and most notably Rosetta have significantly extended our knowledge
    about the nature of these bodies and helped ascertain the mechanisms that lead to production
    of streams of meteoroids from their surface.

\section{Meteoroids} \label{am}
    Th

    A more formal definition is that a meteoroid is a ``solid object moving in interplanetary space,
    of a size considerably smaller than a asteroid and considerably larger than an atom or molecule'' \cite{imo-glossary}.
    The precise size limits have no hard physical constrains and as such are a matter of preference.
    However, no subfield of astronomy or science in general should be left

    A new definition has been proposed by the IAU Commission F1 on Meteors, Meteorites and Interplanetary Dust,
    which elucidates these limits by explicitly stating that a meteoroid is a ``solid natural object
    of a size roughly between \SI{30}{\micro\metre} and \SI{1}{\metre} moving in, or coming from, interplanetary space'' \cite{imo-definitions}.
    Objects smaller than the lower limit are summarily named as \emph{interplanetary dust} while objects
    too large to be considered meteoroids under this definition are already promoted to being called
    asteroids or comets, depending on their other characteristics.

    In colloquial usage in the context of meteor observations, any object producing a visible meteor can be called a meteoroid,
    irrespective of size or origin. This includes small asteroids or comets; pieces of space debris,
    such as fragments of defunct satellites, spent launch vehicles or blobs of fuel combustion products;
    or finally artificial meteors, launched with the expressed purpose of creating meteoroids.

    As of early \nth{21} century, the most prominent meteor showers are the
    \begin{table}[H]
        \begin{tabularx}{\textwidth}{l @{\extracolsep{\fill}} l r r}
            \toprule
                shower name             &   parent body             & solar longitude $\lambda_\Sun$    &   typical ZHR (2020) \\
            \midrule
                Quadrantids             &   2003 EH$_1$             & \ang{270}     & 100 \\
                Lyrids                  &   C/1861 G1 (Thatcher)    & \ang{40}      & 40 \\
                Perseids                &   109P/Swift--Tuttle      & \ang{140}     & 100 \\
                Leonids                 &   55P/Tempel--Tuttle      & \ang{190}      & 40 \\
                Geminids                &   (3200) Phaethon         & \ang{240}     & 150 \\
            \bottomrule
        \end{tabularx}
        \caption{Several prominent meteor showers}
    \end{table}


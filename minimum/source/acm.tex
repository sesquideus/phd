\Epigraph{
    Throughout the ages \\
    Of iron, bronze and stone \\
    We marvelled at the night sky \\
    And what may lie beyond
}{
    Children of the Sun \\
    \textsc{Dead Can Dance}
}

The study of interplanetary matter is 

In comparison to deep-space astronomy it does not produce spectacular discoveries,
such as the discovery of gravity waves or images of black holes, very often.
Yet our ``cosmic backyard'' presents lots 

and, most importantly, is the home system of all life we know.

to know our cosmic home, its origins and our place in it.

Very importantly is mostly within our reach.

The Sun, which consistutes about \num{99.9}\% of the total mass, provides energy and entropy gradient
necessary for life on Earth and gravitationally holds this entire family together.

The four 

and also because of their large destructive potential, should a sufficiently large asteroid take on a collision course.
Unlike powerful solar flares, explosions of supernovae and other dangers lurking in the open space, against which we are
virtually powerless, engineering a space mission to disrupt or deflect an oncoming asteroid is possible with
available technology.

The outlook of cheaper space flight also promises to unlock an enormous commercial and industrial potential in asteroids,
all within the foreseeable future.

Comets and meteors have been known since prehistoric times

Due to their brightness and unusual appearance -- and lack of modern-day light and air pollution -- they
must have been regularly observed by prehistoric people as well.



\section{Asteroids} \label{aa}
    While most of the large bodies of the Solar System were known since times immemorial,
    the first body of the Solar System that was not a planet, a moon or a comet was only discovered at
    the very beginning of the \nth{19} century by Giuseppe Piazzi.

    These objects have formed everywhere where the gravitational attraction was able to overcome chaos \citep{???}

    To our current knowledge, in the regions where gravitational attraction dominated, almost all matter
    coalesced into just several large bodies, which we today know as the planets.
    In a region where the gravitational perturbations by the already-formed big planets, most notably Jupiter,
    did not allow one single body to form, we find a plethora of smaller bodies made of primordial material.

    The most prominent population of asteroids is known as the Main Belt, scattered between the orbits
    of Mars and Jupiter, with typical semi-major axes of about \SIrange{2}{3}{\au}.

    spectral types of asteroids (\emph{C}, \emph{S} and \emph{M}), which are closely reflected in
    spectral types of meteorites.



\section{Comets} \label{ac}
    Comets have historically been frowned upon as harbingers of doom and disaster%
    \footnote{Hence a better term would likely be ``dis\textit{comet}''. The stars are innocent this time.}

    Many comets were soon identified as different apparitions of the same celestial body \todo{who was this?}

    Unlike asteroids, comets are thought to have formed much further from the Sun, where the equilibrium
    temperatures were low enough to allow condensation of volatile substances, most importantly water ice.

    This taxonomical boundary is not entirely set in stone, as there are objects which exhibit
    cometary activity, such as (???). \todo{finish}

    Comets are responsible for the existence of most of the contemporary meteor showers.
    Modern missions, such as Giotto and most notably Rosetta, have significantly extended our knowledge
    about the nature of these bodies and helped ascertain the mechanisms that lead to production
    of streams of meteoroids from their surface.

    The most universally acknowledged mechanism is the release of dust particles from
    volatile base matter upon sublimation during close approaches to the Sun.

\section{Meteoroids} \label{am}
    Apart from these two types of objects, interplanetary space is filled with numerous smaller objects
    of natural origin, which either have not yet coalesced into larger bodies, or are the remnants
    of such objects that have undergone catastrophic collisions.

    These tiny blobs of matter are virtually undetectable by themselves, except \textit{in situ} by passing spacecraft,
    in enormous numbers as zodiacal light, or upon entering the Earth's atmosphere as meteors.

    Due to their sheer numbers, they cannot be described as a collection of bodies.
    However, we should be able to map the spatial density of these bodies in the vicinity of the Earth,
    and perhaps even more importantly, predict its future development.

    Despite the recent surge in the numbers of exploration missions in near space,
    our best detector so far is the Earth's atmosphere.



    A more formal definition is that a meteoroid is a ``solid object moving in interplanetary space,
    of a size considerably smaller than a asteroid and considerably larger than an atom or molecule'' \citep{imo-glossary}.
    The precise size limits have no hard physical constrains and as such are a matter of preference.
    However, no subfield of astronomy or science in general should be left to operate with such vague terms.

    An even more sophisticated definition has been approved by the IAU Commission F1 on Meteors, Meteorites and Interplanetary Dust,
    which clarifies these limits by explicitly stating that a meteoroid is a ``solid natural object
    of a size roughly between \SI{30}{\micro\metre} and \SI{1}{\metre} moving in, or coming from, interplanetary space'' \citep{imo-definitions}.
    Objects smaller than the lower limit are summarily named as \emph{interplanetary dust} while objects
    too large to be considered meteoroids under this definition are referred to
    asteroids or comets, depending on their other characteristics.

    In colloquial usage in the context of meteor observations, any object producing a visible meteor can be called a meteoroid,
    irrespective of size or origin. This includes small asteroids or comets; pieces of space debris,
    such as fragments of defunct satellites, spent launch vehicles or blobs of products of combustion of solid fuel;
    or finally artificial meteors, launched with the expressed purpose of creating meteors \cite{japončíci}.

    \subsection{Origins}
        (pali p5)

        \begin{itemize}
            \item Meteoroids ejected from \emph{comets} are usually highly porous and friable.
                As such they are unable to survive the atmospheric entry, except when embedded into
                stronger material \citep{nittler+2019}.
                Their bulk density is fairly low, with typical values around
                \SIrange{200}{1000}{\kilo\gram\per\cubic\metre} \citep{???} \todo{who?}.
            \item Meteoroids originating from \emph{asteroids} are typically solid,
                rocky bodies, composed primarily of silicate minerals, ???

                They are composed either of primordial matter, which has never been metamorphosed by high pressures
                and temperatures within a larger body, or from upper layers of large bodies,
                where they may have bene melted, weathered or chemically altered.
            \item Rarely, meteoroids may have formed in an catastrophic disruption
                of \emph{large differentiated bodies}, which exposed their metallic cores
                and ejected some of the material into interplanetary space.
                These meteoroids are composed primarily of iron and nickel.
                They are very dense, fairly solid and resist ablation well,
                which means they are often able to survive their atmospheric
                entry and can thus be found as meteorites.
                However, as they are much rarer than stony meteoroids,
                their total numbers are low.
    \end{itemize}

    As of early \nth{21} century, the most prominent meteor showers are the
    \begin{table}[H]
        \begin{tabularx}{\textwidth}{l @{\extracolsep{\fill}} l r r}
            \toprule
                shower name &
                parent body &
                solar longitude at maximum activity $\lambda_\Sun$ &
                typical $\mathrm{ZHR}_\mathrm{max}$ (as of 2020) \\
            \midrule
                Quadrantids             &   2003 EH$_1$             & \ang{270}     & 100 \\
                Lyrids                  &   C/1861 G1 (Thatcher)    & \ang{40}      & 40 \\
                Perseids                &   109P/Swift--Tuttle      & \ang{140}     & 100 \\
                Leonids                 &   55P/Tempel--Tuttle      & \ang{190}     & 40 \\
                Geminids                &   (3200) Phaethon         & \ang{240}     & 150 \\
            \bottomrule
        \end{tabularx}
        \caption{Several prominent meteor showers}
    \end{table}

\section{Meteors} \label{al}
    The exact nature of meteors used to be the subject of lengthy scientific disputes.
    Some considered them to be electric phenomena \citep{???}.
    their astronomical origin was only confirmed on the beginning of the \nth{19} century.

    head
    wake
    train
    smoke trail

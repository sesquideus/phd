\Epigraph{
    Throughout the ages \\
    Of iron, bronze and stone \\
    We marvelled at the night sky \\
    And what may lie beyond
}{
    Children of the Sun \\
    \textsc{Dead Can Dance}
}

The study of interplanetary matter is 

Both comets and meteors have been known since 

they must have been regularly observed by prehistoric people as well.



\section{Asteroids} \label{aa}
    While most of the large bodies of the Solar System were known since times immemorial,
    the first body of the Solar System that was not a planet, a moon or a comet was only discovered at
    the very beginning of the \nth{19} century by Giuseppe Piazzi.

    These objects have formed everywhere where the gravitational attraction was able to overcome chaos \cite{...}

    To our current knowledge, in the regions where gravitational attraction dominated, almost all matter
    coalesced into just several large bodies, which we today know as the planets.
    In a region where the gravitational perturbations by the already-formed big planets, most notably Jupiter,
    did not allow a single body to form, we find a plethora of smaller bodies of primordial material.

    The most prominent population of asteroids is known as the Main Belt, scattered between the orbits
    of Mars and Jupiter, with typical semi-major axes of about \SIrange{2}{3}{\au}.

\section{Comets} \label{ac}
    Comets have historically been frowned upon as harbingers of doom and disaster.\footnote{Hence,
    a better term would likely be ``dis\textit{comet}''. The stars are innocent this time.}
    Many comets were soon identified as different apparitions of the same celestial body (who was this?)

    Unlike asteroids, comets are thought to have formed much further from the Sun, where the equilibrium
    temperatures were low enough to allow condensation of volatiles, most importantly water ice.

    This taxonomical boundary is not entirely set in stone, as there are objects which exhibit
    only very sporadic activity, such as (...).

    Comets are responsible for the existence of most contemporary meteor showers.
    Modern missions, such as Giotto and most notably Rosetta have significantly extended our knowledge
    about the nature of these bodies and helped ascertain the mechanisms that lead to production
    of streams of meteoroids from their surface.

\section{Meteoroids} \label{am}
    Apart from these two types of objects, interplanetary space is filled with numerous smaller objects
    of natural origin, which either have not yet coalesced into larger bodies, or are the remnants
    of such bodies that have undergone catastrophic collisions.

    These tiny blobs of matter are virtually undetectable, except \textit{in situ} by passing spacecraft,
    in enormous numbers as zodiacal light, or upon entering the Earth's atmosphere as meteors.

    Due to their sheer numbers, they cannot be described as a collection of bodies.
    However, we should be able to map the spatial density of these bodies in the vicinity of the Earth,
    and perhaps even more importantly, predict its future development.

    Despite the recent surge in the numbers of exploration missions in near space,
    our best detector so far is the Earth's atmosphere.



    A more formal definition is that a meteoroid is a ``solid object moving in interplanetary space,
    of a size considerably smaller than a asteroid and considerably larger than an atom or molecule'' \cite{imo-glossary}.
    The precise size limits have no hard physical constrains and as such are a matter of preference.
    However, no subfield of astronomy or science in general should be left to operate with such vague terms.

    An even more sophisticated definition has been approved by the IAU Commission F1 on Meteors, Meteorites and Interplanetary Dust,
    which elucidates these limits by explicitly stating that a meteoroid is a ``solid natural object
    of a size roughly between \SI{30}{\micro\metre} and \SI{1}{\metre} moving in, or coming from, interplanetary space'' \cite{imo-definitions}.
    Objects smaller than the lower limit are summarily named as \emph{interplanetary dust} while objects
    too large to be considered meteoroids under this definition are referred to
    asteroids or comets, depending on their other characteristics.

    In colloquial usage in the context of meteor observations, any object producing a visible meteor can be called a meteoroid,
    irrespective of size or origin. This includes small asteroids or comets; pieces of space debris,
    such as fragments of defunct satellites, spent launch vehicles or blobs of fuel combustion products;
    or finally artificial meteors, launched with the expressed purpose of creating meteors \cite{japončíci}.

    As of early \nth{21} century, the most prominent meteor showers are the
    \begin{table}[H]
        \begin{tabularx}{\textwidth}{l @{\extracolsep{\fill}} l r r}
            \toprule
                shower name             &   parent body             & solar longitude $\lambda_\Sun$    &   typical ZHR (2020) \\
            \midrule
                Quadrantids             &   2003 EH$_1$             & \ang{270}     & 100 \\
                Lyrids                  &   C/1861 G1 (Thatcher)    & \ang{40}      & 40 \\
                Perseids                &   109P/Swift--Tuttle      & \ang{140}     & 100 \\
                Leonids                 &   55P/Tempel--Tuttle      & \ang{190}     & 40 \\
                Geminids                &   (3200) Phaethon         & \ang{240}     & 150 \\
            \bottomrule
        \end{tabularx}
        \caption{Several prominent meteor showers}
    \end{table}



\section{Meteors} \label{al}
    Properties of the meteors can be vastly variable depending on their origin:
    \begin{itemize}
        \item meteoroids ejected from \emph{comets} are highly porous and friable,
            their density is fairly low, with typical values around \SIrange{500}{1000}{\kilo\gram\per\cubic\metre}.
            (details)
            Surviving the atmospheric entry is practically impossible.
        \item meteoroids originating from \emph{asteroids} are typically solid,
            rocky bodies, composed primarily of silicate minerals, ...

            They are composed either of primordial matter, which has never been metamorphosed by high pressures
            and temperatures within a larger body,
            (details)
        \item some sporadic meteor may rarely originate from old catastrophic disruptions
            of \emph{large differentiated bodies}, which exposed their metallic cores
            and ejected some of the material into interplanetary space.
            These meteoroids are composed primarily of iron and nickel.
            They are very dense, fairly  and resist ablation well,
            which means they are often able to survive their atmospheric
            entry and can thus be found as meteorites.
            However, as they are much rarer than stony meteoroids,
            their total numbers are low.
    \end{itemize}

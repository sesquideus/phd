\section{Simulation in the Solar System} \label{as}
    The simulation 

    including the gravitational influence of the parent body and then

    on the order of 100 radii.

    With ephemerides taken from the JPL Horizons system \citep{...}.
    The simulation computes the positions of the large bodies of the Solar System, most importantly the Sun,
    the eight planets and the Moon. Then the test particles

    Optinally, positions of massless bodies can be offloaded to a Graphics Processing Unit (GPU).
    Using a GPU is particularly well suited to massively parallel problems, such as simulation of non-interacting test particles.

    Any particle intersecting the cross-section of the Earth is marked as detected and will be examined further.
    Alternatively, in order to reduce the number of trials, we may employ a larger artificial detector.


    as long as the spatial density is only varying on scales much larger than the radius of the Earth,
    temporal variations in activity will be small.
    This assumption is not universally valid, as some younger meteor showers often produce dense, narrow filaments,
    which can be observed as short but intensive outbursts of heightened meteor activity.
    A~good example are the November alpha-Monocerotids, with a period of significantly
    heightened activity spanning only about thirty minutes in 2019 \citep{CBET4692}.

    By comparing the orbital speed and the diameter of the Earth we can see that the Earth moves
    to a completely different position with respect to the stream within several minutes.
    This means that variations in activity are not going to be significant when crossing an old stream,
    whose radius may be on the order of millions of kilometres, but will be of extreme importance with young, narrow filaments.

    The duration of visible activity also depends on the exact shape of the stream and its orientation with respect to the orbit of the Earth.
    The worst case is encountered with streams with high inclinations, where meteors seem to arrive from a direction perpendicular
    to the direction of movement. The 

     and thus is almost perpendicular to the Earth's instantaneous velocity vector.



    termined sphere centered on the Earth but with
    radius increased by about  order of magnitude --

    \subsection{Numerical integration of equations of motion} \label{asi}
        requires computation of $\frac{n\left(n - 1\right)}{2} \sim \mathcal{O}(n^2)$ pairs of forces \cite{...}.
        Due to Newton's third law, opposing pairs of forces are equal in magnitude and opposite in their direction,
        which brings the number of required evaluations to half that value, without changing the order.

        \subsubsection{Massively parallel integration} \label{asip}
            The main difference in simulating small particles, such as meteoroids,
            is that their masses are for all practical purposes completely negligible.
            All gravitational interaction between pairs of test particles may be completely ignored,
            which drastically reduces the number of evaluations.

            Suppose we only need to work with $k$ massive objects:
            in the Solar System we need to consider the Sun, eight planets, their major moons
            and -- at least immediately after the ejection -- also the parent body of the stream.
            This leaves us with $k \sim 10$, which translates to roughly 50--100 evaluations per integration step.
            This is perfectly tractable on a CPU.

            Once the positions of physical bodies are known, we may proceed to computing the forces acting
            on test bodies. Since forces can be added in linear manner as vectors, we need $k$ evaluations
            for each of $n$ test particles, for a total of $nk$ evaluations. As these forces 
            
            for a total of $k^2 + nk$. Since $n \gg k$, this results if

        \subsubsection{Using explicit ephemerides} \label{asie}
            To further simplify matters, we may leave determination of positions and velocities to other software
            and free resources for our business.
            We may directly use ephemerides provided by some external source, such as JPL Horizons \cite{...}.

            This leaves us with only $kn$ interactions that are completely mutually independent.
            Problems of this sort are perfectly suited for parallel computation.

            We hope to achieve sufficient performance to allow simulating about $10^7$ test particles.

\section{Simulation in the atmosphere} \label{aa}
    To simulate the atmospheric entry we used Asmodeus \citep{balaz-thesis}, \citep{balaz+2020}.
    Once a particle is selected for atmospheric entry, its velocity is transformed to the ECEF reference frame
    and 

        the simulation considers four forces acting on the particle:

    In the simulation we consider only two real forces
    \begin{itemize}
        \item the drag force $\vec{F}_{d}$, acting against the instantaneous velocity vector of the particle;
        \item the gravitational force $\vec{F}_{g}$, pulling the meteoroid towards the centre of the Earth;
    \end{itemize}

    
    and two fictitious forces, arising from the fact the simulation is performed in a rotating reference frame:
    \begin{itemize}
        \item the fictitious \emph{centrifugal} force, which pushes the particle away from the axis of rotation of the Earth;
        \item and the fictitious \emph{Coriolis force}.
    \end{itemize}

    In its cosmic stage and during most of the motion in the gravitational sphere of influence of the Earth,
    the particle's motion is dominated by gravitational effects. This is no longer true during the atmospheric entry,
    where the magnitude of the atmospheric drag force exceeds all other effects by several orders of magnitude.
    As the duration of the entire event is on the order of seconds, other forces simply do not act for long enough
    to produce a significant deviation in trajectory. For statistical evaluation of populations it is sufficient
    to consider only the drag force.

    However, should the meteoroid survive the entry as a meteorite,  their influence is important during the dark phase of flight,

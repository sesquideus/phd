\section{Simulation in the Solar System} \label{as}
    The simulation 

    including the gravitational influence of the parent body and then

    on the order of 100 radii.

    With ephemerides taken from the JPL Horizons system \citep{...}.
    The simulation computes the positions of the large bodies of the Solar System, most importantly the Sun,
    the eight planets and the Moon. Then the test particles

    Optinally, positions of massless bodies can be offloaded to a Graphics Processing Unit (GPU).
    Using a GPU is particularly well suited to massively parallel problems, such as simulation of non-interacting test particles.

    Any particle intersecting the cross-section of the Earth is marked as detected and will be examined further.
    Alternatively, in order to reduce the number of trials, we may employ a larger artificial detector.


    as long as the spatial density is only varying on scales much larger than the radius of the Earth,
    temporal variations in activity will be small.
    This assumption is not universally valid, as some younger meteor showers often produce dense, narrow filaments,
    which can be observed as short but intensive outbursts of heightened meteor activity.
    A suitable example are the November alpha-Monocerotids, with a period of significantly
    heightened activity spanning only about thirty minutes in 2019 \citep{CBET4692}.

    By comparing the orbital speed and the diameter of the Earth we can see that the Earth moves
    to a completely different position with respect to the stream within several minutes.
    This means that variations in activity are not going to be significant when crossing an old stream,
    whose radius may be on the order of millions of kilometres, but will be of extreme importance with young, narrow filaments.

    The duration of visible activity also depends on the exact shape of the stream and its orientation with respect to the orbit of the Earth.
    The worst case is encountered with streams with high inclinations, where meteors seem to arrive from a direction perpendicular
    to the direction of movement. The 
    
     and thus is almost perpendicular to the Earth's instantaneous velocity vector.



    termined sphere centered on the Earth but with
    radius increased by about  order of magnitude --

\section{Simulation in the atmosphere}
    To simulate the atmospheric entry we used Asmodeus \citep{balaz-thesis}, \citep{balaz+2020}.
    Once a particle is selected for atmospheric entry, its velocity is transformed to the ECEF reference frame
    and 

    the simulation considers four forces acting on the particle:
    \begin{itemize}
        \item the drag force, acting against the instantaneous velocity vector of the particle;
        \item the gravitational force, pulling the meteoroid towards the centre of the Earth;
        \item the fictitious \emph{centrifugal} force, which pushes the particle away from the axis of rotation of the Earth;
        \item and the fictitious \emph{Coriolis force}.
    \end{itemize}

    On the length and speed scales encountered in the simulation the fictitious forces can be safely neglected,
    their influence is significant only during the dark phase of flight, should the meteoroid survive the entry as a meteorite.

\begin{description}
    \item[CPU]
        \emph{central processing unit}, the general purpose electronic circuit of a computer \cite{...} (rewrite).
        The CPU is universal and can run practically any code, but only with low level of parallelism.
    \item[ECEF]
        \emph{earth-centered, earth-fixed} coordinate frame, a rotating reference frame with origin
        in the centre of mass of the Earth, rotating with the solid surface of the Earth \citep{ecef}
    \item[GPGPU]
        \emph{general-purpose computation on graphics processing units},
        the use of a GPU to handle general computing operations rather than strictly rendering graphical output \citep{techterms}
    \item[GPU]
        \emph{graphics processing unit}, a computer chip specifically designed to handle graphics operations \citep{techterms}.
        GPU cores are less versatile then CPU cores and typically much slower, but since there are thousands of them on a single
        chip, they can execute thousands of identical threads in parallel.
    \item[KDE]
        \emph{kernel density estimation}, a non-parametric method of estimating the probability density function
        of random variables from a number of discrete samples \citep{kde}
    \item[Null Island]
        a nickname for the fictional place at WGS84 coordinates \ang{0}~N, \ang{0}~E, \SI{0}{\metre} \citep{null-island}.
    \item[SIMD]
        \emph{single instruction, multiple data}, the practice of concurrently executing the same operation on
        different data on a multi-processor system or a GPU.
    \item[WGS84]
        \emph{World Geodetic System 1984}, the latest revision of the ellipsoidal Earth model \citep{nima-wgs84}
    \item[zenith attraction]
        a decrease in zenith angle of the radiant of a meteor shower, caused by the deflection of the meteoroids' orbits in the
        gravitational field of the Earth \citep{lovell1954}
    \item[ZHR]
        \emph{zenithal hourly rate}, the number of shower meteors per hour an ideal observer would see
            if his limiting magnitude is \Mag{+6.5} and the radiant is in his zenith \citep{imo-glossary}
\end{description}

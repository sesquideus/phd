\Epigraph[0.4]{
    You saw sagacious Solomon \\
    You know what came of him \\
    To him, complexities seemed plain
}{
    How Fortunate the Man With None \\
    \textsc{Bertolt Brecht} \\
    \textit{translation John Willett}
}

The discovery of meteors..


\section{The need to know}
    In a vast majority

    with the advent of space exploration this has changed. In the open space, the spacecraft and their crews are no longer shielded
    by the protective shell of the atmisphere, and any hypervelocity impact may be easily fatal.
    Even a grain of sand, travelling at typical speed on the order of \SI{50}{\kilo\metre\per\second} can permanently disable
    an expensive space mission, while an impact with a meteoroid the size of a marble is able to release
    destructive energy comparable to being hit by an anti-tank round.

    Technology is not yet advanced enough to offer any form of active protection.
    Passive protection against impacts with small energy has already been developed and tested,
    such as spacers \cite{...} or ...

    While it is theoretically possible to build passive shields that would be able to absorb impacts
    by meteoroids several orders of magnitude heavier, at the current state of technology these would
    be too heavy and thus prohibitively costly to launch.

    This might easily change once we are able to mine asteroids for water or minerals.
    Material that is not commercially valuable by itself, or space-gangue, may be put into use
    for shielding spacecraft or orbital stations.

    The best protection now is relying on luck: the spatial density of meteoroids is fortunately low enough
    to allow space stations to remain in orbit for a very long time. The main threat nowadays comes
    from man-made debris orbiting the Earth. While these objects are numerous, they are on stable orbits around
    the Earth, and it is possible to track them and record their orbits with high precision.
    This is not possible with meteoroids coming from beyond the sphere of influence of the Earth,
    which are usually too small and faint to be detected before their approach.

\section{Aim}
    In this thesis we aim to design a suitable representation of spatial density of meteoroids
    in interplanetary space and subsequently construct a model of their distribution,
    possibly including information about distribution of mass among the bodies and their chemical composition.

    There are two distinct approaches

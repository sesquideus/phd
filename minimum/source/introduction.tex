\Epigraph{
    Throughout the ages \\
    Of iron, bronze and stone \\
    We marvelled at the night sky \\
    And what may lie beyond
}{
    Children of the Sun \\
    \textsc{Dead Can Dance}
}

The research of interplanetary matter is ...

In comparison to deep-space astronomy, it does not produce spectacular discoveries,
such as the discoveries of exoplanets, detection of gravity waves or pictures of black holes.
What makes it different is the fact it is already accessible to us, with our technology,
and that no matter the technological advances, it will always be our home.


    to know our cosmic home, its origin and our place in it.

    Very importantly is mostly within our reach.

\section{The need to know} \label{in}


    A commonly used bound for a body that has a potential for causing mass destruction is on the order of tens of metres,
    depending on the exact geometry of the collision and velocity of the body.


In a vast majority of cases, meteoroids do not pose immediate danger to human activities.
On the surface we are shielded by the Earth's atmosphere, a layer of gases several tens of kilometres
thick, which offer massive resistance to any object trying to enter it at high speeds.
Only a minuscule fraction of bodies are tough and massive enough to survive the violent atmospheric entry.
Even then, atmospheric drag slows most of them down to their terminal speed, which greatly limits the potential destructive effects.
A body of size sufficient to reach the surface with speed much higher than its terminal speed
only arrives very sparsely. There are only a handful known craters younger than ten thousand years.


    An impact of a such a body at speeds of several kilometres per second would have disastrous but localized effects.
    For a global catastrophe the body would need to have size on the order
    of \SI{1}{\kilo\metre}. An impact of this size is thought to occur about once per a million (???) years. \todo{really?}
    While the consequences of an impact of this class would be very dire for the civilisation,
    it is very unlikely to experience one within any reasonable time frame.
    On the level of individuals or organizations it is completely negligible.

    Outside the atmosphere the situation is vastly different.
    The spacecraft operating in the open space and their crews are no longer shielded
    and any hypervelocity collision with a small fragment of interplanetary matter may easily be fatal.
    Even a grain of sand, travelling at typical speed on the order of \SI{50}{\kilo\metre\per\second} can permanently disable
    an expensive space mission or kill a person in a spacesuit.
    An impact of a meteoroid the size of a marble would release
    destructive energy comparable to a hit by an anti-tank round.

    Our technology is not yet advanced enough to offer any form of active protection.
    Passive protection is sufficient against impactors with low total energy.
    A rather simple, cost-effective solution are Whipple shields, consisting of a sacrificial thin outer layer,
    offset from the main hull of the spacecraft \citep{nasa-shield}.
    The outer layer does not aspire to stop the incoming particle,
    but rather to disrupt the meteoroid and reduce it to a debris cloud.
    The force of the impact is thus diluted over a much larger area of the spacecraft wall.

    It is theoretically possible to build passive protection that would be able to absorb impacts
    by meteoroids several orders of magnitude heavier.
    However, at the current state of technology these shields would be very heavy and therefore prohibitively costly to launch.
    This might change once we are able to mine asteroids for water or minerals.
    Most of the material extracted from small asteroids will be of very limited commercial usefulness.
    This ``asteroid gangue'' may however be put into good use: for shielding mining spacecraft or orbital stations.

    Currently, the best available protection is relying on pure luck. The spatial density of meteoroids is
    fortunately low enough to allow space stations to avoid damaging hits for a very long time,
    with more massive -- and thus more dangerous -- bodies being encountered increasingly less often.
    Even so, it is imperative to understand the risks precisely and to be able to assess the frequency
    of the collisions with smaller bodies.

    Another very similar threat comes from man-made debris orbiting the Earth.
    While these objects are relatively numerous, they are on stable orbits around
    the Earth, and it is often possible to track them and record their orbits with high precision.
    This is impossible to do for meteoroids coming from beyond the sphere of influence of the Earth,
    as they are usually too small and faint to be detected before their approach, and the possible
    observation window is very short.

    The useful product of the thesis should thus be a multi-dimensional map of density
    of meteoroids in the space around the Earth, which can be used to assess the risk
    of collision with a small piece of interplanetary matter at any space and time.

    On the other hand, sky maps are useful in predicting the activity levels of
    meteor showers and sporadic background and determination of ideal observing
    directions for narrow-field systems.

\section{Asteroids, comets and meteoroids} \label{ia}
    \todo{finish this here}

  

    The Sun, which constitutes almost \SI{99.9}{\percent} of the total mass, holds the entire
    system together by its gravity. It also provides the energy and entropy gradient necessary
    to support life on Earth.

    The four giant planets, slowly orbiting the Sun between about \SIrange[range-phrase = {\ and\ }]{5}{30}{\au}
    comprise most of the rest. The gravitational effects of the giant planets, most importantly Jupiter,
    have had a profound influence on the current appearance of the Solar System and continue to
    shape the orbits of any body that dares to journey too close to them.

    Then there are the inner planets: four vastly different rocky worlds laid out
    close to their parent star. On one of them there is life,
    which has recently acquired the capability to send robots to other planets as well.
    Apart from these major objects, an enormous number of smaller rocky, metallic and icy
    bodies are bound to the system by gravity as well.

    For us, they are very interesting -- first, out of pure scientific curiosity, as many of them
    have survived the billions of years without significant change; but also because of their destructive capabilities.
    Should a sufficiently large asteroid take on a collision course, it would be by far the most dangerous object
    known by the humanity. Fortunately, unlike with powerful solar flares, explosions of supernovae and other
    dangers lurking in the open space, against which we are virtually powerless,
    engineering a space mission to disrupt or deflect an oncoming asteroid is possible with available technology.
    The outlook of cheaper space flight also promises to unlock an enormous commercial and industrial potential in asteroids,
    all within the foreseeable future.


    \subsection{Asteroids} \label{iaa}
        The Sun, the first six planets and the Moon were known since times immemorial.

        the first body of the Solar System that was not a planet, a moon or a comet was only discovered at
        the very beginning of the \nth{19} century by Giuseppe Piazzi.

        These objects have formed everywhere where the gravitational attraction was able to overcome chaos \citep{???}

        To our current knowledge, in the regions where gravitational attraction dominated, almost all matter
        coalesced into just several large bodies, which we today know as the planets.
        In a region where the gravitational perturbations by the already-formed big planets, most notably Jupiter,
        did not allow one single body to form, we find a plethora of smaller bodies made of primordial material.

        The most prominent population of asteroids is known as the Main Belt, scattered between the orbits
        of Mars and Jupiter, with typical semi-major axes of about \SIrange{2}{3}{\au}.

        Currently about \num{800000} asteroids are known in the main belt \todo{research},
        If we also consider smaller objects, this number increases rapidly.
        It is estimated \citep{???} there are over a million objects with mean diameter of \SI{1}{\kilo\metre} or more,
        and hundred

        spectral types of asteroids (\emph{C}, \emph{S} and \emph{M}), which are closely reflected in
        spectral types of meteorites:

        \begin{itemize}
            \item \emph{C-type} (carbonaceous) asteroids are the most common. They are composed of silicate rocks and... \todo{finish}
            \item \emph{S-type} (silicate) asteroids... \todo{finish}
            \item \emph{M-type} (metallic) asteroids occur least frequently.
                They are thought to originate in dense cores of former large differentiated bodies that have
                undergone a catastrophic collision.
                The most prominent example if \emph{16 Psyche}
        \end{itemize}


    \subsection{Comets} \label{iac}
        Comets have been known to people for thousands of years. Historically they were
        frowned upon as harbingers of doom and disaster%
        \footnote{Hence a better term would likely be ``dis\textit{comet}''. The stars are innocent this time.}
        Due to their brightness and unusual appearance -- and lack of modern-day light and air pollution
        back then -- they must have been regularly observed by prehistoric people as well.
        Many comets later soon identified as different apparitions of the same celestial body.

        Unlike asteroids, comets are thought to have formed much further from the Sun, where the equilibrium
        temperatures were low enough to allow condensation of volatile substances, most importantly water ice.
        This taxonomical boundary is not entirely set in stone\footnote{or ice}.
        There are objects on asteroid-like orbits which exhibit cometary activity.
        They are conveniently named main-belt comets and often bear two designations, such as 7968 Elst--Pizarro,
        also known as 133P/Elst--Pizarro.

        Comets are responsible for the existence of most of the contemporary meteor showers.
        Modern missions, such as Giotto and most notably Rosetta, have significantly extended our knowledge
        about the nature of these bodies and helped ascertain the mechanisms that lead to production
        of streams of meteoroids from their surface.

        From the perspective of planetary defense, comets are much less likely to impact the Earth
        than asteroids. This is both due to their lower total numbers in the inner Solar System
        and large, highly eccentric trajectories, which make close encounters much less frequent.
        However, they are much more difficult to discover and track.
        Many methods of deflection, which require a long time to significantly change the trajectory,
        can be relatively easily applied to NEOs, but not to long-period comets.

    \subsection{Meteoroids} \label{iam}
        Apart from these two types of objects, interplanetary space is filled with numerous smaller objects
        of natural origin, which either have not yet coalesced into larger bodies, or are the remnants
        of such objects that have undergone catastrophic collisions.

        These tiny blobs of matter are virtually undetectable by themselves, except \textit{in situ} by passing spacecraft,
        in enormous numbers as zodiacal light, or upon entering the Earth's atmosphere as meteors.

        As there are countless it is not possible to describe them as a collection of bodies.
        However, we should be able to map the spatial density of these bodies in the vicinity of the Earth,
        and perhaps even more importantly, predict its future development.

        Despite the recent surge in the numbers of exploration missions in near space,
        our largest and most efficient detector so far is the Earth's atmosphere.



        A more formal definition is that a meteoroid is a ``solid object moving in interplanetary space,
        of a size considerably smaller than a asteroid and considerably larger than an atom or molecule'' \citep{imo-glossary}.
        The precise size limits have no hard physical constrains and as such are a matter of preference.
        However, no subfield of astronomy or science in general should be left to operate with such vague terms.

        An even more sophisticated definition has been approved by the IAU Commission F1 on Meteors, Meteorites and Interplanetary Dust,
        which clarifies these limits by explicitly stating that a meteoroid is a ``solid natural object
        of a size roughly between \SI{30}{\micro\metre} and \SI{1}{\metre} moving in, or coming from, interplanetary space'' \citep{imo-definitions}.
        Objects smaller than the lower limit are summarily named as \emph{interplanetary dust} while objects
        too large to be considered meteoroids under this definition are referred to as
        asteroids or comets, depending on their other characteristics.

        In colloquial usage in the context of meteor observations, any object producing a visible meteor can be called a meteoroid,
        irrespective of size or origin. This includes small asteroids or comets; pieces of space debris,
        such as fragments of defunct satellites, spent launch vehicles or blobs of products of combustion of solid fuel;
        or finally artificial meteoroids, launched with the expressed purpose of creating meteors \citep{japončíci?}.

        \subsection{Origins} \label{iamo}
            We usually consider three main sources of meteoroids, loosely corresponding to different
            mineralogical types of meteorites.

            Meteoroids associated with meteor showers are typically of cometary origins.
            Comets are typically composed of volatile material, such as water ice, interspersed
            with rocks or gravel. During the comet's close approaches to the Sun the ice sublimates
            and releases the solid grains. The drag of gas moving relative to the parent body
            is capable of lifting small solid particles with enough speed to overcome
            its weak gravitational force \citep{whipple1951}.
            The particles initially follow the orbit of their parent body,
            however, they are affected by perturbations by other large bodies or non-gravitational effects.
            On the time scale of tens of orbits they disperse along the entire orbit.

            Meteoroids of cometary origins are usually highly porous and friable.
            They are unable to survive the atmospheric entry, except when embedded in some stronger material \citep{nittler+2019}.
            Their bulk density is fairly low, with typical values around
            \SIrange{200}{1000}{\kilo\gram\per\cubic\metre} \citep{???} \todo{cite}.

            Catastrophic collisions of comets may produce secondary nuclei, which expose
            the previously undisturbed volatile material and thus are able to produce
            a large amount of new meteoroids and dust \citep{jenniskens2006}.

            The last important mechanism are the collisions of asteroids with other asteroids
            or with planets, which release a large amount of debris.

            In both meteor spectra and meteorites we may distinguish three major classes of material:
            \begin{itemize}
                \item Simple \emph{primordial matter} left undisturbed since the formation of the Solar System,
                    which has never been metamorphosed by high pressures and temperatures.
                    These bodies are mostly composed of silicate minerals, olivines and pyroxenes,
                    along with small amounts of metals, mostly iron and nickel.
                \item Metamorphosed silicate matter, which has been a part of a larger body,
                    where it may have been melted, shocked or chemically altered.
                \item Former cores of large differentiated bodies, which were disrupted by large collisions
                    that exposed the lower layers.
                    These meteoroids consist mostly of iron and nickel and subsequently are very dense,
                    fairly tough and resist ablation well, which means they are often able to
                    survive the entry into the Earth's atmosphere and can be later found as meteorites.
                    However, they are much rarer than stony meteoroids and their total numbers are low.
            \end{itemize}


    %    As of early \nth{21} century, the most prominent meteor showers are the
    %    \begin{table}[H]
    %        \begin{tabularx}{\textwidth}{l @{\extracolsep{\fill}} l r r}
    %            \toprule
    %                shower name &
    %                parent body &
    %                solar longitude at maximum activity $\lambda_\Sun$ &
    %                typical $\mathrm{ZHR}_\mathrm{max}$ (as of 2020) \\
    %            \midrule
    %                Quadrantids             &   2003 EH$_1$             & \ang{270}     & 100 \\
    %                Lyrids                  &   C/1861 G1 (Thatcher)    & \ang{40}      & 40 \\
    %                Perseids                &   109P/Swift--Tuttle      & \ang{140}     & 100 \\
    %                Leonids                 &   55P/Tempel--Tuttle      & \ang{190}     & 40 \\
    %                Geminids                &   (3200) Phaethon         & \ang{240}     & 150 \\
    %            \bottomrule
    %        \end{tabularx}
    %        \caption{Several prominent meteor showers}
    %    \end{table}

\section{Meteors} \label{il}
    The exact nature of meteors used to be the subject of lengthy scientific disputes.
    Some considered them to be electric phenomena \citep{???}.
    their astronomical origin was only confirmed on the beginning of the \nth{19} century.

    \todo{finish this}
    \todo{max 1 page}

    \todo{head, wake, ionization trail}

    \subsection{AMOS} \label{ila}
        Our primary source of observational data is the network of autonomous all-sky cameras AMOS (All-sky Meteor Orbit System),
        developed and operated by the Division of Astronomy and Astrophysics,
        DAPEM\footnote{Department of Astronomy, Physics of the Earth and Meteorology}
        FMPI\footnote{Faculty of Mathematics, Physics and Informatics} of Comenius University.

        \subsection{Technical specifications} \label{ilat}
            Each station consists of a digital video camera, an image intensifier, a wide angle fish-eye lens
            and various auxiliary mechanical and electronic components effecting
            the reliable operation of the device. The sky is monitored constantly every night
            as long as meteorological conditions are favourable. For a detailed technical report refer to \citet{zigo+2013,toth+2015}.

            The field of view of the cameras is centered on zenith and measures approximately \ang{180} by \ang{140}.
            As the atmosphere at low altitudes is thick, light travelling to the camera from
            meteors near the horizon is significantly attenuated.
            Only very few meteors are outside the covered area and the resulting loss of overall detection ability is minimal.

        \subsection{Operation} \label{ilao}
            The captured video sequences are processed by the \textsc{UFOAnalyzerV2} software package,
            conversion to custom-made software package is underway.
            Objects that are not meteors, such as aeroplanes, satellite flares, flying insects, etc.
            are identified and discarded from the final dataset. Naturally, the processing software is not
            infallible and occasionally reports these objects as meteors, or conversely fails to identify a visible meteor.
            After a meteor is identified, the video sequence is stored and analyzed.
            Properties of the meteor are computed for every frame of the video sequence,
            such as its position on the sky, apparent velocity vector, brightness and angular speed.
            Various auxiliary data are stored as well.

            Under ideal conditions each station is able to detect approximately 20000 meteors per year.
            Multi-station observations are somewhat less frequent, with each pair of stations being able
            to correctly match about 8000 meteors every year.
            With two or more simultaneous observations from different locations it is possible to determine
            the geocentric and heliocentric velocity vectors and derive the original heliocentric orbit of the meteoroid particle.
            For a more detailed description of the methods and algorithm used refer to \citet{fero?}.

        \subsection{Stations} \label{ilas}
            As of 2020, the network comprises eleven stations at three independent locations.
            \sisetup{group-minimum-digits = 8}
            The network in Slovakia consists of five nearly-identical stations:
            \begin{description}[leftmargin = 25mm]
                \item[AGO]      Astronomical and Geophysical Observatory in Modra-Piesok\\
                                since April 2007, coordinates \ang{48.3729}~N, \ang{17.2738}~E, \SI{531}{\metre}
                \item[ARBO]     Arboretum Mlyňany of the Slovak Academy of Sciences\\
                                since September 2009, coordinates \ang{48.3235}~N, \ang{18.3685}~E, \SI{201}{\metre}
                \item[KNM]      observatory in Kysucké Nové Mesto\\
                                since August 2012, coordinates \ang{49.3073}~N, \ang{18.7655}~E, \SI{414}{\metre}
                \item[VAZEC]    village of Važec\\
                                since October 2013, coordinates \ang{49.0543}~N, \ang{19.9899}~E, \SI{812}{\metre}
                \item[SENEC]    Senec Observatory\\
                                since ???, coordinates \ang{48.2203}~N, \ang{17.3950}~E, \SI{137}{\metre} \todo{correct}
            \end{description}

            A pair of stations has been installed on the Canary Islands, with cameras at
            \begin{description}[leftmargin = 25mm]
                \item[LP]       Observatorio del Roque de los Muchachos, La Palma, Canary Islands\\
                                since March 2015, coordinates \ang{28.7600}~N, \ang{17.8823}~W, \SI{2339}{\metre}
                \item[TE]       Observatorio del Teide, Tenerife, Canary Islands\\
                                since March 2015, coordinates \ang{28.3004}~N, \ang{16.5122}~W, \SI{2416}{\metre}
            \end{description}

            The third installation consists of two stations in northern Chile, located at
            \begin{description}[leftmargin = 25mm]
                \item[SP]       San Pedro de Atacama, El Loa, Antofagasta, Chile\\
                                since March 2016, coordinates \ang{22.9534}~S, \ang{68.1793}~W, \SI{2403}{\metre}
                \item[PC]       Paniri Caur Observatory, Chiu Chiu, El Loa, Antofagasta, Chile\\
                                since March 2016, coordinates \ang{22.3360}~S, \ang{68.6442}~W, \SI{2535}{\metre}
            \end{description}

            The last subnetwork is located on Hawai`i, with stations at
            \begin{description}[leftmargin = 25mm]
                \item[HK]       Haleakalā High Altitude Observatory Site, Kula, Maui, Hawaii, USA\\
                                since March 2016, coordinates \ang{20.7075}~N, \ang{156.2560}~W, \SI{3041}{\metre} \todo{fix}
                \item[MK]       Mauna Kea Observatories, Hawai`i County, Hawaii, USA\\
                                since March 2016, coordinates \ang{19.8228}~N, \ang{155.4697}~W, \SI{4203}{\metre} \todo{fix}
            \end{description}

            \sisetup{group-minimum-digits = 5}

            Expansion of the network to more locations, namely South Africa and Australia, is planned.
            With a total of six subnetworks, it will be possible to cover both the Northern and
            the Southern hemispheres continuously, weather permitting.

\Epigraph{
    Throughout the ages \\
    Of iron, bronze and stone \\
    We marvelled at the night sky \\
    And what may lie beyond
}{
    Children of the Sun \\
    \textsc{Dead Can Dance}
}

In comparison to many other branches of astronomy, research of interplanetary matter
does not produce spectacular findings too often.
The unmanned missions to other planets, recent discoveries of exoplanets, detection of gravity waves
or pictures of black holes seem to be much more fascinating.
However, deep space can only be observed, but not reached into, and will continue to remain
inaccessible to us for a very long time -- if not forever.

Within the Solar System, our nearest space neighbourhood, it is different. We are already able not only to observe,
but to reach it directly with our probes and -- probably within foreseeable future -- visit in person.
No matter how fast or unimaginably advances the technological progress will be, it will always be our home.
This is why now, when the technology and our knowledge are already sufficient to explore, map and possibly exploit
the nearest space, we should try hard to deepen and broaden our understanding of our cosmic home, its origin and our place in it.

\section{The need to know} \label{in}
    In a vast majority of cases, interplanetary matter does not pose immediate danger to human activities.
    On the surface we are shielded by the Earth's atmosphere, a layer of gases several tens of kilometres
    thick, which offer massive resistance to any object trying to enter it at high speeds.
    Only a minuscule fraction of bodies are tough and massive enough to survive the violent atmospheric entry.
    Even then, atmospheric drag slows most of them down to their terminal speed, which greatly limits the potential destructive effects.
    A commonly used bound for a body that has a potential for causing mass destruction is on the order of tens of metres,
    depending on the exact geometry of the collision and velocity of the body. Such objects, however, only arrive very sparsely.
    There are only a handful known craters younger than ten thousand years.

%    An impact of a such a body at speeds of several kilometres per second would have disastrous but localized effects.
%    For a global catastrophe the body would need to have size on the order of \SI{1}{\kilo\metre}.
%    An impact of this size is thought to occur about once per a million (???) years. \todo{really?}
%    While the consequences of an impact of this class would have dire consequences for our civilisation,
%    it is very unlikely to experience one within any reasonable time frame.
%    On the level of individuals or organizations the risk completely negligible.

    Outside the atmosphere the situation is vastly different.
    The spacecraft operating in the open space and their crews are no longer shielded
    and any hypervelocity collision with a small fragment of interplanetary matter may easily be fatal.
    Even a grain of sand, travelling at typical speed on the order of \SI{50}{\kilo\metre\per\second} can permanently disable
    an expensive space mission or kill a person in a spacesuit.
    An impact of a meteoroid the size of a marble would release
    destructive energy comparable to a direct hit by an anti-tank round.

    Our technology is not yet advanced enough to offer any form of active protection.
    Passive protection is sufficient against impactors with low total energy.
    A rather simple, cost-effective solution are Whipple shields, consisting of a sacrificial thin outer layer,
    offset from the main hull of the spacecraft.
    The outer layer does not aspire to stop the incoming particle,
    but rather to disrupt the meteoroid and reduce it to a debris cloud.
    The force of the impact is thus diluted over a much larger area of the spacecraft wall.
    For a more in-depth description and classification see \citet{nasa-shield}.

    It is theoretically possible to build a passive protective shield that would be able to absorb impacts
    by meteoroids several orders of magnitude more massive.
    However, at the current state of technology such shielding would be very heavy and therefore prohibitively costly to launch.
    This might change once we are able to mine asteroids for water or minerals.
    As with terrestrial mining, most of the material extracted from small asteroids will be of very limited commercial use.
    This ``asteroid gangue'' may be put into good use: for shielding mining spacecraft or orbital stations.

    Currently, the best available protection is relying on pure luck. The spatial density of meteoroids is
    fortunately low enough to allow space stations to avoid being hit for a very long time,
    with more massive -- and thus more dangerous -- bodies being encountered much more sparsely.
    Even so, it is imperative to understand the risks precisely and to be able to assess the frequency
    of collisions with small bodies.

    Another very similar threat comes from man-made debris orbiting the Earth.
    While these objects are relatively numerous, they are on stable orbits around
    the Earth, and we are often able to track them and record and predict their trajectories with high precision.
    This is impossible to do for meteoroids coming from beyond the sphere of influence of the Earth,
    as they are usually too small and faint to be detected before their approach,
    and the available observation window is very short.

    Apart from its scientific value, a directly useful product of the thesis should thus
    be a multi-dimensional map of density of meteoroids in the space around the Earth,
    which can be used to assess the risk of collisions with small natural objects
    at any space and time.

\section{Asteroids, comets and meteoroids} \label{ia}
    The Solar System consists of many different kinds of matter and provides a multitude
    of environments and opportunities to study and understand.
    The dominating elements is the Sun, which constitutes almost \SI{99.9}{\percent} of the total mass
    and holds the entire system together with its gravity.
    It also provides the energy gradient necessary to support life on Earth.

    The four giant planets, slowly orbiting the Sun between about \SIrange[range-phrase = {\ and\ }]{5}{30}{\au}
    comprise most of the rest. The gravitational effects of the giant planets, most importantly Jupiter,
    have had a profound influence on the current appearance of the Solar System and continue to
    shape the orbits of other bodies bound to the system.

    Then there are the inner planets: four vastly different rocky worlds laid out
    close to their parent star. On one of them there is life,
    which has recently acquired the capability to send robots to other planets as well.

    Apart from these major objects, enormous numbers of smaller rocky, metallic and icy
    bodies are bound to the system by gravity as well. Together they comprise only a minuscule
    fraction of the total solid mass of the Solar System. Yet this category is very populous
    and includes most of the named objects, along with tiny fragments and interplanetary dust,
    which can only be described collectively.

    For us, these objects are very interesting. First, out of pure scientific curiosity, as many of them
    have survived the billions of years without significant change and can thus be considered
    to be unused ingredients of the original recipe from which the Solar System was formed.
    Observations, in-situ analyses or sample retrieval missions such as Hayabusa or OSIRIS-REx
    provide ample data to study and interpret and to formulate hypotheses and theories about our origins.

    Second, it is because of their destructive capabilities.
    Should a sufficiently large asteroid take on a collision course, it would be by far the most dangerous object
    known by the humanity. Fortunately, unlike with powerful solar flares, explosions of supernovae and other
    dangers lurking in the open space, against which we are virtually powerless,
    engineering a space mission to disrupt or deflect an oncoming asteroid is possible with available technology.
    The outlook of cheaper space flight also promises to unlock an enormous commercial and industrial potential in asteroids,
    all within the foreseeable future.

    \subsection{Asteroids} \label{iaa}
        The Sun, the first six planets and the Moon were known since times immemorial.
        The same is true about comets, which, while far less populous and massive,
        are very effective in reflecting the sunlight and thus can be often seen with an unaided eye.
        The darker surface and lack of cometary activity of dwarf planets and asteroids enabled them
        to evade the searching astronomers for a much longer time.
        The first body of the Solar System that was not a planet, a moon or a comet was only discovered at
        the very beginning of the \nth{19} century by Giuseppe Piazzi, after considerable and concentrated effort.

        To our current knowledge, in the regions of the forming Solar System where gravity dominated quickly,
        almost all matter coalesced into just several large bodies, which we today know as the planets.
        In other regions the gravitational perturbations by the already-formed big planets, most notably Jupiter,
        did not allow the formation of a single body that would later clear its orbit of smalled objects,
        we find a plethora of smaller bodies. As most of them have not since coalesced into a larger whole,
        where gravitational compression would heat the material enough to melt, they mostly consist of primordial material.

        The most prominent population of asteroids is known as the Main Belt, scattered between the orbits
        of Mars and Jupiter, with typical semi-major axes of about \SIrange{2}{3.5}{\au} and low inclination
        with respect to the Laplace plane. While the total number of asteroids, even in the Main Belt,
        is very low compared to the volume of the Main Belt, on long time scales collisions are inevitable.
        High-energy collisions are capable of disrupting the bodies completely, forming large numbers
        of fresh fragments. Collisions of bodies differing in mass by orders of magnitudes most often
        only result in cratering of the surface. Nevertheless, as gravitational fields of these bodies
        are relatively weak, debris from collisions is not contained and can disperse into space.
        Even without collisions, asteroids may undergo disintegration due to YORP effect,
        where angular momentum is increased by differential heating of a asymmetric body
        until inertial forces overcome the force gravity holding the body together.

        On the other hand, asteroids are regularly removed from their orbits due to
        close encounters or collisions with large planets; entering forbidden resonances;
        or are slowly moved by differential heating, known as Yarkovsky's effect.
        On long time scales the entire system is in a dynamical balance.

        Currently there are almost a million known asteroids, with surveys discovering more practically every day.
        The number of objects grows rapidly with decreasing size. It is estimated there are about \num{1.5} million
        asteroids larger than \SI{1}{\kilo\metre} in diameter in the Main Belt alone.
        Objects smaller than this are even more numerous, but also increasingly more difficult to discover.
        For even smaller objects it is no longer practical to describe them as a collection of separate bodies,
        but rather only to describe their spatial density.

        %spectral types of asteroids (\emph{C}, \emph{S} and \emph{M}), which are closely reflected in
        %spectral types of meteorites:

        %\begin{itemize}
        %    \item \emph{C-type} (carbonaceous) asteroids are the most common. They are composed of silicate rocks and... \todo{finish}
        %    \item \emph{S-type} (silicate) asteroids... \todo{finish}
        %    \item \emph{M-type} (metallic) asteroids occur least frequently.
        %        They are thought to originate in dense cores of former large differentiated bodies that have
        %        undergone a catastrophic collision.
        %        The most prominent example is \emph{16 Psyche} \citep{???}
        %\end{itemize}


    \subsection{Comets} \label{iac}
        Comets have been known to people for thousands of years. Historically they were
        frowned upon as harbingers of doom and disaster.%
        \footnote{Hence a better term would likely be ``dis\textit{comet}''.}
        Due to their vastly larger brightness compared to asteroids, unusual appearance -- and sadly, also the
        lack of modern-day light and air pollution back then -- they must have been regularly observed by prehistoric people as well.
        Deeper understanding of their nature and origin was only possible with development of celestial mechanics,
        when many comets were identified as different apparitions of the same celestial body \citep{nasa-halley}.
        Unmanned missions, such as Giotto and Rosetta, have significantly extended our knowledge
        about the nature of these bodies and helped ascertain the mechanisms that lead to production of particles from their surfaces.

        Comets are now thought to have formed much further from the Sun than typical asteroids,
        where equilibrium temperatures there were low enough to allow condensation of volatile substances, most importantly water ice,
        along with dust and rocky material \citep{???}. \todo{cite}
        The taxonomical boundary between these is not entirely set in stone -- there
        are objects on asteroid-like orbits which exhibit cometary activity.
        They are conveniently named main-belt comets and often bear two designations, such as the asteroid 7968 Elst--Pizarro,
        also known as the comet 133P/Elst--Pizarro.

        Dust and small debris released from the surface of comets are responsible
        for the existence of most of the contemporary meteor showers.
        of streams of meteoroids from their surface. \citep{???}. \todo{cite}

        From the perspective of planetary defense, comets are much less likely to impact the Earth
        than asteroids. This is both due to their lower total numbers in the inner Solar System
        and highly eccentric trajectories with long periods, which result in close encounters being much less frequent.
        However, they are also much more difficult to discover and track, and it is virtually impossible
        to predict when a new, previously unknown comet might appear.
        Also many methods of deflection, which would require a long time to significantly change the trajectory and
        can be relatively easily applied to NEOs, are not viable with long-period comets.

    \subsection{Meteoroids} \label{iam}
        Apart from these two types of objects, interplanetary space is filled with numerous even smaller objects
        of natural origin, which either have not yet coalesced into larger bodies, or are the remnants
        of such objects that have undergone catastrophic collisions.

        These tiny blobs of matter are virtually undetectable by themselves, except \textit{in situ} by passing spacecraft,
        in enormous numbers as zodiacal light, or upon entering the Earth's atmosphere as meteors.
        Due to their sheer numbers and difficulty of detection it is not possible to describe them as a collection of bodies,
        but rather as a varying continuum, with each particle obeying the laws of gravitation, but also being affected by non-gravitational forces.
        We should be able to map the spatial density of these bodies in the vicinity of the Earth,
        and perhaps even more importantly, predict its future development.

        Despite the recent surge in the numbers of exploration missions in near space,
        our largest and most efficient detector so far is the Earth's atmosphere,
        where the kinetic energy of these bodies can be converted to heat and visible light.
        The resulting phenomena are known as \emph{meteors} and can be readily detected from the surface of the Earth.

        A formal definition is that a meteoroid is a ``solid object moving in interplanetary space,
        of a size considerably smaller than a asteroid and considerably larger than an atom or molecule'' \citep{imo-glossary}.
        The precise size limits have no hard physical constrains and as such are a matter of preference.
        However, no subfield of astronomy or science in general should be left to operate with such vague terms.

        An even more sophisticated definition has been approved\footnote{by the IAU Commission F1 on Meteors, Meteorites and Interplanetary Dust, 2017},
        which clarifies these limits by explicitly stating that a meteoroid is a ``solid natural object
        of a size roughly between $\SI{30}{\micro\metre}$ and \SI{1}{\metre} moving in, or coming from, interplanetary space'' \citep{imo-definitions}.
        Solid objects smaller than the lower limit are summarily named as \emph{interplanetary dust} while objects
        too large to be considered meteoroids under this definition are referred to as
        asteroids or comets, depending on their other characteristics.

        In colloquial usage in the context of meteor observations, any object producing a visible meteor can be called a meteoroid,
        irrespective of size or origin. This includes small asteroids or comets; pieces of space debris,
        such as fragments of defunct satellites, spent launch vehicles or blobs of products of combustion of solid fuel;
        or finally artificial meteoroids, launched with the expressed purpose of creating meteors.

        \subsection{Origins} \label{iamo}
            We usually consider three main sources of meteoroids, loosely corresponding to different
            mineralogical types of meteorites.

            Meteoroids associated with meteor showers are typically of cometary origins.
            Comets are mostly composed of volatile material, such as water ice, interspersed
            with rocks or gravel. During the comet's close approaches to the Sun the ice sublimates
            and releases the solid grains. The drag of gas moving relative to the parent body
            is capable of lifting small solid particles and eventually imparting enough speed
            to overcome the weak gravitational influence of the comet \citep{whipple1951}.
            The particles initially closely follow the orbit of their parent body,
            however, they are affected by perturbations by other large bodies or non-gravitational effects
            and on the time scale of tens of orbits they disperse along the entire orbit.
            Meteoroids of cometary origins are usually highly porous and friable.
            They are unable to survive the atmospheric entry, except when embedded in some stronger material \citep{nittler+2019}.

            Catastrophic collisions of comets may produce secondary nuclei, which expose
            the previously undisturbed volatile material and thus are able to produce
            a large amount of new meteoroids and dust \citep{jenniskens2006}.

            The last important mechanism are the collisions of asteroids with other asteroids
            or with planets, which release a large amount of debris. Orbital elements of the fragments
            may differ from the orbital elements of the original bodies significantly.
            On longer time scales they are also perturbed by other bodies or non-gravitational effects.
            These particles are the main source of the \emph{sporadic background}.

            In both meteor spectra and meteorites we may distinguish three major classes of material:
            \begin{itemize}
                \item Simple \emph{primordial matter} left undisturbed since the formation of the Solar System,
                    which has never been metamorphosed by high pressures and temperatures.
                    These bodies are mostly composed of silicate minerals, olivines and pyroxenes,
                    along with small amounts of metals, mostly iron and nickel.
                \item \emph{Metamorphosed} matter, which has been a part of a larger body,
                    where it may have been melted, shocked or chemically altered.
                \item \emph{Former cores} of large differentiated bodies, which were disrupted by large collisions
                    that exposed the lower layers.
                    These meteoroids consist mostly of iron and nickel and subsequently are very dense,
                    fairly tough and resist ablation well, which means they are often able to
                    survive the entry into the Earth's atmosphere and can be later found as meteorites.
                    However, they are much rarer than stony meteoroids and their total numbers are low.
            \end{itemize}

        \subsection{Evolution} \label{iamf}
            After meteoroids are released from the surface, the forces acting on them are
            by the gravitational force, except for very small particles where the force of solar radiation pressure
            might prevail.

            In case of collisions between asteroids the ejection velocities of debris are often
            large enough to neglect the energy needed to overcome the gravitational pull of the parent body.
            Particles ejected from comets during outgassing have much lower speeds.
            Gravity drag may significantly reduce their kinetic energy, possibly so much they will return to the surface.
            In any case, we may only start tracking the particle after it leaves the sphere of influence of its parent body.
            Differences in initial velocities and ejection time are responsible
            for primary dispersion of their orbits. Variations in particle size and their tendency to be affected
            by non-gravitational forces continue to disperse the particle cloud further.
            In encounters with larger bodies or resonances the orbits are scattered even more.
            Very small meteoroids are also affected by the Poynting-Robertson effect, which
            reduces their orbital energy. This results in particles spiralling inwards and being destroyed by the Sun.

    %    As of early \nth{21} century, the most prominent meteor showers are the
    %    \begin{table}[H]
    %        \begin{tabularx}{\textwidth}{l @{\extracolsep{\fill}} l r r}
    %            \toprule
    %                shower name &
    %                parent body &
    %                solar longitude at maximum activity $\lambda_\Sun$ &
    %                typical $\mathrm{ZHR}_\mathrm{max}$ (as of 2020) \\
    %            \midrule
    %                Quadrantids             &   2003 EH$_1$             & \ang{270}     & 100 \\
    %                Lyrids                  &   C/1861 G1 (Thatcher)    & \ang{40}      & 40 \\
    %                Perseids                &   109P/Swift--Tuttle      & \ang{140}     & 100 \\
    %                Leonids                 &   55P/Tempel--Tuttle      & \ang{190}     & 40 \\
    %                Geminids                &   (3200) Phaethon         & \ang{240}     & 150 \\
    %            \bottomrule
    %        \end{tabularx}
    %        \caption{Several prominent meteor showers}
    %    \end{table}

\section{Meteors} \label{il}
    When a meteoroid enters the atmosphere\footnote{Here we are only concerned with the atmosphere of the Earth,
    but the process is very similar on other planets with atmospheres.}, its kinetic energy
    is converted to heat and visible light in collisions with the gas molecules and subsequent ionization
    and deexcitation. This phenomenon is called a meteor.

    The exact nature of meteors used to be the subject of lengthy scientific disputes.
    Their astronomical origin was only definitely accepted on the beginning of the \nth{19} century \citep{czegka2000}.
    Meteors can be observed with an unaided eye. This simplest method is useful for basic assessment of activity,
    but does not produce verifiable records. Photographic records, video sequences and radiometric measurements
    allow to perform detailed analyses of meteor trails, 
    and in cases of multiple observations of the same meteor, determination of orbits of the meteoroids which produced them.

    Meteoroid speeds with respect to the Earth are on the order of tens of kilometres per second,
    from approximately \SI{11200}{\metre\per\second} for a particle on an Earth-like
    heliocentric orbit attracted solely by the Earth's gravity, to \SI{72800}{\metre\per\second}
    for a head-on collision with a particle on a parabolic orbit (disregarding the rotation of the Earth in both cases).

    \subsection{Atmospheric entry} \label{ile}
        During its atmospheric entry a meteoroid experiences several different regimes in rapid succession.
        For a thorough description of meteor-related phenomena refer to \citep{ceplecha+1998}.

        In the first phase the meteoroid is \emph{pre-heated} -- its surface temperature rises rapidly
        as a result of collisions with the atoms and molecules of the atmosphere,
        approximately in proportion to their density.
        The entire process typically takes several seconds and occurs at typical altitudes
        between \SIrange{100}{300}{\kilo\metre}. The heating is not sufficient to produce a visible meteor yet,
        except in case of very massive bodies. At the same time the meteoroid compresses
        the atmospheric gases, which increases the dynamical pressure. Once the component of the drag force
        tangential to the meteoroid's surface exceeds the strength of the material,
        superficial spallation starts.

        The next stage begins at an altitude of about \SI{100}{\kilo\metre}.
        The density of the atmosphere is sufficient to raise the temperature of the surface
        to about \SI{2500}{\kelvin}, which is enough to evaporate the material and ionize its constituent atoms.
        The recombination of ions is the main source of the visible light that can be observed as a meteor.
        The evaporated material \emph{ablates} as gas, as liquid droplets or as smaller fragments.
        Fragmentation greatly increases the total cross-section, which in turn
        further increases the rate at which the material is ablated away.
        A part of the energy is lost to the surrounding atmosphere as heat.

        The losses of kinetic energy also result in deceleration of the meteoroid.
        Smaller bodies are completely destroyed before they are decelerated significantly.
        More massive meteoroids may survive the ablation stage and enter the third phase, the \emph{dark flight}.
        This generally occurs at speeds on the order of a few kilometres per second and altitudes of about \SIrange{20}{50}{\kilo\metre}.
        From this point we usually speak of a \emph{meteorite}.
        The kinetic energy of the oncoming particles is no longer sufficient to evaporate the material
        and the surface is rapidly cooled to ambient temperatures, forming a fusion crust.

        The dense atmosphere slows the meteorite to its terminal speed of several hundred metres per second,
        which further decreases approximately proportionally to the square root of the density.
        While in theory the trajectory of a falling meteorite can be computed with high precision,
        it is practically not viable, as knowledge of its shape is required, which is not available
        until after it is found.
        Inevitably, the meteorite impacts the surface of the Earth,
        with typical speeds of about \SIrange{10}{100}{\metre\per\second} depending on the density and size.

\section{AMOS} \label{iA}
    Our primary source of observational data is the network of autonomous all-sky cameras AMOS (All-sky Meteor Orbit System),
    developed and operated by the Division of Astronomy and Astrophysics,
    DAPEM\footnote{Department of Astronomy, Physics of the Earth and Meteorology}
    FMPI\footnote{Faculty of Mathematics, Physics and Informatics} of Comenius University.

    \subsection{Technical specifications} \label{iAt}
        Each station consists of a digital video camera, an image intensifier, a wide angle fish-eye lens
        and various auxiliary mechanical and electronic components effecting
        the reliable operation of the device. The sky is monitored constantly every night
        as long as meteorological conditions are favourable. For a detailed technical report refer to \citet{zigo+2013,toth+2015}.

        The field of view of the cameras is centered on zenith and measures approximately \ang{180} by \ang{140}.
        As the atmosphere near the horizon is thick, light travelling to the camera from meteors there horizon is significantly attenuated.
        Only very few meteors are outside the covered area and the resulting loss of overall detection ability is minimal.

    \subsection{Operation} \label{iAo}
        The captured video sequences are processed by the \textsc{UFOAnalyzerV2} software package,
        conversion to custom-made software package is underway.
        Objects that are not meteors, such as aeroplanes, satellite flares, flying insects, etc.
        are identified and discarded from the final dataset. Naturally, the processing software is not
        infallible and occasionally reports these objects as meteors, or conversely fails to identify a visible meteor.
        After a meteor is identified, the video sequence is stored and analyzed.
        Properties of the meteor are computed for every frame of the video sequence,
        such as its position on the sky, apparent velocity vector, brightness and angular speed.
        Various auxiliary data are stored as well.

        Under ideal conditions each station is able to detect approximately 20000 meteors per year.
        Multi-station observations are somewhat less frequent, with each pair of stations being able
        to correctly match about 8000 meteors every year.
        With two or more simultaneous observations from different locations it is possible to determine
        the geocentric and heliocentric velocity vectors and derive the original heliocentric orbit of the meteoroid particle.
        For a more detailed description of the methods and algorithm used refer to \citet{kornos+2015}.

    \subsection{Stations} \label{iAs}
        As of 2020, the network comprises eleven stations at three independent locations.
        \sisetup{group-minimum-digits = 8}
        The network in Slovakia consists of five nearly-identical stations:
        \begin{description}[leftmargin = 25mm]
            \item[AGO]      Astronomical and Geophysical Observatory in Modra-Piesok\\
                            since April 2007, \ang{48.3729}~N, \ang{17.2738}~E, \SI{531}{\metre}
            \item[ARBO]     Arboretum Mlyňany of the Slovak Academy of Sciences\\
                            since September 2009, \ang{48.3235}~N, \ang{18.3685}~E, \SI{201}{\metre}
            \item[KNM]      observatory in Kysucké Nové Mesto\\
                            since August 2012, \ang{49.3073}~N, \ang{18.7655}~E, \SI{414}{\metre}
            \item[VAZEC]    village of Važec\\
                            since October 2013, \ang{49.0543}~N, \ang{19.9899}~E, \SI{812}{\metre}
            \item[SENEC]    Senec Observatory\\
                            since January 2019, \ang{48.2203}~N, \ang{17.3950}~E, \SI{137}{\metre}
        \end{description}

        A pair of stations has been installed on the Canary Islands, with cameras at
        \begin{description}[leftmargin = 25mm]
            \item[LP]       Observatorio del Roque de los Muchachos, La Palma \\
                            since March 2015, \ang{28.7600}~N, \ang{17.8823}~W, \SI{2339}{\metre}
            \item[TE]       Observatorio del Teide, Tenerife \\
                            since March 2015, \ang{28.3004}~N, \ang{16.5122}~W, \SI{2416}{\metre}
        \end{description}

        The third installation consists of two stations in northern Chile, located at
        \begin{description}[leftmargin = 25mm]
            \item[SP]       San Pedro de Atacama, El Loa, Antofagasta \\
                            since March 2016, \ang{22.9534}~S, \ang{68.1793}~W, \SI{2403}{\metre}
            \item[PC]       Paniri Caur Observatory, Chiu Chiu, El Loa, Antofagasta \\
                            since March 2016, \ang{22.3360}~S, \ang{68.6442}~W, \SI{2535}{\metre}
        \end{description}

        The last subnetwork is located on Hawai`i, with stations at
        \begin{description}[leftmargin = 25mm]
            \item[HK]       Haleakalā High Altitude Observatory Site, Kula, Maui \\
                            since September 2018, \ang{20.7075}~N, \ang{156.2560}~W, \SI{3041}{\metre}
            \item[MK]       Mauna Kea Observatories, Hawai`i County \\
                            since September 2018, \ang{19.8228}~N, \ang{155.4697}~W, \SI{4203}{\metre}
        \end{description}

        \sisetup{group-minimum-digits = 5}

        Expansion of the network to more locations, namely South Africa and Australia, is planned.
        With a total of six subnetworks, it will be possible to cover both the Northern and
        the Southern hemispheres continuously, weather permitting.

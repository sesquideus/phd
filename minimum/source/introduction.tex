\Epigraph[0.4]{
    You saw sagacious Solomon \\
    You know what came of him \\
    To him, complexities seemed plain
}{
    How Fortunate the Man With None \\
    \textsc{Bertolt Brecht} \\
    \textit{translation John Willett}
}

The discovery of meteors..


\section{The need to know}
    In a vast majority

    with the advent of space exploration this has changed. In open space, the spacecraft and their crews are no longer shielded
    by the protective shell of the atmosphere. A hypervelocity collision with a small meteoroid may easily become fatal.
    Even a grain of sand, travelling at typical speed on the order of \SI{50}{\kilo\metre\per\second} can permanently disable
    an expensive space mission, while an impact with a meteoroid the size of a marble is able to release
    destructive energy comparable to being hit by an anti-tank round.

    Technology is not yet advanced enough to offer any form of active protection.
    Passive protection against impacts with small energy has already been developed and tested,
    such as spaced shielding \cite{...} or ...

    While it is theoretically possible to build passive shields that would be able to absorb impacts
    by meteoroids several orders of magnitude heavier, at the current state of technology these would
    be very heavy and therefore prohibitively costly to launch.
    This might change once we are able to mine asteroids for water or minerals.
    Most of the material extracted from small asteroids will be of very limited commercial usefulness.
    This ``asteroid gangue'' may however be put into good use: for shielding spacecraft or orbital stations.

    Currently, the best available protection is relying on pure luck. The spatial density of meteoroids is
    fortunately low enough to allow space stations to avoid damaging hits for a very long time,
    with more massive -- and thus more dangerous -- bodies being encountered increasingly less often.
    Even so, it is imperative to understand the risks precisely and to be able to assess the frequency
    of the collisions with smaller bodies.

    Another very similar threat comes from man-made debris orbiting the Earth.
    While these objects are relatively numerous, they are on stable orbits around
    the Earth, and it is often possible to track them and record their orbits with high precision.
    This is impossible to do for meteoroids coming from beyond the sphere of influence of the Earth,
    as they are usually too small and faint to be detected before their approach, and the possible
    observation window is very short.

    The useful product of the thesis should thus be a multi-dimensional map of density
    of meteoroids in the space around the Earth, which can be used to assess the risk
    of collision with a small piece of interplanetary matter at any space and time.

    On the other hand, sky maps are useful in predicting the activity levels of
    meteor showers and sporadic background and determination of ideal observing
    directions for narrow-field systems.


\section{Aim}
    In this thesis we aim to design a suitable representation of spatial density of meteoroids
    in interplanetary space and subsequently construct a model of their distribution,
    possibly including information about distribution of mass among the bodies and their chemical composition.

    There are two notably distinct approaches to the problem.

    First, we are able to analyze the database of observed meteors

    The second approach works in quite the opposite direction. We start with the 

    On the far horizon of our efforts is the synthesis of both methods into one whole:
    we may use the observational data to put rough constrains on the meteoroid flux,
    and then construct an $N$-body numerical simulation which will try to replicate the data,
    accounting for observational bias as well.

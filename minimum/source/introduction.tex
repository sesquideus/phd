\section{The need to know} \label{in}
    In a vast majority of cases meteoroids do not pose immediate danger to human activities.
    On the surface we are shielded by the Earth's atmosphere, a layer of gases several tens of kilometres
    thick, which offer massive resistance to any object trying to enter it at high speeds.
    Only a minuscule fraction of bodies are tough and massive enough to survive the violent atmospheric entry.
    Even then, atmospheric drag slows most of them down to their terminal speed, which greatly limits the potential destructive effects.
    It is estimated that a body of size sufficient to reach the surface with speed much higher than its terminal speed
    only arrives about once per \todo{research this}.

    A commonly used bound for a body that has a potential for causing mass destruction is on the order of tens of metres,
    depending on the exact geometry of the collision and velocity of the body.

    An impact of a such a body at speeds of several kilometres per second would have disastrous but localized effects.
    For a global catastrophe the body would need to have size on the order
    of \SI{1}{\kilo\metre}. An impact of this size is thought to occur about once per a million (???) years. \todo{really?}
    While the consequences of an impact of this class would be very dire for the civilisation,
    it is very unlikely to experience one within any reasonable time frame.
    On the level of individuals or organizations it is completely negligible.

    But outside the atmosphere the situation is vastly different.
    The spacecraft operating in the open space and their crews are no longer shielded
    and any hypervelocity collision with a small fragment of interplanetary matter may easily be fatal.
    Even a grain of sand, travelling at typical speed on the order of \SI{50}{\kilo\metre\per\second} can permanently disable
    an expensive space mission or kill a person in a spacesuit,
    while an impact with a meteoroid the size of a marble could release
    destructive energy comparable to being hit by an anti-tank round.

    Our technology is not yet advanced enough to offer any form of active protection.
    Against impacts with lower energy, passive protection is sufficient.
    A rather simple, cost-effective solution are Whipple shields, consisting of a sacrificial thin outer layer,
    offset from the main hull of the spacecraft \citep{nasa-shield}.
    The outer layer does not aspire to stop the incoming particle,
    but rather to disrupt the projectile and reduce it to a debris cloud.
    The force of the impact is thus diluted over a much larger area of the spacecraft wall.

    It is theoretically possible to build passive protection that would be able to absorb impacts
    by meteoroids several orders of magnitude heavier, at least in laboratory conditions.
    However, at the current state of technology these would be very heavy and therefore prohibitively costly to launch.
    This might change once we are able to mine asteroids for water or minerals.
    Most of the material extracted from small asteroids will be of very limited commercial usefulness.
    This ``asteroid gangue'' may however be put into good use: for shielding mining spacecraft or orbital stations.

    Currently, the best available protection is relying on pure luck. The spatial density of meteoroids is
    fortunately low enough to allow space stations to avoid damaging hits for a very long time,
    with more massive -- and thus more dangerous -- bodies being encountered increasingly less often.
    Even so, it is imperative to understand the risks precisely and to be able to assess the frequency
    of the collisions with smaller bodies.

    Another very similar threat comes from man-made debris orbiting the Earth.
    While these objects are relatively numerous, they are on stable orbits around
    the Earth, and it is often possible to track them and record their orbits with high precision.
    This is impossible to do for meteoroids coming from beyond the sphere of influence of the Earth,
    as they are usually too small and faint to be detected before their approach, and the possible
    observation window is very short.

    The useful product of the thesis should thus be a multi-dimensional map of density
    of meteoroids in the space around the Earth, which can be used to assess the risk
    of collision with a small piece of interplanetary matter at any space and time.

    On the other hand, sky maps are useful in predicting the activity levels of
    meteor showers and sporadic background and determination of ideal observing
    directions for narrow-field systems.

\section{AMOS} \label{iA}
    Our primary source of observational data is the network of autonomous all-sky cameras AMOS (All-sky Meteor Orbit System),
    developed and operated by the Division of Astronomy and Astrophysics,
    DAPEM\footnote{Department of Astronomy, Physics of the Earth and Meteorology}
    FMPI\footnote{Faculty of Mathematics, Physics and Informatics} of Comenius University.

    \subsection{Technical specifications} \label{iAt}
        Each station consists of a digital video camera, an image intensifier, a wide angle fish-eye lens
        and various auxiliary mechanical and electronic components effecting
        the reliable operation of the device. The sky is monitored constantly every night
        as long as meteorological conditions are favourable. For a detailed technical report refer to \citet{zigo+2013,toth+2015}.

        The field of view of the cameras is centered on zenith and measures approximately \ang{180} by \ang{140}.
        As the atmosphere at low altitudes is thick, light travelling to the camera from
        meteors near the horizon is significantly attenuated.
        Only very few meteors are outside the covered area and the resulting loss of overall detection ability is minimal.

    \subsection{Operation} \label{iAo}
        The captured video sequences are processed by the \textsc{UFOAnalyzerV2} software package,
        conversion to custom-made software package is underway.
        Objects that are not meteors, such as aeroplanes, satellite flares, flying insects, etc.
        are identified and discarded from the final dataset. Naturally, the processing software is not
        infallible and occasionally reports these objects as meteors, or conversely fails to identify a visible meteor.
        After a meteor is identified, the video sequence is stored and analyzed.
        Properties of the meteor are computed for every frame of the video sequence,
        such as its position on the sky, apparent velocity vector, brightness and angular speed.
        Various auxiliary data are stored as well.

        Under ideal conditions each station is able to detect approximately 20000 meteors per year.
        Multi-station observations are somewhat less frequent, with each pair of stations being able
        to correctly match about 8000 meteors every year.
        With two or more simultaneous observations from different locations it is possible to determine
        the geocentric and heliocentric velocity vectors and derive the original heliocentric orbit of the meteoroid particle.
        For a more detailed description of the methods and algorithm used refer to \citet{fero?}.

    \subsection{Stations} \label{iAs}
        As of 2020, the network comprises eleven stations at three independent locations.
        \sisetup{group-minimum-digits = 8}
        The network in Slovakia consists of five nearly-identical stations:
        \begin{description}[leftmargin = 25mm]
            \item[AGO]      Astronomical and Geophysical Observatory in Modra-Piesok\\
                            since April 2007, coordinates \ang{48.3729}~N, \ang{17.2738}~E, \SI{531}{\metre}
            \item[ARBO]     Arboretum Mlyňany of the Slovak Academy of Sciences\\
                            since September 2009, coordinates \ang{48.3235}~N, \ang{18.3685}~E, \SI{201}{\metre}
            \item[KNM]      observatory in Kysucké Nové Mesto\\
                            since August 2012, coordinates \ang{49.3073}~N, \ang{18.7655}~E, \SI{414}{\metre}
            \item[VAZEC]    village of Važec\\
                            since October 2013, coordinates \ang{49.0543}~N, \ang{19.9899}~E, \SI{812}{\metre}
            \item[SENEC]    Senec Observatory\\
                            since ???, coordinates \ang{48.2203}~N, \ang{17.3950}~E, \SI{137}{\metre} \todo{correct}
        \end{description}

        A pair of stations has been installed on the Canary Islands, with cameras at
        \begin{description}[leftmargin = 25mm]
            \item[LP]       Observatorio del Roque de los Muchachos, La Palma, Canary Islands\\
                            since March 2015, coordinates \ang{28.7600}~N, \ang{17.8823}~W, \SI{2339}{\metre}
            \item[TE]       Observatorio del Teide, Tenerife, Canary Islands\\
                            since March 2015, coordinates \ang{28.3004}~N, \ang{16.5122}~W, \SI{2416}{\metre}
        \end{description}

        The third installation consists of two stations in northern Chile, located at
        \begin{description}[leftmargin = 25mm]
            \item[SP]       San Pedro de Atacama, El Loa, Antofagasta, Chile\\
                            since March 2016, coordinates \ang{22.9534}~S, \ang{68.1793}~W, \SI{2403}{\metre}
            \item[PC]       Paniri Caur Observatory, Chiu Chiu, El Loa, Antofagasta, Chile\\
                            since March 2016, coordinates \ang{22.3360}~S, \ang{68.6442}~W, \SI{2535}{\metre}
        \end{description}

        The last subnetwork is located on Hawai`i, with stations at
        \begin{description}[leftmargin = 25mm]
            \item[HK]       Haleakalā High Altitude Observatory Site, Kula, Maui, Hawaii, USA\\
                            since March 2016, coordinates \ang{20.7075}~N, \ang{156.2560}~W, \SI{3041}{\metre} \todo{fix}
            \item[MK]       Mauna Kea Observatories, Hawai`i County, Hawaii, USA\\
                            since March 2016, coordinates \ang{19.8228}~N, \ang{155.4697}~W, \SI{4203}{\metre} \todo{fix}
        \end{description}

        \sisetup{group-minimum-digits = 5}

        Expansion of the network to more locations, namely South Africa and Australia, is planned.
        With a total of six subnetworks, it will be possible to cover both the Northern and
        the Southern hemispheres continuously, weather permitting.

\Epigraph[0.4]{
    All work and no play makes Jack a dull boy,\\
    All play and no work makes Jack a mere toy.
}{
    \textit{English proverb} \\
    \textit{enhanced by} \textsc{Maria Edgeworth} \\
    \textit{in} Harry and Lucy Concluded (1825)
}

Once we have ascertained what phbysical objects and phenomena we are going to be working with and what we want to achieve,
we need to understand \emph{how} this can be done and how these phenomena can be captured in a simplified \emph{model}.


\section{The N-body problem} \label{mN}
    At the core of investigation of meteoroid motion in interplanetary space
    there is little more than a fairly complex $N$-body problem:
    we are given the initial positions and velocities of a set of objects,
    mutually interacting by gravitational attraction.
    However, rather than finding their positions and velocities at all future times, we are mostly concerned
    with the question whether some of these particles will ever collide with the Earth.

    Finding a closed-form solution is either not possible at all \cite{...}, or is only expressible
    as a convergent series with way too many terms to be of any practical use \citep{beloriszky-1930}.

    Fortunately, an exact closed-form solution is not needed. First of all, the precision of our results
    is always practically limited by the precision with which the initial conditions are known.
    While for many known parent bodies the observation arcs are very long and allow for very precise
    calculation of Keplerian orbits,

    Furthermore, small particles are often also influenced by non-gravitational forces,
    which are themselves difficult to describe analytically.
    Numerical solutions thus remain the astronomer's best friends.

    With modern integrators and available computational power this is no longer a problem.
    Even a cheap modern computer equipped with suitable software is able to calculate trajectories
    with precision only limited by our knowledge of initial positions and velocities of the bodies.

\section{Properties} \label{mp}

    \section{Additional properties} \label{mpa}
        Apart from the dynamical properties, which describe only the density of particles
        and thus a probability of a collision at a point in space and time,
        we may (and should!) observe and analyze additional properties of the stream.

        \subsection{Mass distribution} \label{mpam}
            The most important characteristic of a meteoroid population is the distribution of masses
            among constituent particles. A shower that is primarily composed of massive particles
            will produce a large number of spectacular fireballs while its ZHR may remain fairly low;
            while another shower may consist of a large number of tiny meteoroids with low cumulative mass,
            yet will result in large ZHR.

            In meteor science, it is customary to assume a specific distribution of masses within a population,
            described by a power law with particle mass as the independent variable:
            \eqn{eq:mas-map}{
                \varrho(m) \propto m^{-s}
            }
            for some constant $s$ named the \emph{mass index}.

            To obtain a distribution, this expression needs to be normalised first,
            effectively yielding a well-known Pareto distribution \citep{arnold1983}.
            In mathematical texts it is usually defined as $\Distribution{\mathrm{Pareto}}{x}{\alpha, x_0}$,
            where $\alpha > 0$ is called \emph{shape} and $x_0 > 0$ is called the \emph{scale} of the distribution.

            To conform to terms commonly used with meteors we substitute these for mass index
            $s = \alpha - 1$, while $x_0 \equiv m_\mathrm{min}$ denotes the \emph{minimum mass} of particles in the stream.
            The probability density function is then 
            \eqn{eq:mas-pareto}{
                \varrho(m) = \mathrm{Pareto}\left(m;\ s-1, {m_\mathrm{min}}\right) \equiv
                \begin{cases}
                    0 &
                        \text{for }m < m_{\mathrm{min}}\text{,} \\
                    \dfrac{\left(s-1\right) m_{\mathrm{min}}^{s-1}}{m^s} &
                        \text{for }m \geq m_{\mathrm{min}}\text{.}
                \end{cases}
            }

            Naturally, the extent to which the model can approximate the physical reality is limited.
            Typical values of $s$ are between \numrange{1.5}{2.5}, with values slightly below 2 being the most common \cite{...}

            In a very simple approximation, the luminosity of a meteor depends on its mass and entry speed as
            \eqn{eq:mas-lum}{
                F_0 \propto mv^5.
            }

            For any meteor shower the speed is almost constant. Hence we know that the luminosity of meteors
            within a certain stream is thus solely a function of particle mass, and that this dependence should
            be approximately linear.

        \subsection{Petrological and chemical classification} \label{mpac}
            While we are not explicitly interested in chemical properties
            or dynamical evolution of meteor streams, these properties
            are correlated with the density and toughness of the material,
            which in turn affect the process of ablation
            and thus also the final luminosity of the meteor.
            All of this contributes to the unwanted selection bias.

            So, in addition to previously mentioned \textit{numerical} data, we might want to
            provide extraneous information about the meteoroids, such as their origin,
            mineralogical or chemical properties.
            For many meteoroids these data are not be available,
            though as they provide a wealth of information on origin
            it is definitely useful to include it when available.

            


            toughness of particles, which in turn affects the penetration depth
            and the absolute brightness of meteors.
            Differences in spectra are responsible for variability in detection efficiency with different cameras.


\section{Backward approach} \label{ab}


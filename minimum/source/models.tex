\Epigraph[0.4]{
    All work and no play makes Jack a dull boy,\\
    All play and no work makes Jack a mere toy.
}{
    \textit{English proverb} \\
    \textit{enhanced by} \textsc{Maria Edgeworth} \\
    \textit{in} Harry and Lucy Concluded (1825)
}

Once we have ascertained what physical objects and phenomena we are going to be working with and what we want to achieve,
we need to understand \emph{how} this can be done and how these phenomena can be described by a simplified \emph{model}.


\section{Generic properties of models} \label{g}
    Some of the quantities are fundamental 

    We can analyze 

    \subsection{Time} \label{mst}
        The first coordinate is the time of the atmospheric entry of the meteoroid.
        First of all, it should be noted that while a meteor is not an completely instantaneous event,
        its duration is on the order of \SI{1}{\second}, which is very short compared to the
        period of rotation of the Earth of the scales on which meteor activity varies.
        Precise timekeeping is thus not very important.
        For our purposes we will be using the time of maximum brightness
        of the meteor for both actual data and in simulations.

        Second, it should be emphasized again that each meteor is treated
        as a random sample from some hidden underlying distribution.
        In the context of this thesis it is immaterial whether any particular meteor
        appeared a minute earlier or later. Since we are only interested in slow variations,
        any such time shift will be invariably smeared during density estimation.
        Differences on the order of tens of minutes could be visible with very narrow filaments.

        In examining the evolution of streams we should anticipate two distinct components:
        \begin{itemize}
            \item the \emph{periodic component}, emerging as the result of the motion of the Earth
                around the Sun and encountering various streams along its orbit.
                This component can be expressed in terms of the \emph{longitude of the sun} $\lambda_\Sun$.

            \item the slowly-varying \emph{secular component}, associated with appearance of new meteoroid
                streams and gradual decay of older streams which are not replenished.
                The peaks produced by meteor showers appear abruptly after the parent body passes through the vicinity of the Earth,
                producing small meteoroid due to heating or outgassing.
                Returns of parents bodies on periodic orbits, result in resupply of meteoroid material,
                and thus form an extra periodic component, included in the secular evolution of the stream.
        \end{itemize}

        Neglecting the precession of the Earth's axis, the time coordinate $t$ may be decomposed into two components as
        \eqn{eq:iot-tyl}{
            t = y + \lambda_\Sun,
        }
        where
        \begin{itemize}
            \item $y$ is the number of whole orbits since some reference time, for instance
                the current \emph{astronomical year} in which the last vernal equinox occurred;
            \item $\lambda_\Sun$ is the \emph{solar longitude}, specified in degrees, which represents the fractional part of the year.
        \end{itemize}

        As we do not expect to cover more than several tens or hundreds of years, effects of precession can be safely ignored.
        Under this convention, August \nth{12}, 2020 roughly corresponds to $y = 2020$ and $\lambda_\Sun \approx \ang{139.4}$,
        while March \nth{14}, 2020 is denoted as $y = 2019$ and $\lambda_\Sun \approx \ang{353.6}$.


\section{Observational model} \label{m}
    The \emph{observational model} is concerned primarily with the properties of meteors,
    as opposed to the \emph{orbital model}, which focuses on the properties of meteoroids in heliocentric orbits.

    Such a model 

    the task is thus reduced to finding the distribution function for meteors
    \eqn{eq:ms-dist}{
        \varrho(\delta_\mathrm{R}, \alpha_\mathrm{R}, t, v_\infty, m)
    }

    and analyze them in greater detail:


    \subsection{Radiants} \label{msr}
        We describe the positions of the meteors

        \todo{solar longitude}
        \todo{motion of the radiant}
        \todo{explain}

    

    \subsection{Magnitude distribution} \label{msm}
        The most important characteristic of a meteoroid population is the distribution of masses
        among constituent particles. A shower that is primarily composed of massive particles
        will produce a large number of spectacular fireballs while its ZHR may remain fairly low;
        while another shower may consist of a large number of tiny meteoroids with low cumulative mass,
        yet will result in large ZHR.

       
        \subsubsection{Photometric mass} \label{msmp}
            From ground observations it is not possible to measure the true mass of the meteoroid directly.
            We may only estimate the mass using a chosen model, using some other quantities we know.
            Two commonly used measures are the \emph{photometric mass} and \emph{dynamic mass} \citep{ceplecha1966}.

            When simulating meteors, we know the original mass of the meteoroid particle.
            With the knowledge of the equations of motion, ablation and luminosity (see \cref{mpam})
            we are able to determine the luminosity of the meteor at any time.
            Photometric mass is obtained by working in the reverse direction: we can estimate the
            ablated mass from the amount of light emitted by the meteoroid during its passage through the atmosphere.

            The standard approach is to invert the equation \cref{eq:sail-???} and integrate it
            through the entire light phase,
            \eqn{eq:msm-photo}{
                \fdiff m = \Int[t_\mathrm{begin}][t_\mathrm{end}]{\frac{F(t)}{\tau v(t)^2}}{t},
            }
            where
            \begin{description}
                \item[$\fdiff m$]
                    is the total mass lost due to ablation;
                \item[$F(t)$]
                    is the instantaneous light flux, corrected for the sensitivity of the camera;
                \item[$\tau$]
                    is the luminous efficiency coefficient;
                \item[$v$]
                    is the measured speed of the meteoroid;
                \item[$t_\mathrm{begin}$]
                    is the time in which the meteor first became visible;
                \item[$t_\mathrm{end}$]
                    is the time in which the meteor stopped being visible.
            \end{description}

            Initial mass is then obtained as
            \eqn{eq:msm-orig}{
                m_\mathrm{begin} = m_\mathrm{end} + \fdiff m.
            }

            Since in most cases the meteoroid is too small and fast to survive the entry, its final mass is zero
            and photometric mass should be equal to true initial mass, at least in an idealized case where our model is perfect.


            In a very simple approximation, the luminosity of a meteor depends on its mass and entry speed as
            \eqn{eq:mpam-lum}{
                F_0 \propto mv^5.
            }

            For any meteor shower the speed is almost constant. Hence we know that the luminosity of meteors
            within a certain stream is thus solely a function of particle mass, and that this dependence should
            be approximately linear.

    \subsection{Processing the observations} \label{msp}
        \subsubsection{Limiting magnitude} \label{mspm}
            The atmosphere is constantly changing due to weather, contamination
            by natural or man-made aerosols or light pollution. The detection efficiency
            of ground-based observers is thus variable and must be included in any analysis.

            In meteor astronomy, the quality of the conditions is usually summarized
            using a single number called the \emph{limiting magnitude}, denoted $m_0$.
            It is defined as ``the apparent magnitude of the faintest stars visible during the observation'' \citep{imo-glossary}.

            In visual observations it is usually assumed that a meteor can be detected if and only if
            its apparent magnitude is at most $m_0$. Mathematically, this is described by a Heaviside step function.
            For photographic and video observations a slightly lower value is typically used.

            Rather than using a sharp limit, a slightly better approach is to assume a sigmoid profile,
            with a smooth transition between very faint meteors (none of which can be recorded)
            and very bright meteors (where it is reasonable to assume they cannot be missed).
            While there is no universally agreed-upon form, a simple yet reasonable parameterised function is
            \eqn{eq:mspm-sigmoid}{
                D(m; m_0, \omega) = \frac{1}{1 + \exp\left(\frac{m-m_0}{\omega}\right)},
            }
            where
            \begin{description}
                \item[$m_0$]    is the \textit{limiting magnitude} where detection probability is \SI{50}{\percent};
                \item[$\omega$] is the width of the function, with higher values representing
                    faster rate of loss of detection efficiency with decreasing brightness.
                    The limit $\omega \to \infty$ corresponds to a sharp cut-off at $m_0$,
                    or the Heaviside step function.
            \end{description}

            Ideally, a functional dependence on the distance from the centre of the field
            of view should be taken into account for human observers, as the eye is much more
            sensitive to light near the yellow spot \todo{who?}.
            Similarly, digital sensors are subject to vignetting and other detrimental effects.
            Further research into the angular response of the human eye would be beneficial.

        \subsubsection{Zenithal hourly rate} \label{mspz}
            The \emph{zenithal hourly rate} (ZHR) is defined as ``[t]he number of shower meteors per hour
            one observer would see if his limiting magnitude is \Mag{+6.5} and the radiant is in his zenith''.
            While it is not an universally valid figure of merit for comparing meteor activity of various showers,
            ZHR fairly simple and widely used as a standard in meteor science.
            Its primary design consideration is to compare the activity of various showers as viewed by human observers,
            but it is perfectly applicable to automatic devices as well.

            ZHR is defined as
            \eqn{eq:ipqz-zhr}{
                \mathrm{ZHR} = \dfrac{\dfrac{N}{T} \cdot \dfrac{1}{k} \cdot r^{\num{6.5} - m_0}}{\sin{\theta_\mathrm{R}}}
            }
            where
            \begin{description}
                \item[$N$]
                    is the number of meteors recorded by the observer;
                \item[$T$]
                    is total effective observation time;
                \item[$k$]
                    is the fraction of the sky that is visible to the observer;
                \item[$r$]
                    is the \emph{population index} of the observed meteor shower;
                \item[$m_0$]
                    is the \emph{limiting magnitude} of the observer;
                \item[$\theta_\mathrm{R}$]
                    is the altitude of the radiant above the horizon.
            \end{description}
            In this definition the limiting magnitude is considered to be a sharp boundary, e. g. all meteors

            In our model the ZHR is not an input, but rather a computed quantity, completely determined by the
            models of the meteoroid population, atmosphere and of the observer's detection efficiency.

        \subsubsection{Population index} \label{mspr}
            Another useful measure of meteor activity is the \emph{population index}.
            It is similar to the mass index (see \cref{msm}) in the sense.
            The definition by \citet{molau2015} is "[the population index] represents the [relative] increase
            in total meteor count when the limiting magnitude $\mathrm{lm}$ improves by one mag[nitude]."

            So if we denote the total number of observed meteors with magnitude at most $\mu$ as $N(\mu)$
            and compare the counts for different values of $\mu$ and $\mu + 1$, we need $N$ to satisfy the requirement
            \eqn{eq:mspr-cdf}{
                r \cdot N(\mu) = N(\mu + 1).
            }

            Solving this for function $N(\mu)$ yields a single family of solutions
            in the form
            \eqn{eq:mspr-cdfsolved}{
                N(\mu) = k \cdot r^\mu
            }
            for some real constant $k$.

            As $N(\mu)$ is -- by definition -- the kernel of the
            cumulative distribution function (CDF) for apparent magnitudes, we can differentiate it
            to obtain the kernel of the probability density function. The kernel of the PDF is then
            \eqn{eq:mspr-pdfk}{
                n(\mu) \propto k \cdot r^\mu \ln r.
            }

            The value or $r$ is assumed to be constant throughout the entire observable magnitude range.%
            \footnote{Without this assumption the population index loses its main purpose.
            Technically, \emph{any} smooth distribution of magnitudes can be described in terms of varying $r$.}

            Note that this function is not a power law, but an exponential (the variable $r$
            is in the base, not the exponent, unlike with mass index $s$).

            To obtain a probability density function, we must find a norming constant
            and set an upper limit on magnitudes. The upper limit is conveniently set as the limiting
            magnitude $m_0$. If we require the integral over the entire visible range to evaluate to one,
            the value of $k$ is unambiguously determined as $k = r^{-m_0}$.
            The PDF is then
            \eqn{eq:mspr-pdf}{
                F_m(m) =
                \begin{cases}
                    r^{\,\mu - m_0} \ln r &
                        \text{for }m \leq m_0\text{,} \\
                    0 &
                        \text{for }m > m_0\text{.}
                \end{cases}
            }

            It should be kept in mind that the mass and population indices are not fundamental physical quantities,
            but only parameters of simple model distribution functions for masses and apparent magnitudes respectively.
            Also, value of $r$ can be measured correctly only when assuming there is a sharp limit of the observer's detection efficiency.
            For sigmoid detection efficiency profiles (such as the one defined in \cref{eq:mspm-sigmoid}) it
            is is not correct.



    Another method has been discussed and implemented in the author's master's thesis \citep{balaz-thesis}
    and used to determine the total flux of the Perseids in 2016. Instead of trying to
    estimate the biases we generated the meteoroids above the atmosphere and simulated
    their entry using a simplified set of equations for motion, ablation and luminosity.
    A collection of stochastic bias functions was then applied to the dataset and
    each meteor was marked as detected or missed. We varied the parameters of the bias functions
    and the mass index until agreement with observational data was achieved.
    The initial population was then declared as the model of the actual population
    and the bias functions were taken as descriptive of the observing system.

    The values of mass index $s$ is an input to the model while the value of $r$
    can be determined from the output.


\section{Simulation in the atmosphere} \label{ma}
    Once it has been determined which meteoroids will enter the Earth's atmosphere, we may proceed to simulating
    the physical processes manifested during their atmospheric entry and obtain the values of physical quantities
    that are observed by ground-based observers.

    The coordinate frame of choice here is the Earth-Centered, Earth-Fixed frame (origin in the centre of mass of the Earth,
    reference plane is the equator and the reference direction is the international prime meridian).
    This reference frame is right-handed

    First of all, the position and velocity of the particle are transformed to the ECEF coordinate frame.

    To simulate the atmospheric entry we used \textsc{Asmodeus},
    a multi-purpose virtual meteor simulator developed as a part of our master's thesis and extended for numerous
    other purposes related to meteor astronomy \citep{balaz-thesis,balaz+2020}.

    Once a particle is selected for atmospheric entry, its velocity is transformed to the ECEF reference frame
    and a numerical integration of the equations of motion is executed.

    \subsection{Numerical integration of the equations of motion} \label{sai}
        Unlike in interplanetary space, where forces are completely dominated by the gravitational force exerted by the Sun,
        a particle entering the Earth's atmosphere switches between several regimes in rapid succession.
        The precise order is subject to variations due to particle's size, entry speed and angle
        and material composition.



        \subsubsection{Acting forces} \label{saia}
            In the simulation we consider only two real forces:
            \begin{itemize}
                \item the \emph{drag force} $\vec{F_d}$, always acting against the instantaneous velocity vector of the particle;
                \item the \emph{gravitational force} $\vec{F_g}$, pulling the meteoroid towards the centre of the Earth.
            \end{itemize}

            The precision of the calculations may be improved if we also account for two fictitious forces,
            arising from the fact the simulation is performed in a rotating reference frame:
            \begin{itemize}
                \item the fictitious \emph{centrifugal force} $\vec{F_{\mathrm{C}}}$,
                    pushing the particle away from the axis of rotation of the Earth;
                \item and the fictitious \emph{Coriolis force} $\vec{F_{\mathrm{c}}}$,
                    which pushes the moving body in a direction perpendicular to its velocity vector.
            \end{itemize}

            Except for the drag force, all forces can be readily expressed in terms of instantaneous
            position and velocity of the meteoroid and several constants.
            Calculating the drag force precisely is computationally very expensive and cannot be done
            without knowing the shape of the particle. Therefore we have to choose an appropriate simplified model.

            In its cosmic stage and during most of the motion in the gravitational sphere of influence of the Earth,
            the forces acting on the particle are dominated by gravitational effects. As the meteoroid penetrates
            deeper into the atmosphere, the density of the surrounding gas increases roughly exponentially.
            At its peak, the atmospheric drag force exceeds all other effects by several orders of magnitude,
            decelerating the meteoroid very rapidly.
            As the duration of the entire event is on the order of seconds, other forces simply do not act for long enough
            to produce a significant deviation in trajectory. For statistical evaluation of artificial meteors
            it is sufficient to consider only the drag force.
            However, should the meteoroid survive the entry as a meteorite, their influence is important during the dark phase of flight.

            \paragraph{Coriolis force} \label{saiC}
                The fictitious Coriolis force arises in a rotating reference frame.
                \eqn{eq:saiC-coriolis}{
                    \vec{F_{\mathrm{C}}} = -2 m \vec{\Omega}_\Earth \times \vec{v}.
                }

                For a meteoroid, its magnitude is at most on the order of
                \eqn{eq:saiC-order}{
                    2 \frac{2\pi}{\SI{86400}{\second}} \cdot \SI{70}{\kilo\metre\per\second} \sim \SI{10}{\metre\per\second\squared},
                }
                roughly the same as gravitational acceleration.

            \paragraph{Centrifugal force} \label{saic}
                The centrifugal force pushes the particles away from the axis of rotation of the Earth.
                \eqn{eq:saic-centrifugal}{
                    \vec{F_{\mathrm{c}}} = - m \vec{\Omega}_\Earth \times \left(\vec{\Omega}_\Earth \times \vec{r}\right),
                }
                where $\vec{\Omega}_\Earth$ is the angular speed of rotation of the Earth,
                $\frac{2\pi}{\SI{86164}{\second}} \approx \SI{7.292e-5}{\per\second}$.

                Since it only depends on the instantaneous position, it can be thought of as a minor correction to the
                gravitational force. A simple calculation shows that its magnitude is on the order
                of \SI{0.03}{\metre\per\second\squared}. Near the surface it can be safely neglected for most purposes.
                It only becomes important when the distance from the axis of rotation is large, for instance if the simulation
                is used to investigate the motion of a meteoroid further from the Earth.
                However, in this case an inertial reference frame should be used instead.

            \paragraph{Gravitational force} \label{saig}
                The real shape of the gravitational potential around the Earth is fairly complex and difficult to describe accurately.
                For particles that are generally moving on highly hyperbolic trajectories the
                simple approximation by a Newtonian point source with mass $M_\Earth$ is more than sufficient.
                The formula used in computation is then simply
                \eqn{eq:saig-gravity}{
                    \vec{F_{\mathrm{G}}} = - \frac{GM_\Earth}{r^3} \vec{r}.
                }

            \paragraph{Drag force} \label{said}
                The precise description of the drag force is crucial to understand the motion
                of meteoroids in the atmosphere. The drag force is dominant between approximately
                the point where meteoroids begin to emit visible light (about \SI{100}{\kilo\metre})
                until the very end of the visible trail. For all but the most massive meteoroids,
                which are able to survive the atmospheric entry, this can be approximated
                by the point where all of the matter has been ablated away; while for the bodies
                surviving as meteorites, during the terminal phase of the flight the drag and
                gravitational forces are nearly balanced out.

                At peak deceleration the magnitude of the drag force can reach values several orders of magnitude
                higher than the sum of all other forces and is thus completely dominant.
                A very simple model of atmospheric flight can thus perform relatively
                well even when all other forces are neglected.

    \subsection{Determination of luminosity} \label{sail}
        Currently we use... \citep{hill+2005} \todo{rmeasureeally?}

        \citep{bronshten1983}
        but more precise models should be developed


\section{Orbital model} \label{mo}
    While the observational model may capture the observed distribution of meteors quite precisely,
    it does not provide any deeper understanding about \textit{why} meteors appear
    at some specified time and from a particular region in the sky.
    To address that, we need to analyze the observed meteoroids and reconstruct their original trajectories.

    For each meteor, it is in principle possible to trace its visible motion back in time
    and derive the trajectory of the original meteoroid in the vicinity of the Earth.
    After correcting for gravitational attraction and subtracting the velocity of the Earth
    we can further extend the trajectory into interplanetary space.

    A fundamental difficulty here is that particle streams whose orbit does not intersect
    the orbit of the Earth cannot be observed at all and thus are not included in the model.
    Determination of flux in these areas is not possible with Earth-based observations at all
    and can only be investigated with in-situ observations. In this thesis we will not analyze them any further.

    A less fundamental difficulty lies in the low precision of these measurements, particularly of meteor speeds.
    The position of the projection of the meteor in the sky can be determined with high precision:
    the resolution of a photograph or a video sequence is typically on the order of \ang{;1;}
    while the distance between the observer and the meteor is typically several hundred kilometres.
    This results in an absolute precision on the order of tens of metres.

    This is particularly important when the convergence angle is small \citep{ceplecha1987} or when the angle between
    the line of sight and the velocity vector of the meteor is small.
    Furthermore, while the trajectory can be computed with very high precision in vacuum,
    meteoroids lose a fraction of their original preatmospheric speed before they become visible \citep{vida+2018}.
    This further adds to the uncertainty of the original orbit.
    One very pronounced misleading effect is the vastly overestimated number of objects
    on interstellar heliocentric orbits in meteor databases \citep{hajdukovajr1994}.

    The natural reference system to work with when investigating meteoroid streams in interplanetary space
    is the \emph{ecliptic coordinate system}.

    \subsection{Distribution of physical quantities} \label{mod}
        The number density distribution function in interplanetary space is simply
        \eqn{eq:po-rv}{
            M(\vec{r}, \vec{v}, t).
        }
        where
        \begin{description}
            \item[$\vec{r}$]    is the instantaneous position in a heliocentric inertial reference frame;
            \item[$\vec{v}$]    is the velocity in the same frame;
            \item[$t$]          is the time;
            \item[$\varrho$]    is the number density of particles.
        \end{description}

        Alternatively this may be expressed in terms of orbital elements as
        \eqn{eq:po-elem}{
            M_\mathrm{oe}(a, e, i, \omega, \Omega, T).
        }

        While particles that collide with the Earth are removed from the stream,
        the total volume swept by the Earth through the stream is negligible for all but the narrowest streams.
        We do not expect any observable reduction in the number of meteoroids during subsequent approaches.

        \subsection{Phase space coordinates} \label{modc}
            The distribution of meteoroid masses of our interest spans six or seven decimal orders of magnitude,
            with most of the samples concentrated clositemsepitemsepe to the lower bound of the interval.
            The upper portion of the mass interval is populated only very sparsely.
            To address this, we may work in logarithmic space instead, using $\log_{10} m$ as the measure of mass.
            The distribution of masses will still be shifted towards lower values, but not so much.

            The same approach is applicable to other quantities, most importantly the semi-major axis $a$.
            Although the region of our interest is located very close to \SI{1}{\au},
            orbits of meteoroids originating from long-period comets have very large $a$.
            The gravitational influence of the Sun is weaker at these distances and
            even a small change in velocity near the perihelion may alter $a$ significantly.

            Therefore we can use either $1/a$ as the measure of orbit size instead,
            and use a Gaussian kernel in estimation of this new coordinate.

            The entire phase space is then
            \eqn{eq:modc-phasespace}{
                M(x, y, z, v_x, v_y, v_z, t, m) \qquad\text{or}\qquad M(1/a, e, i, \omega, \Omega, T, t, \log m).
            }

    \subsection{Physical quantities and properties} \label{mp}
        \subsubsection{Fluxes and densities} \label{msmf}
            The primary measure of activity of a meteor shower (or the sporadic background) is the number
            of meteors encountered per unit area of the surface of the Earth (which is more suitable
            in observational models), or per unit volume in interplanetary space (for orbital models).

            As meteoroids are discrete particles of finite size, it is easier to work with the
            assumed underlying distribution function, so as to avoid artifacts.

            Spatial number density is obtained from the model by integrating over all parameters
            in the space except the spatial ones:
            \eqn{eq:msmf-spatial}{
                \varrho(\vec{r}, t) = \Int[m_0][\infty]{
                    \Int[\Real^3][]{M(\vec{r}, \vec{v}, t, m)}{\vec{v}}
                }{m}
            }

            On the surface of the Earth, or in its close vicinity, we are more concerned with
            the number particles passing through an unit area of the surface.
            To obtain the flux
            \eqn{eq:msmf-flux}{
            }

            Similarly, to determine the total mass influx from a particular population, a simple weighed sum is calculated.
            The weights used are meteoroid masses and \cref{eq:msmf-flux} becomes
            \eqn{eq:msmf-wflux}{
                F(\vec{r}, t) = \Int[m_0][\infty]{
                    \Int[\Real^3][]{M(\vec{r}, \vec{v}, t, m)}{\vec{v}}
                }{m}
            }

        \subsubsection{Mass index and the Pareto distribution} \label{msms}
            While theoretically any distribution of particle masses is possible,

            In meteor science, it is customary to assume a specific distribution of masses within a population,
            described by a power law with particle mass as the independent variable:
            \eqn{eq:mpam-sindex}{
                \varrho(m) \propto m^{-s}
            }
            for some constant $s$ named the \emph{mass index}.

            To obtain a distribution, this expression needs to be normalised first.
            Furthermore, the expression $m^{-s}$ diverges to infinity for $m \to 0$,
            which can be dealt with by setting a lower limit on the mass.
            These requirements are satisfied by the well-known Pareto distribution \citep{arnold1983}.
            In mathematical texts it is usually defined as $\Distribution{\mathrm{Pareto}}{x}{\alpha, x_0}$,
            where $\alpha > 0$ is called \emph{shape} and $x_0 > 0$ is called the \emph{scale} of the distribution.

            To conform with commonly used terminology in meteor science, we substitute the shape with mass index
            $s = \alpha - 1$, and rename $x_0$ to \emph{minimum mass} of particles, $m_\mathrm{min}$.
            The probability density function is then
            \eqn{eq:msms-pareto}{
                \varrho(m) = \mathrm{Pareto}\left(m;\ s-1, {m_\mathrm{min}}\right) \equiv
                \begin{cases}
                    0 &
                        \text{for }m < m_{\mathrm{min}}\text{,} \\
                    \dfrac{\left(s - 1\right) m_{\mathrm{min}}^{s - 1}}{m^s} &
                        \text{for }m \geq m_{\mathrm{min}}\text{.}
                \end{cases}
            }

            Naturally, the extent to which the model can approximate the physical reality is limited.
            Typical values of $s$ are between \numrange{1.5}{2.5}, with values slightly below 2 being most common \cite{...}

        \subsubsection{Additional properties} \label{mpa}
            Apart from the dynamical properties, which describe only the density of particles
            and thus a probability of a collision at a point in space and time,
            we may (and should!) observe and analyze additional properties of the stream.

            \subsection{Petrological and chemical classification} \label{mpac}
                While we are not explicitly interested in chemical properties
                or dynamical evolution of meteor streams, these properties
                are correlated with the density and toughness of the material,
                which in turn affect the process of ablation
                and thus also the final luminosity of the meteor.
                All of this contributes to the unwanted selection bias.

                So, in addition to previously mentioned \textit{numerical} data, we might want to
                provide extraneous information about the meteoroids, such as their origin,
                mineralogical or chemical properties.
                For many meteoroids these data are not be available,
                though as they provide a wealth of information on origin
                it is definitely useful to include it when available.


                toughness of particles, which in turn affects the penetration depth
                and the absolute brightness of meteors.
                Differences in spectra are responsible for variability in detection efficiency with different cameras.


\section{Simulation in the Solar System} \label{ms}
    Lots of work has already been done in this area, such as
    \begin{itemize}
        \item studies of dynamic paths of asteroids and meteoroids
        \item studies of evolution of meteoroids streams from the parent body until collision with the Earth
        \item predictions of activity of meteor showers.
    \end{itemize}

    The simulation computes the positions of the large bodies of the Solar System, most importantly the Sun,
    the eight planets and the Moon. The obtained values may be easily verified by comparing to known databases
    or ephemeris systems.
    Then the test particles are created, each representing a single meteoroid body.

    Optionally, calculating the positions of massless bodies can be offloaded to a Graphics Processing Unit (GPU).
    A GPU is particularly well suited to solving massively parallel problems,
    such as simulation of a large number non-interacting test particles.

    \subsection{The N-body problem} \label{msN}
        At the core of investigation of meteoroid motion in interplanetary space
        there is little more than a fairly complex $N$-body problem:
        we are given the initial positions and velocities of a set of objects,
        mutually interacting by gravitational attraction.
        However, rather than finding their positions and velocities at all future times, we are mostly concerned
        with the question whether some of these particles will ever collide with the Earth.

        Finding a closed-form solution is either not possible at all \cite{...}, or is only expressible
        as a convergent series with way too many terms to be of any practical use \citep{beloriszky-1930}.

        Fortunately, an exact closed-form solution is not needed. First of all, the precision of our results
        is always practically limited by the precision with which the initial conditions are known,
        which is always finite.

        Furthermore, small particles are often also influenced by non-gravitational forces,
        which are themselves difficult to describe analytically.
        Numerical solutions thus remain the astronomer's best friends.

        With modern integrators and available computational power this is no longer a problem.
        Even a cheap modern computer equipped with suitable software is able to calculate trajectories
        with precision only limited by our knowledge of initial positions and velocities of the bodies.


    Any particle intersecting the cross-section of the Earth is marked as detected and will be examined further.
    Alternatively, in order to reduce the number of trials, we may employ a larger artificial detector.


    As long as the spatial density is only varying on scales much larger than the radius of the Earth,
    temporal variations in activity will be slow.
    This assumption is not universally valid, as some younger meteor showers often produce dense, narrow filaments,
    which can be observed as short but intensive outbursts of extraordinary meteor activity.
    A~good example are the November alpha-Monocerotids, with a period of significantly
    heightened activity spanning only about thirty minutes in 2019 \citep{CBET4692}.

    By comparing the orbital speed and the diameter of the Earth we can see that the Earth moves
    to a completely different position with respect to the stream within several minutes.
    This means that variations in activity are not going to be significant when crossing an old stream,
    whose radius may be on the order of millions of kilometres, but will be of extreme importance with young, narrow filaments.

    The duration of visible activity also depends on the exact shape of the stream and its orientation with respect to the orbit of the Earth.
    The worst case are the streams with high inclinations, where meteors seem to arrive from a direction perpendicular
    to the direction of movement.



    It is necessary that at least one of the orbital nodes must lie very close to the orbit of the Earth.
    This can be expressed in terms of orbital elements as
    \eqn{eq:msnodes}{
        \frac{a\left(1 - e^2\right)}{1 \pm e \cos \omega} \approx \SI{1}{\au}
    }


    \subsection{Numerical integration of equations of motion} \label{msi}
        The problem of obtaining a numerical solution to the equations of motion is omnipresent
        in astronomy and as such has been solved many times with increasingly better precision.

        Since every two bodies attract each other, finding all forces acting on the bodies generally
        requires computation of $n\left(n - 1\right) = \BigO{n^2}$ pairs of forces.
        According to Newton's third law opposing forces are equal in magnitude and opposite in their direction,
        and thus one evaluation in every pair is enough.
        This brings the number of required evaluations to half the stated value, without changing the order.
        Further optimizations are not possible in general.

\section{Mathematical foundations} \label{mm}

    The number of parameters encountered and the complexity of the model essentially preclude the use of any parametric models.

    \subsection{Kernel density estimation} \label{mmk}
        Kernel density estimation (KDE) is a non-parametric method of estimating the probability density function
        of a distribution.
        is particularly useful when the underlying distribution is not known or very complicated.

        A comprehensive mathematical description can be found in \citep{hwang+1994}.

        \subsubsection{Kernel} \label{mmkk}
            Surprisingly, the choice of the kernel is not particularly important.
            however, it should be chosen to satisfy two constraints: it must be
            \emph{non-negative} over the entire domain
            \eqn{eq:mmkk-nonneg}{
                \forall \vec{y} \in \Real^d : K(\vec{y}) \geq 0,
            }
            and it should be \emph{normalized} to unity
            \eqn{eq:mmkk-norm}{
                \Int[\Real^d]{K(\vec{y})}{\vec{y}} = 1.
            }
            An optional third constraint is that the kernel should be a \emph{symmetric function},
            \eqn{eq:mmkk-sym}{
                \forall \vec{y} \in \Real^d : K(\vec{y}) = K(-\vec{y}).
            }

            One important consideration is that the explored domain need not be
            a $d$-dimensional Euclidean space. With meteors, two of the coordinates -- declination
            and right ascension of the radiant -- are in fact spherical coordinates.
            In these cases, a different distance metric should be used to determine the true shape of the kernel.

        \subsubsection{Bandwidth} \label{mmkw}
            The most important consideration in KDE is the determination of correct bandwidth.
            When the selected bandwidth is too high, the resulting estimate of the distribution
            becomes very wide and in the limit $h \to \inf$ becomes identical to the original kernel.
            If it is too low, individual kernels do not overlap and in the limit $\left|\vec{H}\right| \to 0$
            a sum of $n$ $\delta$-functions is obtained. Both extremes thus do not provide any useful information.

            Several algorithms for determining the optimal bandwidth exist \citep{bowman1985,jones+1996},
            with larger datasets universally requiring narrower bandwidths.
            A more advanced method is to try to optimize the bandwidth by minimizing the
            \emph{asymptotic mean integral square error} using a synthetic dataset and
            then applying the determined kernel bandwidth to the real dataset.

        \subsubsection{Adaptive kernel density estimation} \label{mmka}
            One possible improvement to the fixed-width KDE is to allow some variation in the bandwidth
            of the kernel in different regions of the domain in some relation to the data.
            Typically we would require a narrower bandwidth where data are abundant,
            such as in meteor showers, and allow a wider bandwidth in areas where data points are sparse.


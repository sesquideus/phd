\Epigraph[0.4]{
    All work and no play makes Jack a dull boy,\\
    All play and no work makes Jack a mere toy.
}{
    \textit{English proverb} \\
    \textit{enhanced by} \textsc{Maria Edgeworth} \\
    \textit{in} Harry and Lucy Concluded (1825)
}

Once we have ascertained what physical objects and phenomena we are going to be working with and what we want to achieve,
we need to understand \emph{how} this can be done and how these phenomena can be described in a simplified \emph{model}.

Since it is not possible to observe meteoroids directly in interplanetary space,
all models obtained from ground-based observations can only describe the close
vicinity of the Earth and the meteor showers which happen to cross its orbit.
Therefore it might be a good idea to produce two kinds of models, each valid within its own domain:
\begin{itemize}
    \item an \emph{observational model}, which describes meteor activity as observed from the Earth,
    \item and an \emph{orbital model}, which represents the density of meteoroids in the interplanetary space.
        These particles can, under right conditions, produce meteors that are described by the observational model.
\end{itemize}

The observational model can be directly related to actual observations, obtained by ground-based
devices placed on various locations around the Earth. The relatively precision of these measurements puts
tight constraints on total number of meteors recorded. Meteor networks are already numerous
and cover the sky practically continuously, at least as long as the weather is favourable.
Such sky maps will be useful in predicting the activity of various meteor showers and of the sporadic background.

On the other side, the orbital model provides understanding of \emph{why} there are meteoroids
encountered at certain point of space and time. Detailed maps of meteoroid density in the past,
and even more so predictions of encounters with meteoroids in the future, are very important
for space agencies operating satellites.

Another simple yet deep distinction is in the fact that the observational model is concerned
primarily with the properties of \emph{meteors} seen from the surface, while the orbital model focuses
on the properties of \emph{meteoroids}, which are still in heliocentric orbits and need not necessarily produce meteors.


% For a presentation we should produce a table:
% model         observational       orbital
% scope         sky                 interplanetary space
% objects       meteors             meteoroids
% validity      high                lower
% etc

\section{Observational model} \label{ms}
    \todo{gentle intro}

    No observational model is able to capture or describe any stream that does not intersect the orbit of the
    Earth.\footnote{This condition is by no means sufficient. In certain pathological cases, such as a fresh stream in resonance
    with the Earth, which has not yet fully spread along its orbit, it is possible no meteors will collide with the Earth even
    if their orbit intersects with the Earth's.}

    The observational model can be \emph{complete}, in the sense that -- at least in principle -- it
    is possible to record all meteors in the Earth's atmosphere and include them in the model.
    As long as there is sufficient coverage of the sky by networks of ground-based stations,
    no statistically significant grouping of meteors should remain unaccounted for.

    \subsection{Physical quantities in the model} \label{msp}
        The design process of any model in astronomy -- or physics in general -- should begin with the analysis
        of physical quantities available for scrutiny. Then we should decide which ones are important enough
        to be included in the model and which are not.

        \subsubsection{Time} \label{mspt}
            The first coordinate is the time of the atmospheric entry of the meteoroid.
            First of all, it should be noted that while a meteor is not an completely instantaneous event,
            its duration is on the order of \SI{1}{\second}, which is very short compared to the
            period of rotation of the Earth of the scales on which meteor activity varies.
            Precise timekeeping is thus not very important.
            For our purposes we will be using the time of maximum brightness
            of the meteor for both actual data and in simulations.

            Second, it should be emphasized again that each meteor is treated
            as a random sample from some hidden underlying distribution.
            In the context of this thesis it is immaterial whether any particular meteor
            appeared a minute earlier or later. Since we are only interested in slow variations,
            any such time shift will be invariably smeared during density estimation.
            Differences on the order of tens of minutes could be visible with very narrow filaments.

            In examining the evolution of streams we should anticipate two distinct components:
            \begin{itemize}
                \item the \emph{periodic component}, emerging as the result of the motion of the Earth
                    around the Sun and encountering various streams along its orbit.
                    This component can be expressed in terms of the \emph{longitude of the sun} $\lambda_\Sun$.

                \item the slowly-varying \emph{secular component}, associated with appearance of new meteoroid
                    streams and gradual decay of older streams which are not replenished.
                    The peaks produced by meteor showers appear abruptly after the parent body passes through the vicinity of the Earth,
                    producing small meteoroid due to heating or outgassing.
                    Returns of parents bodies on periodic orbits, result in resupply of meteoroid material,
                    and thus form an extra periodic component, included in the secular evolution of the stream.
            \end{itemize}

            Neglecting the precession of the Earth's axis, the time coordinate $t$ may be decomposed into two components as
            \eqn{eq:iot-tyl}{
                t = y + \lambda_\Sun,
            }
            where
            \begin{itemize}
                \item $y$ is the number of whole orbits since some reference time, for instance
                    the current \emph{astronomical year} in which the last vernal equinox occurred;
                \item $\lambda_\Sun$ is the \emph{solar longitude}, specified in degrees, which represents the fractional part of the year.
            \end{itemize}

            As we do not expect to cover more than several tens or hundreds of years, effects of precession can be safely ignored.
            Under this convention, August \nth{12}, 2020 roughly corresponds to $y = 2020$ and $\lambda_\Sun \approx \ang{139.4}$,
            while March \nth{14}, 2020 is denoted as $y = 2019$ and $\lambda_\Sun \approx \ang{353.6}$.

        \subsubsection{Position} \label{mspp}
            While position in space it is a fundamental physical quantity,
            it is not a very useful in the observational model.
            On a large scale it is determined by the position of the Earth at the particular time,
            as only particles that enter the atmosphere can be included in the model.
            On the local scale it does not provide much information: most meteoroid streams
            are much wider than the diameter of the Earth and within them,
            we may treat exact position of meteoroids entry as essentially random.
            Even a very small change in the orbital elements of one meteoroid
            would cause it to enter the atmosphere at a different place,
            and the same observer would see it at a vastly difference place in the sky or not at all.

        \subsubsection{Velocity} \label{mspv}
            Another important measurable quantity is the velocity of the meteor.
            Unlike with positions, a very wide range of geocentric velocities is available.
            When combined with the position, which is already known, it completely determines the geocentric
            orbit of the particle, which can in turn be extrapolated into the interplanetary space
            to obtain the original heliocentric orbit.
            This is also the basis for criteria for determination of adherence to streams,
            such as the $D$-criterion defined by \citet{southworth+1963}.

            The meteor's trajectory is approximated by a straight line and extended to infinity.
            The point where this projection intersects the celestial sphere is called the \emph{radiant} of the meteor.
            Unlike the observed position of the meteor, the position of the radiant is not dependent on
            the position of the observer.

            Clustering of meteors based on their radiants and is the basis of determination of
            adherence of meteors to meteor showers: particles on nearly the same trajectory intersecting
            the Earth at the same time will always have (approximately) the same radiant.
            The equality is not perfect due to slight dispersion of orbits,
            and due to the fact that the radiants slowly move across the sky,
            as the geometry of the stream changes with respect to the Earth's orbit.

            Hence we obtain our first three coordinates: the components of the velocity vector $v_x$, $v_y$ and $v_z$
            or, more conveniently, the declination and right ascension of the radiant $\delta_R$ and $\alpha_R$,
            along with the pre-atmospheric speed of the meteoroid $v_\infty$.

            An unwelcome difficulty is encountered with daytime meteor showers, whose radiants
            lie very close to the position of the Sun at the time when they are active.
            Most particles will then enter the atmosphere on the day side
            of the Earth, where most particles inevitably burn up undetected.
            With radar observations it is possible to track the activity as well,
            however this observation technique calls for different methods of processing and debiasing the data.
            We estimate that using purely visual data it is impossible (or at least very difficult) to establish fluxes
            for meteor showers whose radiants are within \ang{15} from the Sun, and significant bias
            is introduced for angles up to \ang{60}.

        \subsubsection{Magnitude distribution} \label{mspm}
            The most important characteristic of a meteoroid population is the distribution of masses
            among constituent particles. A shower that is primarily composed of massive particles
            will produce a large number of spectacular fireballs while its ZHR may remain fairly low;
            while another shower may consist of a large number of tiny meteoroids with low cumulative mass,
            yet will result in large ZHR.

            However, in the observational model it is not possible to measure meteoroid mass directly.
            A much more readily measurable quantity is the apparent magnitude of the meteor, which,
            with the knowledge of the true distance and atmospheric effects, can be readily converted
            to absolute magnitude.

            The relation is not linear and is heavily affected by the meteor speed; with other properties,
            such as the entry angle or chemical composition, having some influence as well.

        \subsubsection{Photometric mass} \label{mspf}
            Although we cannot measure the true mass of the meteoroid directly,
            we may use some of the known quantities to estimate it.
            Two commonly used measures are the \emph{photometric mass} and \emph{dynamic mass} \citep{ceplecha1966}.

            When simulating meteors, we know the original mass of the meteoroid particle.
            With the knowledge of the equations of motion, ablation and luminosity
            we are able to determine the luminosity of the meteor at any time.
            One of the simplest models \citep{hill+2005} assumes that the energy emitted as visible light
            is proportional to change in particle's kinetic energy,
            \eqn{eq:mspf-tau}{
                F_0 = -\tau \frac{\dot{m} v^2}{2}
            }
            for some constant $\tau$ named \emph{luminous efficiency factor}.

            Photometric mass is obtained by working in the reverse direction: we can estimate the
            ablated mass from the amount of light emitted by the meteoroid during its passage through the atmosphere.

            The standard approach is to invert the equation \cref{eq:mspf-tau} and integrate it
            through the entire light phase,
            \eqn{eq:mspf-photo}{
                \fdiff m = \Int[t_\mathrm{begin}][t_\mathrm{end}]{\frac{F(t)}{\tau v(t)^2}}{t},
            }
            where
            \begin{description}
                \item[$\fdiff m$]
                    is the total mass lost due to ablation;
                \item[$F(t)$]
                    is the instantaneous light flux, corrected for the sensitivity of the camera;
                \item[$\tau$]
                    is the luminous efficiency coefficient;
                \item[$v$]
                    is the measured speed of the meteoroid;
                \item[$t_\mathrm{begin}$]
                    is the time in which the meteor first became visible;
                \item[$t_\mathrm{end}$]
                    is the time in which the meteor stopped being visible.
            \end{description}

            Initial mass is then obtained as
            \eqn{eq:mspf-orig}{
                m_\mathrm{begin} = m_\mathrm{end} + \fdiff m.
            }

            Since in most cases the meteoroid is too small and fast to survive the entry, its final mass is zero
            and photometric mass should be equal to true initial mass, at least in an idealized case where our model is perfect.
            The concept of dynamic mass is similar in nature, except that instead of the luminosity,
            the recorded deceleration of the particle is analyzed and fitted.

    \subsection{Additional concepts} \label{msa}
        In addition to quantities that are directly described in the model, several other concepts are useful
        in its construction and verification. Many derived quantities are used in meteor astronomy,
        especially in relation to more fundamental quantities that are difficult to measure.

        \subsubsection{Limiting magnitude} \label{msam}
            The atmosphere is constantly changing due to weather, contamination
            by natural or man-made aerosols or light pollution. The detection efficiency
            of ground-based observers is thus variable and must be included in any analysis.

            In meteor astronomy, the quality of the conditions is usually summarized
            using a single number called the \emph{limiting magnitude}, denoted $m_0$.
            It is defined as ``the apparent magnitude of the faintest stars visible during the observation'' \citep{imo-glossary}.

            In visual observations it is usually assumed that a meteor can be detected if and only if
            its apparent magnitude is at most $m_0$. Mathematically, this is described by a Heaviside step function.
            For photographic and video observations a slightly lower value is typically used.

            Rather than using a sharp limit, a slightly better approach is to assume a sigmoid profile,
            with a smooth transition between very faint meteors (none of which can be recorded)
            and very bright meteors (where it is reasonable to assume they cannot be missed).
            While there is no universally agreed-upon form, a simple yet reasonable parameterised function is
            \eqn{eq:msam-sigmoid}{
                D(m; m_0, \omega) = \frac{1}{1 + \exp\left(\frac{m-m_0}{\omega}\right)},
            }
            where
            \begin{description}
                \item[$m_0$]    is the new \emph{limiting magnitude}, where detection probability is \SI{50}{\percent};
                \item[$\omega$] is the width of the function, with higher values representing
                    faster rate of loss of detection efficiency with decreasing brightness.
                    The limit $\omega \to \infty$ corresponds to a sharp cut-off at $m_0$,
                    or the Heaviside step function.
            \end{description}

            Ideally, a functional dependence on the distance from the centre of the field
            of view should be taken into account for human observers, as the eye is much more
            sensitive to light near the yellow spot.
            Similarly, digital sensors are subject to vignetting and other detrimental effects.
            While with cameras the magnitude of these effects is generally well understood,
            further research into the angular response of the human eye would be beneficial.

        \subsubsection{Zenithal hourly rate} \label{msaz}
            The \emph{zenithal hourly rate} (ZHR) is defined as ``[t]he number of shower meteors per hour
            one observer would see if his limiting magnitude is \Mag{+6.5} and the radiant is in his zenith''.
            While it is not an universally valid figure of merit for comparing meteor activity of various showers,
            ZHR is fairly simple and widely used as a standard in meteor science.
            Its primary design consideration is to be able to compare the activity of various showers
            as viewed by human observers, but it is perfectly applicable to automatic devices as well.

            ZHR is defined as
            \eqn{eq:ipqz-zhr}{
                \mathrm{ZHR} = \dfrac{\dfrac{N}{T} \cdot \dfrac{1}{k} \cdot r^{\num{6.5} - m_0}}{\sin{\theta_\mathrm{R}}}
            }
            where
            \begin{description}
                \item[$N$]
                    is the number of meteors recorded by the observer;
                \item[$T$]
                    is total effective observation time;
                \item[$k$]
                    is the fraction of the sky that is visible to the observer;
                \item[$r$]
                    is the \emph{population index} of the observed meteor shower;
                \item[$m_0$]
                    is the \emph{limiting magnitude} of the observer;
                \item[$\theta_\mathrm{R}$]
                    is the altitude of the radiant above the horizon.
            \end{description}
            In this definition the limiting magnitude is considered to be a sharp boundary -- all
            meteors below the limiting magnitude are detected.

            In our model the ZHR is not an input, but rather a computed quantity, completely determined by the
            models of the meteoroid population, atmosphere and of the observer's detection efficiency.

        \subsubsection{Population index} \label{msar}
            Another useful measure of meteor activity is the \emph{population index}.
            It is similar to the mass index (see \cref{msm}) in the sense.
            The definition by \citet{molau2015} is "[the population index] represents the [relative] increase
            in total meteor count when the limiting magnitude $\mathrm{lm}$ improves by one mag[nitude]."

            So if we denote the total number of observed meteors with magnitude at most $m$ as $N(m)$
            and compare the counts for different values of $m$ and $m + 1$, we need $N$ to satisfy the requirement
            \eqn{eq:msar-cdf}{
                r \cdot N(m) = N(m + 1).
            }

            Solving this for function $N(m)$ yields a single family of solutions
            in the form
            \eqn{eq:msar-cdfsolved}{
                N(m) = k \cdot r^m
            }
            for some real constant $k$.

            As $N(m)$ is -- by definition -- the kernel of the
            cumulative distribution function (CDF) for apparent magnitudes, we can differentiate it
            to obtain the kernel of the probability density function. The kernel of the PDF is then
            \eqn{eq:mspr-pdfk}{
                n(m) \propto r^m \ln r.
            }

            The value or $r$ is assumed to be constant throughout the entire observable magnitude range.%
            \footnote{Without this assumption the population index loses its main purpose,
            to simplify the description of the distribution of magnitudes.
            Technically, \emph{any} smooth distribution of magnitudes can be described in terms of varying $r$.}
            Note that this function is not a power law, but an exponential (the variable $r$
            is in the base, not the exponent, unlike with mass index $s$).

            To obtain a probability density function, we must find a norming constant
            and set an upper limit on magnitudes so that the distribution does not diverge for faint meteors.
            The upper limit is conveniently set to the limiting magnitude $m_0$.
            If we require the integral over the entire visible range to evaluate to one,
            the value of $k$ is unambiguously determined as $k = r^{-m_0}$.
            The PDF is then
            \eqn{eq:msar-pdf}{
                F_m(m) =
                \begin{cases}
                    r^{m - m_0} \ln r &
                        \text{for }m \leq m_0\text{,} \\
                    0 &
                        \text{for }m > m_0\text{.}
                \end{cases}
            }

            It should be kept in mind that the mass and population indices are not fundamental physical quantities,
            but only parameters of simple model distribution functions for masses and apparent magnitudes respectively.
            Also, value of $r$ can be measured correctly only when assuming there is a sharp limit of the observer's detection efficiency.
            For more complex detection efficiency profiles (such as the one defined in \cref{eq:msam-sigmoid})
            it cannot be defined.

            In our model, the value of mass index is always $s$ is treated as an input,
            while the value of $r$ can be computed from the orbital model and simulating the meteoroids entering the atmosphere.

        \subsubsection{Selection bias} \label{msab}
            In many analyses of observational dataset authors tend to overlook the effect of selection bias --  the
            propensity of the selected method of gathering the data to include certain data points with higher probability than other ones.
            This may negatively influence the conclusions that are drawn from the data, if these effects are not understood and corrected for.
            In our specific case it is chiefly the tendency of the cameras to detect brighter meteors more reliably than fainter ones,
            which leads to underestimating the number of faint meteors observable in the sky.
            For a detailed list of possible sources of bias and methods refer to \citep{balaz-thesis}.

    \subsection{Summary} \label{msm}
        The final model would thus be a five-dimensional map of distribution of meteor activity, with independent variables being
        \begin{itemize}
            \item one temporal coordinate $t$, representing the meteor activity in \emph{time};
            \item two spatial coordinates: the \emph{right ascension} $\alpha_R$ and \emph{declination} $\delta_R$ of the radiant;
            \item the \emph{pre-atmospheric geocentric speed} of the particles $v_\infty$;
            \item and the \emph{magnitude spectrum}, or the probability density function of absolute magnitudes of meteors.
        \end{itemize}

        the task is thus reduced to finding the distribution function for meteor radiants
        \eqn{eq:msm-dist}{
            \mathcal{A}(\delta_\mathrm{R}, \alpha_\mathrm{R}, t, v_\infty, m)
        }


        The total observable activity is thus described by the \emph{meteor activity function}
        \eqn{eq:msm-map}{
            \mathcal{A}(t, \vec{v}, m) \equiv \mathcal{A}(t, \delta_R, \alpha_R, v_\infty, m),
        }

        Within such maps we should be able to distinguish two main features:
        \begin{itemize}
            \item the \emph{sporadic background}, which forms the main contour lines of the maps, the ``terrain'';
            \item and several well-defined, sharp peaks, representing the \emph{meteor showers}.
                Due to common origin and orbit, these peaks should be very narrow in all dimensions.
        \end{itemize}

            Due to influences of various acting forces, such as differences
            in initial velocities with respect to the parent body, tidal disruptions,
            perturbations arising from close encounters with large bodies of the Solar System,
            Poynting-Robertson effect, etc., the stream
            gradually widens and disperses in all components until it can no longer be distinguished from the background.

%    Another method has been discussed and implemented in the author's master's thesis \citep{balaz-thesis}
%    and used to determine the total flux of the Perseids in 2016. Instead of trying to
%    estimate the biases we generated the meteoroids above the atmosphere and simulated
%    their entry using a simplified set of equations for motion, ablation and luminosity.
%    A collection of stochastic bias functions was then applied to the dataset and
%    each meteor was marked as detected or missed. We varied the parameters of the bias functions
%    and the mass index until agreement with observational data was achieved.
%    The initial population was then declared as the model of the actual population
%    and the bias functions were taken as descriptive of the observing system.

%\section{Simulation in the atmosphere} \label{ma}
%    Once it has been determined which meteoroids will enter the Earth's atmosphere, we may proceed to simulating
%    the physical processes manifested during their atmospheric entry and obtain the values of quantities
%    that can be observed by ground-based observers.
%
%    To simulate the atmospheric entry we used \textsc{Asmodeus},
%    a multi-purpose virtual meteor simulator developed as a part of our master's thesis and extended for numerous
%    other purposes related to meteor astronomy \citep{balaz-thesis,balaz+2020}.
%
%    Once a particle is selected for atmospheric entry, its velocity is transformed to the ECEF reference frame
%    and a numerical integration of the equations of motion is executed.
%
%    \subsection{Numerical integration of the equations of motion} \label{sai}
%        Unlike in interplanetary space, where forces are completely dominated by the gravitational force exerted by the Sun
%        and only slightly perturbed by other effects,
%        a particle entering the Earth's atmosphere switches between several regimes in rapid succession.
%        The precise order is subject to variations due to particle's size, entry speed and angle
%        and material composition.
%
%        \subsubsection{Acting forces} \label{saia}
%            In the simulation we consider only two real forces:
%            \begin{itemize}
%                \item the \emph{drag force} $\vec{F_d}$, always acting against the instantaneous velocity vector of the particle;
%                \item the \emph{gravitational force} $\vec{F_g}$, pulling the meteoroid towards the centre of the Earth.
%            \end{itemize}
%
%            The precision of the calculations may be improved if we also account for two fictitious forces,
%            arising from the fact the simulation is performed in a rotating reference frame:
%            \begin{itemize}
%                \item the fictitious \emph{centrifugal force} $\vec{F_{\mathrm{C}}}$,
%                    pushing the particle away from the axis of rotation of the Earth;
%                \item and the fictitious \emph{Coriolis force} $\vec{F_{\mathrm{c}}}$,
%                    which pushes the moving body in a direction perpendicular to its velocity vector.
%            \end{itemize}
%
%            Except for the drag force, all forces can be readily expressed in terms of instantaneous
%            position and velocity of the meteoroid and several constants.
%            Calculating the drag force precisely is computationally very expensive and cannot be done
%            without knowing the shape of the particle. Therefore we have to choose an appropriate simplified model.
%
%            In its cosmic stage and during most of the motion in the gravitational sphere of influence of the Earth,
%            the forces acting on the particle are dominated by gravitational effects. As the meteoroid penetrates
%            deeper into the atmosphere, the density of the surrounding gas increases roughly exponentially.
%            At its peak, the atmospheric drag force exceeds all other effects by several orders of magnitude,
%            decelerating the meteoroid very rapidly.
%            As the duration of the entire event is on the order of seconds, other forces simply do not act for long enough
%            to produce a significant deviation in trajectory. For statistical evaluation of artificial meteors
%            it is sufficient to consider only the drag force.
%            However, should the meteoroid survive the entry as a meteorite, their influence is important during the dark phase of flight.
%
%            \paragraph{Coriolis force} \label{saiC}
%                The fictitious Coriolis force arises in a rotating reference frame.
%                \eqn{eq:saiC-coriolis}{
%                    \vec{F_{\mathrm{C}}} = -2 m \vec{\Omega}_\Earth \times \vec{v}.
%                }
%
%                For a meteoroid, its magnitude is at most on the order of
%                \eqn{eq:saiC-order}{
%                    2 \frac{2\pi}{\SI{86400}{\second}} \cdot \SI{70}{\kilo\metre\per\second} \sim \SI{10}{\metre\per\second\squared},
%                }
%                roughly the same as gravitational acceleration.
%
%            \paragraph{Centrifugal force} \label{saic}
%                The centrifugal force pushes the particles away from the axis of rotation of the Earth.
%                \eqn{eq:saic-centrifugal}{
%                    \vec{F_{\mathrm{c}}} = - m \vec{\Omega}_\Earth \times \left(\vec{\Omega}_\Earth \times \vec{r}\right),
%                }
%                where $\vec{\Omega}_\Earth$ is the angular speed of rotation of the Earth,
%                $\frac{2\pi}{\SI{86164}{\second}} \approx \SI{7.292e-5}{\per\second}$ in the direction
%                towards the north celestial pole.
%
%                Since it only depends on the instantaneous position, it can be thought of as a minor correction to the
%                gravitational force. A simple calculation shows that its magnitude is on the order
%                of \SI{0.03}{\metre\per\second\squared}. Near the surface it can be safely neglected for most purposes.
%                It only becomes important when the distance from the axis of rotation is large, for instance if the simulation
%                is used to investigate the motion of a meteoroid further from the Earth.
%                However, in this case an inertial reference frame should be used instead.
%
%            \paragraph{Gravitational force} \label{saig}
%                The real shape of the gravitational potential around the Earth is fairly complex and difficult to describe accurately.
%                For particles that are generally moving on highly hyperbolic trajectories the
%                simple approximation by a Newtonian point source with mass $M_\Earth$ is more than sufficient.
%                The formula used in computation is then simply
%                \eqn{eq:saig-gravity}{
%                    \vec{F_{\mathrm{G}}} = - \frac{GM_\Earth}{r^3} \vec{r}.
%                }
%
%            \paragraph{Drag force} \label{said}
%                The precise description of the drag force is crucial to understand the motion
%                of meteoroids in the atmosphere. The drag force is dominant between approximately
%                the point where meteoroids begin to emit visible light (about \SI{100}{\kilo\metre})
%                until the very end of the visible trail. For all but the most massive meteoroids,
%                which are able to survive the atmospheric entry, this can be approximated
%                by the point where all of the matter has been ablated away; while for the bodies
%                surviving as meteorites, during the terminal phase of the flight the drag and
%                gravitational forces are nearly balanced out.
%
%                At peak deceleration the magnitude of the drag force can reach values several orders of magnitude
%                higher than the sum of all other forces and is thus completely dominant.
%                A very simple model of atmospheric flight can thus perform relatively
%                well even when all other forces are neglected.
%
%    \subsection{Determination of luminosity} \label{sail}
%        Currently we use... \citep{hill+2005} \todo{rmeasureeally?}
%
%        \citep{bronshten1983}
%        but more precise models should be developed
%

\section{Orbital model} \label{mo}
    While the observational model may describe the observed distribution of meteors quite precisely,
    it does not provide any deeper understanding of \textit{why} meteors appear
    at the specified time and from a particular region in the sky.
    To address that, we need to analyze the observed meteors and reconstruct
    the trajectories of the meteoroids which created them.

    For each meteor, it is in principle possible to trace its visible motion back in time
    and derive the trajectory of the original meteoroid in the vicinity of the Earth.
    After correcting for gravitational attraction and subtracting the velocity of the Earth
    we can further extend the trajectory into interplanetary space.

    A fundamental difficulty here is that particle streams whose orbit does not intersect
    the orbit of the Earth cannot be observed at all and thus are not included in the model.
    Determination of flux in these areas is not possible with Earth-based observations at all
    and can only be investigated with in-situ observations. In this thesis we will not analyze them any further.

    A less fundamental, but still important difficulty lies in the low precision of these measurements.
    The position of the projection of the meteor in the sky can be determined with high precision:
    the resolution of a photograph or a video sequence is typically on the order of \SI{1}{\milli\radian}
    while the distance between the observer and the meteor is typically several hundred kilometres.
    This results in an absolute precision on the order of tens of metres.

    With meteor speeds the situation is somewhat more complicated.
    For meteors that are moving in a direction perpendicular to the line of sight, speed can be determined
    with high precision, but the radiant is typically further away.
    Meteors that appear close to the radiant will move slowly in the sky, which introduces large errors,
    however their angular distance from the radiant is small. The effects of geometry thus mostly cancel out.
    Precision of measurement suffers when the convergence angle is small \citep{ceplecha1987}.

    Furthermore, while the trajectory can be computed with very high precision in vacuum,
    meteoroids lose a fraction of their original preatmospheric speed before they become visible \citep{vida+2018}.
    This further adds to the uncertainty of the original orbit.
    One very pronounced misleading effect is the vastly overestimated number of objects
    on interstellar heliocentric orbits in meteor databases \citep{hajdukovajr1994}.

    The orbital model is necessarily incomplete: we cannot see meteoroids that are not intercepted by the atmosphere.
    It is possible there is a dense stream of particles on an orbit not too far from the orbit of the Earth,
    but as long as they do not become visible as meteors, virtually nothing is known about them.

    Therefore meteoroid streams not crossing the Earth's orbit are out of the scope of this thesis.
    On the other hand, this is not too important, as we are -- and likely always
    will be -- much more interested in those which do.

    \subsection{Choice of coordinate space} \label{moc}
        The natural reference system to work with when investigating meteoroid streams in interplanetary space
        is the \emph{heliocentric ecliptic coordinate system}. Its fundamental plane is the orbital plane of the Earth
        and its reference direction is the March equinox.

        As with the observational model, we should begin by determining the whole coordinate space
        in which the distribution function should be defined. For positions of meteoroids we need at least
        three spatial coordinates, three coordinates describing the velocity vector, and the time:
        \eqn{eq:po-rv}{
            \mathcal{M}_\mathrm{rv}(\vec{r}, \vec{v}, t).
        }
        where
        \begin{description}
            \item[$\vec{r}$]
                is the instantaneous position in a heliocentric inertial reference frame;
            \item[$\vec{v}$]
                is the velocity in the same frame;
            \item[$t$]
                is the time;
            \item[$\mathcal{M}_\mathrm{rv}$]
                is the number density of particles in the position-velocity coordinate space.
        \end{description}

        Alternatively this may be expressed in terms of orbital elements as
        \eqn{eq:po-elem}{
            \mathcal{M}_\mathrm{oe}(a, e, i, \omega, \Omega, T, t).
        }

        While in the observational model we were mostly interested in the distribution of apparent and absolute
        magnitudes of the meteors, here the measure of meteoroid size is its true mass, which we will denote as
        $\mu$ (to avoid conflicts with the absolute magnitude of the meteor, which is already denoted by $m$).
        To include it in the model as well, we shall add one more dimension $\mu$.

        \subsubsection{Optimizations} \label{moco}
            The distribution of meteoroid masses of our interest spans six or seven decimal orders of magnitude,
            with most of the samples concentrated near the lower bound of the interval.
            The upper portion of the mass interval is populated only very sparsely.
            To address this, we may work in logarithmic space instead, using $\log \mu$ as the measure of mass.
            The distribution of masses will still be shifted towards lower values, but not so heavily.

            The same approach is applicable to other quantities, most importantly the semi-major axis $a$.
            Although the region of our interest is located very close to \SI{1}{\au},
            orbits of meteoroids originating from long-period comets have very large $a$.
            The gravitational influence of the Sun is weaker at these distances and
            even a small change in velocity near the perihelion may alter $a$ significantly.
            Therefore we can use $1/a$ as the measure of orbit size instead,
            and use a Gaussian kernel in estimation of this new coordinate.

            The entire coordinate space is then
            \eqn{eq:moco-psxyz}{
                \mathcal{M}_c(x, y, z, v_x, v_y, v_z, t, \log \mu),
            }
            or in terms of orbital elements,
            \eqn{eq:moco-psorb}{
                \mathcal{M}_o(1/a, e, i, \omega, \Omega, T, t, \log \mu).
            }

    \subsection{Physical quantities and properties} \label{mp}
        Apart from the position and velocities, which we have already described

        \subsubsection{Time} \label{msmt}

        \subsubsection{Fluxes and densities} \label{msmf}
            While the distribution function in the space should provide complete information
            about the density of particles and its change over time,
            the high number of dimensions it has makes it difficult to comprehend.
            A method of reduction to a lower number of dimensions is necessary.

            The primary measure of activity of a meteor shower (or the sporadic background) is the number
            of meteors encountered per unit area of the surface of the Earth (which is more suitable
            in observational models), or per unit volume in interplanetary space (for orbital models).

            As meteoroids are discrete particles of finite size, it is easier to work with the
            assumed underlying distribution function, so as to avoid artifacts.

            Spatial number density is obtained from the model by integrating over all parameters
            in the space except the spatial ones:
            \eqn{eq:msmf-spatial}{
                \varrho(\vec{r}, t) = \Int[\mu_0][\infty]{
                    \Int[\Real^3][]{M(\vec{r}, \vec{v}, t, \mu)}{\vec{v}}
                }{\mu}
            }

            On the surface of the Earth, or in its close vicinity, we are more concerned with
            the number particles passing through an unit area of the surface.
            To obtain the flux
            \eqn{eq:msmf-flux}{
            }

            Similarly, to determine the total mass influx from a particular population, the first moment
            of the distribution over particle masses is calculated.
            \Cref{eq:msmf-flux} becomes
            \eqn{eq:msmf-wflux}{
                F(\vec{r}, t) = \Int[\mu_0][\infty]{
                    \Int[\Real^3][]{M(\vec{r}, \vec{v}, t, \mu)}{\vec{v}}
                }{\mu}
            }

        \subsubsection{Mass index and the Pareto distribution} \label{msms}
            While theoretically any distribution of particle masses is possible,
            in real meteoroid populations there is always a large excess of small particles,
            at least within the meteoroid mass range which can be covered by visual observations.

            In meteor science, it is customary to assume a specific distribution of masses within a population,
            described by a power law with particle mass as the independent variable:
            \eqn{eq:mpam-sindex}{
                \varrho(\mu) \propto \mu^{-s}
            }
            for some constant $s$ named the \emph{mass index}.

            To obtain a distribution, this expression needs to be normalised first.
            Furthermore, the expression $\mu^{-s}$ diverges to infinity for $\mu \to 0$,
            which can be dealt with by setting a lower limit on the mass.
            These requirements are satisfied by the well-known Pareto distribution \citep{arnold1983}.
            In mathematical texts it is usually defined as $\Distribution{\mathrm{Pareto}}{x}{\alpha, x_0}$,
            where $\alpha > 0$ is called \emph{shape} and $x_0 > 0$ is called the \emph{scale} of the distribution.

            To conform with commonly used terminology in meteor science, we substitute the shape with mass index
            $s = \alpha - 1$, and rename $x_0$ to \emph{minimum mass} of particles, $\mu_\mathrm{min}$.
            The probability density function is then
            \eqn{eq:msms-pareto}{
                \varrho(\mu) = \mathrm{Pareto}\left(\mu;\ s-1, {\mu_\mathrm{min}}\right) \equiv
                \begin{cases}
                    0 &
                        \text{for }\mu < \mu_{\mathrm{min}}\text{,} \\
                    \dfrac{\left(s - 1\right) \mu_{\mathrm{min}}^{s - 1}}{\mu^s} &
                        \text{for }\mu \geq \mu_{\mathrm{min}}\text{.}
                \end{cases}
            }

            Typical values of $s$ are between \numrange{1.5}{2.5}, with values slightly below 2 being most common
            for meteor showers and values around \num{2.2} for sporadic background \citep{blaauw+2011}.
            The collisional equilibrium for particles of equal strength is reached at $s = 11/6$ \citep{dohnanyi1969}.

            The mass index $s$ can be related to the population index $r$ by an empirical relationship by \citet{koschack+1990}
            \eqn{eq:msms-rs}{
                r = 1 + \num{2.3} \log_{10} s.
            }


    \subsection{Additional properties} \label{mpa}
        Apart from the dynamical properties, which describe only the density of particles
        and thus a probability of a collision at a point in space and time,
        we may (and should!) observe and analyze additional properties of the stream.

        \subsubsection{Petrological and chemical classification} \label{mpac}
            While we are not explicitly interested in chemical properties
            or dynamical evolution of meteor streams, these properties
            are correlated with the density and toughness of the material,
            which in turn affect the process of ablation
            and thus also the final luminosity of the meteor.
            All of this contributes to the unwanted selection bias.
            Furthermore, differences in spectra are responsible for
            variability in detection efficiency with different cameras.

            So, in addition to previously mentioned \textit{numerical} data, we might want to
            provide extraneous information about the meteoroids, such as their origin,
            mineralogical or chemical properties.
            For many meteoroids these data are not be available,
            though as they provide a wealth of information on origin
            it is definitely useful to include it when available.


\section{Simulations} \label{mi}
    At the core of both the orbital and the observational models are numerical simulations.
    The simulation starts with some assumed distribution of meteoroid particles

    \subsection{Simulation in the Solar System} \label{mio}
        Simulating the motion of meteoroids in interplanetary space is comparatively easy.
        First of all, it is governed by differential equations describing that are known and theoretically well founded.
        Second, as it is one of the fundamental problems in observational astronomy,
        it has been solved numerous times with ever-increasing precision and performance.
        Modern software packages, such as REBOUND \citep{rein+rebound}, are capable of solving the equations
        with sufficient precision and without any added effort on the part of the user.
        Third, the solutions are not overly dependent on properties of the bodies that are difficult to measure.

        The generic workflow is approximately as follows: the simulation computes the positions of
        the large bodies of the Solar System, most importantly the Sun, the eight planets and their moons.
        The obtained values may be easily verified by comparing to known databases or ephemeris systems.
        Then the test particles are created, each representing a single meteoroid body.

        Since every two bodies attract each other, finding all forces acting on the bodies generally
        requires computation of $n\left(n - 1\right) = \BigO{n^2}$ pairs of forces.
        According to Newton's third law opposing forces are equal in magnitude and opposite in their direction,
        and thus one evaluation in every pair is enough.
        This brings the number of required evaluations to half the stated value, without changing the order.
        Further optimizations are not possible in general.

        Optionally, calculating the positions of massless bodies can be offloaded to a graphics processing unit (GPU).
        A GPU is particularly well suited to solving massively parallel problems,
        such as simulation of a large number of mutually non-interacting test particles.


%\section{Simulation in the atmosphere} \label{ma}
%    Once it has been determined which meteoroids will enter the Earth's atmosphere, we may proceed to simulating
%    the physical processes manifested during their atmospheric entry and obtain the values of quantities
%    that can be observed by ground-based observers.
%
%    To simulate the atmospheric entry we used \textsc{Asmodeus},
%    a multi-purpose virtual meteor simulator developed as a part of our master's thesis and extended for numerous
%    other purposes related to meteor astronomy \citep{balaz-thesis,balaz+2020}.
%
%    Once a particle is selected for atmospheric entry, its velocity is transformed to the ECEF reference frame
%    and a numerical integration of the equations of motion is executed.
%
%    \subsection{Numerical integration of the equations of motion} \label{sai}
%        Unlike in interplanetary space, where forces are completely dominated by the gravitational force exerted by the Sun
%        and only slightly perturbed by other effects,
%        a particle entering the Earth's atmosphere switches between several regimes in rapid succession.
%        The precise order is subject to variations due to particle's size, entry speed and angle
%        and material composition.
%
%        \subsubsection{Acting forces} \label{saia}
%            In the simulation we consider only two real forces:
%            \begin{itemize}
%                \item the \emph{drag force} $\vec{F_d}$, always acting against the instantaneous velocity vector of the particle;
%                \item the \emph{gravitational force} $\vec{F_g}$, pulling the meteoroid towards the centre of the Earth.
%            \end{itemize}
%
%            The precision of the calculations may be improved if we also account for two fictitious forces,
%            arising from the fact the simulation is performed in a rotating reference frame:
%            \begin{itemize}
%                \item the fictitious \emph{centrifugal force} $\vec{F_{\mathrm{C}}}$,
%                    pushing the particle away from the axis of rotation of the Earth;
%                \item and the fictitious \emph{Coriolis force} $\vec{F_{\mathrm{c}}}$,
%                    which pushes the moving body in a direction perpendicular to its velocity vector.
%            \end{itemize}
%
%            Except for the drag force, all forces can be readily expressed in terms of instantaneous
%            position and velocity of the meteoroid and several constants.
%            Calculating the drag force precisely is computationally very expensive and cannot be done
%            without knowing the shape of the particle. Therefore we have to choose an appropriate simplified model.
%
%            In its cosmic stage and during most of the motion in the gravitational sphere of influence of the Earth,
%            the forces acting on the particle are dominated by gravitational effects. As the meteoroid penetrates
%            deeper into the atmosphere, the density of the surrounding gas increases roughly exponentially.
%            At its peak, the atmospheric drag force exceeds all other effects by several orders of magnitude,
%            decelerating the meteoroid very rapidly.
%            As the duration of the entire event is on the order of seconds, other forces simply do not act for long enough
%            to produce a significant deviation in trajectory. For statistical evaluation of artificial meteors
%            it is sufficient to consider only the drag force.
%            However, should the meteoroid survive the entry as a meteorite, their influence is important during the dark phase of flight.



%    \subsection{The N-body problem} \label{msN}
%        At the core of investigation of meteoroid motion in interplanetary space
%        there is little more than a fairly complex $N$-body problem:
%        we are given the initial positions and velocities of a set of objects,
%        mutually interacting by gravitational attraction.
%        However, rather than finding their positions and velocities at all future times, we are mostly concerned
%        with the question whether some of these particles will ever collide with the Earth.
%
%        Finding a closed-form solution is either not possible at all, or is only expressible
%        as a convergent series with way too many terms to be of any practical use \citep{beloriszky-1930}.
%
%        Fortunately, an exact closed-form solution is not needed. First of all, the precision of our results
%        is always practically limited by the precision with which the initial conditions are known,
%        which is always finite.
%
%        Furthermore, small particles are often also influenced by non-gravitational forces,
%        which are themselves difficult to describe analytically.
%        Numerical solutions thus remain the astronomer's best friends.
%
%        With modern integrators and available computational power this is no longer a problem.
%        Even a cheap modern computer equipped with suitable software is able to calculate trajectories
%        with precision only limited by our knowledge of initial positions and velocities of the bodies.


        Any particle intersecting the cross-section of the Earth is marked as captured and will be examined further.
        As we have stated in \cref{msp}, the precise position of atmospheric entry is not important.
        In order to increase the yield and reduce the required computation time, we may employ a larger artificial detector:
        the diameter of the simulated Earth is increased several times in order to capture more meteoroids.
        If the scale on which the density of the distribution function $\mathcal{M}$ changes significantly
        is larger then the radius of the Earth, this is fully justified.
        Similarly, under this assumption temporal variations in activity will be slow.

        This assumption is not universally valid though -- some younger meteor showers often produce dense, narrow filaments,
        which can be observed as short but intensive outbursts of heightened meteor activity.
        A~good example are the November alpha-Monocerotids, with a period of significant
        activity spanning only about thirty minutes in 2019 \citep{CBET4692}.

        By comparing the orbital speed and the diameter of the Earth we can see that the Earth moves
        to a completely different position with respect to the stream within several minutes.
        This means that variations in activity are not going to be significant when crossing an old stream,
        whose radius may be on the order of millions of kilometres, but may cause problems with young, narrow filaments.

        The duration of visible activity also depends on the exact shape of the stream and its orientation with respect to the orbit of the Earth.
        The worst case are fast streams on orbits with high inclinations, where meteors seem to arrive
        from a direction perpendicular to the direction of movement of the Earth, about \ang{90} from the apex and antapex sources.

    \subsection{Simulation in the atmosphere} \label{mia}
        Once it has been determined which meteoroids will enter the Earth's atmosphere, we may proceed to simulating
        the physical processes manifested during their atmospheric entry and obtain the values of quantities
        that can be observed by ground-based observers.

        To simulate the atmospheric entry we plan to use \textsc{Asmodeus},
        a multi-purpose virtual meteor simulator developed as a part of our master's thesis and extended for numerous
        other purposes related to meteor astronomy \citep{balaz-thesis}. A detailed description
        of the equations and algorithms can be found in \citep{balaz+2020}.

\section{Mathematical methods} \label{mm}
    Once we have ascertained what the result should like, we need to describe the mathematical foundation of the methods used to achieve it.
    We have stated that even though meteoroids are discrete particles, it is useful to model their density
    as a smooth function in multi-dimensional coordinate space.
    Actual particles are then treated as random samples obtained from the underlying distribution.

    However, logic dictates we do this in the opposite direction: we need to approximate
    the underlying distribution from a finite set of known samples.
    The number of parameters encountered and the complexity of the model essentially preclude the use of any parametric models.

    The estimate may be computed using a \emph{kernel density estimation} method instead.
    Each actual meteoroid, obtained by backwards integration of motion of an observed meteor,
    is substituted by a multidimensional kernel in the coordinate space. The precise shape of the kernel
    is not overly important; for most purposes a simple multidimensional gaussian is sufficient.
    To make matters even worse, our observational datasets are usually incomplete
    and heavily distorted by observation bias, all of which will have to be addressed before density is estimated.

    \subsection{Kernel density estimation} \label{mmk}
        Kernel density estimation (KDE) is a non-parametric method of estimating the probability density function
        of a distribution. It is particularly useful when the underlying distribution is not known and thus impossible
        to fit with a parametric function; or very complicated, in which case there would be too many parameters
        and possibly many local minima of the fit.

        A comprehensive, mathematically rigorous description can be found in \citep{hwang+1994}.
        In simple terms, we can imagine computing the KDE as in this simple algorithm:
        \begin{enumerate}
            \item determine the bandwidth
            \item for every data point $i$
                \begin{enumerate}[label=\arabic*.]
                \item find the coordinates $\vec{x_i} = (x_{i1}, x_{i2}, ..., x_{id})^T$ of the data point $i$
                \item replace each data point with the kernel, shifted to $\vec{x_i}$
            \end{enumerate}
            \item sum all shifted kernels
            \item return the function as the result
        \end{enumerate}

        Put mathematically,
        \eqn{mmk-kde}{
            \Hat{F}(\vec{x}) = \frac{1}{n \Abs{\vec{H}}} \Sum[i = 1][n]{K\left[\vec{H}^{-1}\left(\vec{x} - \vec{x}_i\right)\right]}
        }
        Alternatively we may define KDE in terms of $d$-dimensional convolution of the kernel with the sum of all data points,
        where each data point is represented by a $\delta$-function:
        \eqn{mmk-kdecont}{
            \Hat{F}(\vec{x}) = K \ast \left(\Sum[i = 1][n]{\delta\left(\vec{H}^{-1}\left(\vec{x} - \vec{x}_i\right)\right)}\right).
        }

        \subsubsection{Kernel} \label{mmkk}
            Surprisingly, the choice of the kernel is not particularly important.
            However, it should be chosen to satisfy two constraints: it must be
            \emph{non-negative} over the entire domain
            \eqn{eq:mmkk-nonneg}{
                \forall \vec{y} \in \Real^d : K(\vec{y}) \geq 0,
            }
            and it should be \emph{normalized} to unity
            \eqn{eq:mmkk-norm}{
                \Int[\Real^d]{K(\vec{y})}{\vec{y}} = 1.
            }

            An optional, but sensible third constraint is that the kernel should be a \emph{symmetric function},
            \eqn{eq:mmkk-sym}{
                \forall \vec{y} \in \Real^d : K(\vec{y}) = K(-\vec{y}),
            }
            though this is affected by the choice of coordinates (see \cref{moc}).
            Some commonly used kernels are box functions, triangles and gaussians.

        \subsubsection{Bandwidth} \label{mmkw}
            The most important consideration in KDE is the determination of correct bandwidth.
            If the selected bandwidth is too high, the resulting estimate of the distribution
            becomes very smeared and does not capture the distribution very well.
            In the limit $\Abs{\vec{H}} \to \infty$ the estimate becomes identical to the original kernel.

            If the bandwidth is too low, individual kernels do not overlap anymore and produce a series
            of kernels at the positions of the original data points. In the limit $\Abs{\vec{H}} \to 0$
            a sum of $n$ $\delta$-functions is obtained.
            Neither of the extremes provides any useful information.

            Several algorithms for determining the optimal bandwidth exist \citep{bowman1985,jones+1996},
            with larger datasets universally requiring narrower bandwidths.
            One possible approach is to try to optimize the bandwidth by minimizing the
            \emph{mean integral square error} on some synthetic dataset.
            The KDE is then calculated on the real dataset with the determined optimal bandwidth.

            To simplify the fit, instead of the covariance matrix $\vec{H}$ we may opt to use a diagonal matrix,
            but with a different metric. With meteors, two of the coordinates -- declination
            and right ascension of the radiant -- are in fact spherical coordinates,
            which cannot be correctly described in Euclidean space anyway.

        \subsubsection{Adaptive kernel density estimation} \label{mmka}
            Another improvement to the fixed-width KDE method is to allow some variation in the bandwidth
            of the kernels in different regions of the domain in some relation to the data.
            Typically we would require a narrower bandwidth where data are abundant,
            such as in meteor showers, and allow a wider bandwidth in areas where data points are sparse.

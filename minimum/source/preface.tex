In this thesis we aim to design a suitable representation of spatial density
and flux of meteoroids in interplanetary space. Subsequently we aspire to
construct a model of their distribution in the inner Solar System,
and of distribution of mass among the bodies in every population.
If possible, we would also like to include information
about their chemical composition or other supplementary properties.
There are two notably distinct approaches to the problem.

In the first method, we analyze the database of observed meteors and establish
the flux in each direction and in any time with recorded observations.
After correcting for observational bias and accounting for data missing due
to incomplete or missing data we should be able to establish the particle
number density and derive the total mass influx.

The second possible approach works in quite the opposite direction. We start with a simulation
of the interplanetary matter in the vicinity of the Earth and an initial guess of its distribution.
The number density and mass flux of meteoroids in the vicinity of the Earth is determined.
We determine which meteoroids are on trajectories that intersect the atmosphere of the Earth
and for those we simulate and record their atmospheric entry. The resulting dataset is statistically compared
to the actual instrumental observations, accounting for detector deficiencies and selection bias.
Finally we vary the initial particle distribution and the model of interaction with the atmosphere
until the best possible agreement with observational data is found.
This is then declared to be the model of the true distribution.

On the far horizon of our efforts is the synthesis of both methods into one whole:
we could use the observational data to put rough constrains on the meteoroid flux,
and then construct an $N$-body numerical simulation which will try to replicate the data,
accounting for observational bias effects as well.

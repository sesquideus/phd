The thesis is a natural extension of the author's master's thesis \citep{balaz-thesis}.
In it, we designed and implemented a parametric Monte-Carlo simulation of meteoroids
and used it to de-bias observational datasets and determine the total flux
of particles into the atmosphere.
The simulation generates virtual meteoroid particles above the atmosphere and
according to pre-defined equations of motion and ablation and determines their luminosity as a function of time.
The resulting meteors are then transformed and after application of various natural and
instrumental effects a synthetic record of an observation is created,
which then can be statistically compared to actual observational data.
The input parameters of the simulation are varied until its output matches to the observational dataset.

The next step, presented in this thesis, is to build upon this system and extend its scope to interplanetary space.
The observational meteor data collected by ground-based automatic stations may be used
to determine the original orbits of meteoroids and to construct a model
of their distribution in the interplanetary space, including information
about distributions of particle mass or composition as well.

Since currently it is not possible to observe meteoroids directly in interplanetary space,
all models obtained from ground-based observations can only describe the close
vicinity of the Earth and the meteor showers which happen to cross its orbit.
Therefore it might be a good idea to produce two kinds of models: an \emph{observational model},
describing meteor activity as observed from the Earth, and an \emph{orbital model},
that represents the density of particles in interplanetary space.
The observational model can be directly related to reality, while the orbital
model provides deeper understand of why there are meteoroids encountered at
certain point of space and time.
We can think of at least two notably distinct methods of building these models.

In the first method, we analyze the database of observed meteors and establish
the flux in each direction and in any time with recorded observations.
After correcting for observational bias and accounting for incomplete or missing
data we should be able to establish the particle
number density and derive the total mass influx.
If possible, we would also like to include information
about chemical composition of the particles or other supplementary properties.

The second approach works in quite the opposite direction. We start with a simulation
of the interplanetary matter in the vicinity of the Earth and an initial guess of its distribution.
The number density and mass flux of meteoroids in the vicinity of the Earth is estimated.
We determine which meteoroids are on trajectories that intersect the atmosphere of the Earth
and for those we simulate and record their atmospheric entry. The resulting dataset is statistically compared
to the actual instrumental observations, accounting for detector deficiencies and selection bias.
Finally we vary the initial particle distribution and the model of interaction with the atmosphere
until the best possible agreement with observational data is found.
This is then declared to be the model of the true distribution.

On the far horizon of our efforts is the synthesis of both methods into one whole:
we could use the observational data to put rough constrains on the meteoroid flux,
and then construct an $N$-body numerical simulation which will try to replicate the data,
accounting for observational bias effects as well.

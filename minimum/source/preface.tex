This thesis is a natural extension of the author's master's thesis \citep{balaz-thesis}.
In it we designed and implemented a parametric Monte-Carlo simulation of meteoroids entering the Earth's atmosphere.
Effect of atmospheric extinctions and limitations of the detecting device, including observational bias, were built into the model.
From the processed data we were able to determine the total flux of meteoroids into the atmosphere.

The simulation generates virtual meteoroid particles above the atmosphere and determines their
luminosity as a function of time, according to pre-defined equations of motion and ablation.
The resulting meteors are then transformed to the horizontal reference frame centered in the virtual camera.
After application of various natural and instrumental effects a synthetic record of an observation is created.
The record then can be statistically compared to actual observational data.
The input parameters of the simulation are varied until its output matches the observational dataset.

In the next step, presented in this thesis, we build upon this simulation and extend its scope to interplanetary space.
The observational meteor data collected by ground-based automatic stations may be used
to determine the original orbits of meteoroids and to construct a model
of their distribution in the interplanetary space, including information
about distributions of particle mass or their mineralogical composition as well.
We can think of at least two notably distinct approaches to building such a model.

In the first method, we analyze the database of observed meteors. From it we determine
the flux of particles coming from each direction in any particular time interval with recorded observations.
After correcting for observational bias and accounting for incomplete or missing
data we should be able to establish the particle number density and derive the total mass influx.
If possible, we would also like to include information
about chemical composition of the particles or other supplementary properties.

The second approach works in quite the opposite direction. We start with a simulation
of interplanetary matter in the vicinity of the Earth and an initial guess of its distribution.
The number density and mass flux of meteoroids in the vicinity of the Earth is computed directly.
We determine which meteoroids are on trajectories that intersect the atmosphere of the Earth.
For those particles we simulate and record the atmospheric entry. The resulting dataset is statistically compared
to the actual observations, accounting for detector deficiencies and selection bias.
Finally, we vary the initial particle distribution and the model of interaction with the atmosphere
until the best possible agreement with observational data is found.
This is then declared to be the model of the true distribution.

On the far horizon of our efforts is the synthesis of both methods into one whole:
we could use the observational data to put constrains on the meteoroid flux,
and then construct a massively parallel $N$-body numerical simulation,
which will try to replicate the data, automatically accounting for observational bias effects as well.

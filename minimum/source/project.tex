\Epigraph[0.4]{
    You saw sagacious Solomon \\
    You know what came of him \\
    To him, complexities seemed plain
}{
    How Fortunate the Man With None \\
    \textsc{Bertolt Brecht} \\
    \textit{translation John Willett}
}

The primary objective of the thesis is to develop a model of meteoroid populations intersecting
the orbit of the Earth -- or more precisely, populations observable from the surface as meteors.
The model must be in agreement with data obtained by visual observers and automated devices watching the sky.

Raw data are invariably skewed by observational bias. Even though elaborate methods for correction of various
effects have been developed,


Typically we do not consider individual particles, but the entire stream and its elliptical orbit around the Sun.

\section{Models}
    Even though meteoroids are discrete particles, it is useful to model their density
    as a smooth function in multi-dimensional phase space.
    Actual particles are then treated as random samples obtained from the underlying distribution.

    However, logic dictates we do this in the opposite direction: we need to approximate
    the underlying distribution from a finite set of known samples.
    This may be done using a \emph{kernel density estimation} method.
    Each actual meteoroid, obtained by backwards integration of motion of an observed meteor,
    is substituted by a multidimensional kernel in the phase space. The precise shape of the kernel
    is not overly important; for most purposes a simple multidimensional gaussian is sufficient \cite{???}.
    To make matters even worse, our observational datasets are usually incomplete
    and heavily distorted by observation bias.

    The final product thus consists of two parts: the \emph{observational model},
    obtained from observations of meteors from the Earth's surface;
    and an extended \emph{orbital model}, which describes the density of matter
    in interplanetary space as a function of position, velocity and time.

\section{Current state of the project} \label{pc}

\section{Observational model} \label{ps}
    We need to use a four-dimensional KDE. Ideally, the kernel should be obtained by a product of five
    independent one-dimensional kernels.

    The final product would thus be a five-dimensional map of distribution of meteor activity, with independent variables being
    \begin{itemize}
        \item one temporal coordinate $t$, representing the meteor activity in \emph{time};
        \item two spatial coordinates: the \emph{right ascension} $\alpha_R$ and \emph{declination} $\delta_R$ of the radiant;
        \item the \emph{pre-atmospheric geocentric speed} of the particles $v_\infty$;
        \item and the \emph{mass spectrum}, or the probability density of distribution of meteoroid masses.
    \end{itemize}

    The total observable activity is thus described by the \emph{meteor activity function}
    \eqn{eq:i-map}{
        \varrho(t, \vec{v}, m) \equiv \varrho(t, \delta_R, \alpha_R, v_\infty, m).
    }

    Necessarily, no observational model is able to capture or describe any stream that does not intersect the orbit of the
    Earth.\footnote{This condition is by no means sufficient. In certain pathological cases, such as a fresh stream in resonance
    with the Earth, which has not yet fully spread along its orbit, it is possible no meteors will collide with the Earth even
    if their orbit intersects with the Earth's.}

    In visualisations it is natural to keep the spatial and temporal dimensions as they are -- that is,
    right ascension and declination are displayed as spatial coordinates using one of suitable projections,
    while time evolution of streams is best displayed in a sequence of images.
    Displaying the secular evolution of one stream can be performed by comparing the state of the function $M$ at
    the same $\lambda_\odot$ in successive years.

    Within such maps we should be able to distinguish two main features:
    \begin{itemize}
        \item the \emph{sporadic background}, which forms the main contour lines of the maps, the ``terrain'';
        \item and several well-defined, sharp peaks, representing the \emph{meteor showers}. Due to common origin and orbit,
            these peaks should be very narrow in all dimensions.
    \end{itemize}

    Due to influences of various acting forces, such as differences
    in initial velocities with respect to the parent body, tidal disruptions,
    perturbations arising from close encounters with large bodies of the Solar System,
    Poynting-Robertson effect, etc., the stream
    gradually widens and disperses in all components until it can no longer be distinguished from the background.

    \subsection{Declination and right ascension} \label{ios}
        The two spatial coordinates mark the position of the \emph{true radiant} of the particles.
        As we are only concerned with the density of the stream as a whole,
        the exact position of the observed meteor in the sky is immaterial.
        Even a minuscule change in velocity or ejection time could have had a drastic effect
        on the precise timing and geometry of the meteor without affecting
        the position of the radiant or the meteoroid's adherence to the stream.

        A significant difficulty is encountered with daytime meteor showers, whose radiants
        lie very close to the current position of the Sun.
        Most particles will then enter the atmosphere on the day side
        of the Earth, where a huge majority of particles burns up unnoticed.
        With radar observations it is possible to track the activity as well,
        however this observation technique calls for different methods of processing and debiasing the data.
        We estimate that using purely visual data it is impossible (or at least very difficult) to establish fluxes
        for meteor showers whose radiants are within \ang{15} from the Sun, and significant bias
        is introduced for angles up to \ang{60}.

    \subsection{Validity of the model} \label{iav}
        Since the model is to be constructed solely with data from visual observations,
        its range of validity will admittedly be rather limited.
        Meteoroids smaller than about \SI{1}{\milli\metre} or \SI{e-7}{\kilo\gram} are usually too faint
        to be detected by automated observations in visible light and should be investigated
        by other methods instead.

        On the other side of the spectrum, particles larger than about \SI{0.05}{\metre} ($\approx$ \SI{0.1}{\kilo\gram})
        are comparatively rare even in the most active showers.
        The error bars on flux derived from only a handful of observations are inevitably going to be very large.
        However, the later stages of the processing pipeline should not be affected by these limitations
        and hence it should be possible to extend the model easily with other data once they are available.

    \subsection{Visualisation} \label{iov}
        High-dimensional data are relatively hard to comprehend. Substantial care should be taken when designing
        a human-oriented visualisation


        The natural solution is to keep spatial coordinates as spatial and the time as time,
        though due to limitations of some media alternative solutions have to be considered.
        The major difficulty lies in displaying the remaining two quantities simultaneously.
        for instance in terms of colour and saturation.

        \subsubsection{Spatial coordinates} \label{iovs}
            The two spatial coordinates 
            The only thing to consider is to how to present spherical data in a flat display,
            which is a standard task in wide-field observational astronomy.
            While not strictly necessary, it makes sense to choose a projection
            that preserves the area, so as not to confuse the viewer.
            One suitable example is the \emph{Mollweide} projection.

        \subsubsection{Time} \label{iovt}
            If possible, the best option is to preserve the time coordinate as time
            and display the development of meteor activity in a video, at some appropriate scale.
            Otherwise a sequence of discrete images is suitable as well.

        \subsubsection{Mass} \label{iovm}
            The mass distribution of meteoroids is universally heavily shifted
            towards smaller particles (see \cref{mpam}), at least within the mass
            range that can be covered by visual observations.
            The KDE method

            it is easier and more meaningful to show the logarithm of mass $\log m$ instead.

            Several problems will have to be addressed as well. For instance, in the case
            of two separate populations with different mass indices mixing the resulting
            mass index 

        \subsubsection{Other properties} \label{iovo}
           The rest of 

        \subsubsection{Example} \label{iove}
            Put all together

            \sfig{example.png}{\textwidth}{fig:iov-mockup}{
                An example of the sky map for one night of observations.
                Individual meteoroid radiants are pictured as dots,
                with larger dots corresponding to more massive particles.
                In the background the underlying distribution of radiants is obtained
                using a spherical KDE method with a Gaussian kernel.

                In the picture are displayed 10000 sporadic meteors with uniform distribution of radiants
                and three fictional meteor showers with different dispersion profiles.
            }



\section{Objectives of the thesis} \label{iO}
    In the thesis we are planning to simulate the orbital evolution and production of meteors
    by several most prominent meteor showers.

    \begin{itemize}
        \item develop a model of meteoroid distribution in the inner Solar System;
        \item reduce the model to the Earth's atmosphere and obtain a sky map
            of observable meteor activity at any place and time,
            such as the one depicted in \cref{fig:iov-mockup};

    \end{itemize}


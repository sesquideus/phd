\Epigraph[0.4]{
    All work and no play makes Jack a dull boy,\\
    All play and no work makes Jack a mere toy.
}{
    \textit{English proverb} \\
    \textit{enhanced by} \textsc{Maria Edgeworth} \\
    \textit{in} Harry and Lucy Concluded (1825)
}

The described design forms the foundation of future work, upon which the actual implementation will be built later.
Naturally, the project is currently open-ended and might change if difficulties are encountered,
or conversely, more appropriate methods are found.

%\section{Models}
%    Even though meteoroids are discrete particles, it is useful to model their density
%    as a smooth function in multi-dimensional coordinate space.
%    Actual particles are then treated as random samples obtained from the underlying distribution.
%
%    However, logic dictates we do this in the opposite direction: we need to approximate
%    the underlying distribution from a finite set of known samples.
%    This may be done using a \emph{kernel density estimation} method.
%    Each actual meteoroid, obtained by backwards integration of motion of an observed meteor,
%    is substituted by a multidimensional kernel in the coordinate space. The precise shape of the kernel
%    is not overly important; for most purposes a simple multidimensional gaussian is sufficient.
%    To make matters even worse, our observational datasets are usually incomplete
%    and heavily distorted by observation bias.
%
%    The final product thus consists of two parts: the \emph{observational model},
%    obtained from observations of meteors from the Earth's surface;
%    and an extended \emph{orbital model}, which describes the density of matter
%    in interplanetary space as a function of position, velocity and time.

\section{Current state of the project} \label{pc}
    In our master's thesis \citep{balaz-thesis} we used a parametric Monte-Carlo simulation
    to determine the total flux of Perseids.
    The simulation in the atmosphere has been improved since, however, its basic principles
    remain unchanged. It can be used without modifications as long as the position of the radiant,
    the true velocity of the particles and the physical properties of the material are known.
    Running the simulation with the completed orbital model would output a sample statistically
    similar to actual observational data.

    In the thesis we assumed a constant radiant and a very brief time interval,
    which essentially represented only a single point in the $M$ coordinate space
    except for the mass, which was sampled from a Pareto distribution for multiple values of $s$,
    specifically
    \eqn{eq:pc-thesis}{
        \mathcal{A}(
            t = 2016 + \ang{140},
            \delta_R = \ang{56},
            \alpha_R = \ang{43},
            v_\infty = \SI{59}{\kilo\metre\per\second},
            \mu \sim \mathrm{Pareto}(\mu_0, s)
        ).
    }
    The virtual sample for this meteor shower were in good agreement with the observational dataset
    obtained by AMOS.

    \subsection{Validity of the model} \label{iav}
        Since the model is to be constructed solely with data from visual observations,
        its range of validity will admittedly be rather limited.
        Meteoroids smaller than about \SI{1}{\milli\metre} or \SI{e-7}{\kilo\gram} are usually too faint
        to be detected by automated observations in visible light and should be investigated
        by other methods instead.

        On the other side of the spectrum, particles larger than about \SI{0.05}{\metre} ($\approx$ \SI{0.1}{\kilo\gram})
        are comparatively rare even in the most active showers.
        The error bars on flux derived from only a handful of observations are inevitably going to be very large.
        However, the later stages of the processing pipeline should not be affected by these limitations
        and hence it should be possible to extend the model easily with other data once they are available.

        As we have mentioned, the orbital model will not include meteoroids whose orbits do not
        intersect the orbit of the Earth. The observational model should be valid for any location on the Earth.
        Due to secular changes in density and positions of the streams, we do not expect the model to cover
        a longer time interval, perhaps on the order of one year.

    \subsection{Visualisation} \label{iov}
        Data spread in more than three dimensions are difficult to comprehend.
        Substantial care should be taken when designing a human-oriented visualisation.

        The natural solution is to keep spatial and temporal dimensions as they are:
        time is best displayed in a video or in a sequence of images.
        In the observational model, right ascension and declination are displayed as spatial coordinates after being
        flattened using one of the suitable projections.
        In the orbital model there are six separate coordinates, which are difficult to display at the same time.
        The options are to either design a three-dimensional animation, or to only display data with reduced dimensions.

        A major difficulty lies in displaying the remaining two quantities -- speed and mass distribution -- simultaneously.
        Colour mappings may be added, or both quantities may be considered separately.
        For an example with simple artificial data, see \cref{e}.

        \subsubsection{Spatial coordinates} \label{iovs}
            The two spatial coordinates, declination $\delta_R$ and right ascension $\alpha_R$ are preserved.
            The only thing to consider is how to present spherical data in a flat display,
            which is a fairly standard task in wide-field observational astronomy.
            The projection used should be an equal-area one, so as to preserve the spatial density of radiants as well.
            One suitable example is the \emph{Mollweide} projection.

            \fig{mollweide.jpg}{\textwidth}{fig:iovs-mollweide}{
                The Mollweide projection used to display the surface of the Earth.
                The projection preserves the area and thus, in our case,
                the density of meteoroid radiants.

                \hfill\textit{Picture by Daniel R. Strebe, 2011, CC BY-SA 3.0}
            }

        \subsubsection{Time} \label{iovt}
            If possible, the best option is to preserve the time coordinate as time
            and display the development of meteor activity in a video, at some appropriate scale.
            Otherwise a sequence of discrete images is suitable as well.
            Displaying the secular evolution of one particular meteor shower stream can be done
            by comparing the function $\mathcal{A}$ at the same $\lambda_\odot$ in successive years.

        \subsubsection{Masses and magnitudes} \label{iovm}
            It is not possible to display the entire information about the spectrum at every point
            in other coordinates. Therefore only the reduced quantities will be displayed,
            namely the values of indices $r$ or $s$ depending on the model.

            Several problems will have to be addressed -- for instance, in the case of
            two separate populations with different mass indices coming from the same radiant,
            it is not possible to describe the resulting population with a single value of $r$ or $s$.

\section{Roadmap} \label{ir}
    At this stage it is difficult to plan progress in detail. The final process may differ significantly.
    However, a rough roadmap may be outlined.

    \begin{itemize}
        \item In the first phase we should develop and revise the model of meteoroid
            distribution in the inner Solar System. A computationally tractable and sufficiently
            precise kernel method to estimate the probability density function must be found.
        \item We should design and implement an $N$-body simulation of small bodies in the Solar System,
            optionally a massively parallel one, which will be used to generate the model.
        \item Next we obtain a database of meteor observations and extract the relevant data from it.
            Here we should also develop and apply methods for removing selection bias from the data.
        \item Finally we use the simulation to approximate the true distribution of meteoroid particles
            in the vicinity of the Earth.
    \end{itemize}

    After the orbital model is complete, we may proceed to the second part, designing the observational model.
    \begin{itemize}
        \item The orbital model is reduced: meteoroids whose trajectories intersect the Earth
            are imported into the virtual observational dataset.
        \item We design another KDE method for determining the probability density function
            in the observational model. The KDE is applied to both the real database and the virtual dataset
            of meteors and compared. An algorithm for comparing the datasets must be devised.
            The virtual dataset is iteratively modified and regenerated until best possible match is found.
            The observational dataset must be consistent with the influx of particles in the orbital model.
            The orbital model will likely need to be modified as well.
        \item Finally we develop and test methods of visualising the data. We should be able to provide
            predictions of meteor activity and derive quantities such as the ZHR, the population index
            and the total number or mass flux.
    \end{itemize}

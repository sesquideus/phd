The primary objective of the thesis is the development of a full-scale orbital model of meteoroid populations
in the inner Solar System.

The fundamental difficulty here is that particle streams whose orbit does not intersect the orbit of the Earth
cannot be observed at all.

Hence the natural reference system to work with when investigating meteor streams in interplanetary space
is the \emph{ecliptic coordinate system}.


It is necessary that at least one of the orbital nodes must lie very close to the orbit of the Earth.
This can be expressed in terms of orbital elements as
\eqn{eq:p-anode}{
    \frac{p}{1 + e \cos \omega} \approx \SI{1}{\au}
}
for the ascending node, or, for the descending node,
\eqn{eq:p-dnode}{
    \frac{p}{1 - e \cos \omega} \approx \SI{1}{\au}.
}

Typically we do not consider individual particles, but the entire stream and its elliptical orbit around the Sun.

It is not entirely clear what constitutes a meteoroid stream.


The final product thus consists of two parts: the actual \emph{orbital model},
describing the density of meteoroid particles in interplanetary space;
and a reduced \emph{observational model}, which predicts the number of meteoroid
entering the Earth's atmosphere as a function of time and celestial coordinates.

\section{Orbital model} \label{po}

    Even though meteoroids are always discrete particles, it is useful to model their density as a smooth
    function in multi-dimensional phase space.
    Actual particles are then treated as samples obtained from the underlying distribution.

    In the opposite direction, the underlying distribution may be approximated using a \emph{kernel density estimation} method.
    Each actual meteoroid, obtained by backwards integration of motion of an observed meteor,
    is substituted by a multidimensional kernel in the phase space. The precise shape of the kernel
    is not overly important; for most purposes a simple multidimensional gaussian is sufficient \cite{...}



    The density function is 
    \eqn{eq:po-rv}{
        \varrho(\vec{r}, \vec{v}, t) \in \left[0, \infty\right).
    }

    or alternatively expressed in terms of orbital elements
    \eqn{eq:po-elem}{
        \varrho(a, e, i, \omega, \Omega, T).
    }

    In meteor science it is customary to assume a specific distribution of masses in a population,
    described by a function that follows a power-law





\section{Reduction to Earth's sky} \label{ir}
    While the primary model describes the actual distribution of meteoroid particles in space around the Earth,
    it is not possible to relate it directly to observations. To do this, we need to \emph{reduce} the model,
    that is, find out which particles will intersect the Earth and produce a meteor, and compute the geometry of this collision.

    Since this is the only observational data are available,
    models developed from data obtained from Earth-based observations are different.

    The final product would thus be a four-dimensional map of distribution of meteor activity, with independent variables being
    \begin{itemize}
        \item one \emph{temporal coordinate}, representing the meteor activity in time;
        \item two spatial coordinates: the \emph{right ascension} $\alpha_R$ and \emph{declination} $\delta_R$ of the radiant;
        \item the \emph{geocentric speed} of the particles;
        \item and the \emph{mass spectrum}, describing the distribution of meteoroid masses in terms of \textit{mass index} $s$.
    \end{itemize}

    described by the \emph{meteor activity function}
    \eqn{eq:i-map}{
        M(t, \vec{v}, m) \equiv \rho(y + \lambda_\odot, \delta_R, \alpha_R, v, m).
    }

    Necessarily, the model is not able to capture or describe any stream that does not intersect the orbit of the
    Earth,\footnote{This condition is by no means sufficient. In certain pathological cases, such as a fresh stream in resonance
    with the Earth, which has not yet fully spread along its orbit, it is possible no meteors will collide with the Earth even
    if their orbit intersects with the Earth's.}

    In visualisations it is natural to keep the spatial and temporal dimensions, that is, right ascension and declination are
    displayed as spatial coordinates using one of suitable projections, while time evolution of streams is best displayed
    in sequence of images.
    Displaying the secular evolution of one stream can be performed by comparing the state of the function $M$ at
    the same $\lambda_\odot$ in successive years.

    \citep{balaz+2020}

    Within this map we are able to distinguish two main features:
    \begin{itemize}
        \item the \emph{sporadic background}, which forms the main contour lines of the maps, the ``terrain'';
        \item and the well-defined, sharp peaks, representing the \emph{meteor showers}. Due to common origin and orbit,
            these peaks should be very narrow in all dimensions.
    \end{itemize}

    Due to influences of various acting forces, such as differences
    in initial velocities with respect to the parent body, tidal disruptions,
    perturbations arising from close encounters with large bodies of the Solar System,
    Poynting-Robertson effect, etc., the stream
    gradually widens and disperses in all components until it can no longer be distinguished from the background.

    \subsection{Time} \label{i}
        In examining the evolution we may recognize two distinct components:
        \begin{itemize}
            \item the \emph{periodic component}, emerging as the result of the Earth orbiting the Sun.
                This coordinate can be thus expressed in terms of the \emph{longitude of the sun} $\lambda_\odot$.

            \item the slowly-varying \emph{secular component}, associated with appearance of new meteoroid streams and gradual decay of older
                streams which are not replenished.

                The peaks produced by meteor showers appear abruptly after the parent body passes through the vicinity of the Earth,
                producing small meteoroid due to heating or outgassing.
                Returns of parents bodies on periodic orbits, such as comets 1P/Halley or 109P/Swift-Tuttle, result
                in resupply of meteoroid material, and thus form an extra periodic component, included in the secular evolution of the stream.

        \end{itemize}

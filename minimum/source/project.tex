\Epigraph[0.4]{
    You saw sagacious Solomon \\
    You know what came of him \\
    To him, complexities seemed plain
}{
    How Fortunate the Man With None \\
    \textsc{Bertolt Brecht} \\
    \textit{translation John Willett}
}

The primary objective of the thesis is to develop a model of meteoroid populations intersecting
the orbit of the Earth -- or more precisely, populations observable from the surface as meteors.
The model must be in agreement with data obtained by visual observers and automated devices watching the sky.

Raw data are invariably skewed by observational bias. Even though elaborate methods for correction of various
effects have been developed,


Typically we do not consider individual particles, but the entire stream and its elliptical orbit around the Sun.

\section{Models}
    Even though meteoroids are discrete particles, it is useful to model their density
    as a smooth function in multi-dimensional coordinate space.
    Actual particles are then treated as random samples obtained from the underlying distribution.

    However, logic dictates we do this in the opposite direction: we need to approximate
    the underlying distribution from a finite set of known samples.
    This may be done using a \emph{kernel density estimation} method.
    Each actual meteoroid, obtained by backwards integration of motion of an observed meteor,
    is substituted by a multidimensional kernel in the coordinate space. The precise shape of the kernel
    is not overly important; for most purposes a simple multidimensional gaussian is sufficient.
    To make matters even worse, our observational datasets are usually incomplete
    and heavily distorted by observation bias.

    The final product thus consists of two parts: the \emph{observational model},
    obtained from observations of meteors from the Earth's surface;
    and an extended \emph{orbital model}, which describes the density of matter
    in interplanetary space as a function of position, velocity and time.

\section{Current state of the project} \label{pc}
    In our master's thesis \citep{balaz-thesis} we used a parametric Monte-Carlo simulation
    to determine the total flux 

    We assumed a constant radiant and a very brief time interval,
    which essentially represented only a single point in the $M$ coordinate space
    except for the mass, which was sampled from a Pareto distribution for multiple values of $s$,
    specifically
    \eqn{eq:pc-thesis}{
        \mathcal{A}(t = 2016 + \ang{140}, \delta_R = \ang{56}, \alpha_R = \ang{43}, v_\infty = \SI{59}{\kilo\metre\per\second}, \mu \sim \mathrm{Pareto}(\mu_0, s)).
    }


    \subsection{Validity of the model} \label{iav}
        Since the model is to be constructed solely with data from visual observations,
        its range of validity will admittedly be rather limited.
        Meteoroids smaller than about \SI{1}{\milli\metre} or \SI{e-7}{\kilo\gram} are usually too faint
        to be detected by automated observations in visible light and should be investigated
        by other methods instead.

        On the other side of the spectrum, particles larger than about \SI{0.05}{\metre} ($\approx$ \SI{0.1}{\kilo\gram})
        are comparatively rare even in the most active showers.
        The error bars on flux derived from only a handful of observations are inevitably going to be very large.
        However, the later stages of the processing pipeline should not be affected by these limitations
        and hence it should be possible to extend the model easily with other data once they are available.

    \subsection{Visualisation} \label{iov}
        Data spread in more than three dimensions are difficult to comprehend.
        Substantial care should be taken when designing a human-oriented visualisation.

        The natural solution is to keep spatial and temporal dimensions as they are:
        time is best displayed in a video or in a sequence of images.
        In the observational model right ascension and declination are displayed as spatial coordinates after being
        flattened using one of the suitable projections.
        In the orbital model there are six separate coordinates, which are difficult to display at the same time.
        The options are to either design a three-dimensional animation, or to only display data with reduced dimensions.

        A major difficulty lies in displaying the remaining two quantities -- speed and mass distribution -- simultaneously.
        Colour mappings to saturation an hue may be added, or both quantities may be considered only separately.
        For an example with simple artificial data see \cref{e}.

        \subsubsection{Spatial coordinates} \label{iovs}
            The two spatial coordinates, declination $\delta_R$ and right ascension $\alpha_R$ are preserved.
            The only thing to consider is to how to present spherical data in a flat display,
            which is a fairly standard task in wide-field observational astronomy.
            The projection used should be an equal-area one, so as to preserve the spatial density of radiants as well.
            One suitable example is the \emph{Mollweide} projection.

            \fig{mollweide.jpg}{\textwidth}{fig:iovs-mollweide}{
                The Mollweide projection used to display the surface of the Earth.
                The projection preserves the area and thus, in our case,
                the density of meteoroid radiants.

                \hfill\textit{Picture by Daniel R. Strebe, 2011, CC BY-SA 3.0}
            }

        \subsubsection{Time} \label{iovt}
            If possible, the best option is to preserve the time coordinate as time
            and display the development of meteor activity in a video, at some appropriate scale.
            Otherwise a sequence of discrete images is suitable as well.
            Displaying the secular evolution of one particular meteor shower stream can be done
            by comparing the function $\mathcal{A}$ at the same $\lambda_\odot$ in successive years.

        \subsubsection{Masses and magnitudes} \label{iovm}
            It is not possible to display the entire information about the spectrum at every point
            in other coordinates. Therefore only the reduced quantities will be displayed,
            namely the values of indices $r$ or $s$ depending on the model.

            Several problems will have to be addressed -- for instance, in the case of
            two separate populations with different mass indices coming from the same radiant,
            it is not possible to describe the resulting population with a single value of $r$ or $s$.

    \todo{finish this}

\section{Prerequisites} \label{ip}
    We need to understand and develop a suitable KDE model, presumably AKDE,
    which will be able to work with sparse data and in non-Euclidean spaces.

\section{Objectives of the thesis} \label{it}
    \begin{itemize}
        \item develop a model of meteoroid distribution in the inner Solar System;
        \item devise a sufficiently precise KDE method 
        \item reduce the model to the Earth's atmosphere and obtain a sky map
            of observable meteor activity at any place and time,
            such as the one depicted in \cref{fig:e-mockup};
    \end{itemize}


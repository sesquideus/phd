The primary objective of the thesis is the development of a model of meteoroid populations
intersecting the orbit of the Earth and hence observable from the surface as meteors.

As our primary data source we are going to use the data from AMOS, the All-sky Meteor Orbit System \citep{zigo+2013}.




The fundamental difficulty here is that particle streams whose orbit does not intersect the orbit of the Earth
cannot be observed at all.

Hence the natural reference system to work with when investigating meteor streams in interplanetary space
is the \emph{ecliptic coordinate system}.



Typically we do not consider individual particles, but the entire stream and its elliptical orbit around the Sun.

\section{Models}
    Even though meteoroids are always discrete particles, it is useful to model their density
    as a smooth function in multi-dimensional phase space.
    Actual particles are then treated as random samples obtained from the underlying distribution.

    However, logic dictates we do this in the opposite direction: we need to approximate
    the underlying distribution from a finite set of known samples.
    This may be done using a \emph{kernel density estimation} method.
    Each actual meteoroid, obtained by backwards integration of motion of an observed meteor,
    is substituted by a multidimensional kernel in the phase space. The precise shape of the kernel
    is not overly important; for most purposes a simple multidimensional gaussian is sufficient \cite{...}.
    To make matters even worse, our observational datasets are usually incomplete
    and heavily distorted by observation bias.

    The final product thus consists of two parts: the \emph{observational model},
    obtained from observations of meteors from the Earth's surface;
    and an extended \emph{orbital model}, which describes the density of matter
    in interplanetary space as a function of position, velocity and time.

\section{Observational model} \label{io}
    We need to use a four-dimensional KDE. Ideally, the kernel should be obtained by a product of five
    independent unidimensional kernels.

    The final product would thus be a five-dimensional map of distribution of meteor activity, with independent variables being
    \begin{itemize}
        \item one \emph{temporal coordinate}, representing the meteor activity in time;
        \item two spatial coordinates: the \emph{right ascension} $\alpha_R$ and \emph{declination} $\delta_R$ of the radiant;
        \item the \emph{pre-atmospheric geocentric speed} of the particles $v_\infty$;
        \item and the \emph{mass spectrum}, or the probability density of distribution of meteoroid masses.
    \end{itemize}

    The total observable activity is thus described by the \emph{meteor activity function}
    \eqn{eq:i-map}{
        \varrho(t, \vec{v}, m) \equiv \varrho(t, \delta_R, \alpha_R, v_\infty, m).
    }

    Necessarily, no observational model is able to capture or describe any stream that does not intersect the orbit of the
    Earth.\footnote{This condition is by no means sufficient. In certain pathological cases, such as a fresh stream in resonance
    with the Earth, which has not yet fully spread along its orbit, it is possible no meteors will collide with the Earth even
    if their orbit intersects with the Earth's.}

    In visualisations it is natural to keep the spatial and temporal dimensions as they are -- that is,
    right ascension and declination are displayed as spatial coordinates using one of suitable projections,
    while time evolution of streams is best displayed in a sequence of images.
    Displaying the secular evolution of one stream can be performed by comparing the state of the function $M$ at
    the same $\lambda_\odot$ in successive years.

    Within such maps we should be able to distinguish two main features:
    \begin{itemize}
        \item the \emph{sporadic background}, which forms the main contour lines of the maps, the ``terrain'';
        \item and several well-defined, sharp peaks, representing the \emph{meteor showers}. Due to common origin and orbit,
            these peaks should be very narrow in all dimensions.
    \end{itemize}

    Due to influences of various acting forces, such as differences
    in initial velocities with respect to the parent body, tidal disruptions,
    perturbations arising from close encounters with large bodies of the Solar System,
    Poynting-Robertson effect, etc., the stream
    gradually widens and disperses in all components until it can no longer be distinguished from the background.

    \subsection{Time} \label{iot}
        The first coordinate is the time of the atmospheric entry of the meteoroid.
        First of all, it should be noted that while a meteor is not an completely instantaneous event,
        its duration is on the order of \SI{1}{\second}, which is very short compared to the
        period of rotation of the Earth of the scales on which meteor activity varies.
        Precise timekeeping is thus not very important.
        For our purposes we will be using the time of maximum brightness
        of the meteor for both actual data and in simulations.

        Second, it should be emphasized again that each meteor is treated
        as a random sample from some hidden underlying distribution.
        In the context of this thesis it is immaterial whether any particular meteor
        appeared a minute earlier or later. Since we are only interested in slow variations,
        any such time shift will be invariably smeared during density estimation.
        Differences on the order of tens of minutes could be visible with very narrow filaments.

        In examining the evolution of streams we should recognize two distinct components:
        \begin{itemize}
            \item the \emph{periodic component}, emerging as the result of the Earth orbiting the Sun.
                This should be thus expressed in terms of the \emph{longitude of the sun} $\lambda_\Sun$.

            \item the slowly-varying \emph{secular component}, associated with appearance of new meteoroid streams and gradual decay of older
                streams which are not replenished.

                The peaks produced by meteor showers appear abruptly after the parent body passes through the vicinity of the Earth,
                producing small meteoroid due to heating or outgassing.
                Returns of parents bodies on periodic orbits, such as comets 1P/Halley or 109P/Swift-Tuttle, result
                in resupply of meteoroid material, and thus form an extra periodic component, included in the secular evolution of the stream.
        \end{itemize}

        The time coordinate $t$ can be thus decomposed to two components,
        \eqn{eq:iot-tyl}{
            t = y + \lambda_\Sun,
        }
        where
        \begin{itemize}
            \item $y$ is the number of whole orbits since some specified time, or in layman's terms, the current \emph{astronomical year} in which
                the last vernal equinox occurred;
            \item $\lambda_\Sun$ is the \emph{solar longitude}, specified in degrees, which represents the fractional part of the year.
        \end{itemize}

        As we do not expect to cover more than several tens or hundreds of years, effects of precession can be safely ignored.
        Under this convention, August \nth{12}, 2020 roughly corresponds to $y = 2020$ and $\lambda_\Sun = \ang{139.4}$,
        while March \nth{14}, 2020 is denoted as $y = 2019$ and $\lambda_\Sun = \ang{353.6}$.

    \subsection{Declination and right ascensions} \label{ios}
        The two spatial coordinates mark the position of the radiant of the particles.

        As we are only concerned with the density of the stream as a whole,
        the exact position in the sky is again not important.
        Even a minuscule change in velocity or ejection time could have had a drastic effect
        on the precise timing and geometry of the meteor without affecting the meteoroid's
        adherence to the stream.

        f the meteoroid which produced the meteor
        had entered the atmosphere several hundred kilometres further,
        which would have drastically changed the position in the sky, while the
        Again, any difference will be lost in kernel smearing.


\section{Orbital model} \label{po}
    For each meteor, it is in principle possible to trace the trajectory back in time
    and derive the trajectory of the original meteoroid particle in the vicinity of Earth.
    After subtracting the velocity of the Earth we can further extend the trajectory
    into interplanetary space.

    Since this is the only observational data are available,
    models developed from data obtained from Earth-based observations are different.

  


    The density function is 
    \eqn{eq:po-rv}{
        \varrho(\vec{r}, \vec{v}, t) \in \left[0, \infty\right).
    }

    or alternatively expressed in terms of orbital elements
    \eqn{eq:po-elem}{
        \varrho(a, e, i, \omega, \Omega, T).
    }


    While practically all particles that collide with the Earth are removed from the stream,
    the total volume swept by the Earth through the stream is negligible for all but the narrowest streams.
    We do not expect any observable reduction in the number of meteoroids during subsequent approaches,
    however, this will have to be confirmed by simulations.

\section{Objectives of the thesis} \label{iO}
    In the thesis we are planning to simulate the orbital evolution and production of meteors
    by several most prominent meteor showers.

    \begin{itemize}
        \item develop a full-scale, reliable model of meteoroid distribution in the inner Solar System,
            which could be used to predict meteor activity 
    \end{itemize}


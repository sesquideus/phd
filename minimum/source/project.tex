The primary objective of the thesis is the development of a model of meteoroid populations
intersecting the orbit of the Earth -- or more precisely, populations observable from the surface as meteors.
The model must be in agreement with data obtained by visual observers and automated devices watching the sky.

Raw data are invariably skewed by observational bias. Even elaborate methods for correction of various
effects have been developed,



Hence the natural reference system to work with when investigating meteor streams in interplanetary space
is the \emph{ecliptic coordinate system}.



Typically we do not consider individual particles, but the entire stream and its elliptical orbit around the Sun.

\section{Models}
    Even though meteoroids are discrete particles, it is useful to model their density
    as a smooth function in multi-dimensional phase space.
    Actual particles are then treated as random samples obtained from the underlying distribution.

    However, logic dictates we do this in the opposite direction: we need to approximate
    the underlying distribution from a finite set of known samples.
    This may be done using a \emph{kernel density estimation} method.
    Each actual meteoroid, obtained by backwards integration of motion of an observed meteor,
    is substituted by a multidimensional kernel in the phase space. The precise shape of the kernel
    is not overly important; for most purposes a simple multidimensional gaussian is sufficient \cite{???}.
    To make matters even worse, our observational datasets are usually incomplete
    and heavily distorted by observation bias.

    The final product thus consists of two parts: the \emph{observational model},
    obtained from observations of meteors from the Earth's surface;
    and an extended \emph{orbital model}, which describes the density of matter
    in interplanetary space as a function of position, velocity and time.

\section{Processing the observations} \label{ip}
    \todo{corrections for observers, limiting magnitude}

    \subsection{Quantities} \label{ipq}
        \subsubsection{Zenithal hourly rate} \label{ipqz}
            The \emph{zenithal hourly rate} (ZHR) is defined as ``[t]he number of shower meteors per hour
            one observer would see if his limiting magnitude is \Mag{+6.5} and the radiant is in his zenith''.

            is somewhat applicable to automatic devices as well.

            ZHR is defined as
            \eqn{eq:ipqz-zhr}{
                \mathrm{ZHR} = \dfrac{\dfrac{N}{T} \cdot \dfrac{1}{k} \cdot r^{\num{6.5} - m_0}}{\sin{\theta_\mathrm{R}}}
            }
            where
            \begin{description}
                \item[$N$]      is the number of meteors recorded by the observer;
                \item[$T$]      is total effective observation time;
                \item[$k$]      is the fraction of the sky that is visible to the observer;
                \item[$r$]      is the \emph{population index} of the meteor shower;
                \item[$m_0$]    is the \emph{limiting magnitude} of the observer.
            \end{description}

            In our  the ZHR is a computed quantity, completely determined by the model.

        \subsubsection{Limiting magnitude} \label{ipqm}
            The atmosphere is constantly changing due to weather, contamination
            by natural or man-made aerosols or light pollution. The detection efficiency
            of ground-based observers is thus variable and must be included in any analysis.

            In meteor astronomy, the quality of the conditions is usually summarized
            using a single number called the \emph{limiting magnitude}, denoted $m_0$.
            It is defined as ``the apparent magnitude of the faintest stars visible during the observation'' \citep{imo-glossary}.

            In visual observations it is usually assumed that a meteor can be detected if and only if
            its apparent magnitude is at most $m_0$. Mathematically, this is described by a Heaviside step function.
            For photographic and video observations a slightly lower value is typically used.

            Rather than using a sharp limit, a slightly better approach is to assume a sigmoid profile,
            with a smooth transition between very faint meteors (none of which can be recorded)
            and very bright meteors (where it is reasonable to assume they cannot be missed).
            While there is no universally agreed-upon function, a simple yet reasonable function is
            \eqn{eq:ipqm-sigmoid}{
                D(m; m_0, \omega) = \frac{1}{1 + \exp\left(\frac{m-m_0}{\omega}\right)},
            }
            where
            \begin{description}
                \item[$m_0$]    is the \textit{limiting magnitude} where detection probability is 50\%;
                \item[$\omega$] is the width of the function, with higher values representing
                    faster rate of loss of detection efficiency with decreasing brightness.
                    The limit $\omega \to \infty$ corresponds to a sharp cut-off at $m_0$,
                    or the Heaviside step function.
            \end{description}

            Ideally, a functional dependence on the distance from the centre of the field
            of view should be taken into account for human observers, as the eye is much more
            sensitive to light near the yellow spot \todo{who?}.
            Similarly, digital sensors are subject to vignetting and other detrimental effects.
            Further research into the angular response of the human eye would be beneficial.

    Another method has been discussed and implemented in the author's master's thesis \citep{balaz-thesis}
    and used to determine the total flux of the Perseids in 2016. Instead of trying to
    estimate the biases we generated the meteoroids above the atmosphere and simulated
    their entry using a simplified set of equations for motion, ablation and luminosity.
    A collection of stochastic bias functions was then applied to the dataset and
    each meteor was marked as detected or missed. We varied the parameters of the bias functions
    and the mass index until agreement with observational data was achieved.
    The initial population was then declared as the model of the actual population
    and the bias functions were taken as descriptive of the observing system.

\section{Observational model} \label{io}
    (...)

    We need to use a four-dimensional KDE. Ideally, the kernel should be obtained by a product of five
    independent one-dimensional kernels.

    The final product would thus be a five-dimensional map of distribution of meteor activity, with independent variables being
    \begin{itemize}
        \item one temporal coordinate $t$, representing the meteor activity in \emph{time};
        \item two spatial coordinates: the \emph{right ascension} $\alpha_R$ and \emph{declination} $\delta_R$ of the radiant;
        \item the \emph{pre-atmospheric geocentric speed} of the particles $v_\infty$;
        \item and the \emph{mass spectrum}, or the probability density of distribution of meteoroid masses.
    \end{itemize}

    The total observable activity is thus described by the \emph{meteor activity function}
    \eqn{eq:i-map}{
        \varrho(t, \vec{v}, m) \equiv \varrho(t, \delta_R, \alpha_R, v_\infty, m).
    }

    Necessarily, no observational model is able to capture or describe any stream that does not intersect the orbit of the
    Earth.\footnote{This condition is by no means sufficient. In certain pathological cases, such as a fresh stream in resonance
    with the Earth, which has not yet fully spread along its orbit, it is possible no meteors will collide with the Earth even
    if their orbit intersects with the Earth's.}

    In visualisations it is natural to keep the spatial and temporal dimensions as they are -- that is,
    right ascension and declination are displayed as spatial coordinates using one of suitable projections,
    while time evolution of streams is best displayed in a sequence of images.
    Displaying the secular evolution of one stream can be performed by comparing the state of the function $M$ at
    the same $\lambda_\odot$ in successive years.

    Within such maps we should be able to distinguish two main features:
    \begin{itemize}
        \item the \emph{sporadic background}, which forms the main contour lines of the maps, the ``terrain'';
        \item and several well-defined, sharp peaks, representing the \emph{meteor showers}. Due to common origin and orbit,
            these peaks should be very narrow in all dimensions.
    \end{itemize}

    Due to influences of various acting forces, such as differences
    in initial velocities with respect to the parent body, tidal disruptions,
    perturbations arising from close encounters with large bodies of the Solar System,
    Poynting-Robertson effect, etc., the stream
    gradually widens and disperses in all components until it can no longer be distinguished from the background.

    \subsection{Time} \label{iot}
        The first coordinate is the time of the atmospheric entry of the meteoroid.
        First of all, it should be noted that while a meteor is not an completely instantaneous event,
        its duration is on the order of \SI{1}{\second}, which is very short compared to the
        period of rotation of the Earth of the scales on which meteor activity varies.
        Precise timekeeping is thus not very important.
        For our purposes we will be using the time of maximum brightness
        of the meteor for both actual data and in simulations.

        Second, it should be emphasized again that each meteor is treated
        as a random sample from some hidden underlying distribution.
        In the context of this thesis it is immaterial whether any particular meteor
        appeared a minute earlier or later. Since we are only interested in slow variations,
        any such time shift will be invariably smeared during density estimation.
        Differences on the order of tens of minutes could be visible with very narrow filaments.

        In examining the evolution of streams we should recognize two distinct components:
        \begin{itemize}
            \item the \emph{periodic component}, emerging as the result of the Earth orbiting the Sun.
                This should be thus expressed in terms of the \emph{longitude of the sun} $\lambda_\Sun$.

            \item the slowly-varying \emph{secular component}, associated with appearance of new meteoroid
                streams and gradual decay of older streams which are not replenished.

                The peaks produced by meteor showers appear abruptly after the parent body passes through the vicinity of the Earth,
                producing small meteoroid due to heating or outgassing.
                Returns of parents bodies on periodic orbits, result in resupply of meteoroid material,
                and thus form an extra periodic component, included in the secular evolution of the stream.
        \end{itemize}

        The time coordinate $t$ can be thus decomposed into two components,
        \eqn{eq:iot-tyl}{
            t = y + \lambda_\Sun,
        }
        where
        \begin{itemize}
            \item $y$ is the number of whole orbits since year 0, or in layman's terms,
                the current \emph{astronomical year} in which the last vernal equinox occurred;
            \item $\lambda_\Sun$ is the \emph{solar longitude}, specified in degrees, which represents the fractional part of the year.
        \end{itemize}

        As we do not expect to cover more than several tens or hundreds of years, effects of precession can be safely ignored.
        Under this convention, August \nth{12}, 2020 roughly corresponds to $y = 2020$ and $\lambda_\Sun \approx \ang{139.4}$,
        while March \nth{14}, 2020 is denoted as $y = 2019$ and $\lambda_\Sun \approx \ang{353.6}$.

    \subsection{Declination and right ascensions} \label{ios}
        The two spatial coordinates mark the position of the radiant of the particles.

        As we are only concerned with the density of the stream as a whole,
        the exact position of the observed meteor in the sky is immaterial.
        Even a minuscule change in velocity or ejection time could have had a drastic effect
        on the precise timing and geometry of the meteor without affecting the meteoroid's
        adherence to the stream.

        If the meteoroid which produced the meteor
        had entered the atmosphere several hundred kilometres further,
        which would have drastically changed the position in the sky, while the
        Again, any difference will be lost in kernel smearing.

        A significant difficulty is encountered with daytime meteor showers, whose radiants
        lie close to the current position of the Sun.
        Most particles will then enter the atmosphere on the day side
        of the Earth, where a huge majority of particles burns up unnoticed.
        With radar observations it is possible to track the activity as well,
        however this observation technique calls for different methods of processing and debiasing the data.
        We estimate that using purely visual data it is impossible (or at least very difficult) to establish fluxes
        for meteor showers whose radiants are within \ang{15} from the Sun.


    \subsection{Validity of the model} \label{iav}
        Since the model is to be constructed solely with data from visual observations,
        its range of validity will admittedly be rather limited.
        Meteoroids smaller than about \SI{1}{\milli\metre} or \SI{e-7}{\kilo\gram} are usually too faint
        to be detected by automated observations in visible light and should be investigated
        by other methods instead.

        On the other side of the spectrum, particles larger than about \SI{0.05}{\metre} ($\approx$ \SI{0.1}{\kilo\gram})
        are comparatively rare even in the most active showers.
        The error bars on flux derived from only a handful of observations are inevitably going to be very large.

        The 
        However, the later stages of the processing pipeline should not be affected by these limitations
        and hence it should be possible to extend the model easily with other data once they are available.

    \subsection{Visualisation} \label{iov}
        High-dimensional data are relatively hard to comprehend. Substantial care should be taken when designing
        a human-oriented visualisation


        The natural solution is to keep spatial coordinates as spatial and the time as time,
        though due to limitations of some media alternative solutions have to be considered.
        The major difficulty lies in displaying the remaining two quantities simultaneously.
        for instance in terms of colour and saturation.

        \subsubsection{Spatial coordinates} \label{iovs}
            The two spatial coordinates 
            The only thing to consider is to how to present spherical data in a flat display,
            which is a standard task in wide-field observational astronomy.
            While not strictly necessary, it makes sense to choose a projection
            that preserves the area, so as not to confuse the viewer.
            One suitable example is the \emph{Mollweide} projection.

        \subsubsection{Time} \label{iovt}
            If possible, the best option is to preserve the time coordinate as time
            and display the development of meteor activity in a video, at some appropriate scale.
            Otherwise a sequence of discrete images is suitable as well.

        \subsubsection{Mass} \label{iovm}
            The mass distribution of meteoroids is universally heavily shifted
            towards smaller particles (see \cref{mpam}), at least within the mass
            range that can be covered by visual observations.
            The KDE method

            it is easier and more meaningful to show the logarithm of mass $\log m$ instead.

            Several problems will have to be addressed as well. For instance, in the case
            of two separate populations with different mass indices mixing the resulting
            mass index 

        \subsubsection{Other properties} \label{iovo}
           The rest of 

        \subsubsection{Example} \label{iove}
            Put all together

            \sfig{example.png}{\textwidth}{fig:iov-mockup}{
                An example of the sky map for one night of observations.
                Individual meteoroid radiants are pictured as dots,
                with larger dots corresponding to more massive particles.
                In the background the underlying distribution of radiants is obtained
                using a spherical KDE method with a Gaussian kernel.

                In this picture there are 10000 sporadic meteors with uniform distribution of radiants
                and three fictional meteor showers with different dispersion profiles.
            }

\section{Orbital model} \label{po}
    For each meteor, it is in principle possible to trace the trajectory back in time
    and derive the trajectory of the original meteoroid in the vicinity of the Earth.
    After correcting for gravitational attraction and subtracting the velocity of the Earth
    we can further extend the trajectory into interplanetary space.

    The fundamental difficulty here is that particle streams whose orbit does not intersect
    the orbit of the Earth cannot be observed at all and thus have to be excluded from the model.
    Determination of flux in these areas is only possible with in situ observations.

    The number density distribution function here is
    \eqn{eq:po-rv}{
        \varrho(\vec{r}, \vec{v}, t).
    }
    where
    \begin{description}
        \item[$\vec{r}$]    is the instantaneous position in a heliocentric inertial reference frame;
        \item[$\vec{v}$]    is the velocity in the same frame;
        \item[$t$]          is the time;
        \item[$\varrho$]    is the number density of particles.
    \end{description}

    Alternatively this may be expressed in terms of orbital elements as
    \eqn{eq:po-elem}{
        \varrho(a, e, i, \omega, \Omega, T).
    }

    While particles that collide with the Earth are removed from the stream,
    the total volume swept by the Earth through the stream is negligible for all but the narrowest streams.
    We do not expect any observable reduction in the number of meteoroids during subsequent approaches.

    \subsection{Estimating the distribution} \label{pod}
        Since the distribution of meteoroid masses of our interest spans about seven decimal orders of magnitude,


        -- while the geometric difference between $a = \SI{1}{\au}$ and $a = \SI{10}{\au}$,
        while the region of our interest is located close to \SI{1}{\au}.
        Therefore we propose to use $\log a$ as the measure of orbit size instead, and use a Gaussian
        kernel in estimation there (which is equivalent to using a log-normal kernel in linear space).

        The same is tr

\section{Objectives of the thesis} \label{iO}
    In the thesis we are planning to simulate the orbital evolution and production of meteors
    by several most prominent meteor showers.

    \begin{itemize}
        \item develop a model of meteoroid distribution in the inner Solar System;
        \item reduce the model to the Earth's atmosphere and obtain a sky map
            of observable meteor activity at any place and time,
            such as the one depicted in \cref{fig:iov-mockup};

    \end{itemize}



\section{Full model} \label{pf}

    Even though meteoroids are always discrete particles, it is useful to model their density as a smooth
    function in multi-dimensional phase space.
    Actual particles are then treated as samples obtained from the underlying distribution.

    In the opposite direction, the underlying distribution may be estimated using a KDE method

    This distribution is not well-defined


    The density function is 
    \eqn{eq:pf-rv}{
        \varrho(\vec{r}, \vec{v}, t) \in \left[0, \infty\right).
    }

    or alternatively expressed in terms of orbital elements
    \eqn{eq:pf-elem}{
        \varrho(a, e, i, \omega, \Omega, T).
    }

    In meteor science it is customary to assume a specific distribution of masses in a population,
    described by a function that follows a power-law
    \eqn{eq:pf-map}{
        \varrho(m) \propto m^{-s}
    }
    for some constant $s$ named the \emph{mass index}.


\section{Reduction to Earth's sky} \label{ir}
    While the primary model describes the actual distribution of meteoroid particles in space around the Earth,
    it is not possible to relate it directly to observations. To do this, 

    models developed from data obtained from Earth-based observations are different.

    Typically we do not consider individual particles, but the entire stream and its elliptical orbit around the Sun.

    the ascending or the descending node of the orbit must lie very close to the orbit of the Earth.




    The final product would thus be a four-dimensional map of distribution of meteor activity, with independent variables being
    \begin{itemize}
        \item two spatial coordinates: the \emph{right ascension} $\alpha_R$ and \emph{declination} $\delta_R$ of the radiant;
        \item one \emph{temporal coordinate}, representing the meteor activity in time;
        \item the \emph{particles' speed}, corresponding to different orbits
        \item the \emph{mass spectrum}, describing the distribution of meteoroid masses, typically in terms of \textit{mass index} $s$;
        \item and the \emph{velocity spectrum}, describing the velocities of objects passing through this point, which can be in turn described by
            \begin{itemize}
                \item two spatial coordinates, the \emph{right ascension and declination} of the radiant, or $\alpha$ and $\delta$;
                \item and \emph{speed} $v$.
            \end{itemize}
    \end{itemize}

    described by the \emph{meteor activity function}
    \eqn{eq:i-map}{
        M(t, \vec{v}, m) \equiv \rho(y + \lambda_\odot, \delta_R, \alpha_R, v, m).
    }

    Necessarily, the model is not able to capture or describe any stream that does not intersect the orbit of the
    Earth,\footnote{This condition is by no means sufficient. In certain pathological cases, such as a fresh stream in resonance
    with the Earth, which has not yet fully spread along its orbit, it is possible no meteors will collide with the Earth.}

    In visualisations it is natural to keep the spatial and temporal dimensions, that is, right ascension and declination are
    displayed as spatial coordinates using one of suitable projections, while time evolution of streams is best displayed
    in sequence of images.
    Displaying the secular evolution of one stream can be performed by comparing the state of the function $M$ at
    the same $\lambda_\odot$ in successive years.

    \citep{balaz+2020}

    Within this map we are able to distinguish two main features:
    \begin{itemize}
        \item the \emph{sporadic background}, which forms the main contour lines of the maps, the ``terrain'';
        \item and the well-defined, sharp peaks, representing the \emph{meteor showers}.
            very narrow in the speed dimensions;
    \end{itemize}

    Due to influences of various acting forces, such as differences
    in initial velocities with respect to the parent body, tidal disruptions,
    perturbations arising from close encounters with large bodies of the Solar System,
    Poynting-Robertson effect, etc., the stream
    gradually widens and disperses in all components until it can no longer be distinguished from the background.

    \subsection{Time} \label{i}
        In examining the evolution we may recognize two distinct components:
        \begin{itemize}
            \item the \emph{periodic component}, emerging as the result of the Earth orbiting the Sun.
                This coordinate can be thus expressed in terms of the \emph{longitude of the sun} $\lambda_\odot$.

            \item the slowly-varying \emph{secular component}, associated with appearance of new meteoroid streams and gradual decay of older
                streams which are not replenished.

                The peaks produced by meteor showers appear abruptly after the parent body passes through the vicinity of the Earth,
                producing small meteoroid due to heating or outgassing.
                Returns of parents bodies on periodic orbits, such as comets 1P/Halley or 109P/Swift-Tuttle, result
                in resupply of meteoroid material, and thus form an extra periodic component, included in the secular evolution of the stream.

        \end{itemize}

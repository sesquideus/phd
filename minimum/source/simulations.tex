


\section{Simulation in the Solar System} \label{as}
    on the order of 100 radii.

    With ephemerides taken from the JPL Horizons system \citep{...}.
    The simulation computes the positions of the large bodies of the Solar System, most importantly the Sun,
    the eight planets and the Moon. Then the test particles

    Optinally, positions of massless bodies can be offloaded to a Graphics Processing Unit (GPU).
    Using a GPU is particularly well suited to massively parallel problems, such as simulation of non-interacting test particles.

    Any particle intersecting the cross-section of the Earth is marked as detected and will be examined further.
    Alternatively, in order to reduce the number of trials, we may employ a larger artificial detector.


    as long as the spatial density is only varying on scales much larger than the radius of the Earth,
    temporal variations in activity will be small.
    This assumption is not universally valid, as some younger meteor showers often produce dense, narrow filaments,
    which can be observed as short but intensive outbursts of extraordinary meteor activity.
    A~good example are the November alpha-Monocerotids, with a period of significantly
    heightened activity spanning only about thirty minutes in 2019 \citep{CBET4692}.

    By comparing the orbital speed and the diameter of the Earth we can see that the Earth moves
    to a completely different position with respect to the stream within several minutes.
    This means that variations in activity are not going to be significant when crossing an old stream,
    whose radius may be on the order of millions of kilometres, but will be of extreme importance with young, narrow filaments.

    The duration of visible activity also depends on the exact shape of the stream and its orientation with respect to the orbit of the Earth.
    The worst case are the streams with high inclinations, where meteors seem to arrive from a direction perpendicular
    to the direction of movement.



    It is necessary that at least one of the orbital nodes must lie very close to the orbit of the Earth.
    This can be expressed in terms of orbital elements as
    \eqn{eq:p-anode}{
        \frac{p}{1 + e \cos \omega} \approx \SI{1}{\au}
    }
    for the ascending node, or, for the descending node,
    \eqn{eq:p-dnode}{
        \frac{p}{1 - e \cos \omega} \approx \SI{1}{\au}.
    }

    \subsection{Numerical integration of equations of motion} \label{asi}
        requires computation of $\frac{n\left(n - 1\right)}{2} = \BigO{n^2}$ pairs of forces \cite{...}.
        Due to Newton's third law, opposing pairs of forces are equal in magnitude and opposite in their direction,
        and thus one evaluation suffices. This brings the number of required evaluations to half that value,
        without changing the order.


        \subsubsection{Massively parallel integration} \label{asip}
            The main difference in simulating small particles, such as meteoroids,
            is that their masses are for all practical purposes completely negligible.
            All gravitational interaction between pairs of test particles may be completely ignored,
            which drastically reduces the number of evaluations.

            Suppose we only need to work with $k$ massive objects:
            in the Solar System we need to consider the Sun, eight planets, their major moons
            and -- at least immediately after the ejection -- also the parent body of the stream.
            This leaves us with $k \sim 10$, which translates to roughly 50--100 evaluations per integration step.
            This is perfectly tractable on a CPU.

            Once the positions of physical bodies are known, we may proceed to computing the forces acting
            on test bodies. Since forces are to be summed as vectors, we need $k$ evaluations
            for each of $n$ test particles, for a total of $nk$ evaluations.
            Generally, there is no further exploitable symmetry.
            However, problems of this sort are perfectly suited for parallel computation.

            The total number of evaluations is
            \eqn{eq:asip-total}{
                E(k, n) = k^2 + nk.
            }

            Since in our case $n \gg k$, this approximately equals $nk$.

            The preferred \textit{modus operandi} at this time seems to be to use off-the-shelf integrators
            running on a CPU for resolving the positions and velocities of planets, and feed those data
            to a massively-parallel GPU integrator, simulating a large number of mutually non-interacting particles.

        \subsubsection{Using explicit ephemerides} \label{asie}
            To further simplify matters, we may leave the determination of positions and velocities
            to other software completely and free resources for other problems.
            We may directly use positions and velocities provided by some external source,
            such as JPL Horizons \cite{...}, or simply use precomputed tabelated values.

            This leaves us with only $kn$ interactions that are completely mutually independent.
            We hope to achieve sufficient performance to allow simulating about $10^7$ test particles within
            a reasonable time frame.

\section{Simulation in the atmosphere} \label{aa}
    Once it has been determined which meteoroids will enter the Earth's atmosphere, we may proceed to simulating
    the physical processes manifested during their atmospheric entry and obtain the values of physical quantities
    that are observed by ground-based observers.

    The coordinate frame of choice here is the Earth-Centered, Earth-Fixed frame (origin in the centre of mass of the Earth,
    reference plane is the equator and the reference direction is the international prime meridian).
    This reference frame is right-handed

    First of all, the position and velocity of the particle are transformed to the ECEF coordinate frame.

    To simulate the atmospheric entry we used \textsc{Asmodeus},
    a multi-purpose virtual meteor simulator developed as a part of our master's thesis and extended for numerous
    other purposes related to meteor astronomy \citep{balaz-thesis,balaz+2020}.

    Once a particle is selected for atmospheric entry, its velocity is transformed to the ECEF reference frame
    and a numerical integration of the equations of motion is executed.

    \subsection{Numerical integration of the equations of motion} \label{aai}
        Unlike in interplanetary space, where forces are completely dominated by the gravitational force
        exerted by the Sun

        a particle entering the Earth's atmosphere switches between several regimes in rapid succession.
        The precise order is subject to variations due to particle's size, entry speed and angle
        and material composition.


        In the simulation we consider only two real forces:
        \begin{itemize}
            \item the \emph{drag force} $\vec{F_d}$, always acting against the instantaneous velocity vector of the particle;
            \item the \emph{gravitational force} $\vec{F_g}$, pulling the meteoroid towards the centre of the Earth.
        \end{itemize}

        Since it is 
        and two fictitious forces, arising from the fact the simulation is performed in a rotating reference frame:
        \begin{itemize}
            \item the fictitious \emph{centrifugal force} $\vec{F_{\mathrm{C}}}$, pushing the particle away from the axis of rotation of the Earth;
            \item and the fictitious \emph{Coriolis force} $\vec{F_{\mathrm{c}}}$, which curves the trajectory to
        \end{itemize}

        Except for the drag force, all forces can be readily expressed in terms of instantaneous position and velocity of the meteoroid,
        along with several constants.
        Calculating the drag force requires a choice of a suitable model.

        In its cosmic stage and during most of the motion in the gravitational sphere of influence of the Earth,
        the particle's motion is dominated by gravitational effects. This is no longer true during the atmospheric entry,
        where the magnitude of the atmospheric drag force exceeds all other effects by several orders of magnitude.
        As the duration of the entire event is on the order of seconds, other forces simply do not act for long enough
        to produce a significant deviation in trajectory. For statistical evaluation of populations it is sufficient
        to consider only the drag force.

        However, should the meteoroid survive the entry as a meteorite, their influence is important during the dark phase of flight.

        \subsubsection{Coriolis force} \label{aafC}
            The fictitious Coriolis force arises in a rotating reference frame.
            \eqn{eq:aafC-coriolis}{
                \vec{F_{\mathrm{C}}} = -2 m \vec{\Omega}_\Earth \times \vec{v}.
            }

            For a meteoroid, its magnitude is at most on the order of
            \eqn{eq:aafC-order}{
                2 \frac{2\pi}{\SI{86400}{\second}} \cdot \SI{70}{\kilo\metre\per\second} \sim \SI{10}{\metre\per\second\squared},
            }
            roughly the same as gravitational acceleration.

        \subsubsection{Centrifugal force} \label{aafc}
            The centrifugal force pushes the particles away from the axis of rotation of the Earth.
            \eqn{eq:aafc-centrifugal}{
                \vec{F_{\mathrm{c}}} = - m \vec{\Omega}_\Earth \times \left(\vec{\Omega}_\Earth \times \vec{r}\right),
            }
            where $\vec{\Omega}_\Earth$ is the angular speed of rotation of the Earth, $\SI[parse-numbers = false]{\frac{2\pi}{86164}}{\per\second}$.

            Since it only depends on the instantaneous position, it can be thought of as a minor correction to the
            gravitational force. A simple calculation shows that its magnitude is on the order
            of \SI{0.03}{\metre\per\second\squared}. For most purposes it can be safely neglected.

        \subsubsection{Gravitational force} \label{aafg}
            The real shape of the gravitational potential around the Earth is fairly difficult to describe accurately
            However, for particles that are generally moving on highly hyperbolic trajectories the
            simple approximation by a Newtonian point source with mass $M_\Earth$ is more than sufficient.
            The formula used in computation is then simply
            \eqn{eq:aafg-gravity}{
                \vec{F_{\mathrm{G}}} = - \frac{GM_\Earth}{r^3} \vec{r}.
            }

        \subsubsection{Drag force} \label{aafd}
            The precise description of the drag force is crucial to understand the motion
            of the particles in the atmosphere. The drag force is dominant between approximately
            the point where meteoroids begin to emit visible light (about \SI{100}{\kilo\metre})
            until the very end of the visible trail. For all but the most massive meteoroids,
            which are able to survive the atmospheric entry, this can be approximated
            by the point where all of the matter has been ablated away; while for the bodies
            surviving as meteorites, during the terminal phase of the flight the drag and
            gravitational forces are nearly balanced out.

            At peak deceleration the magnitude of the drag force can reach values several orders of magnitude
            higher than the sum of all other forces and is thus completely dominant.
            A very simple model of atmospheric flight can perform relatively well even when all other forces are neglected.

    \subsection{Determination of other properties} \label{aap}

        \subsubsection{Absolute and visual magnitudes} \label{aapm}



\section{Henderson model} \label{DH}
    The Henderson model \cite{henderson} explicitly depends on the Mach
    and Reynolds numbers $M_\infty$ and $\Reinf$,
    and on the ratio of wall temperature of the object to the free-flow temperature.

    To simplify to expressions a bit we can substitute
    \eqn{eq:DH-tau}{
        \tau \leftarrow \frac{T_w}{T_\infty},
    }
    and
    \eqn{eq:DH-Rep}{
        \Reinf^\prime \leftarrow \num{0.03}\Reinf + \num{0.48}\sqrt{\Reinf}.
    }

    The model consists of three separate cases depending on the Mach number.

    \subsection{Subsonic case}
        In the subsonic case ($M_\infty \leq 1$), $\Gamma$ is given as
        \aln{eq:DH-G}{
            \Gamma(M_\infty, \Reinf, \tau) &=
                \frac{24}{
                    \Reinf + s \left[
                        \num{4.33} + \left(
                            \frac{\num{3.65} - \num{1.53} \tau}{1 + \num{0.353} \tau}
                        \right) \exp \left(
                            \num{-0.247}\frac{\Reinf}{s}
                        \right)
                    \right]
                } \\ &+ \exp{\left(
                    -\frac{\num{0.5} M_\infty}{\sqrt{\Reinf}}
                \right)}\left[
                    \frac{
                        \num{4.5} + \num{0.38}\Reinf^\prime
                    }{
                        1 + \mathrm{Re}_\infty^\prime
                    } + \num{0.1} M_\infty^2 + \num{0.2} M_\infty^8
                \right] \\ &+ \left[
                    1 - \num{0.6} \exp{\left(-\frac{M_\infty}{\Reinf}\right)} s
                \right],
        }

    \subsection{High-supersonic case} \label{DH-h}
        For higher supersonic and hypersonic speeds ($M_\infty \geq \num{1.75}$), $\Gamma$ is given as
        \eqn{eq:DH-h}{
            \Gamma(M_\infty, \Reinf, \tau) =
            \frac{
                \num{0.9} + \frac{\num{0.34}}{M_\infty^2} + \num{1.86}\sqrt{
                    \frac{M_\infty}{\Reinf}\left[
                        2 + \frac{2}{s} + \frac{\num{1.058}}{s}\sqrt{\tau} - \frac{1}{s^4}
                    \right]
                }
            }{
                1 + \num{1.86}\sqrt{\frac{M_\infty}{\Reinf}}
            }.
        }

    \subsection{Low-supersonic case} \label{DH-l}
        Finally, in the transitional case ($1 < M_\infty < \num{1.75}$) $\Gamma$ is obtained by bridging the boundary values of the expressions for
        the subsonic and high-supersonic case as
        \eqn{eq:DH-Gss}{
            \Gamma(M_\infty, \Reinf, \tau) =
            \Gamma(1, \Reinf, \tau) + \frac{4}{3}\left(M_\infty - 1\right)\left[
                \Gamma(\num{1.75}, \Reinf, \tau) - \Gamma(1, \Reinf, \tau)
            \right]
        }


\section{Loth model} \label{DL}
    The Loth model \citep{loth2008} explicitly depends on free-flow Mach number $M_\infty$,
    free-flow Reynolds number $\Reinf$, free-flow temperature $T_\infty$
    and ??? temperature $T_p$.


    \eqn{eq:DL-s}{
        s = M_\infty \sqrt{\frac{\gamma}{2}}
    }

    \aln{eq:DL-Gfm}{
        \Gamma_\mathrm{fm}^\prime &=
            \frac{\left(1 + 2s^2\right) \exp(-s^2)}{\sqrt{\pi} s^3} +
            \frac{\left(4s^4 + 4s^2 - 1\right) \erf(s)}{2 s^4}, \\
        \Gamma_\mathrm{fm} &=
            \frac{\left(1 + 2s^2\right) \exp(-s^2)}{s^3 \sqrt{\pi}} +
            \frac{\left(4s^4 + 4s^2 - 1\right) \erf(s)}{2 s^4} +
            \frac{2}{3s} \sqrt{\pi \frac{T_p}{T_\infty}}.
    }

    \eqn{eq:DL-GM}{
        G_M =
        \begin{dcases}
            1 - \num{1.525} M_\infty^4
                & M_\infty \leq \num{0.89}, \\
            \num{2e-4} + \num{8e-4} \tanh \left[\num{12.77} \left(M_\infty - \num{2.02}\right)\right]
                & M_\infty > \num{0.89},
        \end{dcases}
    }

    \eqn{eq:DL-CM}{
        C_M =
        \begin{dcases}
            \frac{5 + 2 \tanh \left[3 \ln \left(M_\infty + 1\right)\right]}{3}
                & M_\infty \leq \num{1.45}, \\
            \num{2.044} + \num{0.2} \exp\left[\num{-1.8} \ln\left(\frac{M_\infty}{\num{1.5}}\right)^2\right]
                & M_\infty > \num{1.45}.
        \end{dcases}
    }

    \eqn{eq:DL-HM}{
        H_M = 1 - \frac{\num{0.258} C_M}{1 + 514 G_M}.
    }

    A rarefaction correction $r$ is expressed as a function of Knudsen number
    \eqn{eq:DL-fKn}{
        r(\mathrm{Kn}) = \frac{
            1
        }{
            1 + \mathrm{Kn}_\infty \left[\num{2.514} + \num{0.8} \exp \left(\frac{\num{-0.55}}{\mathrm{Kn}_\infty}\right)\right]
        }
    }

    \eqn{eq:DL-GKnRe}{
        \Gamma_\mathrm{Kn, Re} =
        \frac{24}{\Reinf} \left(1 + \num{0.15} \Reinf^{\num{0.687}}\right) r(\mathrm{Kn}).
    }

    The free-molecular component was determined by \citet{patterson1971} as
    \eqn{eq:DL-GfmRe}{
        \Gamma_\mathrm{fm, Re} =
        \frac{
            \Gamma_\mathrm{fm}
        }{
            1 + \left(\frac{\Gamma_\mathrm{fm}^\prime}{\num{1.63}} - 1\right)\sqrt{\frac{\Reinf}{45}}
        }.
    }

    The final expression for $\Gamma$ is obtained by stitching two functions
    \eqn{eq:DL-Cd}{
        \Gamma =
        \begin{dcases}
            \frac{
                \Gamma_\mathrm{Kn, Re} + M_\infty^4 \Gamma_\mathrm{fm, Re}
            }{
                1 + M_\infty^4
            } & \Reinf \leq 45; \\
            \frac{24}{\Reinf} \left(1 + \num{0.15} \Reinf^{\num{0.687}}\right) H_M
                + \frac{\num{0.42} C_M}{1 + \frac{\num{42500} G_M}{\Reinf^{\num{1.16}}}}
                & \Reinf > 45.
        \end{dcases}
    }

\section{Singh model} \label{DS}
    In the Singh model, the drag coefficient $\Gamma$ explicitly depends on the
    free-flow Mach number $M_\infty$, free-flow Reynolds number $\Reinf$,
    free-flow temperature $T_\infty$ and particle surface temperature $T_p$.

    Furthermore, the correlation requires several constants:
    \begin{itemize}
        \item the \emph{ratio of heat capacities} $\gamma \equiv \frac{c_p}{c_V}$,
            whose value for air is $\frac{7}{5} = \num{1.4}$;
        \item $\omega$, the exponent in the power law dependence
            of viscosity on temperature ($\mu \propto T^\omega$, see \cref{fdv}).
        \item \emph{shock curvature parameter} $\alpha_0 = \num{0.356}$;
        \item \emph{high speed rarefaction correction} $\alpha_\mathrm{hoc} = \num{1.27}$;
        \item \emph{} $\Gamma_{M_\infty} = \num{0.9}$;
        \item \emph{} $\delta_0 = \num{9.4}$;
        \item $A_0 = \frac{24}{\delta_0^2}$.
    \end{itemize}

    First a \emph{non-continuum parameter} is defined as
    \eqn{eq:DS-Wr}{
        W_r = \frac{M_\infty^{2\omega}}{\Reinf} = \mathrm{Kn}_\infty M_\infty^{2\omega - 1} \sqrt{\frac{2}{\pi \gamma}}.
    }
    For supersonic speeds, a correction is computed,
    \eqn{eq:DS-WrT}{
        W_r^T =
        \begin{dcases}
            W_r & M_\infty \leq 1, \\
            W_r \left(1 + \frac{T_p}{T_s}\right) & M_\infty > 1, \\
        \end{dcases}
    }
    where $T_p$ is the surface temperature of the particle and
    $T_s$ is the post-shock temperature of the surrounding atmosphere,
    \eqn{eq:DS-Ts}{
        T_s = T_\infty \frac{
            \left[
                \left(\gamma - 1\right) M_\infty^2 + 1
            \right]
            \left[
                2 \gamma M_\infty^2 - \gamma + 1
            \right]
        }{
            \left(\gamma + 1\right)^2 M_\infty^2
        }.
    }

    

    
    The continuum drag coefficient is expressed as
    \eqn{eq:DS-Gc}{
        \Gamma_c
            = C_1 \left(1 - \alpha \frac{v_s}{v_\infty}\right)
                + C_0 \Theta(M_s)\left(1 + \frac{\delta_0}{\sqrt{\widetilde{\mathrm{Re}_s}}}\right)^2
    }
    where
    \eqn{eq:DS-C1}{
        C_1 = \frac{
            \Gamma_{M_\infty} - C_0 \left[
                1 + \frac{\left(\gamma - 1\right)^2}{4 \gamma}
            \right]^{\frac{\gamma}{\gamma - 1}}
        }{
            1 - \frac{1}{\alpha_0 M_\infty} \frac{\gamma - 1}{\gamma + 1}
        },
    }
    \eqn{eq:DS-ReT}{
        \widetilde{{\mathrm{Re}_s}} =
            \Reinf \left[
                \frac{1}{\alpha^2} \frac{T_\infty}{T_s}
            \right]^\omega
        \Theta(M_s)^{
            \frac{\gamma + 1}{2 \gamma} - \frac{\gamma - 1}{\gamma}\omega
        },
    }
    $\Theta$ is a function of Mach number $M$
    \eqn{eq:DS-Ym}{
        \Theta(M) = \left[1 + \left(\gamma - 1\right) \frac{M^2}{2}\right]^{\frac{\gamma}{\gamma - 1}}
    }
    and $\alpha$ is a free parameter that was determined as
    \eqn{eq:DS-alpha}{
        \alpha = \frac{1}{1 + \alpha_0 M_\infty - \alpha_0}
    }

    For low $M_\infty$ a \emph{rarefaction correction function} $f$ is computed,
    \eqn{eq:DS-fKn}{
        f(\mathrm{Kn}_\infty, W_r^T) =
        \frac{1}{
            1 + \mathrm{Kn}_\infty \left[\num{2.514} + \num{0.8} \exp \left(\frac{\num{-0.55}}{\mathrm{Kn}_\infty}\right)\right]
        }
        \frac{1}{1 + \alpha_\mathrm{hoc} W_r^T}
    }

    The expression for free-molecular flow is the same as in the Loth model, \cref{eq:DL-Gfm},
    \eqn{eq:DS-Gfm}{
        \Gamma_\mathrm{fm} =
            \frac{\left(1 + 2s^2\right) \exp(-s^2)}{s^3 \sqrt{\pi}} +
            \frac{\left(4s^4 + 4s^2 - 1\right) \erf(s)}{2 s^4} +
            \frac{2}{3s} \sqrt{\pi \frac{T_p}{T_\infty}}.
    }

    Finally, the drag coefficient $\Gamma$ is obtained by bridging \labelcref{eq:DS-Gc} and \labelcref{eq:DS-Gfm}
    by a generalized rational polynomial function of the bridging correlation parameter $B$,
    which is defined~as
    \eqn{eq:DS-B}{
        B = W_r^T \frac{M_\infty^{2\omega - 1} + 1}{M_\infty^{2\omega - 1}}.
    }

    The expression for $\Gamma$ is then
    \eqn{eq:DS-G}{
        \Gamma =
            \Gamma_c(M_\infty, \Reinf) f(\mathrm{Kn_\infty}, W_r) \frac{1}{1 + B^\eta} +
            \Gamma_\mathrm{fm} \frac{B^\eta}{1 + B^\eta}.
    }
    where the \emph{bridging modulator} $\eta$ is set to $\eta = \num{1.8}$.

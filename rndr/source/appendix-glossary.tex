\begin{description}[labelindent=0mm, leftmargin=40mm]
    \item[convergence angle]
        the angle between two observers as viewed from the observed object
        (here typically \emph{camera 1} -- meteoroid -- \emph{camera 2})
    \item[ECEF]
        \emph{earth-centered, earth-fixed} coordinate frame, a reference frame with origin
        in the centre of mass of the Earth, rotating with the solid surface of the Earth
    \item[ENU]
        \emph{east, north, up}, a right-handed convention for the \emph{LTPC} coordinate system where
        $x$-axis points locally eastwards, $y$-axis northwards and $z$-axis upwards
        (in the direction of the local plumb line)
    \item[KDE]
        \emph{kernel density estimation}, a non-parametric method of estimating the probability density function
        of random variables from a number of discrete samples
    \item[LTPC]
        \emph{local tangent plane coordinates}, a Cartesian system of coordinates
        for describing locations on the Earth with respect to a fixed observer on the surface
    \item[Null Island]
        a nickname for the fictional place at the origin of the WGS84 coordinate system
        \ang{0}~N, \ang{0}~E, \SI{0}{\metre} \citep{null-island}
    \item[WGS84]
        \emph{World Geodetic System 1984}, the latest revision of the ellipsoidal Earth model \citep{nima-wgs84}
    \item[zenith attraction]
        a decrease in zenith angle of the radiant of a meteor shower,
        caused by the deflection of the meteoroids' orbits in the
        gravitational field of the Earth \citep{lovell1954}
    \item[ZHR]
        \emph{zenithal hourly rate}, the number of shower meteors per hour an ideal observer would see
        if his limiting magnitude was \Mag{+6.5} and the radiant was in his zenith \citep{imo-glossary}
\end{description}

\Epigraph[0.4]{
    You saw sagacious Solomon \\
    You know what came of him \\
    To him, complexities seemed plain
}{
    How Fortunate the Man With None \\
    \textsc{Bertolt Brecht} \\
    \textit{translation John Willett}
}

In this chapter we describe the forces acting on a meteoroid body
and how they affect the resulting motion of the particle in the atmosphere.

\section{Analysis} \label{da}

    We will introduce all relevant constants and properties of the meteoroids and the environment.

    we obtain a tuple which we we call the \emph{state} of the meteoroid.
    This tuple contains the key properties which are bound to the meteoroid body,
    and which can vary during its lifetime and determine its final fate.
    We use the same tuple in all models, with some properties
    kept constant or ignored completely in simpler models where they are not needed.

    The properties of the meteoroid stored in the state are
    \begin{itemize}
        \item \emph{position}, stored as a three-dimensional Cartesian vector $(x, y, z)$
            in the Earth-Centered, Earth-Fixed (ECEF) coordinate system.
            In text or graphical output it is more convenient to represent
            coordinates in the \emph{WGS84} coordinate system, however
            all calculations are performed in Cartesian coordinates;
        \item \emph{velocity}, stored as a three-dimensional vector $(v_x, v_y, v_z)$
            in the ECEF frame;
        \item instantaneous \emph{mass} of the meteoroid. In order to mitigate
            problems with rounding and numerical stability, internally $\log m$
            is stored instead;
        \item surface \emph{temperature} of the meteoroid. In all used models
            it is taken as a constant throughout the volume of the body.
            Surface temperature directly affects the rate of evaporation of the material
            and also affect the drag coefficient of the moving body in more advanced models.
    \end{itemize}

    The remaining properties are constant during the lifetime of the simulated meteoroid.
    They are set at the creation of the particle and do not have to be included in the state.
    Of course, the differential equations describing the time evolution of the state
    may (and will) depend on them.

    \subsection{Phases of meteoroid entry} \label{dap}
        Within the classical categorisation of phases of meteoroid flight \citep{???}
        we 

        \begin{itemize}
            \item For the vast majority of its lifetime the meteoroid remains in
                the \emph{interplanetary phase}, during which it is orbiting the Sun
                far from any planetary atmosphere. Neglecting solar wind and
                the very tenuous interplanetary medium, we may consider this to be vacuum.

                In our simplification, the body does not undergo any changes and its mass remains constant.
                Its surface or average temperature might vary as the distance from
                the Sun changes on its elliptical orbit, but this variation
                is generally slow -- enough to say that immediately before the entry it only
                depends on the properties of the meteoroid material and its current distance from the Sun,
                which is necessarily about \si{1}{au}.

                From the dynamical point of view it is useful to distinguish a truly heliocentric orbit
                (as long as the particle is outside of the gravitational sphere of influence of the Earth)
                and a hyperbolic geocentric orbit (once it has crossed into the Earth's vicinity).
                However, from the perspective of ablation, nothing is changed
                as long as heating from collisions with atmospheric molecules is negligible,
                which does not happen until the meteoroid crossed to within several hundred
                kilometres from the surface of the Earth.
            \item he \emph{pre-heating phase}, which occurs in the exosphere. The  The frequency of these collisions
                and the surface temperature of the meteoroid are still too low to produce a visible
                trail,
        \end{itemize}

        limits of free-molecular flow (which is, in case of a meteoroid,
        encountered during the pre-heating phase at altitudes above \SI{200}{\kilo\metre})


\section{Acting forces} \label{df}
    First, we need to consider which forces are to be included in the simulation.

    In the simulation we consider two real forces,
    \begin{itemize}
        \item the \emph{drag force} $\vec{F_d}$, always acting against the instantaneous velocity vector of the particle;
        \item the \emph{gravitational force} $\vec{F_g}$, pulling the meteoroid towards the centre of the Earth.
    \end{itemize}

    Since the simulation is run in the ECEF reference frame, two fictitious forces are also present.
    Including them may increase the precision of the simulation at a negligible computational cost:
    \begin{itemize}
        \item the fictitious \emph{centrifugal} or \emph{Huygens force} $\vec{F_{\mathrm{Huygens}}}$,
            pushing the particle away from the axis of rotation of the Earth;
        \item and the fictitious \emph{Coriolis force} $\vec{F_{\mathrm{Coriolis}}}$,
            which pushes a moving body in a direction perpendicular to its velocity vector.
    \end{itemize}
%
%            Except for the drag force, all forces can be readily expressed in terms of instantaneous
%            position and velocity of the meteoroid and several constants.
%            Calculating the drag force precisely is computationally very expensive and cannot be done
%            without knowing the shape of the particle. Therefore we have to choose an appropriate simplified model.
%
%            In its cosmic stage and during most of the motion in the gravitational sphere of influence of the Earth,
%            the forces acting on the particle are dominated by gravitational effects. As the meteoroid penetrates
%            deeper into the atmosphere, the density of the surrounding gas increases roughly exponentially.
%            At its peak, the atmospheric drag force exceeds all other effects by several orders of magnitude,
%            decelerating the meteoroid very rapidly.
%            As the duration of the entire event is on the order of seconds, other forces simply do not act for long enough
%            to produce a significant deviation in trajectory. For statistical evaluation of artificial meteors
%            it is sufficient to consider only the drag force.
%            However, should the meteoroid survive the entry as a meteorite, their influence is important during the dark phase of flight.
%
%            \paragraph{Coriolis force} \label{saiC}
%                The fictitious Coriolis force arises in a rotating reference frame.
%                \eqn{eq:saiC-coriolis}{
%                    \vec{F_{\mathrm{C}}} = -2 m \vec{\Omega}_\Earth \times \vec{v}.
%                }
%
%                For a meteoroid, its magnitude is at most on the order of
%                \eqn{eq:saiC-order}{
%                    2 \frac{2\pi}{\SI{86400}{\second}} \cdot \SI{70}{\kilo\metre\per\second} \sim \SI{10}{\metre\per\second\squared},
%                }
%                roughly the same as gravitational acceleration.
%
%            \paragraph{Centrifugal force} \label{saic}
%                The centrifugal force pushes the particles away from the axis of rotation of the Earth.
%                \eqn{eq:saic-centrifugal}{
%                    \vec{F_{\mathrm{c}}} = - m \vec{\Omega}_\Earth \times \left(\vec{\Omega}_\Earth \times \vec{r}\right),
%                }
%                where $\vec{\Omega}_\Earth$ is the angular speed of rotation of the Earth,
%                $\frac{2\pi}{\SI{86164}{\second}} \approx \SI{7.292e-5}{\per\second}$ in the direction
%                towards the north celestial pole.
%
%                Since it only depends on the instantaneous position, it can be thought of as a minor correction to the
%                gravitational force. A simple calculation shows that its magnitude is on the order
%                of \SI{0.03}{\metre\per\second\squared}. Near the surface it can be safely neglected for most purposes.
%                It only becomes important when the distance from the axis of rotation is large, for instance if the simulation
%                is used to investigate the motion of a meteoroid further from the Earth.
%                However, in this case an inertial reference frame should be used instead.
%
%            \paragraph{Gravitational force} \label{saig}
%                The real shape of the gravitational potential around the Earth is fairly complex and difficult to describe accurately.
%                For particles that are generally moving on highly hyperbolic trajectories the
%                simple approximation by a Newtonian point source with mass $M_\Earth$ is sufficient.
%                The formula used in computation is then simply
%                \eqn{eq:saig-gravity}{
%                    \vec{F_{\mathrm{G}}} = - \frac{GM_\Earth}{r^3} \vec{r}.
%                }
%
%            \paragraph{Drag force} \label{said}
%                The precise description of the drag force is crucial to understand the motion
%                of meteoroids in the atmosphere. The drag force is dominant between approximately
%                the point where meteoroids begin to emit visible light (about \SI{100}{\kilo\metre})
%                until the very end of the visible trail. For all but the most massive meteoroids,
%                which are able to survive the atmospheric entry, this can be approximated
%                by the point where all of the matter has been ablated away; while for the bodies
%                surviving as meteorites, during the terminal phase of the flight the drag and
%                gravitational forces are nearly balanced out.
%
%                At peak deceleration the magnitude of the drag force can reach values several orders of magnitude
%                higher than the sum of all other forces and is thus completely dominant.
%                A very simple model of atmospheric flight can thus perform relatively
%                well even when all other forces are neglected.



    is model-dependent. The total force is then
    \eqn{eq:sai-totalforce}{
        \vec{F} = \vec{F_\mathrm{gravity}} + \vec{F_\mathrm{Coriolis}} + \vec{F_\mathrm{Huygens}} + \vec{F_\mathrm{drag}}.
    }

    \subsection{Gravity} \label{dfg}
        

    \subsection{Coriolis force} \label{dfC}


    \subsection{Huygens force} \label{dfH}


    \subsection{Drag force} \label{dfd}
        The last -- but the most important and interesting -- force acting on the meteoroid during
        its atmospheric entry is the drag force. It arises as a consequence of its collisions with
        the molecules of atmospheric air, decreasing the momentum of the particle and at the
        same time convertic its kinetic energy to heat.

        Unlike other forces, which can be readily described in terms of several easily understandable
        quantities, drag force is much more variable and somewhat inscrutable: it depends on the shape
        of the particle, which is not known \textit{a priori}, but can be measured only after
        the meteorite is recovered, which means we generally have to work with assumptions.
        It also depends on atmospheric density, which can be modelled but not measured precisely,
        and varies with time of year, day and solar activity.

        And it gets worse. While certain limits can be described accurately using
        simple concepts and formulae -- for instance, in exosphere the air density is extremely low,
        interactions between constituent particles are very rare and the Knudsen number is virtually infinite.
        The atmospheric drag can be then modelled as a succession of collisions with individual
        molecules that do not acknowledge others' presence anyhow, which is fairly simple.
        On the other hand, a microscopic dust particle falling near the sea level encounters
        Reynolds numbers so low that the situation can be described by the relatively simple Stokes' drag.

        However, between these two the entering meteoroid experiences a vast range of conditions:
        a hypersonic entry of the particle into the atmosphere, whose density grows exponentially with increasing depth;
        then a transonic transition, where all sorts of turbulence and flow separation effects come into play,
        and finally an atmospheric free fall at subsonic terminal velocity.

        Furthermore, the functional dependences describing the exact motion
        of air around the particle are often chaotic and according to modern knowledge
        finding closed-form solutions is not possible at all, so we have to rely


        can only be investigated numerically using CFD simulations, or by semi-empirical
        relations that have to be derived from experiments.


         a complete analytical solution
        is, according to modern knowledge, not possible, numerical simulations

        a tradeoff between accuracy and complexity (and thus computation time).
        The models described here vary from very simple to 

        one particular problem is that the exact shape of the meteoroid is never known beforehand,
        and any simulation working with the real shape can only be performed after the meteorite is
        found -- which somewhat subtracts from the usefulness of running the simulation at all.

        Therefore we always assume that our spherical particles.
        This approximation is not justified physically, but by lack of any better methods (OK this is cruel).

        \subsubsection{No drag} \label{mdn}
            A trivial model in which $\Gamma = 0$ at all times.
            It is not useful for atmospheric modelling, but is computationally
            very fast and makes testing other models easier.

        \subsubsection{Constant drag model} \label{mdc}
            In textbooks and older papers a constant value of $\Gamma$ is often taken.
            For a spherical particle in a turbulent flow a value of \num{0.47}
            is often used \cite{???}. \cite{hrábek} \cite{havrila}

        \subsubsection{Morrison model} \label{mdM}
            The next step is to include the effects of viscous and inertial forces -- or,
            as it is more commonly express, the effect of their ratio or the \emph{Reynolds number}.

            the correlation by \citet{morrison2016}.

            Variable with Reynolds number

            extends both to very low Reynolds number as
            \eqn{eq:mdm-limre}{
                \lim_{\mathrm{Re} \to 0} \Gamma = \frac{24}{\mathrm{Re}},
            }
            yielding the formula for laminar Stokes' drag
            \eqn{eq:mdm-stokes}{
                \lim_{\mathrm{Re} \to 0} F
                    = \lim_{\mathrm{Re} \to 0} \frac{1}{2} \Gamma S \rho_\mathrm{air} v^2
                    = \frac{1}{2} \frac{24 \mu}{2 r \rho_\mathrm{air} v} \rho_\mathrm{air} \pi r^2 v^2
                    = 6 \pi \mu r v.
            }
            It also reproduces the sharp decrease of drag coefficient at high Reynolds
            numbers ($\mathrm{Re} \approx \num{5e5}$), known as the \emph{drag crisis} \cite{???}.

            \fig{drag-crisis.png}{80mm}{ddm-drag-crisis}{Drag crisis}

            However, it is only valid in the incompressible limit and at high Knudsen numbers,
            which are assumptions that are all but valid for a meteoroid in
            hypersonic flight in the rarefied upper atmosphere. Still, it can be useful
            for modelling the atmospheric free fall of a meteorite.

        \subsubsection{Loth model} \label{ddL}
            A further refinement is due to \cite{loth???}.

        \subsubsection{Henderson model} \label{ddH}

        \subsubsection{Singh et al.} \label{ddS}
            A semi-empirical, state-of-the-art model valid throughout the entire range of
            conditions encountered during the atmospheric entry has been developed by
            \citet{singh+2020}. The model is derived from first principles and
            uses rational polynomial bridging functions to interpolate between
            well understood flight regimes a meteoroid experiences during its flight \ref{dap}.


\section{Conclusion} \label{dc}

    \subsubsection{Whipple model} \label{mmw}
        A standard, textbook model of meteoroid flight introduces a change in mass of the meteoroid
        during its flight through the atmosphere. 
        (Öpik) (Ceplecha)

        Again, we assume a constant (spherical or nearly spherical) shape throughout the entire
        lifetime of the meteoroid.\footnote{An advanced fragmentation model could probably assume a non-spherical
        shape in the first moments, which would quickly change back into a spherical one during ablation
        of the newly created particle.}

        The drag coefficient $\Gamma$ is 

        (Gamma) sometimes $C_d$ is used as a symbol instead.

        are generally not available and would require running a proper CFD simulation
        of the shape in question at various densities, orientations and flow speeds.

        Valid across a wide range of conditions, as long as the value of Re is the same.
        ultra-hypersonic flow?

        A single value is used during the entire flight. Typically


        \eqn{eq:reynolds}{
            \mathrm{Re} = \frac{\rho_\infty v d}{\mu}.
        }
        where
        \begin{description}
            \item[$d$] is the \emph{typical size} of the meteoroid in the stream.
                For spherical or nearly-spherical particles the diameter ($d = 2r$) is used;
            \item[$\rho_\infty$] is the \emph{density} of the surrounding fluid, in this case
                air at the location of the meteoroid, ignoring the compression caused
                by the moving body itself;
            \item[$\mu$] is the \emph{dynamic viscosity} of the surrounding medium.
        \end{description}

        Except for $\mu$, all three quantities will change significantly
        during the atmospheric entry: $\rho_\mathrm{air}$ is negligible at first
        and increases roughly exponentially in the range of approximately \SIrange{e-10}{1}{\kilo\gram\per\metre\cubed}
        in the region of our interest, $v$ is typically on the order of \SIrange{1e4}{7e4}{\metre\per\second}
        just above the atmosphere but can drop to about \SI{1}{\metre\per\second}
        for a microscopic particle just above the surface, and the diameter also spans
        about four orders of magnitude (\SIrange{e-4}{1}{\metre}).
        All combined, this means that a wide range of values of $\mathrm{Re}$ has to be investigated.


        To save computation time, two constants can be evaluated in advance as
        \eqn{eq:mconst}{
            C_v = A \rho^{-2/3}
            \QQText{and}
            C_m = \frac{\Lambda A \rho^{-2/3}}{2 Q} = \frac{C_v \Lambda}{2 Q}.
        }


    \subsubsection{Material evaporation model} \label{mwh}
        Another useful extension of the model described in \cref{mmw} is due to
        \cite{hill+2005}, which adds temperature as another variable property of the meteoroid.


        In the code it is referred to as \texttt{HRH2005}.

        \eqn{eq:hrh}{
            \Derivative{T}{t} = \frac{A}{c m^{1/3} \rho^{2/3}}
                \left(\frac{1}{2} \Lambda \rho_\mathrm{air} v^3
                    - 4 \sigma \epsilon \left(T^{\,4} - T_\mathrm{air}^{\,4}\right)
                    + \frac{L}{A} \left(\frac{\rho}{m}\right)^{2/3} \Derivative{m}{t}
                \right),
        }
        where
        \begin{description}
            \item[$T$]
                temperature of the meteoroid ($\mathrm{K}$),
            \item[$t$]
                time (\si{\second}),
            \item[$A$]
                shape factor (dimensionless),
            \item[$c$]
                specific heat of the meteoroid ($\si{\joule\per\kilo\gram\per\kelvin} \equiv \si{\metre\squared\per\second\squared\per\kelvin}$),
            \item[$m$]
                instantaneous mass of the meteoroid (\si{\kilo\gram}),
            \item[$\rho$]
                density of the meteoroid (\si{\kilo\gram\per\metre\cubed}),
            \item[$\Lambda$]
                heat transfer coefficient (\si{1}),
            \item[$\rho_\mathrm{air}$]
                density of the atmosphere at the position of the meteoroid (\si{\kilo\gram\per\metre\cubed}),
            \item[$\sigma$]
                Stefan--Boltzmann constant (\SI{5.67e-8}{\watt\per\metre\squared\per\kelvin\tothe{4}}),
            \item[$\epsilon$]
                emissivity of the meteoroid (dimensionless),
            \item[$v$]
                instantaneous speed of the meteoroid,
            \item[$T$]
                temperature of the meteoroid (\si{kelvin}),
            \item[$T_\mathrm{air}$]
                temperature of the atmosphere at the position of the meteoroid (\si{\kelvin}),
            \item[$L$]
                latent heat of fusion and vaporization ($\si{\joule\per\kilo\gram} \equiv \si{\metre\squared\per\second\squared}$.

        \end{description}

        The state of the meteroid is then defined by the tuple $(\vec{r}, \vec{v}, \ln m, T)$.

        \eqn{eq:hrh2}{
            \Derivative{T}{t} = \frac{1}{c} \left(
                \frac{A}{m^{1/3} \rho^{2/3}}
                \left(
                    \frac{1}{2} \Lambda \rho_\mathrm{air} v^3
                    - 4 \sigma \epsilon \left(T^{\,4} - T_\mathrm{air}^{\,4}\right)
                \right)
                + \frac{L}{m} \Derivative{m}{t}
            \right)
        }

        Again, to increase numerical stability we may substitute $m \to \ln m$,
        by the chain rule
        \eqn{eq:hrh-chain}{
            \Derivative{\left(\ln m\right)}{t} = \frac{1}{m} \Derivative{m}{t}.
        }

        The second term then can be separated and the whole equation \ref{eq:hrh} rewritten as
        \eqn{eq:hrh-fixed}{
            \Derivative{T}{t} = \frac{1}{c} \left(
                A \rho^{-2/3} \exp \left(- \frac{\ln m}{3}\right)
                \left(
                    \frac{1}{2} \Lambda \rho_\mathrm{air} v^3
                    - 4 \sigma \epsilon \left(T^{\,4} - T_\mathrm{air}^{\,4}\right)
                \right)
                + L \Derivative{\left(\ln m\right)}{t}
            \right).
        }

        To further optimize the computation, we may precompute another constant
        $c_T = 4 \sigma \epsilon$ and obtain the computational form of \cref{eq:hrh-fixed}:
        \eqn{eq:hrh-final}{
            \Derivative{T}{t} = \frac{1}{c} \left(
                C_v \exp \left(- \frac{\ln m}{3}\right)
                \left(
                    \frac{1}{2} \Lambda \rho_\mathrm{air} v^3
                    - C_T \left(T^{\,4} - T_\mathrm{air}^{\,4}\right)
                \right)
                + L \Derivative{\left(\ln m\right)}{t}
            \right).
        }



%    Another method has been discussed and implemented in the author's master's thesis \citep{balaz-thesis}
%    and used to determine the total flux of the Perseids in 2016. Instead of trying to
%    estimate the biases we generated the meteoroids above the atmosphere and simulated
%    their entry using a simplified set of equations for motion, ablation and luminosity.
%    A collection of stochastic bias functions was then applied to the dataset and
%    each meteor was marked as detected or missed. We varied the parameters of the bias functions
%    and the mass index until agreement with observational data was achieved.
%    The initial population was then declared as the model of the actual population
%    and the bias functions were taken as descriptive of the observing system.

%\section{Simulation in the atmosphere} \label{ma}
%    Once it has been determined which meteoroids will enter the Earth's atmosphere, we may proceed to simulating
%    the physical processes manifested during their atmospheric entry and obtain the values of quantities
%    that can be observed by ground-based observers.
%
%    To simulate the atmospheric entry we used \textsc{Asmodeus},
%    a multi-purpose virtual meteor simulator developed as a part of our master's thesis and extended for numerous
%    other purposes related to meteor astronomy \citep{balaz-thesis,balaz+2020}.
%
%    Once a particle is selected for atmospheric entry, its velocity is transformed to the ECEF reference frame
%    and a numerical integration of the equations of motion is executed.
%
%    \subsection{Numerical integration of the equations of motion} \label{sai}
%        Unlike in interplanetary space, where forces are completely dominated by the gravitational force exerted by the Sun
%        and only slightly perturbed by other effects,
%        a particle entering the Earth's atmosphere switches between several regimes in rapid succession.
%        The precise order is subject to variations due to particle's size, entry speed and angle
%        and material composition.
%
%        %
%    \subsection{Determination of luminosity} \label{sail}
%        Currently we use... \citep{hill+2005} \todo{measurereally?}
%
%        \citep{bronshten1983}

    \subsection{Physical quantities and properties} \label{mop}
        Apart from the position and velocities, which we have already described, we can express
        several derived quantities which are standard in meteor astronomy. This makes comparison
        with other models easier.

        \subsubsection{Mass index and the Pareto distribution} \label{moms}
            While theoretically any distribution of particle masses is possible,
            in real meteoroid populations there is always a large excess of small particles.
            In meteor science, it is customary to assume a specific distribution of masses within a population,
            described by a power law with particle mass as the independent variable,
            \eqn{eq:moms-sindex}{
                \varrho(\mu) \propto \mu^{-s}
            }
            for some constant $s$ named the \emph{mass index}.

            To obtain a distribution, this expression needs to be normalised first.
            The expression $\mu^{-s}$ diverges to infinity for $\mu \to 0$,
            which can be dealt with by setting a lower limit on the mass.
            These requirements are satisfied by the well-known Pareto distribution \citep{arnold1983}.
            In mathematical texts it is usually defined as $\Distribution{\mathrm{Pareto}}{x}{\alpha, x_0}$,
            where $\alpha > 0$ is called \emph{shape} and $x_0 > 0$ is called the \emph{scale} of the distribution.

            To conform with terminology commonly used in meteor science, we substitute the shape with mass index
            $s = \alpha - 1$, and rename $x_0$ to \emph{minimum mass} of particles, $\mu_\mathrm{min}$.
            The probability density function is then
            \eqn{eq:moms-pareto}{
                \varrho(\mu) = \mathrm{Pareto}\left(\mu;\ s-1, {\mu_\mathrm{min}}\right) \equiv
                \begin{cases}
                    0 &
                        \text{for }\mu < \mu_{\mathrm{min}}\text{,} \\
                    \dfrac{\left(s - 1\right) \mu_{\mathrm{min}}^{s - 1}}{\mu^s} &
                        \text{for }\mu \geq \mu_{\mathrm{min}}\text{.}
                \end{cases}
            }

            Typical values of $s$ are between \numrange{1.5}{2.5}, with values slightly below 2 being most common
            for meteor showers and values around \num{2.2} for the sporadic background \citep{blaauw+2011}.
            The collisional equilibrium for particles of equal strength is reached at $s = 11/6$ \citep{dohnanyi1969}.

            The mass index $s$ can be related to the population index $r$ by an empirical relationship by \citet{koschack+1990},
            \eqn{eq:moms-rs}{
                r = 1 + \num{2.3} \log_{10} s.
            }

    \subsection{Additional properties} \label{moa}
        Apart from the dynamical properties, which describe only the density of particles
        and thus a probability of a collision at a point in space and time,
        we may observe and analyze additional properties of the stream.


\section{Simulations} \label{mi}
    At the core of both the orbital and the observational models are numerical simulations.
    The simulation works with a generic model of the analyzed physical processes and some initial
    distribution of meteoroids and tries to match its output to observational data.


%\section{Simulation in the atmosphere} \label{ma}
%    Once it has been determined which meteoroids will enter the Earth's atmosphere, we may proceed to simulating
%    the physical processes manifested during their atmospheric entry and obtain the values of quantities
%    that can be observed by ground-based observers.
%
%    To simulate the atmospheric entry we used \textsc{Asmodeus},
%    a multi-purpose virtual meteor simulator developed as a part of our master's thesis and extended for numerous
%    other purposes related to meteor astronomy \citep{balaz-thesis,balaz+2020}.
%
%    Once a particle is selected for atmospheric entry, its velocity is transformed to the ECEF reference frame
%    and a numerical integration of the equations of motion is executed.
%
%    \subsection{Numerical integration of the equations of motion} \label{sai}
%        Unlike in interplanetary space, where forces are completely dominated by the gravitational force exerted by the Sun
%        and only slightly perturbed by other effects,
%        a particle entering the Earth's atmosphere switches between several regimes in rapid succession.
%        The precise order is subject to variations due to particle's size, entry speed and angle
%        and material composition.
%
%        \subsubsection{Acting forces} \label{saia}
%            In the simulation we consider only two real forces:
%            \begin{itemize}
%                \item the \emph{drag force} $\vec{F_d}$, always acting against the instantaneous velocity vector of the particle;
%                \item the \emph{gravitational force} $\vec{F_g}$, pulling the meteoroid towards the centre of the Earth.
%            \end{itemize}
%
%            The precision of the calculations may be improved if we also account for two fictitious forces,
%            arising from the fact the simulation is performed in a rotating reference frame:
%            \begin{itemize}
%                \item the fictitious \emph{centrifugal force} $\vec{F_{\mathrm{C}}}$,
%                    pushing the particle away from the axis of rotation of the Earth;
%                \item and the fictitious \emph{Coriolis force} $\vec{F_{\mathrm{c}}}$,
%                    which pushes the moving body in a direction perpendicular to its velocity vector.
%            \end{itemize}
%
%            Except for the drag force, all forces can be readily expressed in terms of instantaneous
%            position and velocity of the meteoroid and several constants.
%            Calculating the drag force precisely is computationally very expensive and cannot be done
%            without knowing the shape of the particle. Therefore we have to choose an appropriate simplified model.
%
%            In its cosmic stage and during most of the motion in the gravitational sphere of influence of the Earth,
%            the forces acting on the particle are dominated by gravitational effects. As the meteoroid penetrates
%            deeper into the atmosphere, the density of the surrounding gas increases roughly exponentially.
%            At its peak, the atmospheric drag force exceeds all other effects by several orders of magnitude,
%            decelerating the meteoroid very rapidly.
%            As the duration of the entire event is on the order of seconds, other forces simply do not act for long enough
%            to produce a significant d964201eviation in trajectory. For statistical evaluation of artificial meteors
%            it is sufficient to consider only the drag force.
%            However, should the meteoroid survive the entry as a meteorite, their influence is important during the dark phase of flight.



\section{Case studies} \label{mc}
    Here we will analyze some case studies.

\Epigraph[0.4]{
    You saw sagacious Solomon \\
    You know what came of him \\
    To him, complexities seemed plain
}{
    How Fortunate the Man With None \\
    \textsc{Bertolt Brecht} \\
    \textit{translation John Willett}
}

In this chapter we describe the forces acting on a body in the atmosphere and how they affect the resulting motion.
At its end, we should obtain several dynamical models of various complexity and precision.
Each such model should comprise a set of differential equations that govern the evolution
of the particle's position and velocity over time until it is destroyed, reaches the surface
or escapes the Earth.

\section{Analysis} \label{da}
    First of all, we need to decide which properties of the meteoroids
    and the environment are relevant to our effort.



    we obtain a tuple which we we call the \emph{state} of the meteoroid.
    This tuple contains the key properties which are bound to the meteoroid body,
    and which can vary during its lifetime and determine its final fate.
    We use the same tuple in all models, with some properties
    kept constant or ignored completely in simpler models where they are not needed.

    The properties of the meteoroid stored in the state are
    \begin{itemize}
        \item \emph{Position}, stored as a three-dimensional Cartesian vector $(x, y, z)$
            in the ECEF\footnote{Earth-Centered, Earth-Fixed (see \cref{G})} coordinate system.
            In text or graphical output it is more convenient to represent
            coordinates in the \emph{WGS84} coordinate system, however
            all calculations are to be performed in Cartesian coordinates.
        \item \emph{Velocity}, stored as a three-dimensional vector
            $(\Derivative{x}{t}, \Derivative{y}{t}, \Derivative{z}{t})$
            in the ECEF frame.
        \item Instantaneous \emph{mass} of the meteoroid. In order to mitigate
            problems with rounding and numerical stability, $\log m$ is stored internally instead.
        \item Surface \emph{temperature} of the meteoroid. In all used models
            it is taken as a constant throughout the volume of the body.
            Surface temperature directly affects the rate of evaporation of the material
            and also affect the drag coefficient of the moving body in more advanced models.
    \end{itemize}

    The remaining properties are constant during the lifetime of the simulated meteoroid.
    They are set at the creation of the particle and do not have to be included in the state.
    Of course, the differential equations describing the time evolution of the state
    may (and will) depend on them.

    \subsection{Phases of meteoroid entry} \label{dap}
        Within the classical categorisation of phases of meteoroid flight \citep{???}
        we 

        \begin{enumerate}
            \item For the vast majority of its lifetime the meteoroid remains in
                the \emph{interplanetary phase}, during which it is orbiting the Sun
                far from any planetary atmosphere. Neglecting solar wind and
                the very tenuous interplanetary medium, we may consider this to be vacuum.

                In our simplification, the body does not undergo any changes and its mass remains constant.
                Its surface or average temperature might vary as the distance from
                the Sun changes on its elliptical orbit, but this variation
                is generally slow -- enough to say that immediately before the entry it only
                depends on the properties of the meteoroid material and its current distance from the Sun,
                which is necessarily about \SI{1}{au}.

                From the dynamical point of view it is useful to distinguish a true heliocentric orbit
                (as long as the particle is outside of the gravitational sphere of influence of the Earth)
                and a hyperbolic geocentric orbit (once it has crossed into the Earth's vicinity).
                However, from the perspective of ablation, nothing is changed
                as long as heating from collisions with atmospheric molecules is negligible,
                which does not happen until the meteoroid crossed to within several hundred
                kilometres from the surface of the Earth.
            \item The \emph{pre-heating phase}, which occurs in the exosphere.
                The frequency of collisions with particles of the Earth's atmosphere
                is no longer negligible but still very low. The surface temperature rises,
                but not enough to produce a visible trail. Some ablation is already possible.
            \item As the density of the atmosphere increases, so does the frequency of collisions
                and the heat flux absorbed by the meteoroid. The surface temperature gradually increases
                until the meteoroid produces a visible glow and thus enters the \emph{light phase}.
                This typically happens at altitudes of about \SIrange{90}{120}{\kilo\metre}.

                The density of the atmosphere at this altitude presents significant drag.


                dynamical pressure may exceed the toughness of the particle,
                so fragmentation is also possible. Peak brightness

                For more massive meteoroids (masses on the order of kilograms to tons) the light phase often continues
                to below \SI{50}{\kilo\metre} or even lower. Very dense and solid meteoroids or small asteroids
                it is possible not to be slowed down and impact the surface at hypersonic speeds,
                emitting light until the last moment.

            \item
                Small particles burn up completely within the upper atmosphere,
                leaving only a trail of ionized material and possibly microscopic dust behind.
                Meteoroids with pre-atmospheric masses of tens of kilograms (or fragments
                of larger bodies which separated from the parents body while still in the atmosphere)
                are gradually decelerated and their surface is cooled enough
                to stop emitting visible light and enters the \emph{dark flight phase}.

                With current technology it is virtually impossible to observe them directly.
                Trajectories can be computed and simulated from a known combination
                of position and velocity during the light phase of flight, and extrapolated
                until they reach the surface.

                The resulting area of possible impacts is called the \emph{strewn field}.
        \end{enumerate}


\section{Acting forces} \label{df}
    First, we need to consider which forces are to be included in the simulation.

    In the simulation we consider two real forces,
    \begin{itemize}
        \item the \emph{gravitational force} $\vec{F_g}$, pulling the meteoroid towards the centre of the Earth;
        \item and the \emph{drag force} $\vec{F_d}$, acting against the instantaneous velocity vector of the particle.
    \end{itemize}

    Other forces, such as aerodynamic lift produced by irregular shape of the meteoroid
    or rocket effect, caused by expulsion of gases during ablation or fragmentation,
    are impossible to be modeled without precise knowledge of meteoroid shape and composition
    and are thus ignored in our simulations.

    However, since the simulation is run in a non-inertial, rotating reference frame,
    two fictitious forces are also present:
    \begin{itemize}
        \item the \emph{centrifugal} or \emph{Huygens force} $\vec{F_{\mathrm{Huygens}}}$,
            pushing a body away from the axis of rotation of the Earth;
        \item and the \emph{Coriolis force} $\vec{F_{\mathrm{Coriolis}}}$,
            which pushes a moving body in a direction perpendicular to its velocity vector.
    \end{itemize}
    Including them in the simulation increases the precision at a negligible computational cost.

    Except for the drag force, all forces can be readily expressed in terms of instantaneous
    position and velocity of the meteoroid and several constants.
%
%            In its cosmic stage and during most of the motion in the gravitational sphere of influence of the Earth,
%            the forces acting on the particle are dominated by gravitational effects. As the meteoroid penetrates
%            deeper into the atmosphere, the density of the surrounding gas increases roughly exponentially.
%            At its peak, the atmospheric drag force exceeds all other effects by several orders of magnitude,
%            decelerating the meteoroid very rapidly.
%            As the duration of the entire event is on the order of seconds, other forces simply do not act for long enough
%            to produce a significant deviation in trajectory. For statistical evaluation of artificial meteors
%            it is sufficient to consider only the drag force.
%            However, should the meteoroid survive the entry as a meteorite, their influence is important during the dark phase of flight.
%

%
%            \paragraph{Drag force} \label{said}
%                The precise description of the drag force is crucial to understand the motion
%                of meteoroids in the atmosphere. The drag force is dominant between approximately
%                the point where meteoroids begin to emit visible light (about \SI{100}{\kilo\metre})
%                until the very end of the visible trail. For all but the most massive meteoroids,
%                which are able to survive the atmospheric entry, this can be approximated
%                by the point where all of the matter has been ablated away; while for the bodies
%                surviving as meteorites, during the terminal phase of the flight the drag and
%                gravitational forces are nearly balanced out.
%
%                At peak deceleration the magnitude of the drag force can reach values several orders of magnitude
%                higher than the sum of all other forces and is thus completely dominant.
%                A very simple model of atmospheric flight can thus perform relatively
%                well even when all other forces are neglected.



    \subsection{Gravity} \label{dfg}
        The true shape of the gravitational potential field around the Earth is complex.
        Fortunately, meteoroids are mostly moving on highly hyperbolic trajectories,
        and a simple approximation by a Newtonian point source with mass $M_\Earth$ is sufficient.
        The formula used in computation of gravitational force is
        \eqn{eq:dfg-gravity}{
            \vec{F_{\mathrm{G}}}(\vec{r}) = - \frac{GM_\Earth}{r^3} \vec{r}.
        }

        Its magnitude is at most about \SI{10}{\metre\per\second\squared} near the surface.

    \subsection{Coriolis force} \label{dfC}
        The fictitious Coriolis force arises in a rotating reference frame.
        It depends explicitly on the velocity vector of the particle,
        \eqn{eq:dfg-coriolis}{
            \vec{F_{\mathrm{C}}}(\vec{v}) = -2 m \vec{\Omega}_\Earth \times \vec{v},
        }
        where $\vec{\Omega}_\Earth$ is the angular speed of rotation of the Earth,
        $\frac{2\pi}{\SI{86164}{\second}} \approx \SI{7.292e-5}{\per\second}$
        with the rotation axis in the direction towards the north celestial pole.

        For a meteoroid, its magnitude is at most on the order of
        \eqn{eq:dfg-order}{
            2 \frac{2\pi}{\SI{86164}{\second}} \cdot \SI{70}{\kilo\metre\per\second} \approx \SI{10}{\metre\per\second\squared},
        }
        roughly the same as gravitational acceleration.

    \subsection{Huygens force} \label{dfH}
        In a rotating reference frame, the centrifugal Huygens force pushes
        the particles away from the axis of rotation. It is equal to
        \eqn{eq:saic-centrifugal}{
            \vec{F_{\mathrm{c}}}(\vec{r})
                = - m \vec{\Omega}_\Earth \times \left(\vec{\Omega}_\Earth \times \vec{r}\right).
        }

        Since it only depends on the instantaneous position, it can be thought of as a minor correction to the
        gravitational force. A simple calculation shows that near the surface, its magnitude is on the order
        of \SI{0.03}{\metre\per\second\squared} and it can be safely ignored in the simulation.
        It only becomes important when the distance from the axis of rotation is large -- for
        instance if the simulation is used to investigate the motion of a meteoroid further from the Earth.
        However, for orbital simulations it is more advantageous to use an inertial reference frame instead.

\section{Drag force} \label{fd}
        The last -- but the most important and least understood -- force acting on the meteoroid during
        its atmospheric entry is the drag force. It arises as a consequence of its collisions with
        the molecules of atmospheric air, decreasing the momentum of the meteoroid and at the
        same time converting its kinetic energy to heat.

        Unlike other forces, which can be readily described in terms of several easily understandable
        quantities, drag force is much more variable and somewhat inscrutable: it depends on the shape
        of the particle, which is not known \textit{a priori}. It can be measured after
        a meteorite is recovered, but for most part we have to work with assumptions.
        Drag also depends on atmospheric density, which can be modelled but not measured precisely,
        and varies with time of day, time of year and solar activity.

        And it gets worse. Although certain limits can be described accurately using
        simple concepts and formulae, in general the behaviour is chaotic.
        For instance in exosphere the air density is extremely low,
        and interactions between constituent particles are very rare.
        Atmospheric drag can be then modelled as a succession of collisions with individual
        molecules, which is fairly simple.
        On the other hand, a microscopic dust particle falling near the sea level encounters
        Reynolds numbers so low that the situation can be described by the relatively simple Stokes' drag.
        \todo{is this actually true?}

        However, inbetween these two extremes an entering meteoroid experiences a vast range of conditions:
        a hypersonic entry of the particle into the atmosphere, whose density grows exponentially with increasing depth;
        then a transonic transition, where all sorts of turbulence and flow separation effects come into play,
        and finally an atmospheric free fall at subsonic terminal velocity.

        Furthermore, the functional dependences describing the exact motion
        of air around the particle are often chaotic and according to modern knowledge
        finding closed-form solutions is not possible at all, so we have to rely
        on numerical investigation using CFD simulations, or by semi-empirical
        relations that have to be derived from experiments.


        a tradeoff between accuracy and complexity (and thus computation time).
        The models described here vary from very simple to 

        one particular problem is that the exact shape of the meteoroid is never known beforehand,
        and any simulation working with the real shape can only be performed after the meteorite is
        found -- which somewhat subtracts from the usefulness of running the simulation at all.

        Therefore we always assume that our spherical particles.
        This approximation is not justified physically, but by lack of any better methods (OK this is cruel).

    \subsection{Viscosity} \label{fdv}
        Another parameter is the viscosity of the medium in which
        motion is taking place, in our case atmospheric air.
        However, viscosity is only important under two conditions: that the flow stays
        in the continuum regime, and that it is not fully turbulent.

        In the free-molecular flow limit $\mathrm{Kn} \to \infty$ the drag force does not
        depend on viscosity. The concept of viscosity is not even applicable here -- the
        molecules only interact with the meteoroid but not with each other.
        Despite that, the drag force is nonzero as long as there are collisions
        between the particles of the medium and the meteoroid.

        In the turbulent limit ($\mathrm{Re} \to \infty$) viscous forces are dominated
        by inertial forces and intermolecular forces can be ignored.

        \subsubsection{Formulae} \label{fdvf}
            The only medium we need to consider is regular air, composed of roughly 78\%~N,
            21\%~O, 1\%~Ar and trace amounts of other gases.

            \begin{example}[Homosphere and heterosphere]
                The atmosphere can be divided into two regions based on its chemical composition:
                \begin{itemize}
                    \item From the sea level to approximately \SI{100}{\kilo\metre} the chemical
                        composition does not vary appreciably (see \cref{fig:fdvf-species}),
                        as the constituents are constantly mixed by atmospheric turbulence.
                        The composition is thus the same as near the surface, where it can be readily analysed.
                        This part of the atmosphere is called the \emph{homosphere}.
                    \item Above about \SI{100}{\kilo\metre}, lighter species are more abundant
                        relative to heavier species. The dominant mixing mechanism is molecular diffusion.
                        Mixing depends on thermal velocities of molecules, which decrease
                        with increasing molecular mass. This region is called the \emph{heterosphere}.

                        Variations in chemical composition affect viscosity and other properties
                        of the flow -- however, mean free paths of molecules at these altitudes are
                        very long and the flow around the meteoroid is essentially always free-molecular.
                        Viscous drag can be safely neglected there.
                \end{itemize}
            \end{example}

            \fig{msis-species.pdf}{0.875\textwidth}{fig:fdvf-species}[ht]{
                The volume fraction of the main constituents of the Earth's atmosphere
                as a function of height, based on the MSIS-E-90 atmospheric model.
                The homosphere extends to about \SI{100}{\kilo\metre} and is clearly visible
                in the graph.

                \textit{Credit: Wikipedia, user Amaurea, based on the data from the MSIS-E-90 model.}
            }

            For simple calculations a constant value is customarily used,
            \eqn{eq:fdvf-const}{
                \mu_\mathrm{air} = \SI{1.8e-5}{\pascal\second}.
            }

            A better model also considers the change of viscosity with temperature as well.
            The exact dependence cannot be derived easily, but various empirical correlations
            exist, usually in the form $\mu \propto T^\omega$.
            For air, one such approximation \citep{tec-science-viscosity}
            uses $\omega = \num{0.7355}$. Viscosity is then determined~as
            \eqn{eq:fdvf-exp}{
                \mu_\mathrm{air}(T) = \num{2.791e-7} \left(\frac{T}{\si{\kelvin}}\right)^{\num{0.7355}} \si{\pascal\second}.
            }

            The correlation is considered valid for $\SI{253}{\kelvin} \leq T \leq \SI{673}{\kelvin}$
            and yields the value from the simpler model for $T \approx \SI{288.5}{\kelvin}$.
            Since temperatures encountered in the atmosphere mostly fall within this range
            and the model is fairly simple, we use it in all our calculations.

    \subsection{Atmospheric density} \label{fdd}

        \subsubsection{No atmosphere} \label{fddn}
            The simplest possible model does not account for the presence of the atmosphere at all.
            It is not suitable for modelling of meteoroid entry, but it is useful for testing.

        \subsubsection{Isothermal atmosphere} \label{fddt}
            The simplest at least somewhat realistic model that allows limited closed-form
            solutions of equations of motion is that of \emph{isothermal atmosphere}.
            The atmosphere is modelled as having constant composition and temperature
            throughout its entire altitudinal profile. The density then only depends on pressure.
            Such model is called \textit{barotropic} \citep{shames1992}.
            These models are customarily computed not in terms of \textit{geometric altitude},
            but of \textit{geopotential altitude}.

            \begin{example}[Geometric and geopotential altitude]
                In numerical simulations we prefer to work with true geometric coordinates.
                However, in atmospheric modelling it is advantageous to transform the coordinates first
                in order to eliminate the slight changes in gravitational acceleration with altitude.

                The ISA model is defined in terms of geopotential meters \citep{nasa-isa}.
                Instead of measuring the true distance between two points and taking its radial component,
                the difference in actual gravitational potential between these points
                is measured. The obtained value is divided by standard acceleration due to gravity,
                $g_0 = \SI{9.80665}{\metre\per\second\squared}$. The resulting value has a dimension of length
                and is termed \textit{geopotential height}.
            \end{example}

            Pressure and density then decrease exponentially with altitude as
            \eqn{eq:fddt-denspress}{
                p(h) = p(0) e^{\frac{-mgh}{kT}}
                \qquad\text{and}\qquad
                \rho(h) = \rho(0) e^{\frac{-mgh}{kT}},
            }
            where
            \begin{itemize}
                \item $m$ is the mean molecular mass of the air ($\approx \SI{29}{\atomicmass}$),
                \item $g$ is the acceleration due to gravity ($\approx \SI{9.81}{\metre\per\second\squared}$),
                \item $h$ is the altitude above mean sea level,
                \item $k$ is the Boltzmann constant, $\SI{1.380649}{\joule\per\kelvin}$ and
                \item $T$ is the constant temperature, usually set to \SI{287.15}{\kelvin}.
            \end{itemize}

            Additionally, $p(0) = \SI{101325}{\pascal}$ and
            $\rho(0) = p(0) \frac{m}{kT} \approx \SI{1.23}{\kilo\gram\per\cubic\metre}$.

            If we consider the atmosphere to be thin compared to the radius of the Earth,
            gravitational acceleration can also be assumed to be constant, in which case
            geopotential and geometric altitude are identical on a spherical Earth model;
            and the introduced errors remain tolerably small even with the WGS84 model.

        \subsubsection{International Standard Atmosphere} \label{fddi}
            The next step towards a more precise model involves allowing the temperature to change with altitude.
            Of course, temperature also varies with time of day, time of year and is influenced by weather,
            so any model that is not directly based on current data will differ from reality.
            Most of these variations are caused by weather and are contained within the troposphere.

            However, atmospheric density at altitudes which are most important to us
            (\SIrange{60}{120}{\kilo\metre}) can vary significantly due to various factors including
            solar activity,

        \subsubsection{MSIS-E-90} \label{fdd9}
            daily variations
            yearly variations
            solar activity
            recommended for studies across several atmospheric layers \cite{...}

            For generic simulations, such as those of meteor showers,
            it is reasonable to use one output of the model that is considered valid for the entire simulation.
            In case higher precision is needed (for instance when modelling a particular bolide),
            the model output can be downloaded for the precise time and location of the event.

        \subsubsection{NRLMSISE-00} \label{fdd0}
            The next update of the MSIS-E-90 model was released in 2000 by \citet{picone+nrlmsise00}
            after updating with new satellite drag data. NRLMSISE-00 is considered to be
            the reference model for all meteor simulations.

    \subsection{Analysed drag models} \label{fdm}
        In the thesis we will compare six different drag models of varying precision
        and computational cost.

        research: see which region of combinations of Re, Kn and Ma are normally attained during flight

        six distinct drag models: \hyperref[fdmn]{no drag}, \hyperref[fdmc]{constant drag}, and four advanced
        models by \hyperref[fdmM]{Morrison}, \hyperref[fdmL]{Loth}, \hyperref[fdmH]{Henderson} and \hyperref[fdmS]{Singh et al}.

        \subsubsection{No drag} \label{fdmn}
            A trivial model in which $\Gamma = 0$ at all times.
            Again, it is not useful for atmospheric modelling, but is computationally
            very fast and makes testing other models easier.

            While atmospheric density is still non-zero in the exosphere,
            and result in minuscule drag force acting on particles on low-Earth orbits.
            Over long time periods (months to millenia, depending on altitude) it is able to slow
            an orbiting particle enough to bring it down to the surface.

            However, in meteor science any effects of this very tenuous layer can be safely ignored:
            Meteoroids are always on hyperbolic trajectories with respect to the Earth
            and pass through this volume within seconds. Any long-term effects
            are too small to appreciably alter the meteoroids' trajectories.
            Therefore all models can safely fall back to zero drag above the
            altitude of about \SI{400}{\kilo\metre}.

        \subsubsection{Constant drag model} \label{fdmc}
            In older textbooks and papers a constant value of $\Gamma$ is often taken as a first approximation.
            For a spherical particle in a turbulent flow a value of \num{0.47} is often used,
            which corresponds to a smooth sphere at a Reynolds number of \num{e5} \cite{???}.
            One of the primary objectives of this thesis is to quantify how large is the resulting
            and whether it is justifiable to use such simple models at all.

        \subsubsection{Morrison model} \label{fdmM}
            The next step is to include the effects of viscous and inertial forces -- or,
            as it is more commonly expressed, the effect of their ratio, the \emph{Reynolds number}.
            Precise values can be only obtained by direct measurement or CFD simulations,
            which is clearly not viable in our case.

            As a rough approximation we can use the correlation by \citet{morrison2016},
            which is an empirical fit of experimental data for a smooth sphere.
            The model extends from very low to very high Reynolds numbers.
            In the viscous limit it approaches
            \eqn{eq:fdM-limre}{
                \lim_{\mathrm{Re} \to 0} \Gamma = \frac{24}{\mathrm{Re}},
            }
            yielding the classic formula for laminar Stokes' drag
            \eqn{eq:fdM-stokes}{
                \lim_{\mathrm{Re} \to 0} F
                    = \lim_{\mathrm{Re} \to 0} \frac{1}{2} \Gamma S \rho_\mathrm{air} v^2
                    = \frac{1}{2} \frac{24 \mu}{2 r \rho_\mathrm{air} v} \rho_\mathrm{air} \pi r^2 v^2
                    = 6 \pi \mu r v.
            }

            It also reproduces the sharp decrease of drag coefficient at high Reynolds
            numbers ($\mathrm{Re} \approx \num{e5}$), known as the \emph{drag crisis}
            or the \emph{Eiffel paradox} \citep{eiffel1912}.

            \fig{drag-crisis.pdf}{90mm}{ddm-drag-crisis}{%
                Dependence of drag coefficient $\Gamma$ for a smooth sphere (teal) and rough sphere (green).
                For a smooth sphere, the drag coefficient decreases abruptly from about \num{0.5} to
                \num{0.1} at $\mathrm{Re} \approx \num{3e5}$, which corresponds to the transition from
                laminar to turbulent boundary flow \citep{roshko1961}.

                For a rough sphere, the magnitude of the decrease is significantly smaller, but the change
                occurs at a much lower Reynolds number, about \num{e5}. Although meteoroids are
                never perfectly smooth spheres, a rough sphere is a reasonable
                approximation where there is no prior knowledge about the shape.

                \textit{Credit: Original image by \citet{nasa-drag}, vectorized by%
                Wikipedia user Kraaiennest, edits by Olivier Cleynen}
            }

            The problem with this model is that it is only valid in the incompressible limit.
            For air, this is justifiable at low speeds ($M \lessapprox \numrange{0.2}{0.3}$ or
            about \SIrange{70}{100}{\metre\per\second}) and only in continuum flow ($\mathrm{Kn} < \num{0.01}$).
            These conditions are not met during most of an atmospheric entry,
            but the model can be useful for simulating the atmospheric free fall.

            \begin{example}[A note on the term ``free fall'']
                The terms here are a bit confusing. We distinguish true \emph{free fall},
                the motion of a body in the absence of external forces other than gravity;
                and an \emph{atmospheric free fall}, the motion of a body where the atmospheric
                drag force is already in equilibrium with the gravitational force.

                To make matters even more complicated, an object falling from rest
                (or being slowed down from a higher-than-terminal speed, as in the case of a meteoroid)
                can also be said to be in ``free fall'' all the time even if neither
                of the stricter definitions applies at any moment.
            \end{example}

        \subsubsection{Henderson model} \label{fdmH}
            One of the first high-precision drag models for spherical bodies,
            valid across a wide range of Reynolds and Knudsen numbers, was published by \citet{henderson1976}.
            It was specifically designed to be utilized in computer simulations
            and displayed high agreement with experimental data.

        \subsubsection{Loth model} \label{fdmL}
            A further refinement is due to \citet{loth2008}.
            The Loth model accounts for effects of compression of the gas
            during subsonic and supersonic flight.

            The full model is described in \cref{DH}.

        \subsubsection{Singh et al.} \label{fdmS}
            A semi-empirical, state-of-the-art model with a wide range of validity has been developed by
            \citet{singh+2020}. The model is derived from first principles and
            uses rational polynomial bridging functions to interpolate between
            the limits of well understood flight regimes (as described in \cref{dap}).

            At the time of writing it is the most elaborate and complex model available.
            It is considered valid throughout the entire range of
            conditions encountered during the atmospheric entry of a meteoroid
            ($0 < \mathrm{Kn} < \infty$, $0 < \mathrm{Re} < \num{e6}$, $0 < M < 200$)
            (N. Singh, personnal communication, 2021--05--13).
            For a full set of equations see \cref{DS}.

\section{Conclusion} \label{dc}

    \subsubsection{Whipple model} \label{mmw}
        A standard, textbook model of meteoroid flight introduces a change in mass of the meteoroid
        during its flight through the atmosphere.
        (Öpik) (Ceplecha)

        Again, we assume a constant (spherical or nearly spherical) shape throughout the entire
        lifetime of the meteoroid.\footnote{An advanced fragmentation model could probably assume a non-spherical
        shape in the first moments, which would quickly change back into a spherical one during ablation
        of the newly created particle.}

        The drag coefficient $\Gamma$ is 

        (Gamma) sometimes $C_d$ is used as a symbol instead.

        are generally not available and would require running a proper CFD simulation
        of the shape in question at various densities, orientations and flow speeds.

        Valid across a wide range of conditions, as long as the value of Re is the same.
        ultra-hypersonic flow?

        A single value is used during the entire flight. Typically


        \eqn{eq:reynolds}{
            \mathrm{Re} = \frac{\rho_\infty v d}{\mu}.
        }
        where
        \begin{description}
            \item[$d$] is the \emph{typical size} of the meteoroid in the stream.
                For spherical or nearly-spherical particles the diameter ($d = 2r$) is used;
            \item[$\rho_\infty$] is the \emph{density} of the surrounding fluid, in this case
                air at the location of the meteoroid, but ignoring the compression caused
                by the moving body itself;
            \item[$\mu$] is the \emph{dynamic viscosity} of the surrounding medium.
        \end{description}

        Except for $\mu$, all three quantities will change significantly
        during the atmospheric entry: $\rho_\mathrm{air}$ is negligible at first
        and increases roughly exponentially in the range of approximately \SIrange{e-10}{1}{\kilo\gram\per\metre\cubed}
        in the region of our interest, $v$ is typically on the order of \SIrange{1e4}{7e4}{\metre\per\second}
        just above the atmosphere but can drop to about \SI{1}{\metre\per\second}
        for a microscopic particle just above the surface, and the diameter also spans
        about four orders of magnitude (\SIrange{e-4}{1}{\metre}).
        All combined, this means that a wide range of values of $\mathrm{Re}$ has to be investigated.


        To save computation time, two constants can be evaluated in advance as
        \eqn{eq:mconst}{
            C_v = A \rho^{-2/3}
            \QQText{and}
            C_m = \frac{\Lambda A \rho^{-2/3}}{2 Q} = \frac{C_v \Lambda}{2 Q}.
        }


    \subsubsection{Material evaporation model} \label{mwh}
        Another useful extension of the model described in \cref{mmw} is due to
        \cite{hill+2005}, which adds temperature as another variable property of the meteoroid.


        In the code it is referred to as \texttt{HRH2005}.

        \eqn{eq:hrh}{
            \Derivative{T}{t} = \frac{A}{c m^{1/3} \rho^{2/3}}
                \left(\frac{1}{2} \Lambda \rho_\mathrm{air} v^3
                    - 4 \sigma \epsilon \left(T^{\,4} - T_\mathrm{air}^{\,4}\right)
                    + \frac{L}{A} \left(\frac{\rho}{m}\right)^{2/3} \Derivative{m}{t}
                \right),
        }
        where
        \begin{description}
            \item[$T$]
                temperature of the meteoroid ($\mathrm{K}$),
            \item[$t$]
                time (\si{\second}),
            \item[$A$]
                shape factor (dimensionless),
            \item[$c$]
                specific heat of the meteoroid ($\si{\joule\per\kilo\gram\per\kelvin} \equiv \si{\metre\squared\per\second\squared\per\kelvin}$),
            \item[$m$]
                instantaneous mass of the meteoroid (\si{\kilo\gram}),
            \item[$\rho$]
                density of the meteoroid (\si{\kilo\gram\per\metre\cubed}),
            \item[$\Lambda$]
                heat transfer coefficient (\si{1}),
            \item[$\rho_\mathrm{air}$]
                density of the atmosphere at the position of the meteoroid (\si{\kilo\gram\per\metre\cubed}),
            \item[$\sigma$]
                Stefan--Boltzmann constant (\SI{5.67e-8}{\watt\per\metre\squared\per\kelvin\tothe{4}}),
            \item[$\epsilon$]
                emissivity of the meteoroid (dimensionless),
            \item[$v$]
                instantaneous speed of the meteoroid,
            \item[$T$]
                temperature of the meteoroid (\si{\kelvin}),
            \item[$T_\mathrm{air}$]
                temperature of the atmosphere at the position of the meteoroid (\si{\kelvin}),
            \item[$L$]
                latent heat of fusion and vaporization ($\si{\joule\per\kilo\gram} \equiv \si{\metre\squared\per\second\squared}$).

        \end{description}

        The state of the meteroid is then defined by the tuple $(\vec{r}, \vec{v}, \ln m, T)$.

        \eqn{eq:hrh2}{
            \Derivative{T}{t} = \frac{1}{c} \left(
                \frac{A}{m^{1/3} \rho^{2/3}}
                \left(
                    \frac{1}{2} \Lambda \rho_\mathrm{air} v^3
                    - 4 \sigma \epsilon \left(T^{\,4} - T_\mathrm{air}^{\,4}\right)
                \right)
                + \frac{L}{m} \Derivative{m}{t}
            \right)
        }

        Again, to increase numerical stability we may substitute $m \to \ln m$,
        by the chain rule
        \eqn{eq:hrh-chain}{
            \Derivative{\left(\ln m\right)}{t} = \frac{1}{m} \Derivative{m}{t}.
        }

        The second term then can be separated and the whole equation \ref{eq:hrh} rewritten as
        \eqn{eq:hrh-fixed}{
            \Derivative{T}{t} = \frac{1}{c} \left(
                A \rho^{-2/3} \exp \left(- \frac{\ln m}{3}\right)
                \left(
                    \frac{1}{2} \Lambda \rho_\mathrm{air} v^3
                    - 4 \sigma \epsilon \left(T^{\,4} - T_\mathrm{air}^{\,4}\right)
                \right)
                + L \Derivative{\left(\ln m\right)}{t}
            \right).
        }

        To further optimize the computation, we may precompute another constant
        $c_T = 4 \sigma \epsilon$ and obtain the computational form of \cref{eq:hrh-fixed}:
        \eqn{eq:hrh-final}{
            \Derivative{T}{t} = \frac{1}{c} \left(
                C_v \exp \left(- \frac{\ln m}{3}\right)
                \left(
                    \frac{1}{2} \Lambda \rho_\mathrm{air} v^3
                    - C_T \left(T^{\,4} - T_\mathrm{air}^{\,4}\right)
                \right)
                + L \Derivative{\left(\ln m\right)}{t}
            \right).
        }



%    Another method has been discussed and implemented in the author's master's thesis \citep{balaz-thesis}
%    and used to determine the total flux of the Perseids in 2016. Instead of trying to
%    estimate the biases we generated the meteoroids above the atmosphere and simulated
%    their entry using a simplified set of equations for motion, ablation and luminosity.
%    A collection of stochastic bias functions was then applied to the dataset and
%    each meteor was marked as detected or missed. We varied the parameters of the bias functions
%    and the mass index until agreement with observational data was achieved.
%    The initial population was then declared as the model of the actual population
%    and the bias functions were taken as descriptive of the observing system.

%\section{Simulation in the atmosphere} \label{ma}
%    Once it has been determined which meteoroids will enter the Earth's atmosphere, we may proceed to simulating
%    the physical processes manifested during their atmospheric entry and obtain the values of quantities
%    that can be observed by ground-based observers.
%
%    To simulate the atmospheric entry we used \textsc{Asmodeus},
%    a multi-purpose virtual meteor simulator developed as a part of our master's thesis and extended for numerous
%    other purposes related to meteor astronomy \citep{balaz-thesis,balaz+2020}.
%
%    Once a particle is selected for atmospheric entry, its velocity is transformed to the ECEF reference frame
%    and a numerical integration of the equations of motion is executed.
%
%    \subsection{Numerical integration of the equations of motion} \label{sai}
%        Unlike in interplanetary space, where forces are completely dominated by the gravitational force exerted by the Sun
%        and only slightly perturbed by other effects,
%        a particle entering the Earth's atmosphere switches between several regimes in rapid succession.
%        The precise order is subject to variations due to particle's size, entry speed and angle
%        and material composition.
%
%        %
%    \subsection{Determination of luminosity} \label{sail}
%        Currently we use... \citep{hill+2005} \todo{measurereally?}
%
%        \citep{bronshten1983}

    \subsection{Physical quantities and properties} \label{mop}
        Apart from the position and velocities, which we have already described, we can express
        several derived quantities which are standard in meteor astronomy. This makes comparison
        with other models easier.

        \subsubsection{Mass index and the Pareto distribution} \label{moms}
            While theoretically any distribution of particle masses is possible,
            in real meteoroid populations there is always a large excess of small particles.
            In meteor science, it is customary to assume a specific distribution of masses within a population,
            described by a power law with particle mass as the independent variable,
            \eqn{eq:moms-sindex}{
                \varrho(\mu) \propto \mu^{-s}
            }
            for some constant $s$ named the \emph{mass index}.

            To obtain a distribution, this expression needs to be normalised first.
            The expression $\mu^{-s}$ diverges to infinity for $\mu \to 0$,
            which can be dealt with by setting a lower limit on the mass.
            These requirements are satisfied by the well-known Pareto distribution \citep{arnold1983}.
            In mathematical texts it is usually defined as $\Distribution{\mathrm{Pareto}}{x}{\alpha, x_0}$,
            where $\alpha > 0$ is called \emph{shape} and $x_0 > 0$ is called the \emph{scale} of the distribution.

            To conform with terminology commonly used in meteor science, we substitute the shape with mass index
            $s = \alpha - 1$, and rename $x_0$ to \emph{minimum mass} of particles, $\mu_\mathrm{min}$.
            The probability density function is then
            \eqn{eq:moms-pareto}{
                \varrho(\mu) = \mathrm{Pareto}\left(\mu;\ s-1, {\mu_\mathrm{min}}\right) \equiv
                \begin{cases}
                    0 &
                        \text{for }\mu < \mu_{\mathrm{min}}\text{,} \\
                    \dfrac{\left(s - 1\right) \mu_{\mathrm{min}}^{s - 1}}{\mu^s} &
                        \text{for }\mu \geq \mu_{\mathrm{min}}\text{.}
                \end{cases}
            }

            Typical values of $s$ are between \numrange{1.5}{2.5}, with values slightly below 2 being most common
            for meteor showers and values around \num{2.2} for the sporadic background \citep{blaauw+2011}.
            The collisional equilibrium for particles of equal strength is reached at $s = 11/6$ \citep{dohnanyi1969}.

            The mass index $s$ can be related to the population index $r$ by an empirical relationship by \citet{koschack+1990},
            \eqn{eq:moms-rs}{
                r = 1 + \num{2.3} \log_{10} s.
            }

    \subsection{Additional properties} \label{moa}
        Apart from the dynamical properties, which describe only the density of particles
        and thus a probability of a collision at a point in space and time,
        we may observe and analyze additional properties of the stream.


\section{Simulations} \label{mi}
    At the core of both the orbital and the observational models are numerical simulations.
    The simulation works with a generic model of the analyzed physical processes and some initial
    distribution of meteoroids and tries to match its output to observational data.


%\section{Simulation in the atmosphere} \label{ma}
%    Once it has been determined which meteoroids will enter the Earth's atmosphere, we may proceed to simulating
%    the physical processes manifested during their atmospheric entry and obtain the values of quantities
%    that can be observed by ground-based observers.
%
%    To simulate the atmospheric entry we used \textsc{Asmodeus},
%    a multi-purpose virtual meteor simulator developed as a part of our master's thesis and extended for numerous
%    other purposes related to meteor astronomy \citep{balaz-thesis,balaz+2020}.
%
%    Once a particle is selected for atmospheric entry, its velocity is transformed to the ECEF reference frame
%    and a numerical integration of the equations of motion is executed.
%
%    \subsection{Numerical integration of the equations of motion} \label{sai}
%        Unlike in interplanetary space, where forces are completely dominated by the gravitational force exerted by the Sun
%        and only slightly perturbed by other effects,
%        a particle entering the Earth's atmosphere switches between several regimes in rapid succession.
%        The precise order is subject to variations due to particle's size, entry speed and angle
%        and material composition.
%
%        \subsubsection{Acting forces} \label{saia}
%            In the simulation we consider only two real forces:
%            \begin{itemize}
%                \item the \emph{drag force} $\vec{F_d}$, always acting against the instantaneous velocity vector of the particle;
%                \item the \emph{gravitational force} $\vec{F_g}$, pulling the meteoroid towards the centre of the Earth.
%            \end{itemize}
%
%            The precision of the calculations may be improved if we also account for two fictitious forces,
%            arising from the fact the simulation is performed in a rotating reference frame:
%            \begin{itemize}
%                \item the fictitious \emph{centrifugal force} $\vec{F_{\mathrm{C}}}$,
%                    pushing the particle away from the axis of rotation of the Earth;
%                \item and the fictitious \emph{Coriolis force} $\vec{F_{\mathrm{c}}}$,
%                    which pushes the moving body in a direction perpendicular to its velocity vector.
%            \end{itemize}
%
%            Except for the drag force, all forces can be readily expressed in terms of instantaneous
%            position and velocity of the meteoroid and several constants.
%            Calculating the drag force precisely is computationally very expensive and cannot be done
%            without knowing the shape of the particle. Therefore we have to choose an appropriate simplified model.
%
%            In its cosmic stage and during most of the motion in the gravitational sphere of influence of the Earth,
%            the forces acting on the particle are dominated by gravitational effects. As the meteoroid penetrates
%            deeper into the atmosphere, the density of the surrounding gas increases roughly exponentially.
%            At its peak, the atmospheric drag force exceeds all other effects by several orders of magnitude,
%            decelerating the meteoroid very rapidly.
%            As the duration of the entire event is on the order of seconds, other forces simply do not act for long enough
%            to produce a significant d964201eviation in trajectory. For statistical evaluation of artificial meteors
%            it is sufficient to consider only the drag force.
%            However, should the meteoroid survive the entry as a meteorite, their influence is important during the dark phase of flight.



\section{Case studies} \label{mc}
    Here we will analyze some case studies.

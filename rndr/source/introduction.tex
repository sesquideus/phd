\Epigraph{
    Throughout the ages \\
    Of iron, bronze and stone \\
    We marvelled at the night sky \\
    And what may lie beyond
}{
    Children of the Sun \\
    \textsc{Dead Can Dance}
}


Raises a number of interesting questions:

\begin{itemize}
    \item Where we can find these particles?
    \item How well do these models fit the observational data?
    \item What be done to improve them?
\end{itemize}


%\section{AMOS} \label{iA}
%    The model of meteor activity and meteoroid distribution does not depend on the observational method
%    which was used to build it, as long as instrumental and selection bias effects are corrected.
%    Our primary source of observational data will be the network of autonomous all-sky cameras AMOS (All-sky Meteor Orbit System),
%    developed and operated by the Division of Astronomy and Astrophysics,
%    DAPEM\footnote{Department of Astronomy, Physics of the Earth and Meteorology}
%    FMPI\footnote{Faculty of Mathematics, Physics and Informatics} of Comenius University in Bratislava.
%
%    \subsection{Technical specifications} \label{iAt}
%        Each station consists of a digital video camera, an image intensifier, a wide angle fish-eye lens
%        and various auxiliary mechanical and electronic components effecting
%        the reliable operation of the device. The sky is monitored constantly every night
%        as long as meteorological conditions are favourable. For a detailed technical report refer to \citet{zigo+2013,toth+2015}.
%
%        The field of view of the cameras is centered on zenith and measures approximately \ang{180} by \ang{140}.
%        As the atmosphere near the horizon is thick, light travelling to the camera from meteors
%        in this area of the sky is significantly attenuated.
%        Only very few meteors are outside the covered area and the resulting loss of overall detection ability is minimal.
%
%    \subsection{Operation} \label{iAo}
%        The captured video sequences are processed by the \textsc{UFOAnalyzerV2} software package.
%        Conversion to a custom-made software package is underway.
%        Objects that are not meteors, such as aeroplanes, satellite flares, flying insects, etc.
%        are identified and discarded from the final dataset. Naturally, the processing software is not
%        infallible and occasionally reports these objects as meteors, or conversely fails to identify a visible meteor.
%        After a meteor is identified, the video sequence is stored and analyzed.
%        Properties of the meteor are computed for every frame of the video sequence,
%        such as its position on the sky, apparent velocity vector, brightness and angular speed.
%        Various auxiliary data are stored as well.
%
%        Under ideal conditions each station is able to detect approximately 20000 meteors per year.
%        Multi-station observations are somewhat less frequent, with each pair of stations being able
%        to correctly match about 8000 meteors every year.
%        With two or more simultaneous observations from different locations it is possible to determine
%        the geocentric and heliocentric velocity vectors and derive the original heliocentric orbit of the meteoroid particle.
%        For a more detailed description of the methods and algorithm used refer to \citet{kornos+2015}.
%
%    \subsection{Stations} \label{iAs}
%        As of 2020, the network comprises eleven stations at four independent locations.
%        \sisetup{group-minimum-digits = 8}
%        The network in Slovakia consists of five nearly-identical stations:
%        \begin{description}[leftmargin = 25mm]
%            \item[AGO]      Astronomical and Geophysical Observatory in Modra-Piesok\\
%                            since April 2003, \ang{48.3729}~N, \ang{17.2738}~E, \SI{531}{\metre}
%            \item[ARBO]     Arboretum Mlyňany of the Slovak Academy of Sciences\\
%                            since September 2009, \ang{48.3235}~N, \ang{18.3685}~E, \SI{201}{\metre}
%            \item[KNM]      observatory in Kysucké Nové Mesto\\
%                            since August 2012, \ang{49.3073}~N, \ang{18.7655}~E, \SI{414}{\metre}
%            \item[VAZEC]    village of Važec\\
%                            since October 2013, \ang{49.0543}~N, \ang{19.9899}~E, \SI{812}{\metre}
%            \item[SENEC]    Senec Observatory\\
%                            since January 2019, \ang{48.2203}~N, \ang{17.3950}~E, \SI{137}{\metre}
%        \end{description}
%
%        A pair of stations has been installed on the Canary Islands, with cameras at
%        \begin{description}[leftmargin = 25mm]
%            \item[LP]       Observatorio del Roque de los Muchachos, La Palma \\
%                            since March 2015, \ang{28.7600}~N, \ang{17.8823}~W, \SI{2339}{\metre}
%            \item[TE]       Observatorio del Teide, Tenerife \\
%                            since March 2015, \ang{28.3004}~N, \ang{16.5122}~W, \SI{2416}{\metre}
%        \end{description}
%
%        The third installation consists of two stations in northern Chile, located at
%        \begin{description}[leftmargin = 25mm]
%            \item[SP]       San Pedro de Atacama, El Loa, Antofagasta \\
%                            since March 2016, \ang{22.9534}~S, \ang{68.1793}~W, \SI{2403}{\metre}
%            \item[PC]       Paniri Caur Observatory, Chiu Chiu, El Loa, Antofagasta \\
%                            since March 2016, \ang{22.3360}~S, \ang{68.6442}~W, \SI{2535}{\metre}
%        \end{description}
%
%        The last subnetwork is located on Hawai`i, with stations at
%        \begin{description}[leftmargin = 25mm]
%            \item[HK]       Haleakalā High Altitude Observatory Site, Kula, Maui \\
%                            since September 2018, \ang{20.7075}~N, \ang{156.2560}~W, \SI{3041}{\metre}
%            \item[MK]       Mauna Kea Observatories, Hawai`i County \\
%                            since September 2018, \ang{19.8228}~N, \ang{155.4697}~W, \SI{4126}{\metre}
%        \end{description}
%
%        \sisetup{group-minimum-digits = 5}
%
%        Expansion of the network to more locations, namely South Africa and Australia, is planned.
%        With a total of six subnetworks, it will be possible to cover both the Northern and
%        the Southern hemispheres continuously, weather permitting.

\Epigraph[0.4]{
    It is not what you think of \\
    when I say I'm doing modelling.
}{
}




\subsection{Models of meteoroid dynamics} \label{mm}





\section{Models of the Earth}


    \subsection{Locally flat Earth} \label{mmf}


    \subsection{Spherical Earth} \label{mms}
        As the Earth is roughly spherical, 

        In mathematics, it is custom to denote spherical coordinates by a triplet
        $(\theta, \phi, r)$

        (picture)

        However, for describing location in the ECEF frame it is more convenient
        to use the \emph{geographical convention} with coordinates $(\phi, \lambda, h)$:
        \begin{itemize}
            \item latitude $\phi \in \IntervalCC{\ang{-90}}{\ang{+90}}$, measured from the equator,
                with values increasing towards the North Pole;
            \item longitude $\lambda \in \IntervalCC{\ang{-180}}{\ang{+180}}$, measured from the IERS Reference Meridian,
                with values increasing towards east;
            \item altitude $h \in \IntervalCC{-R_\Earth}{\infty}$, measured from the surface of
                a sphere with diameter equal to the radius of the model of the Earth $R_\Earth$, increasing outwards.
                At this level of approximation the value of $R_0$ can be chosen somewhat arbitrarily.
                The equatorial radius of \SI{6378137}{\metre} or mean radius of \SI{6371008}{\metre} are commonly used.
                For the sake of clarity and consistency we always use the
                rounded value of \SI{6371000}{\metre} in our models.
        \end{itemize}

        (picture)

        Tre transformations functions are then
        \aln{eq:mms-xyz-to-sph}{
            h       &= \sqrt{x^2 + y^2 + z^2} - R_0, \\
            \phi    &= \arccos{\frac{z}{x^2 + y^2 + z^2}}, \\
            \lambda &= \mathrm{atan2}(y, x).
        }
        and the inverse transformation is given as
        \aln{eq:mss-sph-to-xyz}{
            x &= \left(h + R_\Earth\right) \cos\phi \cos\lambda, \\
            y &= \left(h + R_\Earth\right) \cos\phi \sin\lambda. \\
            z &= \left(h + R_\Earth\right) \sin\phi.
        }

        While absolute errors are on the order of kilometres, typical scales of interest
        in meteor science are not spanning the entire diameter of the Earth,
        but only hundreds of kilometres, which keeps relative errors considerably lower.

        For a camera observing a meteor at a typical altitude of \SI{100}{\kilo\metre},
        the 

        (conduct experiment here)


    \subsection{Oblate spheroid} \label{mmw}
        An even better approximation is to use an oblate spheroid,
        that is, an ellipsoid with a shorter polar axis and two
        equal longer equatorial axes.

        The standard model to be used is \emph{WGS84}, 
        The difference from the geoid is on the order of +110/-80 metres,
        which translates to a negligible local error.

        The transformation formulae are mathematically complex and beyond the scope of this thesis.
        In our models we use the implementation by \cite{osen2017},
        for further information including a description of the code, refer to this report.

    \subsection{Conclusion}
        With currently available computational power it is no longer reasonable to
        consider a locally-flat Earth model, even for the simplest use cases.
        The transformations involved in the use of the spherical model
        are simple and computationally inexpensive enough to warrant its employment everywhere.

        We recommend to use the WGS84 model for most applications, with a possible
        fallback to the spherical Earth model where computational speed is crucial.


The number of possible combinations



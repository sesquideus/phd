\Epigraph[0.4]{
    You saw sagacious Solomon \\
    You know what came of him \\
    To him, complexities seemed plain
}{
    How Fortunate the Man With None \\
    \textsc{Bertolt Brecht} \\
    \textit{translation John Willett}
}

Once we have ascertained what physical objects and phenomena we are going to be working with,
we need to understand \emph{how} this can be done and how meteors and meteoroids can be described in a simplified \emph{model}.
The entire phenomenon is too complex to be described by a single theory: there are very different physical processes
at play during the cosmic age of the meteoroids and in the very short-lived phase where it can be detected as a meteor.
A complete analysis requires knowledge of mechanisms of production of meteoroids and the evolution of their orbits,
just as much as understanding the processes behind ablation, fragmentation, production of light and techniques used in observations.

Therefore we propose to separate the concerns and produce two distinct (but related) models:
\begin{itemize}
    \item an \emph{observational model}, which describes activity of meteors in space and time,
        as observed from the Earth;
    \item and an extended \emph{orbital model}, which represents the density of meteoroids in the interplanetary space.
        These particles can, under right conditions, produce meteors that are described by the observational model.
        The orbital model can be deduced from the observational model, since the distribution of meteoroids in space
        must be consistent with the distribution of meteors.
\end{itemize}

The observational model can be directly compared to actual observations, obtained by ground-based
devices placed on various locations around the Earth. The relatively high precision of these measurements puts
tight constraints on total number and distribution of recorded meteors. There are already numerous
meteor-observing networks which cover the sky continuously, at least as long as the weather is favourable.
The sky maps obtained from this model can be used to predict the activity of meteor showers and of the sporadic background.

On the other side, the orbital model provides understanding of \emph{why} there are meteors
encountered at certain point of space and time. Detailed maps of meteoroid density in the past,
and even more so predictions of possible encounters with meteoroid streams in the future,
are very important for space agencies operating satellites.

Even though meteoroids and meteor are discrete entities, it is useful not to treat them as such,
but rather as random samples from some smooth distribution function in a multi-dimensional coordinate space.
Actual particles are then treated as random samples obtained from the underlying distribution.

In the project, logic dictates that we do this in the opposite direction: we need to approximate
the underlying distribution from a finite set of known samples.
The essential steps in building the model are to find a suitable domain of this distribution,
to design a method of deducing it from available observations, and finally to
determine its values over the domain.

% For a presentation we should produce a table:
% model         observational       orbital
% scope         sky                 interplanetary space
% objects       meteors             meteoroids
% validity      high                lower
% etc

\section{Observational model} \label{ms}
    In the observational model we will aim to describe the observed distribution of meteors
    in terms of their apparent geocentric trajectories, time of entry into the atmosphere and other
    observable properties.

    This model can be \emph{complete}: it is possible -- at least in principle -- to record all
    meteors in the Earth's atmosphere and include them in the model.
    As long as there is sufficient coverage of the sky by networks of ground-based stations,
    no statistically significant grouping of meteors should remain unaccounted for.
    In reality there might be discrepancies caused by selection bias,
    lower coverage of the southern hemisphere or difficulties in detecting daytime meteor showers.

    \subsection{Physical quantities in the model} \label{msp}
        The design process of any model in astronomy -- or physics in general -- begins with the selection
        of physical quantities available for analysis. Then we should decide which ones are important enough
        to be included in the model and which can be ignored or treated as constants.

        Each of the important quantities then corresponds to a distinct dimension in a virtual coordinate space.
        The task of building the model is reduced to finding the values of some multivariate
        function $\mathcal{A}(x_1, x_2, \ldots, x_n)$ describing the expected number density of observable meteors.
        Measurable quantities are for instance time, position, velocity or apparent magnitude.

        \subsubsection{Time} \label{mspt}
            The first measurable quantity is the time of the atmospheric entry of the meteoroid.
            First of all, it should be noted that while a meteor is not an instantaneous event,
            its typical duration is on the order of \SI{1}{\second}, which is very short compared to the
            period of rotation of the Earth or the scales on which meteor activity varies.
            Highly precise timekeeping is thus not very important in the model.
            For our purposes we will be using the time of maximum brightness
            of the meteor for both actual data and in simulations.

            In the context of this thesis it is immaterial whether any particular meteor
            appeared a minute earlier or later. Since we are only interested in slow variations,
            any such time shift will be invariably smeared in kernel density estimations.
            Differences on the order of tens of minutes could be visible with very narrow filaments.
            The precision of AMOS data is more than sufficient.

            In examining the evolution of streams we should anticipate two distinct components:
            \begin{itemize}
                \item the \emph{periodic component}, emerging as the result of the motion of the Earth
                    around the Sun and encountering various streams along its orbit.
                    This component can be expressed in terms of the \emph{solar longitude} $\lambda_\Sun$.

                \item the slowly-varying \emph{secular component}, associated with appearance of new meteoroid
                    streams and gradual decay of older streams which are not replenished.
                    The peaks produced by meteor showers appear abruptly after the parent body
                    passes through the vicinity of the Earth, producing small meteoroids due to heating or outgassing.
                    Returns of parents bodies on periodic orbits, result in resupply of meteoroid material,
                    and thus form an extra periodic component, included in the secular evolution of the stream.
            \end{itemize}

            Neglecting the precession of the Earth's axis, the time coordinate $t$ may be decomposed into two components as
            \eqn{eq:iot-tyl}{
                t = y + \lambda_\Sun,
            }
            where
            \begin{description}
                \item[$y$]
                    is the number of whole orbits since some reference time,
                    for instance the current \emph{astronomical year} in which the last vernal equinox occurred;
                \item[$\lambda_\Sun$]
                    is the \emph{solar longitude}, specified in degrees, which represents the fractional part of the year.
            \end{description}

            \begin{example}
                Under this convention, August \nth{12}, 2020 roughly corresponds to $y = 2020$ and $\lambda_\Sun \approx \ang{139.4}$,
                while March \nth{14}, 2020 is denoted as $y = 2019$ and $\lambda_\Sun \approx \ang{353.6}$.
            \end{example}

        \subsubsection{Position} \label{mspp}
            While position in space is a fundamental physical quantity, it is not very useful in the observational model.
            On the large scale it is determined by the position of the Earth at the particular time,
            as only particles that enter the atmosphere can be included in the model.
            On the local scale it does not provide much information: most meteoroid streams
            are much wider than the diameter of the Earth and within them,
            we may treat exact positions of meteoroids entry as essentially random.
            Even a minuscule change in the orbital elements of one meteoroid
            would cause it to enter the atmosphere at a different place,
            and the same observer would see it at a vastly different position in the sky or not at all;
            while its adherence to the stream or its radiant would only be changed very slightly.

        \subsubsection{Velocity} \label{mspv}
            Another important measurable quantity is the velocity of the meteor.
            Unlike with positions, a very wide range of geocentric velocities is possible.
            When combined with the position, which is already known, it completely determines the geocentric
            orbit of the particle, which can in turn be extrapolated into the interplanetary space
            to obtain the original heliocentric orbit.
            This is also the basis for criteria for determination of adherence to streams,
            such as the $D$-criterion defined by \citet{southworth+1963} and other criteria.

            The meteor's trajectory is approximated by a straight line and extended to infinity.
            The point where this projection intersects the celestial sphere is called the \emph{radiant} of the meteor.
            Unlike the observed position of the meteor in the sky,
            the position of the radiant is not dependent on the position of the observer.

            Clustering of meteors based on their radiants and geocentric velocities
            is the basis of identification of meteor showers: particles on nearly the same trajectory intersecting
            the Earth at the same time will always have (approximately) the same radiant.
            The equality is not perfect due to slight dispersion of orbits,
            and due to the fact that the radiants slowly move across the sky,
            as the geometry of the stream changes with respect to the Earth's orbit.

            Hence we obtain our first three coordinates: the components of the velocity vector $v_x$, $v_y$ and $v_z$
            or, more conveniently, the declination and right ascension of the radiant $\delta_R$ and $\alpha_R$,
            along with the pre-atmospheric speed of the meteoroid $v_\infty$.

            An unwelcome difficulty is encountered with daytime meteor showers, whose radiants
            lie very close to the position of the Sun at the time when they are active.
            These meteoroids enter the atmosphere on the day side of the Earth,
            where most particles inevitably burn up undetected.
            With radar observations it is possible to track the activity of these showers as well,
            however this observation technique calls for different methods of processing and debiasing the data.
            We estimate that using purely visual data it is impossible (or at least very difficult) to establish fluxes
            for meteor showers whose radiants are within \ang{15} from the Sun, and significant bias
            is introduced even for larger angles.

        \subsubsection{Absolute magnitude} \label{mspm}
            Another important characteristic of a single meteoroid population is the distribution of masses
            among constituent particles. A shower that is primarily composed of massive particles
            will produce a large number of spectacular fireballs while its ZHR\footnote{zenithal hourly rate, see \cref{msaz}} may remain fairly low;
            while another shower may consist of a large number of tiny meteoroids with low cumulative mass,
            yet will result in large ZHR.

            However, in the observational model it is not possible to measure meteoroid mass directly.
            A much more readily measurable quantity is the apparent magnitude of the meteor, which,
            with the knowledge of the true distance and atmospheric effects, can be readily converted
            to absolute magnitude.

            The relation is not linear and is heavily affected by the meteor speed; with other properties,
            such as the entry angle or chemical composition, having some effect as well.

        \subsubsection{Population index} \label{msar}
            A useful derived measure of meteor activity is the \emph{population index}, commonly denoted~$r$.
            The definition by \citet{molau2015} is "[the population index] represents the [relative] increase
            in total meteor count when the limiting magnitude $lm$ improves by one mag[nitude]."

            So if we denote the total number of observed meteors with magnitude at most $m$ as $N(m)$
            and compare the counts for different values of $m$ and $m + 1$, we need $N$ to satisfy the requirement
            \eqn{eq:msar-cdf}{
                r \cdot N(m) = N(m + 1).
            }

            Solving this for function $N(m)$ yields a single family of solutions
            in the form
            \eqn{eq:msar-cdfsolved}{
                N(m) = k \cdot r^m
            }
            for some real constant $k$.

            As $N(m)$ is -- by definition -- the kernel of the
            cumulative distribution function (CDF) for apparent magnitudes, we can differentiate it
            to obtain the kernel of the probability density function. The kernel of the PDF is then
            \eqn{eq:mspr-pdfk}{
                n(m) \propto r^m \ln r.
            }

            The value or $r$ is assumed to be constant throughout the entire observable magnitude range.%
            \footnote{Without this assumption the population index loses its main purpose,
            to simplify the description of the distribution of magnitudes.
            Technically, \emph{any} smooth distribution of magnitudes can be described in terms of varying $r$.}
            Note that this function is not a power law, but an exponential (the variable $r$
            is in the base, not the exponent, unlike with mass index $s$).

            To obtain a probability density function, we must find a norming constant
            and set an upper limit on magnitudes so that the distribution does not diverge for faint meteors.
            The upper limit is conveniently set to the limiting magnitude $m_0$.
            If we require the integral over the entire visible range to evaluate to one,
            the value of $k$ is unambiguously determined as $k = r^{-m_0}$.
            The PDF is then
            \eqn{eq:msar-pdf}{
                F_m(m) =
                \begin{cases}
                    r^{m - m_0} \ln r &
                        \text{for }m \leq m_0\text{,} \\
                    0 &
                        \text{for }m > m_0\text{.}
                \end{cases}
            }

            It should be kept in mind that the mass and population indices are not fundamental physical quantities,
            but only parameters of simple model distribution functions for masses and apparent magnitudes respectively.
            Also, value of $r$ can be measured correctly only assuming there is a sharp limit of the observer's detection efficiency.
            For more complex detection efficiency profiles (such as the one defined in \cref{eq:msam-sigmoid})
            it cannot be defined.

        \subsubsection{Distribution function} \label{mspd}
            The final model would thus be a five-dimensional map of distribution of meteor activity, with independent variables being
            \begin{itemize}
                \item one temporal coordinate $t$, representing the meteor activity in \emph{time};
                \item two angular coordinates: the \emph{right ascension} $\alpha_R$ and \emph{declination} $\delta_R$ of the radiant;
                \item the \emph{pre-atmospheric geocentric speed} of the particles $v_\infty$;
                \item and the \emph{magnitude spectrum}, or the probability density function of absolute magnitudes of meteors.
            \end{itemize}

            Our task is reduced to finding the distribution function for meteor radiants, or the \emph{meteor activity function}
            \eqn{eq:msm-dist}{
                \mathcal{A}(t, \vec{v}, m) \equiv \mathcal{A}(t, \delta_\mathrm{R}, \alpha_\mathrm{R}, v_\infty, m).
            }

            Within this function we expect to be able to distinguish two main features:
            \begin{itemize}
                \item the \emph{sporadic background}, which forms the main contour lines of the maps, the ``terrain'';
                \item and several well-defined, sharp peaks, representing the \emph{meteor showers}.
                    Due to their common origin and orbit, these peaks should be typically very narrow in all dimensions
                    except for the distribution of magnitudes.
            \end{itemize}

    \subsection{Additional concepts} \label{msa}
        In addition to quantities that are directly described in the model, several other concepts are useful
        in its construction and verification. Many derived quantities are used in meteor astronomy,
        especially in relation to more fundamental quantities that are difficult to measure.

        \subsubsection{Limiting magnitude} \label{msam}
            The atmosphere is constantly changing due to weather, contamination
            by natural or man-made aerosols or light pollution. The detection efficiency
            of ground-based observers is thus variable and must be included in any analysis.

            In meteor astronomy, the quality of the conditions is usually summarized
            using a single number called the \emph{limiting magnitude}, denoted $m_0$.
            It is defined as ``the apparent magnitude of the faintest stars visible during the observation'' \citep{imo-glossary}.

            In visual observations it is usually assumed that a meteor can be detected if and only if
            its apparent magnitude is at most $m_0$. Mathematically, this can be described by a
            detection probability function in the form of a Heaviside step function.
            In photographic and video observations the limiting magnitude is usually slightly lower than for visual observers.

            A slightly better approach is to assume a gradual decrease of detection efficiency.
            This can be described by a sigmoid function, where the transition occurs on some interval
            between very faint meteors (none of which can be recorded, $p \to 0$)
            and very bright meteors (where it is justifiable to assume they cannot be missed, $p \to 1$).
            While there is no universally agreed-upon form, a simple yet reasonable parameterised function is \citep{jedicke+1997}
            \eqn{eq:msam-sigmoid}{
                D(m; m_0, \omega) = \frac{1}{1 + \exp\left(\frac{m-m_0}{\omega}\right)},
            }
            where
            \begin{description}
                \item[$m_0$]    is the new \emph{limiting magnitude}, where the detection probability is \SI{50}{\percent};
                \item[$\omega$] is the width of the function, with higher values representing
                    faster rate of loss of detection efficiency with decreasing brightness.
                    The limit $\omega \to \infty$ corresponds to a sharp cut-off at $m_0$,
                    or the Heaviside step function.
            \end{description}

            Ideally, a functional dependence on the distance from the centre of the field
            of view should be taken into account for human observers, as the eye is much more
            sensitive to light near the yellow spot.
            Similarly, digital sensors are subject to vignetting and other detrimental effects.
            While with cameras the magnitude of these effects is generally well understood,
            further research into the angular response of the human eye would be beneficial.

        \subsubsection{Zenithal hourly rate} \label{msaz}
            The \emph{zenithal hourly rate} (ZHR) is defined as ``[t]he number of shower meteors per hour
            one observer would see if his limiting magnitude is \Mag{+6.5} and the radiant is in his zenith''.
            While it is not an universally valid figure of merit for comparing meteor activity of various showers,
            ZHR is fairly simple and widely used as a standard in meteor science.
            Its primary design consideration is to be able to compare the activity of various showers
            as viewed by human observers, but it is perfectly applicable to automatic devices as well.

            ZHR is defined as
            \eqn{eq:ipqz-zhr}{
                \mathrm{ZHR} = \frac{\frac{N}{T} \cdot \frac{1}{k} \cdot r^{\num{6.5} - m_0}}{\sin{\theta_\mathrm{R}}}
            }
            where
            \begin{description}
                \item[$N$]
                    is the number of meteors recorded by the observer;
                \item[$T$]
                    is total effective observation time;
                \item[$k$]
                    is the fraction of the sky that is visible to the observer;
                \item[$r$]
                    is the \emph{population index} of the observed meteor shower;
                \item[$m_0$]
                    is the \emph{limiting magnitude} of the observer;
                \item[$\theta_\mathrm{R}$]
                    is the altitude of the radiant above the horizon.
            \end{description}
            In this definition the limiting magnitude is considered to be a sharp boundary -- all
            meteors below the limiting magnitude are detected.

            In our model the ZHR is not an input, but rather a computed quantity, completely determined by the
            models of the meteoroid population, atmosphere and of the observer's detection efficiency.

        \subsubsection{Photometric mass} \label{msaf}
            Although we cannot measure the true mass of the meteoroid directly,
            we may use some of the known quantities to estimate it.
            Two commonly used measures are the \emph{photometric mass} and \emph{dynamic mass} \citep{ceplecha1966}.

            When simulating meteors, we know the original mass of the meteoroid particle.
            With the knowledge of the equations of motion, ablation and luminosity
            we are able to determine the luminosity of the meteor at any time.
            One of the simplest models \citep{hill+2005} assumes that the energy emitted as visible light
            is proportional to change in particle's kinetic energy,
            \eqn{eq:msaf-tau}{
                F_0 = -\tau \frac{\dot{\mu} v^2}{2}
            }
            for some $\tau$ named \emph{luminous efficiency factor}. This can be either a constant
            or a function of speed or other properties of the meteoroid.

            Photometric mass is obtained by working in the reverse direction: we can estimate the
            ablated mass from the amount of light emitted by the meteoroid during its passage through the atmosphere.
            The standard approach is to invert the equation \cref{eq:msaf-tau} and integrate it
            throughout the light phase,
            \eqn{eq:msaf-photo}{
                \fdiff \mu = \Int[t_\mathrm{begin}][t_\mathrm{end}]{\frac{F(t)}{\tau v(t)^2}}{t},
            }
            where
            \begin{description}
                \item[$\fdiff \mu$]
                    is the total mass lost due to ablation;
                \item[$F(t)$]
                    is the instantaneous light flux, corrected for the sensitivity of the camera;
                \item[$\tau$]
                    is the luminous efficiency coefficient;
                \item[$v$]
                    is the measured speed of the meteoroid;
                \item[$t_\mathrm{begin}$]
                    is the time in which the meteor first became visible;
                \item[$t_\mathrm{end}$]
                    is the time in which the meteor stopped being visible.
            \end{description}

            Initial mass is then obtained as
            \eqn{eq:msaf-orig}{
                \mu_\mathrm{begin} = \mu_\mathrm{end} + \fdiff \mu.
            }

            Since in most cases the meteoroid is too small and fast to survive the entry, its final mass is zero
            and photometric mass should be equal to true initial mass, at least in an idealized case where our model is perfect.
            Another similar concept is the \emph{dynamic mass}, in which the recorded deceleration of the particle
            is analyzed and fitted instead of its luminosity.

%        \subsubsection{Flux} \label{msax}
%            While the distribution function \todo{...}

        \subsubsection{Selection bias} \label{msab}
            In many analyses of observational dataset authors tend to overlook the effect of selection bias --  the
            propensity of the selected method of gathering the data to include certain data points with higher probability than other ones.
            This may negatively influence the conclusions that are drawn from the data, if these effects are not understood and corrected for.
            In our specific case it is chiefly the tendency of the cameras to detect brighter meteors more reliably than fainter ones,
            which leads to underestimating the number of faint meteors observable in the sky.
            For a detailed list of possible sources of bias and methods refer to \citet{balaz-thesis}.


            % make this a separate chapters
\subsection{Models of meteoroid dynamics} \label{mm}
    We will introduce all relevant constants and properties of the meteoroids and the environment.

    we obtain a tuple which we we call the \emph{state} of the meteoroid.
    This tuple contains the key properties which are bound to the meteoroid body,
    and which can vary during its lifetime and determine its final fate.
    We use the same tuple in all models, with some properties
    kept constant or ignored completely in models where they are not needed.

    \subsection{Drag} \label{md}
        The that can be independetly varies is the model of drag. While a complete analytical solution
        is, according to modern knowledge, not possible, numerical simulations

        a tradeoff between accuracy and complexity (and thus computation time).
        The models described here vary from very simple to 

        one particular problem is that the exact shape of the meteoroid is never known beforehand,
        and any simulation working with the real shape can only be performed after the meteorite is
        found -- which somewhat subtracts from the usefulness of running the simulation at all.

        Therefore we always use spherical particles. This approximation is not justified
        physically, but by lack of any better methods (OK this is cruel).

        \subsubsection{No drag} \label{mdn}
            A trivial model in which $\Gamma = 0$ at all times.
            It is not useful for atmospheric modelling, but is computationally
            very fast and makes testing other models easier.

        \subsubsection{Constant drag} \label{mdc}
            In textbooks and older papers a constant value of $\Gamma$ is often taken.
            For a spherical particle in a turbulent flow a value of \num{0.47}
            is used \cite{???}. \cite{hrábek} \cite{havrila}

        \subsubsection{Morrison} \label{mdm}
            The next step is the correlation by \cite{morrison}.

            Variable with Reynolds number

            extends both to very low Reynolds number as
            \eqn{eq:mdm-limre}{
                \lim_{\mathrm{Re} \to 0} \Gamma = \frac{24}{\mathrm{Re}},
            }
            yielding the formula for Stokes drag
            \eqn{eq:mdm-stokes}{
                F = \frac{1}{2} \Gamma S \rho_\mathrm{air} v^2
                    = \frac{1}{2} \frac{24 \mu}{2 r \rho_\mathrm{air} v} \rho_\mathrm{air} \pi r^2 v^2
                    = 6 \pi \mu r v;
            }
            and reproduces the effect of a sharp decrease of drag coefficient at high Reynolds
            numbers (n the order of \num{5e5}, known as the \emph{drag crisis} \cite{???}.

            \fig{}{

            }

            However, it is only valid in the incompressible limit and at high Knudsen numbers,
            which are assumptions that are all but valid for a meteoroid in
            hypersonic flight in the rarefied upper atmosphere.


    \subsubsection{Whipple model} \label{mmw}
        A standard, textbook model of meteoroid flight introduces a change in mass of the meteoroid
        during its flight through the atmosphere. 
        (Öpik) (Ceplecha)

        Again, we assume a constant (spherical or nearly spherical) shape throughout the entire
        lifetime of the meteoroid.\footnote{An advanced fragmentation model could probably assume a non-spherical
        shape in the first moments, which would quickly change back into a spherical one during ablation
        of the newly created particle.}

        The drag coefficient $\Gamma$ is 

        (Gamma) sometimes $c_d$ is used as a symbol instead.

        are generally not available and would require running a proper CFD simulation
        of the shape in question at various densities, orientations and flow speeds.

        Valid across a wide range of conditions, as long as the value of Re is the same.
        ultra-hypersonic flow?

        A single value is used during the entire flight. Typically


        \eqn{eq:reynolds}{
            \mathrm{Re} = \frac{\rho_\mathrm{air} v d}{\mu}.
        }
        where
        \begin{description}
            \item[$d$] is the typical size of the meteoroid in the stream.
                For spherical or nearly-spherical particles the diameter ($d = 2r$) is used;
            \item[$\mu$] is the \emph{dynamic viscosity of the air},
                $\SI{1.8e-5}{\pascal\second} = \SI{1.8e-5}{\kilo\gram\per\metre\per\second}$.
        \end{description}

        Except for $\mu$, all three quantities will change significantly
        during the atmospheric entry: $\rho_\mathrm{air}$ is negligible at first
        and increases roughly exponentially in the range of roughly \SIrange{e-10}{1}{\kilo\gram\per\metre\cubed}
        in the region of our interest, $v$ is typically on the order of \SIrange{1e4}{7e4}{\metre\per\second}
        just above the atmosphere but can drop to about \SI{1}{\metre\per\second}
        for a microscopic particle in the troposphere, and the diameter also spans
        about four orders of magnitude (\SIrange{e-4}{1}{\metre}).

        Combined, a wide range of values of Re has to be investigated,


        To save computation time, two constants can be evaluated in advance as
        \eqn{eq:mconst}{
            C_v = A \rho^{-2/3}
            \QQText{and}
            C_m = \frac{\Lambda A \rho^{-2/3}}{2 Q} = \frac{C_v \Lambda}{2 Q}.
        }


    \subsubsection{Material evaporation model} \label{mwh}
        Another useful extension of the model described in \cref{mmw} is due to
        \cite{hill+2005}, which adds temperature as another variable property of the meteoroid.


        In the code it is referred to as \texttt{HRH2005}.

        \eqn{eq:hrh}{
            \Derivative{T}{t} = \frac{A}{c m^{1/3} \rho^{2/3}}
                \left(\frac{1}{2} \Lambda \rho_\mathrm{air} v^3
                    - 4 \sigma \epsilon \left(T^{\,4} - T_\mathrm{air}^{\,4}\right)
                    + \frac{L}{A} \left(\frac{\rho}{m}\right)^{2/3} \Derivative{m}{t}
                \right),
        }
        where
        \begin{description}
            \item[$T$]
                temperature of the meteoroid ($\mathrm{K}$),
            \item[$t$]
                time (\si{\second}),
            \item[$A$]
                shape factor (dimensionless),
            \item[$c$]
                specific heat of the meteoroid ($\si{\joule\per\kilo\gram\per\kelvin} \equiv \si{\metre\squared\per\second\squared\per\kelvin}$),
            \item[$m$]
                instantaneous mass of the meteoroid (\si{\kilo\gram}),
            \item[$\rho$]
                density of the meteoroid (\si{\kilo\gram\per\metre\cubed}),
            \item[$\Lambda$]
                heat transfer coefficient (\si{1}),
            \item[$\rho_\mathrm{air}$]
                density of the atmosphere at the position of the meteoroid (\si{\kilo\gram\per\metre\cubed}),
            \item[$\sigma$]
                Stefan--Boltzmann constant (\SI{5.67e-8}{\watt\per\metre\squared\per\kelvin\tothe{4}}),
            \item[$\epsilon$]
                emissivity of the meteoroid (dimensionless),
            \item[$v$]
                instantaneous speed of the meteoroid,
            \item[$T$]
                temperature of the meteoroid (\si{kelvin}),
            \item[$T_\mathrm{air}$]
                temperature of the atmosphere at the position of the meteoroid (\si{\kelvin}),
            \item[$L$]
                latent heat of fusion and vaporization ($\si{\joule\per\kilo\gram} \equiv \si{\metre\squared\per\second\squared}$.

        \end{description}

        The state of the meteroid is then defined by the tuple $(\vec{r}, \vec{v}, \ln m, T)$.

        \eqn{eq:hrh2}{
            \Derivative{T}{t} = \frac{1}{c} \left(
                \frac{A}{m^{1/3} \rho^{2/3}}
                \left(
                    \frac{1}{2} \Lambda \rho_\mathrm{air} v^3
                    - 4 \sigma \epsilon \left(T^{\,4} - T_\mathrm{air}^{\,4}\right)
                \right)
                + \frac{L}{m} \Derivative{m}{t}
            \right)
        }

        Again, to increase numerical stability we may substitute $m \to \ln m$,
        by the chain rule
        \eqn{eq:hrh-chain}{
            \Derivative{\left(\ln m\right)}{t} = \frac{1}{m} \Derivative{m}{t}.
        }

        The second term then can be separated and the whole equation \ref{eq:hrh} rewritten as
        \eqn{eq:hrh-fixed}{
            \Derivative{T}{t} = \frac{1}{c} \left(
                A \rho^{-2/3} \exp \left(- \frac{\ln m}{3}\right)
                \left(
                    \frac{1}{2} \Lambda \rho_\mathrm{air} v^3
                    - 4 \sigma \epsilon \left(T^{\,4} - T_\mathrm{air}^{\,4}\right)
                \right)
                + L \Derivative{\left(\ln m\right)}{t}
            \right).
        }

        To further optimize the computation, we may precompute another constant
        $c_T = 4 \sigma \epsilon$ and obtain the computational form of \cref{eq:hrh-fixed}:
        \eqn{eq:hrh-final}{
            \Derivative{T}{t} = \frac{1}{c} \left(
                C_v \exp \left(- \frac{\ln m}{3}\right)
                \left(
                    \frac{1}{2} \Lambda \rho_\mathrm{air} v^3
                    - C_T \left(T^{\,4} - T_\mathrm{air}^{\,4}\right)
                \right)
                + L \Derivative{\left(\ln m\right)}{t}
            \right).
        }



%    Another method has been discussed and implemented in the author's master's thesis \citep{balaz-thesis}
%    and used to determine the total flux of the Perseids in 2016. Instead of trying to
%    estimate the biases we generated the meteoroids above the atmosphere and simulated
%    their entry using a simplified set of equations for motion, ablation and luminosity.
%    A collection of stochastic bias functions was then applied to the dataset and
%    each meteor was marked as detected or missed. We varied the parameters of the bias functions
%    and the mass index until agreement with observational data was achieved.
%    The initial population was then declared as the model of the actual population
%    and the bias functions were taken as descriptive of the observing system.

%\section{Simulation in the atmosphere} \label{ma}
%    Once it has been determined which meteoroids will enter the Earth's atmosphere, we may proceed to simulating
%    the physical processes manifested during their atmospheric entry and obtain the values of quantities
%    that can be observed by ground-based observers.
%
%    To simulate the atmospheric entry we used \textsc{Asmodeus},
%    a multi-purpose virtual meteor simulator developed as a part of our master's thesis and extended for numerous
%    other purposes related to meteor astronomy \citep{balaz-thesis,balaz+2020}.
%
%    Once a particle is selected for atmospheric entry, its velocity is transformed to the ECEF reference frame
%    and a numerical integration of the equations of motion is executed.
%
%    \subsection{Numerical integration of the equations of motion} \label{sai}
%        Unlike in interplanetary space, where forces are completely dominated by the gravitational force exerted by the Sun
%        and only slightly perturbed by other effects,
%        a particle entering the Earth's atmosphere switches between several regimes in rapid succession.
%        The precise order is subject to variations due to particle's size, entry speed and angle
%        and material composition.
%
%        \subsubsection{Acting forces} \label{saia}
%            In the simulation we consider only two real forces:
%            \begin{itemize}
%                \item the \emph{drag force} $\vec{F_d}$, always acting against the instantaneous velocity vector of the particle;
%                \item the \emph{gravitational force} $\vec{F_g}$, pulling the meteoroid towards the centre of the Earth.
%            \end{itemize}
%
%            The precision of the calculations may be improved if we also account for two fictitious forces,
%            arising from the fact the simulation is performed in a rotating reference frame:
%            \begin{itemize}
%                \item the fictitious \emph{centrifugal} or \emph{Huygens force} $\vec{F_{\mathrm{C}}}$,
%                    pushing the particle away from the axis of rotation of the Earth;
%                \item and the fictitious \emph{Coriolis force} $\vec{F_{\mathrm{c}}} \equiv{-\vec{\Omega} \times \vec{\Omega} \times \vec{v}}$,
%                    which pushes the moving body in a direction perpendicular to its velocity vector.
%            \end{itemize}
%
%            Except for the drag force, all forces can be readily expressed in terms of instantaneous
%            position and velocity of the meteoroid and several constants.
%            Calculating the drag force precisely is computationally very expensive and cannot be done
%            without knowing the shape of the particle. Therefore we have to choose an appropriate simplified model.
%
%            In its cosmic stage and during most of the motion in the gravitational sphere of influence of the Earth,
%            the forces acting on the particle are dominated by gravitational effects. As the meteoroid penetrates
%            deeper into the atmosphere, the density of the surrounding gas increases roughly exponentially.
%            At its peak, the atmospheric drag force exceeds all other effects by several orders of magnitude,
%            decelerating the meteoroid very rapidly.
%            As the duration of the entire event is on the order of seconds, other forces simply do not act for long enough
%            to produce a significant deviation in trajectory. For statistical evaluation of artificial meteors
%            it is sufficient to consider only the drag force.
%            However, should the meteoroid survive the entry as a meteorite, their influence is important during the dark phase of flight.
%
%            \paragraph{Coriolis force} \label{saiC}
%                The fictitious Coriolis force arises in a rotating reference frame.
%                \eqn{eq:saiC-coriolis}{
%                    \vec{F_{\mathrm{C}}} = -2 m \vec{\Omega}_\Earth \times \vec{v}.
%                }
%
%                For a meteoroid, its magnitude is at most on the order of
%                \eqn{eq:saiC-order}{
%                    2 \frac{2\pi}{\SI{86400}{\second}} \cdot \SI{70}{\kilo\metre\per\second} \sim \SI{10}{\metre\per\second\squared},
%                }
%                roughly the same as gravitational acceleration.
%
%            \paragraph{Centrifugal force} \label{saic}
%                The centrifugal force pushes the particles away from the axis of rotation of the Earth.
%                \eqn{eq:saic-centrifugal}{
%                    \vec{F_{\mathrm{c}}} = - m \vec{\Omega}_\Earth \times \left(\vec{\Omega}_\Earth \times \vec{r}\right),
%                }
%                where $\vec{\Omega}_\Earth$ is the angular speed of rotation of the Earth,
%                $\frac{2\pi}{\SI{86164}{\second}} \approx \SI{7.292e-5}{\per\second}$ in the direction
%                towards the north celestial pole.
%
%                Since it only depends on the instantaneous position, it can be thought of as a minor correction to the
%                gravitational force. A simple calculation shows that its magnitude is on the order
%                of \SI{0.03}{\metre\per\second\squared}. Near the surface it can be safely neglected for most purposes.
%                It only becomes important when the distance from the axis of rotation is large, for instance if the simulation
%                is used to investigate the motion of a meteoroid further from the Earth.
%                However, in this case an inertial reference frame should be used instead.
%
%            \paragraph{Gravitational force} \label{saig}
%                The real shape of the gravitational potential around the Earth is fairly complex and difficult to describe accurately.
%                For particles that are generally moving on highly hyperbolic trajectories the
%                simple approximation by a Newtonian point source with mass $M_\Earth$ is sufficient.
%                The formula used in computation is then simply
%                \eqn{eq:saig-gravity}{
%                    \vec{F_{\mathrm{G}}} = - \frac{GM_\Earth}{r^3} \vec{r}.
%                }
%
%            \paragraph{Drag force} \label{said}
%                The precise description of the drag force is crucial to understand the motion
%                of meteoroids in the atmosphere. The drag force is dominant between approximately
%                the point where meteoroids begin to emit visible light (about \SI{100}{\kilo\metre})
%                until the very end of the visible trail. For all but the most massive meteoroids,
%                which are able to survive the atmospheric entry, this can be approximated
%                by the point where all of the matter has been ablated away; while for the bodies
%                surviving as meteorites, during the terminal phase of the flight the drag and
%                gravitational forces are nearly balanced out.
%
%                At peak deceleration the magnitude of the drag force can reach values several orders of magnitude
%                higher than the sum of all other forces and is thus completely dominant.
%                A very simple model of atmospheric flight can thus perform relatively
%                well even when all other forces are neglected.

                is model-dependent. The total force is then
                \eqn{eq:sai-totalforce}{
                    \vec{F} = \vec{F_\mathrm{gravity}} + \vec{F_\mathrm{Coriolis}} + \vec{F_\mathrm{Huygens}} + \vec{F_\mathrm{drag}}
                }
%
%    \subsection{Determination of luminosity} \label{sail}
%        Currently we use... \citep{hill+2005} \todo{measurereally?}
%
%        \citep{bronshten1983}

    \subsection{Physical quantities and properties} \label{mop}
        Apart from the position and velocities, which we have already described, we can express
        several derived quantities which are standard in meteor astronomy. This makes comparison
        with other models easier.

        \subsubsection{Mass index and the Pareto distribution} \label{moms}
            While theoretically any distribution of particle masses is possible,
            in real meteoroid populations there is always a large excess of small particles.
            In meteor science, it is customary to assume a specific distribution of masses within a population,
            described by a power law with particle mass as the independent variable,
            \eqn{eq:moms-sindex}{
                \varrho(\mu) \propto \mu^{-s}
            }
            for some constant $s$ named the \emph{mass index}.

            To obtain a distribution, this expression needs to be normalised first.
            The expression $\mu^{-s}$ diverges to infinity for $\mu \to 0$,
            which can be dealt with by setting a lower limit on the mass.
            These requirements are satisfied by the well-known Pareto distribution \citep{arnold1983}.
            In mathematical texts it is usually defined as $\Distribution{\mathrm{Pareto}}{x}{\alpha, x_0}$,
            where $\alpha > 0$ is called \emph{shape} and $x_0 > 0$ is called the \emph{scale} of the distribution.

            To conform with terminology commonly used in meteor science, we substitute the shape with mass index
            $s = \alpha - 1$, and rename $x_0$ to \emph{minimum mass} of particles, $\mu_\mathrm{min}$.
            The probability density function is then
            \eqn{eq:moms-pareto}{
                \varrho(\mu) = \mathrm{Pareto}\left(\mu;\ s-1, {\mu_\mathrm{min}}\right) \equiv
                \begin{cases}
                    0 &
                        \text{for }\mu < \mu_{\mathrm{min}}\text{,} \\
                    \dfrac{\left(s - 1\right) \mu_{\mathrm{min}}^{s - 1}}{\mu^s} &
                        \text{for }\mu \geq \mu_{\mathrm{min}}\text{.}
                \end{cases}
            }

            Typical values of $s$ are between \numrange{1.5}{2.5}, with values slightly below 2 being most common
            for meteor showers and values around \num{2.2} for the sporadic background \citep{blaauw+2011}.
            The collisional equilibrium for particles of equal strength is reached at $s = 11/6$ \citep{dohnanyi1969}.

            The mass index $s$ can be related to the population index $r$ by an empirical relationship by \citet{koschack+1990},
            \eqn{eq:moms-rs}{
                r = 1 + \num{2.3} \log_{10} s.
            }

    \subsection{Additional properties} \label{moa}
        Apart from the dynamical properties, which describe only the density of particles
        and thus a probability of a collision at a point in space and time,
        we may observe and analyze additional properties of the stream.


\section{Simulations} \label{mi}
    At the core of both the orbital and the observational models are numerical simulations.
    The simulation works with a generic model of the analyzed physical processes and some initial
    distribution of meteoroids and tries to match its output to observational data.


%\section{Simulation in the atmosphere} \label{ma}
%    Once it has been determined which meteoroids will enter the Earth's atmosphere, we may proceed to simulating
%    the physical processes manifested during their atmospheric entry and obtain the values of quantities
%    that can be observed by ground-based observers.
%
%    To simulate the atmospheric entry we used \textsc{Asmodeus},
%    a multi-purpose virtual meteor simulator developed as a part of our master's thesis and extended for numerous
%    other purposes related to meteor astronomy \citep{balaz-thesis,balaz+2020}.
%
%    Once a particle is selected for atmospheric entry, its velocity is transformed to the ECEF reference frame
%    and a numerical integration of the equations of motion is executed.
%
%    \subsection{Numerical integration of the equations of motion} \label{sai}
%        Unlike in interplanetary space, where forces are completely dominated by the gravitational force exerted by the Sun
%        and only slightly perturbed by other effects,
%        a particle entering the Earth's atmosphere switches between several regimes in rapid succession.
%        The precise order is subject to variations due to particle's size, entry speed and angle
%        and material composition.
%
%        \subsubsection{Acting forces} \label{saia}
%            In the simulation we consider only two real forces:
%            \begin{itemize}
%                \item the \emph{drag force} $\vec{F_d}$, always acting against the instantaneous velocity vector of the particle;
%                \item the \emph{gravitational force} $\vec{F_g}$, pulling the meteoroid towards the centre of the Earth.
%            \end{itemize}
%
%            The precision of the calculations may be improved if we also account for two fictitious forces,
%            arising from the fact the simulation is performed in a rotating reference frame:
%            \begin{itemize}
%                \item the fictitious \emph{centrifugal force} $\vec{F_{\mathrm{C}}}$,
%                    pushing the particle away from the axis of rotation of the Earth;
%                \item and the fictitious \emph{Coriolis force} $\vec{F_{\mathrm{c}}}$,
%                    which pushes the moving body in a direction perpendicular to its velocity vector.
%            \end{itemize}
%
%            Except for the drag force, all forces can be readily expressed in terms of instantaneous
%            position and velocity of the meteoroid and several constants.
%            Calculating the drag force precisely is computationally very expensive and cannot be done
%            without knowing the shape of the particle. Therefore we have to choose an appropriate simplified model.
%
%            In its cosmic stage and during most of the motion in the gravitational sphere of influence of the Earth,
%            the forces acting on the particle are dominated by gravitational effects. As the meteoroid penetrates
%            deeper into the atmosphere, the density of the surrounding gas increases roughly exponentially.
%            At its peak, the atmospheric drag force exceeds all other effects by several orders of magnitude,
%            decelerating the meteoroid very rapidly.
%            As the duration of the entire event is on the order of seconds, other forces simply do not act for long enough
%            to produce a significant d964201eviation in trajectory. For statistical evaluation of artificial meteors
%            it is sufficient to consider only the drag force.
%            However, should the meteoroid survive the entry as a meteorite, their influence is important during the dark phase of flight.



\section{Case studies} \label{mc}
    Here we will analyze some case studies.

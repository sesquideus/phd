\Epigraph[0.4]{
    It is not what you think of \\
    when I say I'm doing modelling.
}{
}




\section{Models of meteoroid dynamics} \label{mm}





\section{Models of the Earth}
    Apart from the model of meteoroid dynamics, an important consideration is the level of precision
    of the model of the Earth. We distinguish three separate models:

    \begin{itemize}
        \item A \emph{locally flat} Earth, which describes the close vicinity of the observer,
            generally in Cartesian coordinates. It is useful for describing strictly local events
            in the coordinate system of the observer, such as the simulation of the dark flight phase of a meteor.
        \item A \emph{spherical} Earth, which is a suitable approximation for global coverage,
            where high precision of trajectories and observation is not a concern, such as determination
            of particle flux or density maps.
        \item An \emph{oblate spheroid} model of the Earth, which provides higher
            precision but is computationally more expensive.
    \end{itemize}

    Further refinement is possible by implementing a geoid model. This is usually done by means of
    a spherical harmonics expansion \cite{???}, which accounts for variations in local gravity
    and deviations of the local vertical.
    In meteor science, this level of precision is generally not needed as any improvement over the
    oblate spheroid model will be dominated by other sources of measurement errors.
    More precise models thus will not be investigated in this thesis.


    \subsection{Locally flat Earth} \label{mmf}
        Some simulations, such as those of a single meteor, or a determination of
        the number of meteors visible during one night, can be performed for a single virtual observer.
        As the radius of the Earth -- and subsequently the curvature of its radius -- are
        sufficiently large compared to the typical distances between the observer and meteors,
        in these cases the Earth can be considered to be locally flat.

        The natural system of referencing locations is then \emph{horizontal coordinate system},
        centered in the observer and aligned with the cardinal directions.
        A local tangent plane is laid out so that it touches the surface of the Earth
        at the position of the observer. Special consideration has to be given to computation of altitudes:
        unlike in deep-sky astronomy, where the altitude of the observer with respect to the sea level
        is negligibly small compared to distances to objects and can be ignored, in meteor astronomy
        this difference is significant (for instance when computing atmospheric density).

        (picture)

        The coordinates are denoted by a triplet $(x, y, z)$, where
        \begin{description}
            \item[$x$]  is the \emph{northing}, the distance in the direction of local true north;
            \item[$y$]  is the \emph{easting}, the distance in the direction of local true east;
            \item[$z$]  is the \emph{altitude}, the distance in the direction straight up
                (away from the centre of the Earth);
        \end{description}
        or expressed as spherical coordinates $(r, h, a)$, where
        \begin{description}
            \item[$r$]  is the \emph{radial distance} from the observer,
                $r \in \IntervalCO{0}{\infty}$;
            \item[$h$]  is the \emph{altitude} of the object, measured from the horizon upwards,
                $h \in \IntervalCO{\ang{-90}}{\ang{90}}$,
            \item[$a$]  is the \emph{azimuth}, measured from true north, east positive
                ($a \in \IntervalCO{\ang{0}}{\ang{360}}$).
        \end{description}


    \subsection{Spherical Earth} \label{mms}
        The natural next step is to consider a spherical Earth model.
        The true shape of the Earth's zero-altitude surface differs from
        a perfect sphere by no more than one part in 300, which introduces errors
        that are for many purposes negligible.
        The origin of the coordinate system lies in the centre of mass of the Earth.        

        In mathematics, it is custom to denote spherical coordinates by the triplet
        $(r, \theta, \phi)$, where
        \begin{description}
            \item[$r$] is the \emph{radial distance} from the Earth's centre of mass,
                $r \in \IntervalCO{0}{\infty}$;
            \item[$\theta$] is the \emph{polar angle} or \emph{inclination}, measured
                from a chosen reference direction called the \textit{zenith},
                $\theta \in \IntervalCC{0}{\pi}$;
            \item[$\phi$] is the \emph{azimuthal angle}, the oriented angle between the projection
                of $\vec{r}$ into the reference plane, which is perpendicular to the zenith direction, and
                a chosen direction with that plane, called the \textit{azimuth reference direction}.
                By convention, $\phi \in \IntervalCO{0}{2\pi}$.
        \end{description}

        (picture)

        For describing locations in the ECEF frame, it is customary to use the
        more convenient \emph{geographical convention}. The coordinates are substituted
        \aln{mms-subst}{
            h &:= r - R_\Earth, \\
            \phi &:= \ang{90} - \theta, \\
            \lambda &:= \phi
        }
        and reordered as $(\phi, \lambda, h)$.
        Also, angular measures are expressed in degrees instead of radians.
        The three coordinates then become
        \begin{itemize}
            \item \emph{latitude} $\phi$,
                measured from the Earth's equator, with values increasing towards the North Pole.
                Hence $\phi \in \IntervalCC{\ang{-90}}{\ang{+90}}$;
            \item \emph{longitude} $\lambda$, measured from the IERS Reference Meridian,
                with values increasing towards east. By convention, values over \ang{180} are
                reduced by one full revolution, so that $\lambda \in \IntervalCC{\ang{-180}}{\ang{+180}}$;
            \item \emph{altitude} $h$, measured from the surface of a sphere
                with diameter equal to the radius of the model of the Earth $R_\Earth$, increasing outwards,
                which yields $h \in \IntervalCO{-R_\Earth}{\infty}$.
                Negative values represent points below the surface of the Earth --
                while theoretically possible, they are of no interest in meteor
                astronomy.\footnote{Of course, unless we're observing near the Dead Sea
                or a meteorite is found inside a mine shaft.}
                The value of $R_\Earth$ can be chosen somewhat arbitrarily:
                equatorial radius of \SI{6378137}{\metre} or mean radius of \SI{6371008}{\metre} are commonly used.
                For the sake of clarity and consistency we use the value of \SI{6371008}{\metre} in our models.
        \end{itemize}

        (picture)

        The transformation functions are then
        \aln{eq:mms-xyz-to-sph}{
            h       &= \sqrt{x^2 + y^2 + z^2} - R_\Earth, \\
            \phi    &= \arccos{\frac{z}{x^2 + y^2 + z^2}}, \\
            \lambda &= \mathrm{atan2}(y, x)
        }
        and the inverse transformation is given as
        \aln{eq:mss-sph-to-xyz}{
            x &= \left(h + R_\Earth\right) \cos\phi \cos\lambda, \\
            y &= \left(h + R_\Earth\right) \cos\phi \sin\lambda, \\
            z &= \left(h + R_\Earth\right) \sin\phi.
        }

    \subsection{Oblate spheroid} \label{mmo}
        An even better approximation is to use an oblate spheroid,
        that is, an ellipsoid with a shorter polar axis and two
        equal longer equatorial axes.

        The standard model to be used is \emph{WGS84}, defined as an ellipsoid
        with a semi-major axis $a = \SI{6378137}{\metre}$, inverse flattening
        $1/f = \num{298.257223563}$, which results in a semi-minor axis of
        $b \approx \SI{6356752.314245}{\metre}$ \cite{???}.
        The difference from the geoid is at most -105/+85 metres \cite{???}
        which translates to a negligible local error.
        Deviations of the local plumb line, caused by density inhomogeneities
        inside the Earth, can be safely ignored in our use case.

        The coordinates used are the same as in the spherical model. (Plumb line is always perpendicular to equipotential surface)

        The transformation formulae are mathematically complex and beyond the scope of this thesis.
        In our models we use the optimized implementation by \citet{osen2017}.

    \subsection{Comparison} \label{mmc}
        While absolute errors of the spherical model are on the order of kilometres,
        typical scales of interest in meteor science are not spanning the entire diameter of the Earth,
        but are on the order of hundreds of kilometres. This keeps the relative errors considerably lower.

        For a camera observing a meteor at a typical altitude of \SI{100}{\kilo\metre},
        the deviation is generally lowest for meteors in the zenith
        and gradually increases towards the horizon.

        (conduct simulation experiment here)

        flat vs spherical

    \subsection{Conclusion}
        With currently available computational power it is no longer reasonable to
        use a locally-flat Earth model, not even for the simplest cases.
        The transformations involved in the use of the spherical model
        are simple and computationally inexpensive enough to warrant its employment everywhere.

        We recommend to use the WGS84 model for most applications, with a possible
        fallback to the spherical Earth model where computational speed is crucial.

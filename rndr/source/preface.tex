\Epigraph[0.4]{
    I gazed into the abyss \\
    and the abyss gazed back.
}{
    Friedrich Nietzsche, paraphrase
}

The problem of determining the exact

observational science, where experiments cannot be performed at will.

The theoretical foundation of the process and the differential equations involved
are not hard, at least not in terms of mathematics involved. In fact, in could in principle already be fully understood by Newton.
Still, precise modelling of atmospheric entry remains a fairly difficult problem due to its inherent complexity.
There are many parameters that cannot be easily measured or that cannot be known at the time when they are relevant.
Any attempt at modelling of these phenomena, either with the purpose of finding meteorites from particular meteors,
or for evaluating the influx of particles from outer space statistically,
still includes lots of approximations that are not always justified.

The fundamental problem of meteor astronomy can be stated as follows:
Given the position, velocity, mass, shape and chemical composition of a meteoroid,
and knowing the properties of the Earth's atmosphere, calculate its trajectory, luminosity and final fate.
However, usually we are presented with an inverted problem: given an observation of a
meteor -- its position in the sky, angular velocity, apparent luminosity and in best possible case also
a meteorite found on the surface -- we want to find the orbital and body characteristics of the original meteoroid.

This can be naturally extended to observations of meteor showers or storms,
where selection bias comes into play: some meteors are not as easy to detect as other
and what we see is not all that is actually there. With the knowledge of the characteristics
of the detection system we are nonetheless able to quantify and correct these effects.

A popular approach to problems like this is to harness the available computational
power and let computer programs do all the work. Several models of vastly different precision and complexity
can be designed and implemented, which predict the behaviour of meteors with varying success.

%What is perhaps even more important, these techniques and algorithms can be transposed to other fields
%of astronomy or observational science in general. One such use is in asteroid observation,
%where proper de-biasing of observational datasets is crucial to being able to determine
%how many potentially hazardous asteroids there are.

I took this rigorous thesis as an opportunity to investigate the foundations
of such models, and to discover if there is some room for improvement in this subfield or meteor science.

\hfill Martin Baláž

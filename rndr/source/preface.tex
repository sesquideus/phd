\Epigraph[0.4]{
    I gazed into the abyss \\
    and the abyss gazed back.
}{
    Friedrich Nietzsche, paraphrase
}


The problem of determining the exact

While the theoretical foundation of the process and the differential equations involved
are not hard -- in terms of mathematics involved -- and could in principle already be fully understood by Newton,
precise modelling of atmospheric entry remains a fairly difficult problem due to its inherent complexity.
There are many parameters that cannot be easily measured or cannot be known at the time when they are relevant.

Yet modelling of these phenomena, either with the purpose of finding meteorites
or determining the total influx of particles from outer space
still includes lots of unknowns or approximations that are not always justified.

The fundamental problem of meteor astronomy can be then stated as follows:

Given the position, velocity, mass, shape and chemical composition of a meteoroid,
and knowing the state of the Earth's atmosphere, calculate its trajectory, luminosity and final fate.

However, usually we are presented with an inverted problem: given an observation of a
meteor -- its position in the sky, angular velocity, apparent luminosity and sometimes
a meteorite found on the surface -- find the orbital and body characteristics of the original meteoroid.

This can be naturally extended to observations of meteor showers or storms,
where selection bias comes into play: some meteors are not as easy to detect as other
and what we see is not all that is actually there. With the knowledge of the characteristics
of the detection system we are nonetheless able to quantify these effects and estimate
corrections which, when applied to our datasets, approximate the true meteor population.

Using modern-day computational techniques it is then possible to use simulation-based optimization
techniques to invert the problem, and estimate the 

models of variable precision and complexity

observational science, where experiments cannot be performed at will.

What is perhaps even more important, these techniques and algorithms can be transposed to other fields
of astronomy or observational science in general. One such use as asteroid observation 


I took this rigorous thesis as an opportunity to investigate the basis
of these models, and to discover if there is some room for improvement
in this subfield or meteor science.

\hfill Martin Baláž

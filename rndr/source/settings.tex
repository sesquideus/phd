\usepackage[
    paper                   = a4paper,
    left                    = 35mm,
    right                   = 20mm,
    top                     = 25mm,
    bottom                  = 25mm,
    headheight              = 16pt,
    footskip                = 36pt,
]{geometry}

\usepackage{
    mathtools,
    amssymb,
    siunitx,
    xparse,
    enumitem,
    graphicx,
    rotating,
    booktabs,
    cancel,
    tabularx,
    epigraph,
    float,
    etoolbox,
    setspace,
    longtable,
    subcaption,
    csquotes,
}

\usepackage[
    pdfpagelabels,
    hidelinks,
    breaklinks              = true,
]{hyperref}
\usepackage[xindy,nonumberlist]{glossaries}
\usepackage[many]{tcolorbox}
\usepackage[all]{nowidow}                                       % Prevent widows and orphans (single line at the beginning or end of page)
\usepackage[bottom]{footmisc}
\usepackage{fancyhdr}                                           % Fancy headers
\usepackage{titlesec}
\usepackage[titletoc, title, toc, page]{appendix}
\usepackage[
    format                  = hang,
    labelfont               = {small, stretch=1.2},
    textfont                = {small, stretch=1.3},
    parskip                 = 7pt,
]{caption}
\usepackage[super]{nth}

%\usepackage[f]{esvect}
\usepackage[
    final
]{pdfpages}
\usepackage[noabbrev]{cleveref}
\usepackage[
    colorinlistoftodos,
    prependcaption,
    textsize                = scriptsize,
]{todonotes}\reversemarginpar
\usepackage[MnSymbol]{mathspec}                                 % includes fontspec
\usepackage{
    polyglossia,
    xunicode,
}
\setdefaultlanguage{english}
\usepackage[nottoc]{tocbibind}

\usepackage[
    backend         = biber,
    style           = iso-authoryear,
    natbib          = true,
    maxcitenames    = 2,
]{biblatex}

\addbibresource{../kvik.bib}

% Hack XeLaTeX loader to enable 256 alphabets instead of 16
\makeatletter
    \def\new@mathgroup{\alloc@8\mathgroup\mathchardef\@cclvi}
    \patchcmd{\document@select@group}{\sixt@@n}{\@cclvi}{}{}
    \patchcmd{\select@group}{\sixt@@n}{\@cclvi}{}{}
\makeatother

\defaultfontfeatures{
    Mapping         = tex-text,
    Scale           = MatchLowercase,
    Ligatures       = TeX
}

\newfontfamily{\semibold}{Adobe Garamond Pro Semibold}
\newfontfamily{\greek}{Minion Pro Italic}
\setallmainfonts[
    Path            = fonts/ ,
    Extension       = .otf ,
    UprightFont     = *-r ,
    BoldFont        = *-b ,
    ItalicFont      = *-i ,
    BoldItalicFont  = *-bi
]{adobegaramondpro}
\setmonofont[
    Path            = fonts/ ,
    Extension       = .ttf ,
    UprightFont     = *-r ,
    BoldFont        = *-b ,
    ItalicFont      = *-i ,
    BoldItalicFont  = *-bi
]{ubuntumono}
\setsansfont[
    Path            = fonts/ ,
    Extension       = .ttf ,
    UprightFont     = *-r ,
    BoldFont        = *-b ,
    ItalicFont      = *-i ,
    BoldItalicFont  = *-bi
]{verdana}
\setmathsfont(Digits,Latin,Greek){Minion Pro}
\setmathrm{Minion Pro}
\setmonofont{Consolas}


\titleformat{\paragraph}[hang]{\semibold}{}{0pt}{}

\NewDocumentCommand{\Sun}{}{}
\NewDocumentCommand{\Earth}{}{}

\DeclareSIUnit\au{AU}
\DeclareSIUnit\pixel{px}
\DeclareSIUnit\lightyear{ly}
\DeclareSIUnit\parsec{pc}
\DeclareSIUnit\earthmass{M_{\earth}}
\DeclareSIUnit\speedoflight{c}
\DeclareSIUnit\foe{foe}
\DeclareSIUnit\year{yr}
\DeclareSIUnit\eur{€}
\DeclareSIUnit\solarmass{M_{\astrosun}}
\DeclareSIUnit\solarluminosity{L_{\astrosun}}
\DeclareSIUnit{\byte}{B}


/home/kvik/dgs/core/tex/math.tex

\numberwithin{equation}{section}

\setcounter{tocdepth}{2}
\setcounter{secnumdepth}{2}

\setlength\bibitemsep{3mm}
\setlength{\parindent}{0cm}
\setlength{\parskip}{3mm}
\setlength{\abovedisplayskip}{0mm}
\setlength{\belowdisplayskip}{0mm}
\setlength{\abovedisplayshortskip}{0mm}
\setlength{\belowdisplayshortskip}{5mm}
\setlength{\epigraphwidth}{0.400\textwidth}
\setlength{\afterepigraphskip}{6mm}
\setlength{\textfloatsep}{10mm}
\setlength{\intextsep}{10mm}
\setlength{\belowcaptionskip}{0mm}
\RenewDocumentCommand{\epigraphsize}{}{\footnotesize}

\setstretch{1.5}
\renewcommand{\arraystretch}{1.7}
\renewcommand{\chaptername}{}

\fancypagestyle{title}{
    \fancyhf{}
    \RenewDocumentCommand{\headrulewidth}{}{0pt}
    \RenewDocumentCommand{\footrulewidth}{}{0pt}
    \fancyfoot[LE,RO]{\thepage}
}

\hyphenpenalty=1000
\tolerance=2000
\emergencystretch=10pt

\fancypagestyle{main}{
    \pagestyle{title}
    \RenewDocumentCommand{\headrulewidth}{}{0.5pt}
    \RenewDocumentCommand{\sectionmark}{m}{\markright{\thesection\ ##1}}
    \fancyhead[LO,RE]{\small{\nouppercase{\leftmark}}}
    \fancyhead[LE,RO]{\small{\textit{\nouppercase{\rightmark}}}}
}

\fancypagestyle{plain}{
    \pagestyle{title}
}

\fancypagestyle{bibliography}{
    \pagestyle{main}
    \fancyhead[R]{}
}

% Set list environments

\setlist[enumerate]{
    itemsep             = -1mm,
}

\setlist[description]{
    style               = multiline,
    labelindent         = 6mm,
    leftmargin          = 25mm,
    itemsep             = 0mm,
    font                = \normalfont\semibold,
}

\setlist[itemize]{
    topsep              = 0mm,
    itemsep             = 0mm,
}

\newcounter{example}
\NewDocumentEnvironment{example}{O{}}{
    \vspace{4mm}
    \colorlet{colexam}{black} % Global example color
    \newtcolorbox[use counter=example]{examplebox}{
        empty,% Empty previously set parameters
        title={\raisebox{5mm}{\kern1.5pt \ifstrempty{#1}{Example}{#1}}},% use \thetcbcounter to access the testexample counter text
        % Attaching a box requires an overlay
        attach boxed title to top left,
           % Ensures proper line breaking in longer titles
           minipage boxed title,
        % (boxed title style requires an overlay)
        boxed title style={empty,size=minimal,toprule=0pt,top=4pt,left=3mm,overlay={}},
        coltitle=colexam,fonttitle=\bfseries,
        before=\par\medskip\noindent,parbox=false,boxsep=0pt,left=3mm,right=0mm,top=2pt,breakable,pad at break=0mm,
           before upper=\csname @totalleftmargin\endcsname0pt, % Use instead of parbox=true. This ensures parskip is inherited by box.
        % Handles box when it exists on one page only
        overlay unbroken={\draw[colexam,line width=.5pt] ([xshift=-0pt]title.north west) -- ([xshift=-0pt]frame.south west); },
        overlay first={\draw[colexam,line width=.5pt] ([xshift=-0pt]title.north west) -- ([xshift=-0pt]frame.south west); },
        overlay middle={\draw[colexam,line width=.5pt] ([xshift=-0pt]frame.north west) -- ([xshift=-0pt]frame.south west); },
        overlay last={\draw[colexam,line width=.5pt] ([xshift=-0pt]frame.north west) -- ([xshift=-0pt]frame.south west); },%
    }
    \begin{examplebox}
}{%
    \end{examplebox}\endlist%
}

\makeatletter
% Render percent sign with nice font, not ugly Computer modern
    \mathcode`\%="7025

% Fixes mathspec bug -- URL numbers are rendered with wrong font
    \ernewcommand\eu@MathPunctuation@symfont{Latin:m:n}
    \DeclareMathSymbol{,}{\mathpunct}{\eu@MathPunctuation@symfont}{`,}
    \DeclareMathSymbol{.}{\mathord}{\eu@MathPunctuation@symfont}{`.}
    \DeclareMathSymbol{<}{\mathrel}{\eu@MathPunctuation@symfont}{`<}
    \DeclareMathSymbol{>}{\mathrel}{\eu@MathPunctuation@symfont}{`>}
    \DeclareMathSymbol{/}{\mathord}{\eu@MathPunctuation@symfont}{`/}
    \DeclareMathSymbol{;}{\mathpunct}{\eu@MathPunctuation@symfont}{`;}
    \XeTeXDeclareMathSymbol{^^^^2026}{\mathinner}{\eu@MathPunctuation@symfont}{"2026}[\mathellipsis]
    \DeclareMathSymbol{0}{\mathalpha}{\eu@DigitsArabic@symfont}{`0}
    \DeclareMathSymbol{1}{\mathalpha}{\eu@DigitsArabic@symfont}{`1}
    \DeclareMathSymbol{2}{\mathalpha}{\eu@DigitsArabic@symfont}{`2}
    \DeclareMathSymbol{3}{\mathalpha}{\eu@DigitsArabic@symfont}{`3}
    \DeclareMathSymbol{4}{\mathalpha}{\eu@DigitsArabic@symfont}{`4}
    \DeclareMathSymbol{5}{\mathalpha}{\eu@DigitsArabic@symfont}{`5}
    \DeclareMathSymbol{6}{\mathalpha}{\eu@DigitsArabic@symfont}{`6}
    \DeclareMathSymbol{7}{\mathalpha}{\eu@DigitsArabic@symfont}{`7}
    \DeclareMathSymbol{8}{\mathalpha}{\eu@DigitsArabic@symfont}{`8}
    \DeclareMathSymbol{9}{\mathalpha}{\eu@DigitsArabic@symfont}{`9}
    \DeclareMathSymbol{,}{\mathpunct}{\eu@MathPunctuation@symfont}{`,}
\makeatother

\RenewDocumentCommand{\vec}{m}{\BoldVector{#1}}

